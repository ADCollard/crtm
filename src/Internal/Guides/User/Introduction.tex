\chapter{Introduction}
%=====================


\section{Conventions}
%====================
\label{sec:conventions}
The following are conventions that have been adhered to in the current release of the CRTM framework. They are guidelines intended to make understanding the code at a glance easier, to provide a recognisable ``look and feel'', and to minimise name space clashes.



\subsection{Naming of Structure Types and Instances of Structures}
%-----------------------------------------------------------------

The derived data type, or structure\footnote{The terms ``derived type'' and ``structure'' are used interchangably in this document.} type, naming convention adopted for use in the CRTM is, 

\hspace{0.5cm}\f{[CRTM\_]}\textit{name}\f{\_type} 

where \textit{name} is an identifier that indicates for what a structure is to be used. All structure type names are suffixed with ``\f{\_type}'' and CRTM-specific structure types are prefixed with ``\f{CRTM\_}''. Some examples are,

\begin{alltt}
  \hyperref[sec:atmosphere_structure]{CRTM_Atmosphere_type}
  \hyperref[sec:rtsolution_structure]{CRTM_RTSolution_type}\end{alltt}

An instance of a structure is then referred to via its \textit{name}, or some sort of derivate of its \textit{name}. Some structure declarations examples are,

\begin{alltt}
  TYPE(\hyperref[sec:atmosphere_structure]{CRTM_Atmosphere_type}) :: atm, atm_K
  TYPE(\hyperref[sec:rtsolution_structure]{CRTM_RTSolution_type}) :: rts, rts_K\end{alltt}

where the K-matrix structure variables are identified with a ``\f{\_K}'' suffix. Similarly, tangent-linear and adjoint variables are suffixed with ``\f{\_TL}'' or ``\f{\_AD}'' respectively.



\subsection{Naming of Definition Modules}
%----------------------------------------

Modules containing structure type definitions are termed \textit{definition modules}. These modules contain the actual structure definitions as well as various utility procedures that operate on the structure of the designated type. The naming convention adopted for definition modules in the CRTM is, 

\hspace{0.5cm}\f{[CRTM\_]}\textit{name}\f{\_Define} 

where, as with the structure type names, all definition module names are suffixed with ``\f{\_Define}'' and CRTM-specific definition modules are prefixed with ``\f{CRTM\_}''. Some examples are,

\begin{alltt}
  CRTM_Atmosphere_Define
  CRTM_RTSolution_Define\end{alltt}

The actual source code files for these modules have the same name with a ``\f{.f90}'' suffix.



\subsection{Standard Definition Module Procedures}
%-------------------------------------------------

The definition modules for the user-accessible CRTM structures (\Atmosphere, \Cloud, \Aerosol, \Surface, \Geometry, \RTSolution, and \Options) contain a standard set of procedures for use with the structure being defined. The naming convention for these procedures is,

\hspace{0.5cm}\f{CRTM\_}\textit{name}\f{\_}\textit{action}

where the available default actions for each procedure are listed in table \ref{tab:definition_module_default_procedures}. This is not an exhaustive list but procedures for the actions listed in table \ref{tab:definition_module_default_procedures} are guaranteed to be present.

Note, however, that the \ChannelInfo structure does \emph{not} have any I/O procedures available for it. This is to ensure that the \ChannelInfo structure can only be populated during initialization of the CRTM.

\begin{table}[htp]
  \centering
  \caption{Default action procedures available in structure definition modules. $^{\dagger}$ I/O functions not available for the \ChannelInfo{} structure.}
  \begin{tabular}{r c l}
    \hline\\[-0.1cm]
    \sffamily\textbf{Action} & \sffamily\textbf{Type} & \sffamily\textbf{Description} \\
    \hline\hline\\[-0.2cm]
    \texttt{OPERATOR(==)}             & Elemental function   & Tests the equality of two structures. \\
    \texttt{Associated}               & Elemental function   & Tests if the structure components have been allocated. \\
    \texttt{Destroy}                  & Elemental subroutine & Deallocates any allocated structure components. \\
    \texttt{Create}                   & Elemental subroutine & Allocates any allocatable structure components. \\
    \texttt{Inspect}                  & Subroutine           & Displays structure contents to \texttt{stdout}. \\
    \texttt{InquireFile}$^{\dagger}$  & Function             & Inquires an existing file for dimensions. \\
    \texttt{WriteFile}$^{\dagger}$    & Function             & Write an instance of a structure to file. \\
    \texttt{ReadFile}$^{\dagger}$     & Function             & Loads an instance of a structure with data read from file. \\
  \hline
  \end{tabular}
  \label{tab:definition_module_default_procedures}
\end{table}

Some examples of these procedure names are,

\begin{alltt}
  CRTM_Atmosphere_Associated
  CRTM_Surface_Inspect
  CRTM_Geometry_WriteFile
  CRTM_RTsolution_Destroy\end{alltt}

The relational operator, \f{==}, is implemented via an overloaded \f{Equal} action procedure, as is shown below for the \f{Atmosphere} structure,

\begin{alltt}
  INTERFACE OPERATOR(==)
    MODULE PROCEDURE CRTM_Atmosphere_Equal
  END INTERFACE OPERATOR(==)\end{alltt}

For a complete list of the definition module procedures for use with the publicly available structures, see section \ref{sec:structure_and_interface_definition}.



\subsection{Naming of Application Modules}
%-----------------------------------------

Modules containing the routines that perform the calculations for the various components of the CRTM are termed \textit{application modules}. The naming convention adopted for application modules in the CRTM is, 

\hspace{0.5cm}\f{CRTM\_}\textit{name}

Some examples are,

\begin{alltt}
  CRTM_AtmAbsorption
  CRTM_SfcOptics
  CRTM_RTSolution\end{alltt}

However, in this case, \textit{name} does not necessarilty refer just to a structure type. Separate application modules are used as required to split up tasks in manageable (and easily maintained) chunks. For example, separate modules have been provided to contain the cloud and aerosol optical property retrieval; similarly separate modules handle different surface types for different instrument types in computing surface optics.

Again, the actual source code files for these modules have the same name with a ``\f{.f90}'' suffix. Note that not all definition modules have a corresponding application module since some structures (e.g. \SpcCoeff{} structures) are simply data containers.



\section{Components}
%===================

The CRTM is designed around three broad categories: atmospheric optics, surface optics and radiative transfer.



\subsection{Atmospheric Optics}
%------------------------------

(\AtmOptics) This category includes computation of the absorption by atmospheric gases (\AtmAbsorption) and scattering and absorption by both clouds (\CloudScatter) and aerosols (\AerosolScatter).

The gaseous absorption component computes the optical depth of the absorbing constituents in the atmosphere given the pressure, temperature, water vapour, ozone, and -- for the hyperspectral infrared sensors -- trace gas\footnote{CO\subscript{2}, CH\subscript{4}, CO, and N\subscript{2}O} profiles.

The scattering component simply interpolates look-up-tables (LUTs) of optical properties -- such as mass extinction coefficient and single scatter albedo -- for cloud and aerosol types that are then used in the radiative transfer component. See tables \ref{tab:cloud_type} and \ref{tab:aerosol_type} for the current valid cloud and aerosol types, respectively, that are valid in the CRTM.



\subsection{Surface Optics}
%------------------------------

(\SfcOptics) This category includes the computation of surface emissivity and reflectivity for four main surface categories (land, water, snow, and ice). The surface optics models are implemented differently for different surface categories based upon the spectral region of a sensor. Thus, each surface category may have a number of surface types associated with it. This is fully discussed in section \ref{sec:fill_step-surface}.



\subsection{Radiative Transfer Solution}
%---------------------------------------
(\RTSolution) This category takes the \AtmOptics{} and \SfcOptics{} data and solves the radiative transfer problem in either clear or scattering atmospheres.


\section{Models}
%===============
The CRTM is composed of four models: a forward model, a tangent-linear model, an adjoint model, and a K-matrix model. These can be represented as shown in equations \ref{eqn:fwd} to \ref{eqn:k}.
\begin{subequations}
  \begin{eqnarray}
    \mathbf{T_{B}},\mathbf{R} &=& \mathbf{F}(\mathbf{T},\mathbf{q},T_{s},...)\label{eqn:fwd}\\\nonumber\\
    \mathbf{\delta{T_{B}}},\mathbf{\delta{R}} &=& \mathbf{H}(\mathbf{T},\mathbf{q},T_{s},...\mathbf{\delta{T}},\mathbf{\delta{q}},\delta{T_{s}},...)\label{eqn:tl}\\\nonumber\\
    \mathbf{\dstar{T}},\mathbf{\dstar{q}},\dstar{T_{s}},... &=& \mathbf{H^T}(\mathbf{T},\mathbf{q},T_{s},...\mathbf{\dstar{T_{B}}})\label{eqn:ad}\\\nonumber\\
    \mathbf{\dstar{T}}_l,\mathbf{\dstar{q}}_l,\dstar{T_{s,l}},... &=& \mathbf{K}(\mathbf{T},\mathbf{q},T_{s},...\mathbf{\dstar{T_{B}}})\textrm{ for }l=1,2,...,L\label{eqn:k}
  \end{eqnarray}
\end{subequations}
Here $\mathbf{F}$ is the forward operator that, given the atmospheric temperature and absorber profiles ($\mathbf{T}$ and $\mathbf{q}$), surface temperature ($T_{s}$), etc., produces a vector of channel brightness temperatures ($\mathbf{T_{B}}$) and radiances ($\mathbf{R}$).

The tangent-linear operator, $\mathbf{H}$, represents a linearisation of the forward model about $\mathbf{T}$, $\mathbf{q}$, $T_{s}$, etc. and when also supplied with perturbations about the linearisation point (quantities represented by the $\delta$'s) produces the expected perturbations to the brightness temperature and channel radiances.

The adjoint operator, $\mathbf{H^T}$, is simply the transpose of the tangent-linear operator and produces gradients (the quantities represented by the $\dstar$'s). It is worth noting that, in the CRTM, these adjoint gradients are accumulated over channel and thus do not represent channel-specific Jacobians.

The K-matrix operator\footnote{The term K-matrix is used because references to this operation in the literature commonly use the symbol $\mathbf{K}$}, $\mathbf{K}$, is effectively the same as the adjoint but with the results preserved by channel (indicated via the subscript $l$). In the CRTM, the adjoint and K-matrix results are related by,
\begin{equation}
  \dstar{x} = \sum_{l=1}^{L}\dstar{x}_l
\end{equation}
Thus, the K-matrix results are the derivatives of the diagnostic variables with respect to the prognostic variables, e.g.
\begin{equation}
  \dstar{x_{l}} = \frac{\partial{T_{B,l}}}{\partial{x}}
\end{equation}
Typically, only the forward or K-matrix models are used in applications. However, the intermediate models are generated and retained for maintenance and testing purposes. Any changes to the CRTM forward model are translated to the tangent-linear model and the latter tested against the former. When the tangent-linear model changes have been verified, the changes then translated to the adjoint model and, as before, the latter is tested against the former. This process is repeated for the adjoint-to-K-matrix models also.


\section{Design Framework}
%=========================
This document is not really the place to fully discuss the design framework of the CRTM, so it will only be briefly mentioned here. Where appropriate, different physical processes are isolated into their own modules. The CRTM interfaces presented to the user are, at their core, simply drivers for the individual parts. This is shown schematically in the forward and K-matrix model flowcharts of figure \ref{fig:fwd_k_flowchart}.

A fundamental tenet of the CRTM design is that each component define its own structure definition and application modules to facilitate independent development of an algorithm outside of the mainline CRTM development. By isolating different processes, we can more easily identify requirements for an algorithm with a view to minimise or eliminate potential software conflicts and/or redundancies. The end result sought via this approach is that components developed by different groups can more easily be added into the framework leading to faster implementation of new science and algorithms.

\begin{figure}[htp]
  \centering
  \input{graphics/Flowcharts/CRTM_Flowcharts.pstex_t}
  \caption{Flowchart of the CRTM Forward and K-Matrix models.}
  \label{fig:fwd_k_flowchart}
\end{figure}

