\section{Code Testing}
%=====================
\label{sec:code_testing}


\subsection{Simple test}
%-----------------------

To verify the implementation of the various overlap assumptions, the simple example shown in figure 1 of \citet{HoganIllingworth_2000}, here shown in figure \ref{fig:cloud_overlap_example}, was used. The result of the computation is shown in figure \ref{fig:cloud_overlap_comparison}.

\begin{figure}[H]
  \caption{Simple example from figure 1 of Hogan and Illingworth [2000].}
  \label{fig:cloud_overlap_example}
  \centering
  \includegraphics[trim={20 20 10 20},clip,scale=0.4]{graphics/hogan_and_illingworth_fig1.eps}
\end{figure}

\begin{figure}[H]
  \caption{Result of the cloud overlap assumptions for comparison with the figure 1 schematic in Hogan and Illingworth [2000].}
  \label{fig:cloud_overlap_comparison}
  \centering
  \includegraphics[trim={40 30 10 50},clip,scale=0.7]{graphics/cloud_overlap_comparison.eps}
\end{figure}

The final, near-surface cloud cover value is equivalent to the total cloud cover (TCC). As expected, the maximum overlap assumption results in a minimum cloud cover, the random overlap assumption maximises the cloud cover, and the maximum-random assumption falls between those two extremes. 


\subsection{Testing using ECMWF5K profile set}
%---------------------------------------------

To more fully exercise test the cloud fraction capability, the ECMWF 5K profile set \citep{ECMWF5K_profile_set} was used. The dataset individual cloud water content spans are shown in figure \ref{fig:ECMWF5K.wcspan}.

\begin{figure}[htp]
  \caption{The cloud water content span of the ECMWF 5K profile set. The dashed line is the average. Minimum value is zero.}
  \label{fig:ECMWF5K.wcspan}
  \centering
  \includegraphics[trim={20 0 20 20},clip,scale=0.8]{graphics/wcspan.eps}
\end{figure}

The ECMWF5K dataset does not contain cloud fraction or a variable cloud particle effective radius for the various cloud types.

A cloud fraction profile was generated for each cloud type by scaling a random number, $A \in [0,1)$, by the fractional cloud water content, $q$. So, for the $n^{th}$ cloud in the $k^{th}$ atmospheric layer,
\begin{equation}
  f_{n,k} = \left\{
  \begin{array}{cl}
    \frac{\displaystyle{A}}{\displaystyle{c}}\frac{\displaystyle{q_{n,k}}}{\displaystyle{\max(\mathbf{q}_n)}} & \text{if } q_{n,k} > 10^{-6}\\\\
    0 & \text{otherwise}
  \end{array} \right.
  \label{eqn:f_individual}
\end{equation}
where $c$ is an additional variable used to control the maximum cloud fraction in any profile. Values of 2 and 5 were used in testing.

The total cloud fraction in the $k^{th}$ layer, $f_k$, was then simply set to the maximum value of the individual cloud fractions in that layer,
\begin{equation}
  f_k = \max_{n=1,N}f_{n,k}
  \label{eqn:f_total}
\end{equation}

Figure \ref{fig:ECMWF5K.cfspan} shows the resulting span of generating cloud fraction profiles for the set.

\begin{figure}[htp]
  \caption{The cloud fraction span generated for the ECMWF 5K profile set. The dashed line is the average. Minimum value is zero.}
  \label{fig:ECMWF5K.cfspan}
  \centering
  \includegraphics[trim={20 0 20 50},clip,scale=0.8]{graphics/cfspan.eps}
\end{figure}

Similarly for the cloud effective radius. A fixed ``reference'' value, $R_{\mathit{eff}}$, was assigned for each cloud type, $n$. To generate a variable effective radius for each layer, this reference value was scaled by the individual cloud water contents, $q_n$, at each layer, $k$,
\begin{equation}
  r_{n,k} = \left\{
  \begin{array}{cl}
    R_{\mathit{eff},n} \frac{\displaystyle{q_{n,k}}}{\displaystyle{\max(\mathbf{q_n})}} & \text{if } q_{n,k} > 10^{-6}\\\\
    0 & \text{otherwise}
  \end{array} \right.
  \label{eqn:r_individual}
\end{equation}

This gave the spread of effective radii for the dataset cloud types shown in figure \ref{fig:ECMWF5K.effrspan}.

\begin{figure}[htp]
  \caption{The effective radius span generated for the ECMWF 5K profile set. The dashed line is the average. Minimum value is zero.}
  \label{fig:ECMWF5K.effrspan}
  \centering
  \includegraphics[trim={20 0 20 20},clip,scale=0.8]{graphics/effrspan.eps}
\end{figure}

The fractional cloudy atmosphere data was then fed into the CRTM, once for each overlap/cloud cover  methodology implemented. A selection of some generated cloud cover profiles for maximum cloud fractions of 0.5 and 0.2 (for $c = 2$ and $c = 5$ respectively per equation \ref{eqn:f_individual}) is shown in figure \ref{fig:cc_profile_selection}.

\begin{figure}[H]
  \caption{Selection of computed cloud cover profiles from the ECMWF5K profile set for two different maximum cloud fractions. \textbf{(Left)} Maximum cloud fraction of 0.5. \textbf{(Right)} Maximum cloud fraction of 0.2. }
  \label{fig:cc_profile_selection}
  \centering
  \begin{tabular}{c c}
    \includegraphics[scale=0.3]{graphics/profiles/c-eq-2.cc_profile-0001.eps} &
    \includegraphics[scale=0.3]{graphics/profiles/c-eq-5.cc_profile-0001.eps}\\
    \includegraphics[scale=0.3]{graphics/profiles/c-eq-2.cc_profile-2001.eps} &
    \includegraphics[scale=0.3]{graphics/profiles/c-eq-5.cc_profile-2001.eps}\\
    \includegraphics[scale=0.3]{graphics/profiles/c-eq-2.cc_profile-4501.eps} &
    \includegraphics[scale=0.3]{graphics/profiles/c-eq-5.cc_profile-4501.eps}
  \end{tabular}
\end{figure}

The frequency distributions of the computed cloud cover for the different methodologies are shown in figure \ref{fig:cloud_cover_frequency} for the maximum cloud fractions of 0.5 (cloudier) and 0.2 (less cloudy). The random overlap assumption consistently yields a much higher total cloud cover, in most cases producing the equivalent of an overcast profile.

\begin{figure}[H]
  \caption{Frequency distribution of total cloud cover for the different overlap/cloud cover assumptions using the ECMWF5K set. \textbf{(Top)} Maximum cloud fraction of 0.5. \textbf{(Bottom)} Maximum cloud fraction of 0.2. }
  \label{fig:cloud_cover_frequency}
  \centering
  \begin{tabular}{c}
    \sffamily\textbf{Maximum cloud fraction of 0.5} \\
    \includegraphics[trim={110 30 60 59},clip,scale=0.5]{graphics/cc_stats/c-eq-2.cc_stats.eps} \\\\
    \sffamily\textbf{Maximum cloud fraction of 0.2} \\
    \includegraphics[trim={110 30 60 59},clip,scale=0.5]{graphics/cc_stats/c-eq-5.cc_stats.eps}
  \end{tabular}
\end{figure}

To get an idea of the relative impacts of the various cloud cover methodologies, the ECMWF5K profile set was used to compute brightness temperatures for the NPP ATMS and CrIS(399) sensors for each method, and the results for the overlap assumption methods (maximum, random, maximum-random) differenced from the averaging approach of \citet{Geer_2009}. Two runs were made, using maximum cloud fractions of 0.5 and 0.2.

The brightness temperature residuals for the ATMS and CrIS are shown in figures \ref{fig:atms_npp.tb_stats} and \ref{fig:cris399_npp.tb_stats} respectively. The most evident result is that the random overlap assumption produces brightness temperatures quite different from the other methods, as well as being relatively insensitive to the maximum cloud fraction in a profile (due to the cloud cover quickly ``saturating''). The other methods behave as expected with the residuals decreasing roughly similar with the maximum cloud fraction.

\begin{figure}[H]
  \caption{NPP ATMS average brightness temperature differences for cloud cover methods compared to the averaging approach using the ECMWF5K profile set. The error bars represent $\pm$ one standard deviation. \textbf{(Top)} Maximum cloud fraction of 0.5. \textbf{(Bottom)} Maximum cloud fraction of 0.2. }
  \label{fig:atms_npp.tb_stats}
  \centering
  \begin{tabular}{c}
    \\
    \sffamily\textbf{Maximum cloud fraction of 0.5} \\
    \includegraphics[trim={110 30 60 59},clip,scale=0.5]{graphics/tb_stats/c-eq-2.atms_npp.tb_stats.eps} \\\\
    \sffamily\textbf{Maximum cloud fraction of 0.2} \\
    \includegraphics[trim={110 30 60 59},clip,scale=0.5]{graphics/tb_stats/c-eq-5.atms_npp.tb_stats.eps}
  \end{tabular}
\end{figure}

\begin{figure}[H]
  \caption{NPP CrIS average brightness temperature differences (399 channel subset) for cloud cover methods compared to the averaging approach using the ECMWF5K profile set. The thick line is the mean, the thin lines represent $\pm$ one standard deviation. \textbf{(Top)} Maximum cloud fraction of 0.5. \textbf{(Bottom)} Maximum cloud fraction of 0.2. }
  \label{fig:cris399_npp.tb_stats}
  \centering
  \begin{tabular}{c}
    \\
    \sffamily\textbf{Maximum cloud fraction of 0.5} \\
    \includegraphics[trim={110 30 60 59},clip,scale=0.5]{graphics/tb_stats/c-eq-2.cris399_npp.tb_stats.eps} \\\\
    \sffamily\textbf{Maximum cloud fraction of 0.2} \\
    \includegraphics[trim={110 30 60 59},clip,scale=0.5]{graphics/tb_stats/c-eq-5.cris399_npp.tb_stats.eps}
  \end{tabular}
\end{figure}


\subsection{Finite-difference and K-matrix Jacobian comparisons}
%---------------------------------------------------------------
\label{sec:fd_vs_km}

This section shows some comparisons of Jacobians computed using finite differences, via the CRTM forward model, with those produced by the CRTM K-matrix model.

The finite difference $T_B$ Jacobian for an atmospheric state variable, $x$, was constructed by perturbing the state variable in a single layer to produce perturbed radiances,

\begin{equation}
  J_{FD,k} = \frac{T_B(x_k + \Delta x) - T_B(x_k - \Delta x)}{2\Delta x}\qquad\mbox{for every}\quad k = 1,\ldots,K
  \label{eqn:fd_jacobian}
\end{equation}
for each channel and where $x$ is temperature, gaseous absorber concentration, or cloud water content. The perturbations, $\Delta x$, applied at each layer were 0.5K for temperature and 2.5\% of the layer value for the other state variables.

Both the finite difference and K-matrix Jacobians were computed for the NPP ATMS instrument using the ECMWF5K profile set for each of the cloud cover computation methods. Brightness temperature Jacobian comparison plots for ATMS channel 2, profile 1 using the maximum overlap, random overlap, maximum-random overlap, and the cloud water amount weighted average method \citep{Geer_2009} to compute cloud cover are shown in figures \ref{fig:fractional.maximum.atms_npp.profile1.chidx2.TB-atmjac} to \ref{fig:fractional.average.atms_npp.profile1.chidx2.TB-atmjac}. In all cases the unperturbed forward model brightness temperatures are shown in the bottom panel for all channels (note that the Jacobian x-axis and $T_B$ y-axis ranges for these plots differs across the figures).

For comparison, the clear sky result for the same channel/profile is shown in figure \ref{fig:clear.atms_npp.profile1.chidx2.TB-atmjac}.

The first thing to note about the Jacobian comparisons is that the finite difference and K-matrix versions agree very well. The exceptions are the the cloud water content Jacobians when the averaging method is used to compute cloud cover (figure \ref{fig:fractional.maximum.atms_npp.profile1.chidx2.TB-atmjac}).

In all cases the temperature Jacobians agree well. Comparison across cloud cover methodology are also good, with the magnitude differences due to the different amount of cloudiness in each case.

Similarly for the water vapour Jacobians. However, when comparing across cloud cover methodologies, the random-overlap case (figure \ref{fig:fractional.random.atms_npp.profile1.chidx2.TB-atmjac}) yields a differently shaped water vapour Jacobian with a maximum magnitude about an order of magnitude less.

As for the cloud water content Jacobians, there is generally good agreement between the finite difference (FD) and K-matrix (KM) Jacobians, both within the run and for the different cloud cover methodologies \emph{except} for the averaging approach (figure \ref{fig:fractional.average.atms_npp.profile1.chidx2.TB-atmjac}). Recall that the averaging approach is the only cloud cover method that uses the cloud water content (see equation \ref{eqn:average_cc}). The most prominent difference between the averaging method FD and KM cloud water content Jacobians is the fact that the KM Jacobians retain a non-zero value even for layers where there are no clouds present. Comparing the cloud water content Jacobians in figure \ref{fig:fractional.average.atms_npp.profile1.chidx2.TB-atmjac} to the cloud water content profiles shown in figure \ref{fig:profile1.cloud_content} it is clear that whenever the cloud water content for a particular cloud type goes to zero the same happens to the respective FD cloud water content Jacobian. The water and rain cloud FD Jacobians go to zero at ~400hPa, the ice cloud FD Jacobian at ~700hPa and 200hPa, and the snow cloud FD Jacobian at ~750hPa and 200hPa. This is not the case for the KM Jacobians, and is particularly apparent above 200hPa where there are no clouds at all yet the KM cloud water content Jacobians plateau to the same value.


\begin{figure}[H]
  \caption{Comparison of NPP ATMS channel 2 finite-difference and K-matrix $T_B$ Jacobians for ECMWF5K profile 1 using the \emph{maximum overlap} method to compute fractional cloud cover. \textbf{(Top)} Temperature, water vapour, and cloud water content Jacobians. The colours for the latter represent \textcolor{green}{\textbf{water}}, \textcolor{magenta}{\textbf{ice}}, \textcolor{blue}{\textbf{rain}}, and \textcolor{orange}{\textbf{snow}} clouds. \textbf{(Bottom)} Channel brightness temperatures. Channel 2 is highlighted by the vertical green line.}
  \label{fig:fractional.maximum.atms_npp.profile1.chidx2.TB-atmjac}
  \centering
  \includegraphics[trim={0 0 20 45},clip,scale=0.5]{graphics/fd_vs_km/fractional.maximum.atms_npp.profile1.chidx2.TB-atmjac.eps}
\end{figure}


\begin{figure}[H]
  \caption{Comparison of NPP ATMS channel 2 finite-difference and K-matrix $T_B$ Jacobians for ECMWF5K profile 1 using the \emph{random overlap} method to compute fractional cloud cover. \textbf{(Top)} Temperature, water vapour, and cloud water content Jacobians. The colours for the latter represent \textcolor{green}{\textbf{water}}, \textcolor{magenta}{\textbf{ice}}, \textcolor{blue}{\textbf{rain}}, and \textcolor{orange}{\textbf{snow}} clouds. \textbf{(Bottom)} Channel brightness temperatures. Channel 2 is highlighted by the vertical green line.}
  \label{fig:fractional.random.atms_npp.profile1.chidx2.TB-atmjac}
  \centering
  \includegraphics[trim={0 0 20 45},clip,scale=0.5]{graphics/fd_vs_km/fractional.random.atms_npp.profile1.chidx2.TB-atmjac.eps}
\end{figure}


\begin{figure}[H]
  \caption{Comparison of NPP ATMS channel 2 finite-difference and K-matrix $T_B$ Jacobians for ECMWF5K profile 1 using the \emph{maximum-random overlap} method to compute fractional cloud cover. \textbf{(Top)} Temperature, water vapour, and cloud water content Jacobians. The colours for the latter represent \textcolor{green}{\textbf{water}}, \textcolor{magenta}{\textbf{ice}}, \textcolor{blue}{\textbf{rain}}, and \textcolor{orange}{\textbf{snow}} clouds. \textbf{(Bottom)} Channel brightness temperatures. Channel 2 is highlighted by the vertical green line.}
  \label{fig:fractional.maxran.atms_npp.profile1.chidx2.TB-atmjac}
  \centering
  \includegraphics[trim={0 0 20 45},clip,scale=0.5]{graphics/fd_vs_km/fractional.maxran.atms_npp.profile1.chidx2.TB-atmjac.eps}
\end{figure}


\begin{figure}[H]
  \caption{Comparison of NPP ATMS channel 2 finite-difference and K-matrix $T_B$ Jacobians for ECMWF5K profile 1 using the \emph{cloud water amount weighted average} method to compute fractional cloud cover. \textbf{(Top)} Temperature, water vapour, and cloud water content Jacobians. The colours for the latter represent \textcolor{green}{\textbf{water}}, \textcolor{magenta}{\textbf{ice}}, \textcolor{blue}{\textbf{rain}}, and \textcolor{orange}{\textbf{snow}} clouds. \textbf{(Bottom)} Channel brightness temperatures. Channel 2 is highlighted by the vertical green line.}
  \label{fig:fractional.average.atms_npp.profile1.chidx2.TB-atmjac}
  \centering
  \includegraphics[trim={0 0 20 45},clip,scale=0.5]{graphics/fd_vs_km/fractional.average.atms_npp.profile1.chidx2.TB-atmjac.eps}
\end{figure}

\begin{figure}[H]
  \caption{Comparison of NPP ATMS channel 2 finite-difference and K-matrix $T_B$ Jacobians for ECMWF5K profile 1 with no clouds. \textbf{(Top)} Temperature and water vapour Jacobians. \textbf{(Bottom)} Clear sky channel brightness temperatures. Channel 2 is highlighted by the vertical green line.}
  \label{fig:clear.atms_npp.profile1.chidx2.TB-atmjac}
  \centering
  \includegraphics[trim={0 0 20 45},clip,scale=0.5]{graphics/fd_vs_km/clear.atms_npp.profile1.chidx2.TB-atmjac.eps}
\end{figure}

\begin{figure}[H]
  \caption{The cloud water contents for ECMWF5K profile 1.}
  \label{fig:profile1.cloud_content}
  \centering
  \begin{tabular}{c c}
    \\
    \quad\sffamily\textbf{Water cloud} & \quad\sffamily\textbf{Ice cloud} \\
    \includegraphics[trim={350 30 10 50},clip,scale=0.65]{graphics/fd_vs_km/profile1.water_cloud.eps} & \includegraphics[trim={350 30 10 50},clip,scale=0.65]{graphics/fd_vs_km/profile1.ice_cloud.eps} \\
    \quad\sffamily\textbf{Rain cloud} & \quad\sffamily\textbf{Snow cloud} \\
    \includegraphics[trim={350 30 10 50},clip,scale=0.65]{graphics/fd_vs_km/profile1.rain_cloud.eps} & \includegraphics[trim={350 30 10 50},clip,scale=0.65]{graphics/fd_vs_km/profile1.snow_cloud.eps} \\
  \end{tabular}
\end{figure}


\subsection{Review of forward, tangent-linear, and adjoint averaging code}
%-------------------------------------------------------------------------

So, why does the averaging cloud cover method produce the peculair cloud water content Jacobians seen in section \ref{sec:fd_vs_km}? To answer that let's look at the actual code. Figures \ref{fig:Compute_Average_Overlap_subroutine} to \ref{fig:Compute_Average_Overlap_AD_subroutine} show the forward, tangent-linear, and adjoint procedures of the averaging method to compute fractional cloud cover. For reference, the \f{CRTM\_Cloudcover\_type} object and the internal variable object definitions are shown in appendix \ref{sec:cloudcover_object}, figures \ref{fig:CRTM_CloudCover_object} and \ref{fig:iVar_object} respectively.

The forward procedure implements equation \ref{eqn:average_cc}, repeated here:
\begin{equation}
  CC_{k,ave} = \frac{\displaystyle \sum_{i=1}^k{\left( \textstyle \sum_{n=1}^N{q_{n,i}} \right) f_i}}{\displaystyle \sum_{i=1}^k{\left( \textstyle \sum_{n=1}^N{q_{n,i}} \right)}}\qquad\mbox{for each}\quad k = 1,\ldots,K
  \label{eqn:average_cc_repeat}
\end{equation}
The variable \f{self\%iVar\%lwc} is the value of $\sum_{n=1}^N{q_{n,k}}$ for all $N$ clouds at every layer, $k$. The variables \f{self\%iVar\%cwc\_sum} and \f{self\%iVar\%wc\_sum} represent the numerator and denominator of eqution \ref{eqn:average_cc_repeat} respectively. These summed variables are retained to allow the computation to be ``undone'' in the adjoint code.

The first line of code we'll inspect is the generation of the cloud fraction weighted water content sum, the numerator of equation \ref{eqn:average_cc_repeat}, \f{cwc\_sum},

\begin{alltt}
  self%iVar%cwc_sum(k) = self%iVar%cwc_sum(k-1) + (self%Cloud_Fraction(k) * self%iVar%lwc(k))
\end{alltt}
which we'll recast in equation form where, hopefully, the shortened variable names are clear,
\begin{equation}
  cwc_k = cwc_{k-1} + f_k.lwc_k
  \label{eqn:cwc_sum}
\end{equation}
The tangent-linear form of this is,
\begin{equation}
  \delta cwc_k = \delta cwc_{k-1} + (f_k.\delta lwc_k) + (lwc_k.\delta f_k)
  \label{eqn:cwc_sum_tl}
\end{equation}
Writing equation \ref{eqn:cwc_sum_tl} in matrix form we get,
\begin{equation}
  \left[ 
    \begin{array}{c}
      \delta cwc_k \\
      \delta cwc_{k-1} \\
      \delta lwc_k \\
      \delta f_k
    \end{array}
  \right]^{l} =
  \left[ 
    \begin{array}{cccc}
      0 & 1 & f_k & lwc_k \\
      0 & 1 & 0   & 0 \\
      0 & 0 & 1   & 0 \\
      0 & 0 & 0   & 1
    \end{array}
  \right]
  \left[ 
    \begin{array}{c}
      \delta cwc_k \\
      \delta cwc_{k-1} \\
      \delta lwc_k \\
      \delta f_k
    \end{array}
  \right]^{l-1}
  \label{eqn:cwc_sum_tl_matrix}
\end{equation}
where $l-1$ and $l$ represent the state of the tangent-linear variables before and after assignment. The adjoint form of equation \ref{eqn:cwc_sum_tl_matrix} simply uses the transpose of the matrix to invert to give,
\begin{equation}
  \left[ 
    \begin{array}{c}
      \delta^* cwc_k \\
      \delta^* cwc_{k-1} \\
      \delta^* lwc_k \\
      \delta^* f_k
    \end{array}
  \right]^{l-1} =
  \left[ 
    \begin{array}{cccc}
      0     & 0 & 0 & 0 \\
      1     & 1 & 0 & 0 \\
      f_k   & 0 & 1 & 0 \\
      lwc_k & 0 & 0 & 1
    \end{array}
  \right]
  \left[ 
    \begin{array}{c}
      \delta^* cwc_k \\
      \delta^* cwc_{k-1} \\
      \delta^* lwc_k \\
      \delta^* f_k
    \end{array}
  \right]^{l}
  \label{eqn:cwc_sum_ad_matrix}
\end{equation}
yielding the series of adjoint equations,
\begin{equation}
  \begin{array}{rcrcl}
    \delta^* f_k       & = & lwc_k.\delta^* cwc_k & + & \delta^* f_k       \\
    \delta^* lwc_k     & = &   f_k.\delta^* cwc_k & + & \delta^* lwc_k     \\
    \delta^* cwc_{k-1} & = &       \delta^* cwc_k & + & \delta^* cwc_{k-1} \\
    \delta^* cwc_k     & = &                      & 0 &
  \end{array}
  \label{eqn:cwc_sum_ad}
\end{equation}
which are implemented as seen in figure \ref{fig:Compute_Average_Overlap_AD_subroutine},
\begin{alltt}
self_AD%Cloud_Fraction(k) = self_AD%Cloud_Fraction(k) + (cc_FWD%iVar%lwc(k)       * cwc_sum_AD(k))
lwc_AD(k)                 = lwc_AD(k)                 + (cc_FWD%Cloud_Fraction(k) * cwc_sum_AD(k))
cwc_sum_AD(k-1)           = cwc_sum_AD(k-1)           + cwc_sum_AD(k)
cwc_sum_AD(k)             = ZERO
\end{alltt}
Similarly for the denominator of \ref{eqn:average_cc_repeat}, the cloud water content sum, \f{wc\_sum},
\begin{alltt}
  self%iVar%wc_sum(k) = self%iVar%wc_sum(k-1) + self%iVar%lwc(k)
\end{alltt}
recasting in equation form,
\begin{equation}
  wc_k = wc_{k-1} + lwc_k
  \label{eqn:wc_sum}
\end{equation}
Going straight to the matrix tangent-linear form,
\begin{equation}
  \left[ 
    \begin{array}{c}
      \delta wc_k \\
      \delta wc_{k-1} \\
      \delta lwc_k
    \end{array}
  \right]^{l} =
  \left[ 
    \begin{array}{ccc}
      0 & 1 & 1 \\
      0 & 1 & 0 \\
      0 & 0 & 1
    \end{array}
  \right]
  \left[ 
    \begin{array}{c}
      \delta wc_k \\
      \delta wc_{k-1} \\
      \delta lwc_k
    \end{array}
  \right]^{l-1}
  \label{eqn:wc_sum_tl_matrix}
\end{equation}
we take the transpose of equation \ref{eqn:wc_sum_tl_matrix} for the adjoint form,
\begin{equation}
  \left[ 
    \begin{array}{c}
      \delta^* wc_k \\
      \delta^* wc_{k-1} \\
      \delta^* lwc_k 
    \end{array}
  \right]^{l-1} =
  \left[ 
    \begin{array}{ccc}
      0 & 0 & 0 \\
      1 & 1 & 0 \\
      1 & 0 & 1
    \end{array}
  \right]
  \left[ 
    \begin{array}{c}
      \delta^* wc_k \\
      \delta^* wc_{k-1} \\
      \delta^* lwc_k
    \end{array}
  \right]^{l}
  \label{eqn:wc_sum_ad_matrix}
\end{equation}
yielding the series of adjoint equations,
\begin{equation}
  \begin{array}{rcrcl}
    \delta^* lwc_k    & = & \delta^* wc_k & + & \delta^* lwc_k    \\
    \delta^* wc_{k-1} & = & \delta^* wc_k & + & \delta^* wc_{k-1} \\
    \delta^* wc_k     & = &               & 0 &
  \end{array}
  \label{eqn:wc_sum_ad}
\end{equation}
which are implemented in figure \ref{fig:Compute_Average_Overlap_AD_subroutine} as,
\begin{alltt}
lwc_AD(k)      = lwc_AD(k)      + wc_sum_AD(k)
wc_sum_AD(k-1) = wc_sum_AD(k-1) + wc_sum_AD(k)
wc_sum_AD(k)   = ZERO
\end{alltt}
Up to this point the actual cloud water contents in these equations have been represented by the intermediate $lwc$ variable. The next steps in the adjoint directly impact the cloud water content Jacobians. Starting with the forward model summation to obtain the total layer water content,
\begin{alltt}
DO n = 1, atm%n_Clouds
  self%iVar%lwc = self%iVar%lwc + atm%Cloud(n)%Water_Content(1:n_layers)
END DO
\end{alltt}
which is executed for all $k = 1,\ldots,K$ layers simultaneously as it loops over each of the $N$ cloud types. The line of code doing the actual summing is expressed simply as,
\begin{equation}
  lwc_k = lwc_k + q_{k,n}\qquad\mbox{for each}\quad n = 1,\ldots,N
  \label{eqn:lwc_sum}
\end{equation}
Going directly to the matrix tangent-linear form of equation \ref{eqn:lwc_sum} we get,
\begin{equation}
  \left[ 
    \begin{array}{c}
      \delta lwc_k \\
      \delta q_{k,n}
    \end{array}
  \right]^{l} =
  \left[ 
    \begin{array}{ccc}
      1 & 1 \\
      0 & 1
    \end{array}
  \right]
  \left[ 
    \begin{array}{c}
      \delta lwc_k \\
      \delta q_{k,n}
    \end{array}
  \right]^{l-1}
  \label{eqn:lwc_sum_tl_matrix}
\end{equation}
Taking the transpose of equation \ref{eqn:lwc_sum_tl_matrix} for the adjoint form,
\begin{equation}
  \left[ 
    \begin{array}{c}
      \delta^* lwc_k \\
      \delta^* q_{n,k} 
    \end{array}
  \right]^{l-1} =
  \left[ 
    \begin{array}{ccc}
      1 & 0 \\
      1 & 1
    \end{array}
  \right]
  \left[ 
    \begin{array}{c}
      \delta^* lwc_k \\
      \delta^* q_{n,k}
    \end{array}
  \right]^{l}
  \label{eqn:lwc_sum_ad_matrix}
\end{equation}
giving the adjoint equation,
\begin{equation}
  \delta^* q_{n,k} = \delta^* lwc_k + \delta^* q_{n,k}\qquad\mbox{for each}\quad n = 1,\ldots,N
  \label{eqn:lwc_sum_ad}
\end{equation}
which is implemented in figure \ref{fig:Compute_Average_Overlap_AD_subroutine} as,
\begin{alltt}
DO n = 1, atm_AD%n_Clouds
  atm_AD%Cloud(n)%Water_Content(1:n_layers) = atm_AD%Cloud(n)%Water_Content(1:n_layers) + lwc_AD
END DO
\end{alltt}
and this is where the ``plateau'' value seen in the cloud water amount Jacobians above 200hPa in figure \ref{fig:fractional.average.atms_npp.profile1.chidx2.TB-atmjac} originates. The value of \f{lwc\_AD}, which is determined from equations \ref{eqn:cwc_sum_ad} and \ref{eqn:wc_sum_ad}, is added to each cloud water content Jacobian. This is shown graphically in figure \ref{fig:lwc_q_ad} where the cloud water content Jacobians are shown before and after the \f{lwc\_AD} summation. It is clear the \f{lwc\_AD} variable is the major contributor to the cloud water content Jacobians in the averaging method for computing cloud cover.

\begin{figure}[H]
  \caption{Plot of the cloud water content Jacobian variables before and after the \f{lwc\_AD} profile summation in the \f{Compute\_Average\_Overlap\_AD()} procedure.}
  \label{fig:lwc_q_ad}
  \centering
  \includegraphics[trim={40 20 30 50},clip,scale=0.8]{graphics/lwc_ad.eps}
\end{figure}

The question still remains: is this the correct behaviour, or a bug?

\begin{figure}[H]
  \caption{\f{Compute\_Average\_Overlap()} subroutine.}
  \label{fig:Compute_Average_Overlap_subroutine}
  \lineacrosspage
  \centering
    \begin{alltt}
SUBROUTINE Compute_Average_Overlap()
  INTEGER  :: k, n

  ! The total layer water content
  self%iVar%lwc = ZERO
  DO n = 1, atm%n_Clouds
    self%iVar%lwc = self%iVar%lwc + atm%Cloud(n)%Water_Content(1:n_layers)
  END DO

  ! The cloud cover profile (not strictly needed)
  self%iVar%wc_sum(0)  = ZERO
  self%iVar%cwc_sum(0) = ZERO
  DO k = 1, n_layers
    self%iVar%wc_sum(k)  = self%iVar%wc_sum(k-1)  + self%iVar%lwc(k)
    self%iVar%cwc_sum(k) = self%iVar%cwc_sum(k-1) + (self%Cloud_Fraction(k) * self%iVar%lwc(k))
    IF ( self%iVar%wc_sum(k) > ZERO ) self%Cloud_Cover(k) = self%iVar%cwc_sum(k)/self%iVar%wc_sum(k)
  END DO
  self%Total_Cloud_Cover = self%Cloud_Cover(n_layers)
END SUBROUTINE Compute_Average_Overlap\end{alltt}
  \lineacrosspage
\end{figure}

\begin{figure}[H]
  \caption{\f{Compute\_Average\_Overlap\_TL()} (tangent-linear) subroutine.}
  \label{fig:Compute_Average_Overlap_TL_subroutine}
  \lineacrosspage
  \centering
  \begin{alltt}
SUBROUTINE Compute_Average_Overlap_TL()
  INTEGER  :: k, n
  REAL(fp) :: lwc_TL(n_Layers), wc_sum_TL(0:n_Layers), cwc_sum_TL(0:n_Layers)
  REAL(fp) :: denom

  ! The total layer water content
  lwc_TL = ZERO
  DO n = 1, atm_TL%n_Clouds
    lwc_TL = lwc_TL + atm_TL%Cloud(n)%Water_Content(1:n_layers)
  END DO

  ! The cloud cover profile (not strictly needed)
  wc_sum_TL(0)  = ZERO
  cwc_sum_TL(0) = ZERO
  DO k = 1, n_layers
     wc_sum_TL(k) =  wc_sum_TL(k-1) +  lwc_TL(k)
    cwc_sum_TL(k) = cwc_sum_TL(k-1) + (cc_FWD%Cloud_Fraction(k) * lwc_TL(k)                ) + &
                                      (cc_FWD%iVar%lwc(k)       * self_TL%Cloud_Fraction(k))
    IF ( cc_FWD%iVar%wc_sum(k) > ZERO ) THEN
      denom = ONE/cc_FWD%iVar%wc_sum(k)
      self_TL%Cloud_Cover(k) = (denom                             * cwc_sum_TL(k)) - &
                               (denom**2 * cc_FWD%iVar%cwc_sum(k) * wc_sum_TL(k) )
    END IF
  END DO
  self_TL%Total_Cloud_Cover = self_TL%Cloud_Cover(n_layers)
END SUBROUTINE Compute_Average_Overlap_TL\end{alltt}
  \lineacrosspage
\end{figure}


\begin{figure}[H]
  \caption{\f{Compute\_Average\_Overlap\_AD()} (adjoint) subroutine.}
  \label{fig:Compute_Average_Overlap_AD_subroutine}
  \lineacrosspage
  \centering
  \begin{alltt}
SUBROUTINE Compute_Average_Overlap_AD()
  INTEGER  :: k, n
  REAL(fp) :: lwc_AD(n_Layers), wc_sum_AD(0:n_Layers), cwc_sum_AD(0:n_Layers)
  REAL(fp) :: denom

  ! Initialise local adjoint variables
  lwc_AD     = ZERO
  wc_sum_AD  = ZERO
  cwc_sum_AD = ZERO

  ! Adjoint of the cloud cover profile
  self_AD%Cloud_Cover(n_layers) = self_AD%Cloud_Cover(n_layers) + self_AD%Total_Cloud_Cover
  self_AD%Total_Cloud_Cover = ZERO
  DO k = n_layers, 1, -1
    IF ( cc_FWD%iVar%wc_sum(k) > ZERO ) THEN
      denom = ONE/cc_FWD%iVar%wc_sum(k)
      wc_sum_AD(k)  = wc_sum_AD(k)  - (denom**2 * cc_FWD%iVar%cwc_sum(k) * self_AD%Cloud_Cover(k))
      cwc_sum_AD(k) = cwc_sum_AD(k) + (denom                             * self_AD%Cloud_Cover(k))
      self_AD%Cloud_Cover(k) = ZERO
    END IF

    self_AD%Cloud_Fraction(k) = self_AD%Cloud_Fraction(k) + (cc_FWD%iVar%lwc(k)       * cwc_sum_AD(k))
    lwc_AD(k)                 = lwc_AD(k)                 + (cc_FWD%Cloud_Fraction(k) * cwc_sum_AD(k))
    cwc_sum_AD(k-1)           = cwc_sum_AD(k-1)           + cwc_sum_AD(k)
    cwc_sum_AD(k)             = ZERO

    lwc_AD(k)      = lwc_AD(k)      + wc_sum_AD(k)
    wc_sum_AD(k-1) = wc_sum_AD(k-1) + wc_sum_AD(k)
    wc_sum_AD(k)   = ZERO
  END DO
  wc_sum_AD(0)  = ZERO
  cwc_sum_AD(0) = ZERO

  ! Adjoint of the total layer water content
  DO n = 1, atm_AD%n_Clouds
    atm_AD%Cloud(n)%Water_Content(1:n_layers) = atm_AD%Cloud(n)%Water_Content(1:n_layers) + lwc_AD
  END DO
  lwc_AD = ZERO
END SUBROUTINE Compute_Average_Overlap_AD\end{alltt}
  \lineacrosspage
\end{figure}


\subsection{Testing in the GSI}
%------------------------------

TBD


