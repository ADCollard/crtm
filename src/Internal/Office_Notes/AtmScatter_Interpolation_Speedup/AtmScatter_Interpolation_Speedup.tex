% The generic preamble
\documentclass[10pt,letterpaper,fleqn,titlepage]{article}

% Define packages to use
\usepackage{natbib}
\usepackage[dvips]{graphicx,color}
\usepackage{amsmath,amssymb}
\usepackage{bm}
\usepackage{caption}
\usepackage{xr}
\usepackage{ifthen}
\usepackage[dvipdfm,colorlinks,linkcolor=blue,citecolor=blue,urlcolor=blue]{hyperref}
\usepackage{fancybox}
\usepackage{textcomp}
\usepackage{alltt}
%\usepackage{floatflt}
%\usepackage{svn}


% Redefine default page
\setlength{\textheight}{9in}  % 1" above and below
\setlength{\textwidth}{6.75in}   % 0.5" left and right
\setlength{\oddsidemargin}{-0.25in}

% Redefine default paragraph
\setlength{\parindent}{0pt}
\setlength{\parskip}{1ex plus 0.5ex minus 0.2ex}

% Define caption width and default fonts
\setlength{\captionmargin}{0.5in}
\renewcommand{\captionfont}{\sffamily}
\renewcommand{\captionlabelfont}{\bfseries\sffamily}

% Define commands for super- and subscript in text mode
\newcommand{\superscript}[1]{\ensuremath{^\textrm{#1}}}
\newcommand{\subscript}[1]{\ensuremath{_\textrm{#1}}}

% Derived commands
\newcommand{\invcm}{\textrm{cm\superscript{-1}}}
\newcommand{\micron}{\ensuremath{\mu\textrm{m}}}

\newcommand{\df}{\ensuremath{\delta f}}
\newcommand{\Df}{\ensuremath{\Delta f}}
\newcommand{\dx}{\ensuremath{\delta x}}
\newcommand{\Dx}{\ensuremath{X_{max}}}
\newcommand{\Xeff}{\ensuremath{X_{eff}}}

\newcommand{\water}{\textrm{H\subscript{2}O}}
\newcommand{\carbondioxide}{\textrm{CO\subscript{2}}}
\newcommand{\ozone}{\textrm{O\subscript{3}}}

\newcommand{\taup}[1]{\ensuremath{\tau_{#1}}}
\newcommand{\efftaup}[1]{\ensuremath{\tau_{#1}^{*}}}

\newcommand{\textbfm}[1]{\boldmath\ensuremath{#1}\unboldmath}

\newcommand{\rb}[1]{\raisebox{1.5ex}[0pt]{#1}}

\newcommand{\f}[1]{\texttt{#1}}

% Define how equations are numbered
\numberwithin{equation}{section}
\numberwithin{figure}{section}
\numberwithin{table}{section}

% Define a command for title page author email footnote
\newcommand{\email}[1]
{%
  \renewcommand{\thefootnote}{\alph{footnote}}%
  \footnote{#1}
  \renewcommand{\thefootnote}{\arabic{footnote}}
}

% Define a command to print the Office Note subheading
\newcommand{\notesubheading}[1]
{%
  \ifthenelse{\equal{#1}{}}{}
  { {\Large\bfseries Office Note #1\par}%
    {\scriptsize \sc This is an unreviewed manuscript, primarily intended for informal}\\ 
    {\scriptsize \sc exchange of information among JCSDA researchers\par}%
  }
}

% Redefine the maketitle macro
\makeatletter
\def\docseries#1{\def\@docseries{#1}}
\def\docnumber#1{\def\@docnumber{#1}}
\renewcommand{\maketitle}
{%
  \thispagestyle{empty}
  \vspace*{1in}
  \begin{center}%
     \sffamily
     {\huge\bfseries Joint Center for Satellite Data Assimilation\par}%
     \notesubheading{\@docnumber}
  \end{center}
  \begin{flushleft}%
     \sffamily
     \vspace*{0.5in}
     {\Large\bfseries\ifthenelse{\equal{\@docseries}{}}{}{\@docseries: }\@title\par}%
     \medskip
     {\large\@author\par}%
     \medskip
     {\large\@date\par}%
     \bigskip\hrule\vspace*{2pc}%
  \end{flushleft}%
  \newpage
  \setcounter{footnote}{0}
}
\makeatother
\docseries{}
\docnumber{}


% Define a command for a DRAFT watermark
\usepackage{eso-pic}
\newcommand{\draftwatermark}
{
  \AddToShipoutPicture{%
    \definecolor{lightgray}{gray}{.85}
    \setlength{\unitlength}{1in}
    \put(2.5,3.5){%
      \rotatebox{45}{%
        \resizebox{4in}{1in}{%
          \textsf{\textcolor{lightgray}{DRAFT}}
        }
      }
    }
  }
}




\includeonly{Introduction,CloudScatter_Profile,AerosolScatter_Profile,CRTM_Profile}
% Title info
\title{Cloud and Aerosol Optical Property Lookup Table Interpolation Speedup for REL-1.2}
\author{Paul van Delst\email{paul.vandelst@noaa.gov}\\JCSDA/EMC/SAIC}
\date{December, 2008}
\docnumber{(unassigned)}
\docseries{CRTM}


%-------------------------------------------------------------------------------
%                            Ze document begins...
%-------------------------------------------------------------------------------
\begin{document}
\maketitle

%\draftwatermark

% The front matter
%=================
\thispagestyle{empty}
\vspace*{10cm}
\begin{center}
  {\sffamily\Large\bfseries Change History}
  \begin{table}[htp]
    \centering
    \begin{tabular}{|p{2cm}|p{3cm}|p{8cm}|}
      \hline
      \sffamily\textbf{Date} & \sffamily\textbf{Author} & \sffamily\textbf{Change}\\
      \hline\hline
      2008-12-04 & P.van Delst & Initial release.\\
      \hline
      2008-12-31 & P.van Delst & Consolidated the CloudScatter and AerosolScatter timing results. Added the CRTM ComponentTest linux/gfortran timing results\\
      \hline
    \end{tabular}
  \end{table}
\end{center}
\clearpage
\pagenumbering{arabic}
\setcounter{page}{1}


% The main matter
%================
\chapter{Introduction}
%=====================


\section{Components}
%===================
\label{sec:components}

The LBLRTM I/O library is constructed around five data constructs, described in table \ref{tab:component_definitions}:
\begin{table}[htp]
  \centering
  \caption{The data constructs of the LBLRTM I/O library}
  \begin{tabular}{p{2.5cm} p{12cm}}
    \hline\\[-0.1cm]
    \sffamily\textbf{Component Name} & \sffamily\textbf{Description} \\
    \hline\hline\\[-0.2cm]
    \texttt{Fhdr}  & The file header construct that is present at the start of each layer of data. \\\\
    \texttt{Phdr}  & This is the panel header construct that is present at the start of each ``chunk'' of data (usually referred to as a ``panel''. See following.) \\\\
    \texttt{Panel} & This construct corresponds to a ``chunk'' of spectral data. An LBLRTM datafile is referred to as being a single- or double-panel file. The former means a single spectral quantity is present (e.g. optical depth), and the latter means that two spectral quantities are present (e.g. radiance and transmittances). \\\\
    \texttt{Layer} & This construct contains spectral data for the entire frequency range of an LBLRTM calculation for a single layer. The concept ``layer'' can correspond to the spectral data for individual atmospheric layers of the input profile, or to the final result for an entire atmosphere. \\\\
    \texttt{File}  & This contruct is true to its name. It corresponds to an entire datafile of data which may consist of a single layer or multiple layers, and for single- or double-panel spectral data. \\
  \hline
  \end{tabular}
  \label{tab:component_definitions}
\end{table}

Each component has a definition module to define the object and some basic methods to manipulate it, and an I/O module to read and write instances of the objects from/to file.

Two components -- the file header and panel header -- are standalone, but the others contain other components, i.e. the panel object contains panel headers; the layer object contains file headers; and the file object contains layers. A schematic illustration of how the actual LBLRTM datafile format relates to the component definitions is shown in figure \ref{fig:lblrtm_format}.

Note, however, that when an LBLRTM file is read, the individual panel ``chunks'' of spectral data are concatenated into a single spectrum. Thus the \Panel{} object itself is only used when reading from an LBLRTM file and is not used in the \File{} or \Layer{} objects.

\begin{figure}[htp]
  \centering
  \input{graphics/lblrtm_format.pstex_t}
  \caption{Schematic illustration of the LBLRTM single- and double-panel datafile format. A datafile can contain one, or multiple, layers of data.}
  \label{fig:lblrtm_format}
\end{figure}



\section{Conventions}
%====================
\label{sec:conventions}
The following are conventions that have been adhered to in the current release of the LBLRTM I/O library. They are guidelines intended to make understanding the code at a glance easier, to provide a recognisable ``look and feel'', and to minimise name space clashes.



\subsection{Naming of Objects and Instances of Objects}
%------------------------------------------------------

The object\footnote{The terms ``derived type'' and ``structure'' can also be used as the code is not yet fully OO - that's for future updates.} naming convention adopted for use in the LBLRTM I/O library is, 

\hspace{0.4cm}\f{LBLRTM\_}\textit{name}\f{\_type} 

where \textit{name} is an identifier for the particular component (e.g. panel header, layer, file, etc as listed in table \ref{tab:component_definitions}). All object type names are suffixed with ``\f{\_type}''. The ``\f{LBLRTM\_}'' prefix is to define a namespace to minimise name clashes. An instance of a object is then referred to via its \textit{name}, or some sort of derivate of its \textit{name}. Some object declarations examples are,

\begin{alltt}
  TYPE(\hyperref[fig:LBLRTM_File_type_structure]{LBLRTM_File_type}) :: sp_file, dp_file
  TYPE(\hyperref[fig:LBLRTM_Layer_type_structure]{LBLRTM_Layer_type}) :: layer\end{alltt}



\subsection{Naming of Definition Modules}
%----------------------------------------

Modules containing object definitions are termed \textit{definition modules}. These modules contain the actual object definitions as well as various utility procedures that operate on the object. The naming convention adopted for definition modules in the LBLRTM I/O library is, 

\hspace{0.4cm}\f{LBLRTM\_}\textit{name}\f{\_Define} 

where all definition module names are suffixed with ``\f{\_Define}''. The actual source code files for these modules have the same name with a ``\f{.f90}'' suffix.



\subsection{Naming of I/O Modules}
%---------------------------------

Modules containing all the object I/O procedures are termed, surprise, surprise, \textit{I/O modules}. These modules contain function to read and write LBLRTM format datafiles. The naming convention adopted for I/O modules in the LBLRTM I/O library is, 

\hspace{0.4cm}\f{LBLRTM\_}\textit{name}\f{\_IO} 

where all I/O module names are suffixed with ``\f{\_IO}''. As with the definition modules, the actual source code files for these modules have the same name with a ``\f{.f90}'' suffix.



\subsection{Standard Definition Module Procedures}
%-------------------------------------------------

The definition modules for the user-accessible objects (for practical purposes just \File, although \Layer, \Fhdr, \Panel, and \Phdr are accessible for now) contain a standard set of procedures for use with the object being defined. The naming convention for these procedures is,

\hspace{0.4cm}\f{LBLRTM\_}\textit{name}\f{\_}\textit{action}

where the available default actions for each procedure are listed in table \ref{tab:definition_module_default_procedures}. This is not an exhaustive list but procedures for the actions listed in table \ref{tab:definition_module_default_procedures} are generally going to be present.

The exception is that the objects with no allocatable components do not have a creation procedure.

\begin{table}[htp]
  \centering
  \caption{Default action procedures available in object definition modules. $^{\dagger}$Procedures not available for the \Fhdr{} and \Phdr{} objects. $^{\ddagger}$Procedure available only for the \Layer{} object.}
  \begin{tabular}{p{2.5cm} p{3.5cm} p{8.5cm}}
    \hline\\[-0.1cm]
    \sffamily\textbf{Action} & \sffamily\textbf{Type} & \sffamily\textbf{Description} \\
    \hline\hline\\[-0.2cm]
    \texttt{OPERATOR(==)}             & Elemental function   & Tests the equality of two structures. \\
    \texttt{OPERATOR(/=)}             & Elemental function   & Tests the inequality of two structures. \\
    \texttt{Associated}$^{\dagger}$   & Elemental function   & Tests if the object components have been allocated. \\
    \texttt{Create}$^{\dagger}$       & Elemental subroutine & Allocates any allocatable object components. \\
    \texttt{Destroy}                  & Elemental subroutine & Reinitialises an object. \\
    \texttt{DefineVersion}            & Subroutine           & Returns the module version information. \\
    \texttt{Frequency}$^{\ddagger}$   & Subroutine           & Compute and return the spectral frequency grid. \\
    \texttt{Inspect}                  & Subroutine           & Displays object contents to \texttt{stdout}. \\
    \texttt{IsValid}                  & Elemental function   & Tests if the object contains valid data. \\
    \texttt{SetValid}                 & Elemental subroutine & Flags the object as containing valid data. \\
  \hline
  \end{tabular}
  \label{tab:definition_module_default_procedures}
\end{table}

\begin{table}[htp]
  \centering
  \caption{Default action procedures available in object I/O modules.}
  \begin{tabular}{p{2.5cm} p{3.5cm} p{8.5cm}}
    \hline\\[-0.1cm]
    \sffamily\textbf{Action} & \sffamily\textbf{Type} & \sffamily\textbf{Description} \\
    \hline\hline\\[-0.2cm]
    \texttt{IOVersion} & Subroutine  & Returns the module version information. \\
    \texttt{Read}      & Function    & Loads an instance of an object with data read from file. \\
    \texttt{Write}     & Function    & Write an instance of an object to file. \\
  \hline
  \end{tabular}
  \label{tab:io_module_default_procedures}
\end{table}

Some examples of these procedure names are,

\begin{alltt}
  \hyperref[sec:LBLRTM_File_Associated_interface]{LBLRTM_File_Associated}
  \hyperref[sec:LBLRTM_File_IsValid_interface]{LBLRTM_File_IsValid}
  \hyperref[sec:LBLRTM_Layer_Destroy_interface]{LBLRTM_Layer_Destroy}
  \hyperref[sec:LBLRTM_Layer_Frequency_interface]{LBLRTM_Layer_Frequency}
  \hyperref[sec:LBLRTM_File_Inspect_interface]{LBLRTM_File_Inspect}\end{alltt}

The relational operators, \f{==} and \f{/=}, are implemented via overloaded \f{Equal} and \f{NotEqual} action procedures respectively, as is shown below for the \File{} structure,

\begin{alltt}
  INTERFACE OPERATOR(==)
    MODULE PROCEDURE LBLRTM_File_Equal
  END INTERFACE OPERATOR(==)

  INTERFACE OPERATOR(/=)
    MODULE PROCEDURE LBLRTM_File_NotEqual
  END INTERFACE OPERATOR(/=)\end{alltt}

For a complete list of the definition and I/O module procedures for use with the available objects, see appendix \ref{app:object_and_interface_definition}.


\section{CloudScatter Forward Model Profile Results}
%===================================================
The forward model test results are shown separately for the microwave and infrared cloud optical property LUT interpolation. The microwave interpolation is done at different dimensionalities dependeing on the cloud type. The infrared interpolation is always two-dimensional.

No significant timing differences are seen in the forward model test since the number of calculations is the same, just that some of them are being retained in the \texttt{csi} structure as shown in figure \ref{fig:CSVariables_structure}.

\subsection{Linux test}
%----------------------

\begin{table}[ht]
  \centering
  \begin{tabular}{p{0.25cm} p{3.55cm} *{2}{r@{.}l} c *{2}{r@{.}l}}
    \hline
                    &                    & \multicolumn{4}{c}{\textbf{Baseline}} & \hspace{1.0em} & \multicolumn{4}{c}{\textbf{Modified}} \\
    \multicolumn{2}{c}{\textbf{Routine}} & \multicolumn{2}{c}{\textbf{self}} & \multicolumn{2}{c}{\textbf{called}} & & \multicolumn{2}{c}{\textbf{self}} & \multicolumn{2}{c}{\textbf{called}} \\
    \hline\hline
    \multicolumn{2}{l}{\texttt{get\_cloud\_opt\_mw}} &  76&96 & 578&58   & &    53&13 & 571&72 \vspace{0.5em}\\
    &\texttt{interp\_3d}                             &  26&10 & 214&69   & &    26&90 & 210&44 \\
    &\texttt{find\_random\_index}                    & 136&65 &   0&00   & &   141&96 &   0&00 \\
    &\texttt{lpoly}                                  &  20&74 &  96&16   & &    17&99 &  91&97 \\
    &\texttt{interp\_2d}                             &  24&73 &  56&77   & &    23&12 &  56&60 \\
    &\texttt{interp\_1d}                             &   2&74 &   0&00   & &     2&74 &   0&00 \\
    \hline
  \end{tabular}
  \caption{gfortran CloudScatter Forward model profile results for the \texttt{Get\_Cloud\_Opt\_MW} subroutine. All times in seconds.}
  \label{tab:fwd_cs_test_get_cloud_opt_mw_gfortran}
\end{table}


\begin{table}[ht]
  \centering
  \begin{tabular}{p{0.25cm} p{3.55cm} *{2}{r@{.}l} c *{2}{r@{.}l}}
    \hline
                    &                    & \multicolumn{4}{c}{\textbf{Baseline}} & \hspace{1.0em} & \multicolumn{4}{c}{\textbf{Modified}} \\
    \multicolumn{2}{c}{\textbf{Routine}} & \multicolumn{2}{c}{\textbf{self}} & \multicolumn{2}{c}{\textbf{called}} & & \multicolumn{2}{c}{\textbf{self}} & \multicolumn{2}{c}{\textbf{called}} \\
    \hline\hline
    \multicolumn{2}{l}{\texttt{get\_cloud\_opt\_ir}} & 31&69 & 189&73   & &  23&17 & 186&95 \vspace{0.5em}\\
    &\texttt{interp\_2d}                             & 30&76 &  70&61   & &  28&76 &  70&40 \\
    &\texttt{find\_random\_index}                    & 47&62 &   0&00   & &  49&47 &   0&00 \\
    &\texttt{lpoly}                                  &  7&23 &  33&51   & &   6&27 &  32&05 \\
    \hline
  \end{tabular}
  \caption{gfortran CloudScatter Forward model profile results for the \texttt{Get\_Cloud\_Opt\_IR} subroutine. All times in seconds.}
  \label{tab:fwd_cs_test_get_cloud_opt_ir_gfortran}
\end{table}

                
\subsection{IBM test}
%--------------------

\begin{table}[ht]
  \centering
  \begin{tabular}{p{0.25cm} p{3.55cm} *{2}{r@{.}l} c *{2}{r@{.}l}}
    \hline
                    &                    & \multicolumn{4}{c}{\textbf{Baseline}} & \hspace{1.0em} & \multicolumn{4}{c}{\textbf{Modified}} \\
    \multicolumn{2}{c}{\textbf{Routine}} & \multicolumn{2}{c}{\textbf{self}} & \multicolumn{2}{c}{\textbf{called}} & & \multicolumn{2}{c}{\textbf{self}} & \multicolumn{2}{c}{\textbf{called}} \\
    \hline\hline
    \multicolumn{2}{l}{\texttt{get\_cloud\_opt\_mw}} & 463&88 & 4633&15   & &   399&15 & 4664&71 \vspace{0.5em}\\
    &\texttt{interp\_3d}                             & 412&34 & 2133&74   & &   390&48 & 2148&70 \\
    &\texttt{interp\_2d}                             & 383&67 &  498&79   & &   375&41 &  512&72 \\
    &\texttt{lpoly}                                  & 165&98 &  598&86   & &   168&51 &  600&09 \\
    &\texttt{find\_random\_index}                    & 432&79 &    0&00   & &   461&62 &    0&00 \\
    &\texttt{interp\_1d}                             &   6&98 &    0&00   & &     7&18 &    0&00 \\
    \hline
  \end{tabular}
  \caption{IBM CloudScatter Forward model profile results for the \texttt{Get\_Cloud\_Opt\_MW} subroutine. All times in seconds.}
  \label{tab:fwd_cs_test_get_cloud_opt_mw_ibm}
\end{table}


\begin{table}[ht]
  \centering
  \begin{tabular}{p{0.25cm} p{3.55cm} *{2}{r@{.}l} c *{2}{r@{.}l}}
    \hline
                    &                    & \multicolumn{4}{c}{\textbf{Baseline}} & \hspace{1.0em} & \multicolumn{4}{c}{\textbf{Modified}} \\
    \multicolumn{2}{c}{\textbf{Routine}} & \multicolumn{2}{c}{\textbf{self}} & \multicolumn{2}{c}{\textbf{called}} & & \multicolumn{2}{c}{\textbf{self}} & \multicolumn{2}{c}{\textbf{called}} \\
    \hline\hline
    \multicolumn{2}{l}{\texttt{get\_cloud\_opt\_ir}} & 186&83 & 1514&98   & &  164&67 & 1533&38 \vspace{0.5em}\\
    &\texttt{interp\_2d}                             & 477&22 &  620&41   & &  466&94 &  637&73 \\
    &\texttt{lpoly}                                  &  57&84 &  208&69   & &   58&72 &  209&12 \\
    &\texttt{find\_random\_index}                    & 150&82 &    0&00   & &  160&87 &    0&00 \\
    \hline
  \end{tabular}
  \caption{IBM CloudScatter Forward model profile results for the \texttt{Get\_Cloud\_Opt\_IR} subroutine. All times in seconds.}
  \label{tab:fwd_cs_test_get_cloud_opt_ir_ibm}
\end{table}

\section{AerosolScatter Profile Results}
%=======================================

\subsection{AerosolScatter Forward Model Profile Results}
%--------------------------------------------------------
Aerosol optical properties are only used in the infrared spectral region, so only one routine is profiled here -- the generic \texttt{get\_aerosol\_opt}. As with the CloudScatter forward model results, no significant timing differences are seen since the number of calculations is essentially the same.


% Linux test

\begin{table}[ht]
  \centering
  \begin{tabular}{p{0.25cm} p{3.55cm} *{2}{r@{.}l} c *{2}{r@{.}l}}
    \hline
                    &                    & \multicolumn{4}{c}{\textbf{Baseline}} & \hspace{1.0em} & \multicolumn{4}{c}{\textbf{Modified}} \\
    \multicolumn{2}{c}{\textbf{Routine}} & \multicolumn{2}{c}{\textbf{self}} & \multicolumn{2}{c}{\textbf{called}} & & \multicolumn{2}{c}{\textbf{self}} & \multicolumn{2}{c}{\textbf{called}} \\
    \hline\hline
    \multicolumn{2}{l}{\texttt{get\_aerosol\_opt}} & 39&52 & 230&06   & &   34&28 & 245&13 \vspace{0.5em}\\
    &\texttt{interp\_2d}                           & 40&73 &  91&91   & &   37&85 & 103&73 \\
    &\texttt{find\_random\_index}                  & 49&10 &   0&00   & &   53&28 &   0&00 \\
    &\texttt{lpoly}                                &  8&01 &  40&15   & &    9&09 &  40&75 \\
    &\texttt{aerosol\_type\_index}                 &  0&18 &   0&00   & &    0&44 &   0&00 \\
    \hline
  \end{tabular}
  \caption{gfortran AerosolScatter Forward model profile results for the \texttt{Get\_Aerosol\_Opt} subroutine. All times in seconds.}
  \label{tab:fwd_as_test_get_aerosol_opt_gfortran}
\end{table}



% IBM test

\begin{table}[ht]
  \centering
  \begin{tabular}{p{0.25cm} p{3.55cm} *{2}{r@{.}l} c *{2}{r@{.}l}}
    \hline
                    &                    & \multicolumn{4}{c}{\textbf{Baseline}} & \hspace{1.0em} & \multicolumn{4}{c}{\textbf{Modified}} \\
    \multicolumn{2}{c}{\textbf{Routine}} & \multicolumn{2}{c}{\textbf{self}} & \multicolumn{2}{c}{\textbf{called}} & & \multicolumn{2}{c}{\textbf{self}} & \multicolumn{2}{c}{\textbf{called}} \\
    \hline\hline
    \multicolumn{2}{l}{\texttt{get\_aerosol\_opt}} & 267&45 & 1855&09   & &   234&69 & 1889&82 \vspace{0.5em}\\
    &\texttt{interp\_2d}                           & 579&35 &  817&80   & &   583&05 &  795&91 \\
    &\texttt{lpoly}                                &  74&27 &  260&30   & &    73&78 &  263&76 \\
    &\texttt{find\_random\_index}                  & 122&63 &    0&00   & &   172&29 &    0&00 \\
    &\texttt{aerosol\_type\_index}                 &   0&74 &    0&00   & &     1&03 &    0&00 \\
    \hline
  \end{tabular}
  \caption{IBM AerosolScatter Forward model profile results for the \texttt{Get\_Aerosol\_Opt} subroutine. All times in seconds.}
  \label{tab:fwd_as_test_get_aerosol_opt_ibm}
\end{table}



\subsection{AerosolScatter Tangent-Linear Model Profile Results}
%============================================================
Here we see the decreased execution time of the modified code due to the reuse of the intermediate interpolation results in the tangent-linear model. The speedup is around 3-4x.


% Linux test

\begin{table}[ht]
  \centering
  \begin{tabular}{p{0.25cm} p{3.55cm} *{2}{r@{.}l} c *{2}{r@{.}l}}
    \hline
                    &                    & \multicolumn{4}{c}{\textbf{Baseline}} & \hspace{1.0em} & \multicolumn{4}{c}{\textbf{Modified}} \\
    \multicolumn{2}{c}{\textbf{Routine}} & \multicolumn{2}{c}{\textbf{self}} & \multicolumn{2}{c}{\textbf{called}} & & \multicolumn{2}{c}{\textbf{self}} & \multicolumn{2}{c}{\textbf{called}} \\
    \hline\hline
    \multicolumn{2}{l}{\texttt{get\_aerosol\_opt\_tl}} & 152&97 & 549&00   & &   41&19 & 498&31 \vspace{0.5em}\\
    &\texttt{interp\_2d\_tl}                           &  85&45 & 285&46   & &   98&44 & 315&26 \\
    &\texttt{lpoly\_tl}                                &  15&46 &  66&40   & &   15&01 &  69&23 \\
    &\texttt{find\_random\_index}                      &  51&30 &   0&00   & &   \multicolumn{2}{c}{-} & \multicolumn{2}{c}{-} \\
    &\texttt{lpoly}                                    &   8&45 &  36&11   & &   \multicolumn{2}{c}{-} & \multicolumn{2}{c}{-} \\
    &\texttt{aerosol\_type\_index}                     &   0&38 &   0&00   & &    0&37 &   0&00 \\
    \hline
  \end{tabular}
  \caption{gfortran AerosolScatter Tangent-linear model profile results for the \texttt{Get\_Aerosol\_Opt\_TL} subroutine. All times in seconds.}
  \label{tab:tl_as_test_get_aerosol_opt_gfortran}
\end{table}



% IBM test

\begin{table}[ht]
  \centering
  \begin{tabular}{p{0.25cm} p{3.55cm} *{2}{r@{.}l} c *{2}{r@{.}l}}
    \hline
                    &                    & \multicolumn{4}{c}{\textbf{Baseline}} & \hspace{1.0em} & \multicolumn{4}{c}{\textbf{Modified}} \\
    \multicolumn{2}{c}{\textbf{Routine}} & \multicolumn{2}{c}{\textbf{self}} & \multicolumn{2}{c}{\textbf{called}} & & \multicolumn{2}{c}{\textbf{self}} & \multicolumn{2}{c}{\textbf{called}} \\
    \hline\hline
    \multicolumn{2}{l}{\texttt{get\_aerosol\_opt\_tl}} & 1944&73 & 5353&09   & &    436&08 & 4861&03 \vspace{0.5em}\\
    &\texttt{interp\_2d\_tl}                           & 1380&02 & 2944&39   & &   1423&29 & 2896&04 \\
    &\texttt{lpoly\_tl}                                &   85&19 &  480&71   & &     86&00 &  454&49 \\
    &\texttt{lpoly}                                    &   73&51 &  264&22   & &   \multicolumn{2}{c}{-} & \multicolumn{2}{c}{-} \\
    &\texttt{find\_random\_index}                      &  124&16 &    0&00   & &   \multicolumn{2}{c}{-} & \multicolumn{2}{c}{-} \\
    &\texttt{aerosol\_type\_index}                     &    0&88 &    0&00   & &      1&21 &    0&00 \\
    \hline
  \end{tabular}
  \caption{IBM AerosolScatter Tangent-linear model profile results for the \texttt{Get\_Aerosol\_Opt\_TL} subroutine. All times in seconds.}
  \label{tab:tl_as_test_get_aerosol_opt_ibm}
\end{table}

\subsection{AerosolScatter Adjoint Model Profile Results}
%=====================================================
As with the AerosolScatter tangent-linear model, we see the decreased execution time of the modified code due to the reuse of the intermediate interpolation results in the adjoint model. The speedup is around 2-3x, although the total test time is relatively short so care must be taken in interpreting those factors. As with the CloudScatter updates, the main point here is that the modifications do not make the code slower.


% Linux test

\begin{table}[ht]
  \centering
  \begin{tabular}{p{0.25cm} p{3.55cm} *{2}{r@{.}l} c *{2}{r@{.}l}}
    \hline
                    &                    & \multicolumn{4}{c}{\textbf{Baseline}} & \hspace{1.0em} & \multicolumn{4}{c}{\textbf{Modified}} \\
    \multicolumn{2}{c}{\textbf{Routine}} & \multicolumn{2}{c}{\textbf{self}} & \multicolumn{2}{c}{\textbf{called}} & & \multicolumn{2}{c}{\textbf{self}} & \multicolumn{2}{c}{\textbf{called}} \\
    \hline\hline
    \multicolumn{2}{l}{\texttt{get\_aerosol\_opt\_ad}} & 28&84 & 117&95   & &    9&86 &  92&47 \vspace{0.5em}\\
    &\texttt{interp\_2d\_ad}                           & 19&48 &  57&33   & &   19&22 &  53&72 \\
    &\texttt{lpoly\_ad}                                &  3&26 &  16&92   & &    3&39 &  14&75 \\
    &\texttt{lpoly}                                    &  1&80 &   7&77   & &   \multicolumn{2}{c}{-} & \multicolumn{2}{c}{-} \\
    &\texttt{find\_random\_index}                      &  9&38 &   0&00   & &   \multicolumn{2}{c}{-} & \multicolumn{2}{c}{-} \\
    &\texttt{aerosol\_type\_index}                     &  0&05 &   0&00   & &    0&05 &   0&00 \\
    \hline
  \end{tabular}
  \caption{gfortran AerosolScatter Adjoint model profile results for the \texttt{Get\_Aerosol\_Opt\_AD} subroutine. All times in seconds.}
  \label{tab:ad_as_test_get_aerosol_opt_gfortran}
\end{table}



% IBM test

\begin{table}[ht]
  \centering
  \begin{tabular}{p{0.25cm} p{3.55cm} *{2}{r@{.}l} c *{2}{r@{.}l}}
    \hline
                    &                    & \multicolumn{4}{c}{\textbf{Baseline}} & \hspace{1.0em} & \multicolumn{4}{c}{\textbf{Modified}} \\
    \multicolumn{2}{c}{\textbf{Routine}} & \multicolumn{2}{c}{\textbf{self}} & \multicolumn{2}{c}{\textbf{called}} & & \multicolumn{2}{c}{\textbf{self}} & \multicolumn{2}{c}{\textbf{called}} \\
    \hline\hline
    \multicolumn{2}{l}{\texttt{get\_aerosol\_opt\_ad}} & 290&81 & 1064&11   & &    88&80 &  985&50 \vspace{0.5em}\\
    &\texttt{interp\_2d\_ad}                           & 269&14 &  561&43   & &   276&07 &  563&02 \\
    &\texttt{lpoly\_ad}                                &  15&95 &  124&51   & &    15&52 &  121&79 \\
    &\texttt{lpoly}                                    &  13&22 &   48&02   & &   \multicolumn{2}{c}{-} & \multicolumn{2}{c}{-} \\
    &\texttt{find\_random\_index}                      &  22&79 &    0&00   & &   \multicolumn{2}{c}{-} & \multicolumn{2}{c}{-} \\
    &\texttt{aerosol\_type\_index}                     &   0&19 &    0&00   & &     0&16 &    0&00 \\
    \hline
  \end{tabular}
  \caption{IBM AerosolScatter Adjoint model profile results for the \texttt{Get\_Aerosol\_Opt\_AD} subroutine. All times in seconds.}
  \label{tab:ad_as_test_get_aerosol_opt_ibm}
\end{table}

\section{CRTM ComponentTest Profile Results}
%===========================================

The CRTM ComponentTest test programs run the CRTM models for a variety of sensors (both microwave and infrared) and input profiles (including both cloud and aersol data in the input atmospheric profiles. These tests were run using both the baseline and modified cloud and aerosol optical property interpolation codes to determine what the timing changes would be for a complete model run. Currently only linux/gfortran test cases are available, but the results of the tests performed indicate that the modified interpolation code has no significant impact on the code speed.

The results of the forward/tangent-linear and tangent-linear/adjoint ComponentTest runs are shown in tables \ref{tab:fwdtl_test_gfortran} and \ref{tab:tlad_test_gfortran} respectively. In both cases the gas absorption and matrix inversion procedures (the latter used in the CRTM ADA RTSolution computation) were the most expensive procedures. Only those procedures that \emph{did not} call any child processes are shown. For example, the time spent in the \texttt{crtm\_rtsolution::crtm\_doubling\_layer} procedure was greater than that of \texttt{crtm\_utility::matinv} procedure, but the former calls the latter.

\subsection{Forward/Tangent-linear ComponentTest Profile Results}
%----------------------------------------------------------------

% Linux test

\begin{table}[ht]
  \centering
  \begin{tabular}{l *{2}{r@{.}l} c *{2}{r@{.}l} }
    \hline
                     & \multicolumn{4}{c}{\textbf{Baseline}} & \hspace{1.0em} & \multicolumn{4}{c}{\textbf{Modified}} \\
    \textbf{Routine} & \multicolumn{2}{c}{\textbf{\%}} & \multicolumn{2}{c}{\textbf{self time}} & & \multicolumn{2}{c}{\textbf{\%}} & \multicolumn{2}{c}{\textbf{self time}} \\
    \hline\hline
\texttt{crtm\_utility::matinv}                                 & 66&7 & 284849&04 & & 66&0 & 285413&03 \\ 
\texttt{crtm\_atmabsorption::crtm\_compute\_atmabsorption}     &  9&0 &  38424&78 & &  8&9 &  38527&22 \\
... & \multicolumn{2}{c}{...} & \multicolumn{2}{c}{...} & &  \multicolumn{2}{c}{...} & \multicolumn{2}{c}{...} \\ 
\texttt{crtm\_interpolation::interp\_1d}                       &  0&9 &   4009&04 & &  1&0 &   4356&03 \\
\texttt{crtm\_interpolation::interp\_1d\_tl}                   &  0&5 &   2059&62 & &  0&5 &   2232&06 \\
    \hline
  \end{tabular}
  \caption{gfortran CRTM Forward/Tangent-linear ComponentTest profile results comparing the most expensive routines with the two most expensive interpolation routines. The routine names are prefixed with their containing module name. All times in seconds.}
  \label{tab:fwdtl_test_gfortran}
\end{table}


\subsection{Tangent-linear/Adjoint ComponentTest Profile Results}
%----------------------------------------------------------------

% Linux test

\begin{table}[ht]
  \centering
  \begin{tabular}{l *{2}{r@{.}l} c *{2}{r@{.}l} }
    \hline
                     & \multicolumn{4}{c}{\textbf{Baseline}} & \hspace{1.0em} & \multicolumn{4}{c}{\textbf{Modified}} \\
    \textbf{Routine} & \multicolumn{2}{c}{\textbf{\%}} & \multicolumn{2}{c}{\textbf{self time}} & & \multicolumn{2}{c}{\textbf{\%}} & \multicolumn{2}{c}{\textbf{self time}} \\
    \hline\hline
\texttt{crtm\_atmabsorption::crtm\_compute\_atmabsorption}     & 19&5 & 618&28 & & 19&8 & 620&95 \\ 
\texttt{crtm\_utility::matinv}                                 & 11&3 & 359&16 & & 11&9 & 372&32 \\
... & \multicolumn{2}{c}{...} & \multicolumn{2}{c}{...} & &  \multicolumn{2}{c}{...} & \multicolumn{2}{c}{...} \\ 
\texttt{crtm\_interpolation::interp\_1d\_tl}                   &  2&1 &  67&15 & &  2&1 &  65&79 \\
\texttt{crtm\_interpolation::interp\_1d}                       &  1&8 &  57&65 & &  2&1 &  66&17 \\
    \hline
  \end{tabular}
  \caption{gfortran CRTM Tangent-linear/Adjoint ComponentTest profile results comparing the most expensive routines with the two most expensive interpolation routines. The routine names are prefixed with their containing module name. All times in seconds.}
  \label{tab:tlad_test_gfortran}
\end{table}





% The references section
%=======================
%\clearpage
%\bibliographystyle{plainnat}
%\bibliography{bibliography}


% The appendices section
%=======================
%\begin{appendix}
%\end{appendix}

\end{document}

