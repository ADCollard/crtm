\section{Radiometric impact of SRF interpolation}
%================================================
To determine the radiometric impact of the various forms of the SRF (due to different data sources or different interpolation methods), effective temperatures\footnote{See appendix \ref{app:band_correction_coefficients} for a definition of effective temperature.} for each sensor channel were computed for a blackbody temperature of 285K. The result for the spline interpolated SRF with a tension of 5.0 (spline5) was chosen as the reference. Comparisons were then made with results for a linearly interpolated SRF (linear), a spline interpolated SRF with tension of 0.1 (spline0.1), a spline interpolated SRF with tension of 20 (spline20), the original uninterpolated SRF itself (original), and the current SRF used in CRTM processing (current). The temperature residuals for an SRF type, ``x'', are defined as,
\begin{equation}
  \Delta T(\textrm{x}) = T_{eff}(\textrm{spline5}) - T_{eff}(\textrm{x})
\end{equation}
and are shown in table \ref{tab:teff_comparison}. It is apparent that the type of interpolation used on the SRFs has minimal radiometric impact.

The computed central frequencies and band correction coefficients for the NOAA-16, -17, -18, and MetOp-A AVHRR sensors are shown in appendix \ref{app:band_correction_coefficients}.

\begin{table}[htp]
  \centering
  \begin{tabular}{l c *{6}{r@{.}l}}
    \hline
    \multicolumn{2}{c}{ } & \multicolumn{2}{c}{\textbfm{T_{eff}}} & \multicolumn{2}{c}{\textbfm{\Delta T}} & \multicolumn{2}{c}{\textbfm{\Delta T}} & \multicolumn{2}{c}{\textbfm{\Delta T}} & \multicolumn{2}{c}{\textbfm{\Delta T}} & \multicolumn{2}{c}{\textbfm{\Delta T}} \\
    \textbf{Platform} & \textbf{Channel} & \multicolumn{2}{c}{spline5} & \multicolumn{2}{c}{linear} & \multicolumn{2}{c}{spline0.1} & \multicolumn{2}{c}{spline20} & \multicolumn{2}{c}{original} & \multicolumn{2}{c}{current}\\
    \multicolumn{2}{c}{ } & \multicolumn{2}{c}{(K)} & \multicolumn{2}{c}{(K)} & \multicolumn{2}{c}{(K)} & \multicolumn{2}{c}{(K)} & \multicolumn{2}{c}{(K)}  & \multicolumn{2}{c}{(K)} \\
    \hline\hline
            &  3B & \hspace{0.2em}286&12 & -4&72e-04 &  2&04e-04 & -3&05e-04 &  1&51e-04 &  2&92e-02 \\ 
    NOAA-16 &  4  &               285&01 & -3&43e-06 &  1&73e-06 & -2&28e-06 &  1&76e-06 &  1&26e-05 \\   
            &  5  &               284&98 &  1&08e-05 & -4&00e-06 &  6&84e-06 & -3&62e-06 & -8&87e-05 \vspace{0.75em}\\ 
            &  3B &               286&13 & -4&55e-04 &  1&85e-04 & -2&91e-04 & -1&86e-03 &  6&35e-02 \\   
    NOAA-17 &  4  &               285&01 & -6&02e-06 &  2&70e-06 & -3&92e-06 & -7&89e-05 &  1&35e-03 \\   
            &  5  &               284&98 &  8&72e-06 & -3&24e-06 &  5&52e-06 &  1&60e-04 & -9&14e-04 \vspace{0.75em}\\ 
            &  3B &               286&10 & -4&45e-04 &  1&79e-04 & -2&84e-04 & -3&66e-03 & -3&91e-03 \\   
    NOAA-18 &  4  &               285&01 & -5&17e-06 &  2&29e-06 & -3&36e-06 &  1&15e-05 &  4&70e-06 \\   
            &  5  &               284&98 &  1&09e-05 & -3&91e-06 &  6&89e-06 &  6&55e-06 &  8&00e-06 \vspace{0.75em}\\ 
            &  3B &               286&06 & -4&75e-04 &  1&67e-04 & -2&98e-04 &  1&68e-04 &  0&00e-00 \\   
    NOAA-19 &  4  &               285&01 & -5&11e-06 &  2&24e-06 & -3&32e-06 &  2&25e-06 &  0&00e-00 \\   
            &  5  &               284&98 &  1&08e-05 & -3&90e-06 &  6&78e-06 & -3&56e-06 &  0&00e-00 \vspace{0.75em}\\
            &  3B &               286&33 & -4&28e-04 &  1&82e-04 & -2&75e-04 & -6&51e-03 & -8&11e-03 \\   
    MetOp-A &  4  &               285&01 & -4&68e-06 &  2&15e-06 & -3&06e-06 &  2&98e-06 &  9&08e-07 \\   
            &  5  &               284&98 &  1&17e-05 & -4&24e-06 &  7&43e-06 & -7&09e-06 & -5&24e-06 \\ 
    \hline
  \end{tabular}
  \caption{Effective temperature residuals for a blackbody temperature of 285K between the reference AVHRR SRFs (derived from the NESDIS/STAR AVHRR SRFs \citep{NESDIS_AVHRR_SRFs} via spline interpolation with tension 5.0), different interpolation methods (including none at all), and the current SRF used in CRTM processing.}
  \label{tab:teff_comparison}
\end{table}

