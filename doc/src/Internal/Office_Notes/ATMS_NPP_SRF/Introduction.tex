\section{Introduction}
%=====================

Typically, when preparing the CRTM for microwave sensors, the only available channel frequency data consists of central frequencies ($f_0$), sideband offsets ($df_1$,$df_2$), and channel bandwidths ($\Delta f$). We start with the instrument specification of those quantities and later, once the instrument has been built and undergone testing, we end up with measured values for those quantities.

In generating CRTM coefficients, these frequency values are used to contruct ``boxcar'' frequency responses. These boxcar responses are then used in the convolution of monochromatic quantities such as Planck radiances or transmittances to produce such things as polychromatic correction coefficients and instrument resolution transmittances. The latter are then regressed against a set of predictors to produce the fast transmittance model coefficients used by the CRTM. As such, the accuracy of the CRTM for a particular sensor is sensitive to the shape of the channel spectral responses used in generating the CRTM coefficients.

This Office Note investigates the radiometric impact of using actual microwave sensor spectral response functions (SRFs) for the NPP ATMS sensor. Digistisations of the ATMS channel SRFs have been made for the sensor nominal operations case from G. De Amici (NGAS), as well as for digitisations of channel SRFs for non-nominal operations (different baseplate temperatures and bias voltages), at both low and high spectral resolution (i.e. with and without SRF wings beyond -3dB insertion loss), from C.L. Chidester (SDL/USU).

