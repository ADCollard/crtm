% The generic preamble
\documentclass[10pt,letterpaper,fleqn,titlepage]{article}

% Define packages to use
\usepackage{natbib}
\usepackage[dvips]{graphicx,color}
\usepackage{amsmath,amssymb}
\usepackage{bm}
\usepackage{caption}
\usepackage{xr}
\usepackage{ifthen}
\usepackage[dvipdfm,colorlinks,linkcolor=blue,citecolor=blue,urlcolor=blue]{hyperref}
\usepackage{fancybox}
\usepackage{textcomp}
\usepackage{alltt}
%\usepackage{floatflt}
%\usepackage{svn}


% Redefine default page
\setlength{\textheight}{9in}  % 1" above and below
\setlength{\textwidth}{6.75in}   % 0.5" left and right
\setlength{\oddsidemargin}{-0.25in}

% Redefine default paragraph
\setlength{\parindent}{0pt}
\setlength{\parskip}{1ex plus 0.5ex minus 0.2ex}

% Define caption width and default fonts
\setlength{\captionmargin}{0.5in}
\renewcommand{\captionfont}{\sffamily}
\renewcommand{\captionlabelfont}{\bfseries\sffamily}

% Define commands for super- and subscript in text mode
\newcommand{\superscript}[1]{\ensuremath{^\textrm{#1}}}
\newcommand{\subscript}[1]{\ensuremath{_\textrm{#1}}}

% Derived commands
\newcommand{\invcm}{\textrm{cm\superscript{-1}}}
\newcommand{\micron}{\ensuremath{\mu\textrm{m}}}

\newcommand{\df}{\ensuremath{\delta f}}
\newcommand{\Df}{\ensuremath{\Delta f}}
\newcommand{\dx}{\ensuremath{\delta x}}
\newcommand{\Dx}{\ensuremath{X_{max}}}
\newcommand{\Xeff}{\ensuremath{X_{eff}}}

\newcommand{\water}{\textrm{H\subscript{2}O}}
\newcommand{\carbondioxide}{\textrm{CO\subscript{2}}}
\newcommand{\ozone}{\textrm{O\subscript{3}}}

\newcommand{\taup}[1]{\ensuremath{\tau_{#1}}}
\newcommand{\efftaup}[1]{\ensuremath{\tau_{#1}^{*}}}

\newcommand{\textbfm}[1]{\boldmath\ensuremath{#1}\unboldmath}

\newcommand{\rb}[1]{\raisebox{1.5ex}[0pt]{#1}}

\newcommand{\f}[1]{\texttt{#1}}

% Define how equations are numbered
\numberwithin{equation}{section}
\numberwithin{figure}{section}
\numberwithin{table}{section}

% Define a command for title page author email footnote
\newcommand{\email}[1]
{%
  \renewcommand{\thefootnote}{\alph{footnote}}%
  \footnote{#1}
  \renewcommand{\thefootnote}{\arabic{footnote}}
}

% Define a command to print the Office Note subheading
\newcommand{\notesubheading}[1]
{%
  \ifthenelse{\equal{#1}{}}{}
  { {\Large\bfseries Office Note #1\par}%
    {\scriptsize \sc This is an unreviewed manuscript, primarily intended for informal}\\ 
    {\scriptsize \sc exchange of information among JCSDA researchers\par}%
  }
}

% Redefine the maketitle macro
\makeatletter
\def\docseries#1{\def\@docseries{#1}}
\def\docnumber#1{\def\@docnumber{#1}}
\renewcommand{\maketitle}
{%
  \thispagestyle{empty}
  \vspace*{1in}
  \begin{center}%
     \sffamily
     {\huge\bfseries Joint Center for Satellite Data Assimilation\par}%
     \notesubheading{\@docnumber}
  \end{center}
  \begin{flushleft}%
     \sffamily
     \vspace*{0.5in}
     {\Large\bfseries\ifthenelse{\equal{\@docseries}{}}{}{\@docseries: }\@title\par}%
     \medskip
     {\large\@author\par}%
     \medskip
     {\large\@date\par}%
     \bigskip\hrule\vspace*{2pc}%
  \end{flushleft}%
  \newpage
  \setcounter{footnote}{0}
}
\makeatother
\docseries{}
\docnumber{}


% Define a command for a DRAFT watermark
\usepackage{eso-pic}
\newcommand{\draftwatermark}
{
  \AddToShipoutPicture{%
    \definecolor{lightgray}{gray}{.85}
    \setlength{\unitlength}{1in}
    \put(2.5,3.5){%
      \rotatebox{45}{%
        \resizebox{4in}{1in}{%
          \textsf{\textcolor{lightgray}{DRAFT}}
        }
      }
    }
  }
}




% Define included documents
\includeonly{ATMS,CrIS,VIIRS}


% Definitions for tables
\newcommand{\rb}[1]{\raisebox{1.5ex}[0pt]{#1}}
\newcommand{\frequency}[1]{\ensuremath{f_{#1}}}
\newcommand{\bfrequency}[1]{\boldmath\frequency{#1}\unboldmath}
\newcommand{\bdf}[1]{\boldmath\df{#1}\unboldmath}
\newcommand{\sideband}[1]{\ensuremath{df_{#1}}}
\newcommand{\bsideband}[1]{\boldmath\sideband{#1}\unboldmath}

% Title info
\title{Instrument Information for NPP/NPOESS Instruments: ATMS, CrIS, and VIIRS}
\author{Paul van Delst\email{paul.vandelst@noaa.gov}\\JCSDA/EMC/SAIC}
\date{June, 2008}
\docnumber{(unassigned)}
\docseries{CRTM}


%-------------------------------------------------------------------------------
%                            Ze document begins...
%-------------------------------------------------------------------------------
\begin{document}
\maketitle

\draftwatermark

\begin{abstract}
This document describes the necessary NPP/NPOESS instrument information for ATMS, CrIS, and VIIRS to allow computation of instrument resolution transmittances from a line-by-line transmittance model. These instrument transmittances are then used to derive the fast model coefficients for the CRTM.
\textbf{Keywords}: NPP, NPOESS, ATMS, CrIS, VIIRS, CRTM, fast model.
\end{abstract}

\section{Introduction}
%=====================

\subsection{Sensor identification}
%---------------------------------
The CRTM and WMO\cite{WMO_Common_Code_Tables2007} sensor identifications for the NPP instruments are shown in table \ref{tab:npp_sensor_id}.
\begin{table}[htp]
  \centering
  \begin{tabular}{|c|c|c|c|}
    \hline
    \textbf{Sensor} & \textbf{CRTM Sensor Id} & \textbf{WMO Sensor Id} & \textbf{WMO Satellite Id} \\
    \hline\hline
    ATMS  & \texttt{atms\_c1}  & 621 &            \\
    CrIS  & \texttt{cris\_c1}  & 620 & Unassigned \\
    VIIRS & \texttt{viirs\_c1} & 616 &            \\
    \hline
  \end{tabular}
  \caption{Sensor and satellite identification for NPP}
  \label{tab:npp_sensor_id}
\end{table}


% Include all the other various sections
%=======================================
\section{ATMS (Advanced Technology Microwave Sounder)}
%=====================================================

\subsection{Channel type definitions}
%------------------------------------
In generating the line-by-line transmittances for microwave instruments, the ``intermediate frequencies'' are used to define the sensor responses. The ATMS will be treated similarly to heterodyne receivers like the AMSU-A instruments where the received radio frequency (RF) is downconverted to a lower intermediate frequency (IF). The relationship between the intermediate and measurement frequencies is shown in figure \ref{fig:broadband_frequency_translation}. Note that this treatment is purely schematic for explanatory purposes only - it should not be construed as a description of the construction or behaviour of the ATMS itself.
\begin{figure}[htp]
  \centering
  \input{graphics/broadband_frequency_translation.pstex_t}
  \caption{Frequency translation in heterodyne reception for a broadband signal. Adapted from fig.1.9b in \cite{Janssen1993}}
  \label{fig:broadband_frequency_translation}
\end{figure}

For modeling the ATMS, the channels are divided into three types,
\begin{itemize}
  \item{Single passband channels (1-5,7-10,16,17)}
  \item{Double sideband channels (6,11,18-22)}
  \item{Quadruple sideband channels (12-15)}
\end{itemize}
Single passband channels are defined as those whose bandwidth span the channel centre frequency, as shown in figure \ref{fig:single_passband}. Typically for these channels stopbands are specified to reduce the effects of local oscillator noise, although no information is currently available on whetrher or not the ATMS single passband channels have stopbands. Double sideband channels are shown schematically in figure \ref{fig:double_sideband}. These channels are also referred to as folded passbands with the lower frequency sideband referred to as the lower sideband and the higher frequency sideband being the upper sideband. Quadruple sideband channels are shown schematically in figure \ref{fig:quadruple_sideband}. Note that all of the schematic channel definitions in figures \ref{fig:double_sideband} and \ref{fig:quadruple_sideband} assume bandwidth symmetry about the central and first sideband offset frequencies.
\begin{figure}[htp]
  \centering
  \input{graphics/single_passband.pstex_t}
  \caption{Schematic illustration of a single passband microwave channel}
  \label{fig:single_passband}
\end{figure}
\begin{figure}[htp]
  \centering
  \input{graphics/double_sideband.pstex_t}
  \caption{Schematic illustration of a double sideband microwave channel}
  \label{fig:double_sideband}
\end{figure}
\begin{figure}[htp]
  \centering
  \input{graphics/quadruple_sideband.pstex_t}
  \caption{Schematic illustration of a quadruple sideband microwave channel}
  \label{fig:quadruple_sideband}
\end{figure}

\subsection{Channel Frequencies}
%-------------------------------
The central and sideband offset frequencies listed here are taken from \cite{Muth_etal_2004}, and the bandwidth frequencies are taken from \cite{ATMS_PFM_CalLog}; these data are shown in table \ref{tab:atms_fo_sb_and_df}.
\begin{table}[htp]
  \centering
  \begin{tabular}{|c|c|c|c|c|}
    \hline
                     & \textbf{Central Frequency}\superscript{a} & \textbf{Sideband 1 Offset}\superscript{a} & \textbf{Sideband 2 Offset}\superscript{a} & \textbf{Bandwidth}\superscript{b} \\
    \textbf{Channel} & \bfrequency{0}             & \sideband{1}               & \sideband{2}               & $\Delta f$         \\
                     & (GHz)                      & (GHz)                      & (GHz)                      & (MHz)              \\
    \hline\hline
            1        &           23.800000        & 0.0                        & 0.0                        & 0.258         \\
            2        &           31.400000        & 0.0                        & 0.0                        & 0.172         \\
            3        &           50.300000        & 0.0                        & 0.0                        & 0.173         \\
            4        &           51.760000        & 0.0                        & 0.0                        & 0.381         \\
            5        &           52.800000        & 0.0                        & 0.0                        & 0.366         \\
            6        &           53.596000        & 0.115                      & 0.0                        & 0.1587,0.1648\superscript{c} \\
            7        &           54.400000        & 0.0                        & 0.0                        & 0.387         \\
            8        &           54.940000        & 0.0                        & 0.0                        & 0.387         \\
            9        &           55.500000        & 0.0                        & 0.0                        & 0.317         \\
           10        &           57.290344        & 0.0                        & 0.0                        & 0.151         \\
           11        &           57.290344        & 0.217                      & 0.0                        & 0.0763        \\
           12        &           57.290344        & 0.3222                     & 0.048                      & 0.0351        \\
           13        &           57.290344        & 0.3222                     & 0.022                      & 0.01547       \\
           14        &           57.290344        & 0.3222                     & 0.010                      & 0.0078,0.0079\superscript{c} \\
           15        &           57.290344        & 0.3222                     & 0.0045                     & 0.0029        \\
           16        &           88.200000        & 0.0                        & 0.0                        & 1.9282        \\
           17        &          165.500000        & 0.0                        & 0.0                        & 1.1251        \\
           18        &          183.310000        & 7.0                        & 0.0                        & 1.9302        \\
           19        &          183.310000        & 4.5                        & 0.0                        & 1.9519        \\
           20        &          183.310000        & 3.0                        & 0.0                        & 0.9799        \\
           21        &          183.310000        & 1.8                        & 0.0                        & 0.9823        \\
           22        &          183.310000        & 1.0                        & 0.0                        & 0.4940        \\
    \hline
  \end{tabular}
  \caption{Central, sideband offset, and bandwidth frequencies for ATMS. \superscript{a}Data from \cite{Muth_etal_2004}. \superscript{b}Data from \cite{ATMS_PFM_CalLog}. \superscript{c}Different lower and upper sideband widths reported. }
  \label{tab:atms_fo_sb_and_df}
\end{table}
Typically, it is assumed the upper and lower sideband bandwidths are the same for the double and quadruple sideband channels. However, for ATMS channels 6 and 14, different bandwidth for the sidebands were reported (e.g. the lower and upper sidebands for channel 6 differ by about 3.7\%). Thus, rather than determining the intermediate frequency ranges (under the assumption of bandwidth symmetry about the central frequency or sideband offsets), the band edge frequencies relative to the channel central frequencies will be specified, as shown in figure \ref{fig:broadband_frequency_translation}. For the single passband channels, assuming no stopbands), the frequency range $f_1 \rightarrow f_2$ is simply given by,
\begin{eqnarray*}
  f_1 & = & f_0 - \frac{\Delta f}{2} \\
  f_2 & = & f_0 + \frac{\Delta f}{2}
\end{eqnarray*}
The band edge frequency ranges for the ATMS single passband channels are shown in table \ref{tab:atms_single_f}.
\begin{table}[htp]
  \centering
  \begin{tabular}{|c|c|}
    \hline
    \textbf{Channel} & \bfrequency{1}$\rightarrow$\bfrequency{2} \\
                     & (GHz) \\
    \hline\hline
    1   &    23.671000 - 23.929000  \\  
    2   &    31.314000 - 31.486000  \\  
    3   &    50.213500 - 50.386500  \\  
    4   &    51.569500 - 51.950500  \\  
    5   &    52.617000 - 52.983000  \\  
    7   &    54.206500 - 54.593500  \\  
    8   &    54.746500 - 55.133500  \\  
    9   &    55.341500 - 55.658500  \\  
    10  &    57.214844 - 57.365844  \\  
    16  &    87.235900 - 89.164100  \\  
    17  &    164.93745 - 166.06255  \\
    \hline
  \end{tabular}
  \caption{Computed band edge frequencies for the ATMS single passband channels}
  \label{tab:atms_single_f}
\end{table}

The frequency ranges for the double sideband channels are computed assuming the bandwidths for the lower and upper sidebands (as defined in figure \ref{fig:double_sideband}) are different,
\begin{eqnarray*}
  f_{L1} & = & f_0 - df_1 - \frac{\Delta f_L}{2} \\
  f_{L2} & = & f_0 - df_1 + \frac{\Delta f_L}{2} \\
  f_{U1} & = & f_0 + df_1 - \frac{\Delta f_U}{2} \\
  f_{U2} & = & f_0 + df_1 + \frac{\Delta f_U}{2}
\end{eqnarray*}
The band edge frequency ranges for the ATMS double sideband channels are shown in table \ref{tab:atms_double_f}.
\begin{table}[htp]
  \centering
  \begin{tabular}{|c|c|c|}
    \hline
    \textbf{Channel} & \bfrequency{L1}$\rightarrow$\bfrequency{L2} & \bfrequency{U1}$\rightarrow$\bfrequency{U2} \\
                     & (GHz) & (GHz) \\
    \hline\hline
    6   &    53.401650 - 53.560350   &   53.628600 - 53.793400 \\
    11  &    57.035194 - 57.111494   &   57.469194 - 57.545494 \\
    18  &    175.34490 - 177.27510   &   189.34490 - 191.27510 \\
    19  &    177.83405 - 179.78595   &   186.83405 - 188.78595 \\
    20  &    179.82005 - 180.79995   &   185.82005 - 186.79995 \\
    21  &    181.01885 - 182.00115   &   184.61885 - 185.60115 \\
    22  &    182.06300 - 182.55700   &   184.06300 - 184.55700 \\
    \hline
  \end{tabular}
  \caption{Computed band edge frequencies for the ATMS double sideband channels}
  \label{tab:atms_double_f}
\end{table}

For the quadruple sideband channels, we begin to run into nomenclature issues. For the purposes of this document, it will be assumed that the lower and upper sidebands shown in figure \ref{fig:quadruple_sideband} are symmetric about the central frequency, $f_0$. That is, the bandwidths of the outermost (furtherest from $f_0$) sidebands of figure \ref{fig:quadruple_sideband} are both given by $\Delta f_U$, and the bandwidths of the innermost (closest to $f_0$) sidebands of figure \ref{fig:quadruple_sideband} are both given by $\Delta f_L$. To distinguish between sidebands less than and greater than $f_0$, the subscripts - and + shall be used respectively. The frequency ranges for the quadruple sideband channels are then given by,
\begin{eqnarray*}
  f_{U1-} & = & f_0 - df_1 - df_2 - \frac{\Delta f_U}{2} \\
  f_{U2-} & = & f_0 - df_1 - df_2 + \frac{\Delta f_U}{2} \\
  f_{L1-} & = & f_0 - df_1 + df_2 - \frac{\Delta f_L}{2} \\
  f_{L2-} & = & f_0 - df_1 + df_2 + \frac{\Delta f_L}{2} \\
  f_{L1+} & = & f_0 + df_1 + df_2 - \frac{\Delta f_L}{2} \\
  f_{L2+} & = & f_0 + df_1 + df_2 + \frac{\Delta f_L}{2} \\
  f_{U1-} & = & f_0 + df_1 - df_2 - \frac{\Delta f_U}{2} \\
  f_{U2-} & = & f_0 + df_1 - df_2 + \frac{\Delta f_U}{2}   
\end{eqnarray*}
The band edge frequency ranges for the ATMS quadruple sideband channels are shown in table \ref{tab:atms_quadruple_f}.
\begin{table}[htp]
  \centering
  \begin{tabular}{|c|c|c|c|c|}
    \hline
    \textbf{Channel} & \bfrequency{U1-}$\rightarrow$\bfrequency{U2-} & \bfrequency{L1-}$\rightarrow$\bfrequency{L2-} & \bfrequency{L1+}$\rightarrow$\bfrequency{L2+} & \bfrequency{U1+}$\rightarrow$\bfrequency{U2+} \\
       & (GHz)     & (GHz)     & (GHz)     & (GHz) \\
    \hline\hline
    12 & 56.902594 - 56.937694 & 56.998594 - 57.033694 & 57.546994 - 57.582094 & 57.642994 - 57.678094 \\
    13 & 56.938409 - 56.953879 & 56.982409 - 56.997879 & 57.582809 - 57.598279 & 57.626809 - 57.642279 \\
    14 & 56.954194 - 56.962094 & 56.974244 - 56.982044 & 57.598644 - 57.606444 & 57.618594 - 57.626494 \\
    15 & 56.962194 - 56.965094 & 56.971194 - 56.974094 & 57.606594 - 57.609494 & 57.615594 - 57.618494 \\
    \hline
  \end{tabular}
  \caption{Computed band edge frequencies for the ATMS quadruple sideband channels}
  \label{tab:atms_quadruple_f}
\end{table}


\section{CrIS (Cross-track Infrared Sounder)}
%============================================

\subsection{Spectral and interferometric domain transforms}
%----------------------------------------------------------
In generating the line-by-line transmittances for a Fourier Transform Interferometer (FTS), we assume an idealised instrument. The relationships between the spectral frequency interval and bandwidth, {\df} and {\Df}, and the effective interferogram sampling interval and maximum optical path difference, {\dx} and {\Dx}, can be simply expressed by,
\begin{equation}\df = \frac{1}{2.\Dx}\label{eqn:df_deltax}\end{equation}
\begin{equation}\Df = \frac{1}{2.\dx}\label{eqn:deltaf_dx}\end{equation}
(Fully general definitions are more nuanced than this, but for this exercise, the above will suffice. See \citet{Bell_1972} for an exhaustive treatment.)

The relationship between the number of spectral (SPC) and double-sided interferogram (IFG) points is given by,
\begin{equation}N_{IFG} = 2.(N_{SPC}-1)\end{equation}
\begin{equation}N_{SPC} = \left(\frac{N_{IFG}}{2}\right) - 1\end{equation}

These relationships are shown schematically in figure \ref{fig:X_F_defn}. Note that maximum positive abscissa at $x = \Dx$, or $f = f_{2}$, does not have an equivalent negative point pairing.
\begin{figure}[htp]
  \centering
  \input{graphics/X_F_definition.pstex_t}
  \caption{Schematic illustration of interferogram and spectrum point ordering. The circles represent positive delays or frequencies, and the squares negative. (ZPD=Zero Path Difference)}
  \label{fig:X_F_defn}
\end{figure}

\subsection{Band Frequencies}
%----------------------------
It assumed that the CrIS radiances that are to be modeled have been corrected for self-apodisation effects and detector nonlinearities, and have been resampled to the frequency grids shown in table \ref{tab:cris_band_f}.
\begin{table}[htp]
  \centering
  \begin{tabular}{|c|c|c|c|c|c|}
    \hline
    \textbf{Band} & \bfrequency{min} & \bfrequency{max} & \boldmath$\Delta f$\unboldmath & \boldmath$\delta f$\unboldmath & \boldmath$N_{SPC}$\unboldmath       \\
                  & (\invcm)         & (\invcm)         & (\invcm)   & (\invcm)   & \textbf{Number of points} \\
    \hline\hline
    1 (LW) &  650.0 & 1095.0 & 445.0 & 0.625 & 713 \\
    2 (MW) & 1210.0 & 1750.0 & 540.0 & 1.25  & 433 \\
    3 (SW) & 2155.0 & 2550.0 & 395.0 & 2.5   & 159 \\
    \hline
  \end{tabular}
  \caption{Frequency range and spacing for the CrIS bands. From \cite{CrIS_SDR_ATBD}.}
  \label{tab:cris_band_f}
\end{table}

\section{VIIRS (Visible Infrared Imager/Radiometer Suite)}
%=========================================================

No information.



% The references section
%=======================
\bibliographystyle{plain}
\bibliography{bibliography}	
%\begin{thebibliography}{99}
%  \bibitem{ref:WMO_codes} WMO Operational Codes as of 7 November, 2007; Common Code Tables C-5 and C-8 from http://www.wmo.int/pages/prog/www/WMOCodes/OperationalCodes.html  
%  \bibitem{ref:Janssen93} Atmospheric remote sensing by microwave radiometry, edited by Michael A. Janssen. Wiley, New York, c1993.
%\end{thebibliography}


%% The appendices section
%%=======================
%\begin{appendix}
%
%\end{appendix}


\end{document}

