\chapter*{What's New in v2.0.1}
%==============================
\addcontentsline{toc}{chapter}{What's New in v2.0.1}

The v2.0.1 update to the CRTM was done to
\begin{itemize}
  \item Fix defects of varying severity
  \item Refactor some modules to work around compiler bugs
  \item Reorganise the testing/example program.
\end{itemize}

\section*{Bug Fixes}
%-------------------
\addcontentsline{toc}{section}{Bug Fixes}

\begin{description}

\item[Replacing \f{CRTM\_Atmosphere\_IsValid} \f{WARNING} message for missing ozone with \f{FAILURE}] The CRTM contains two different transmittance model algorithm: the Optical Depth in Absorber Space (ODAS) algorithm and the Optical Depth in Pressure Space (ODPS) algorithm. The ODPS algorithm was constructed to handle ``missing'' profiles of major trace gas absorbers (e.g. ozone). The ODAS algorithm, however, cannot yet handle a missing ozone profile. As such, we have switched back to missing ozone being a \f{FAILURE} error, regardless of whether or not the ODAS or ODPS transmittance algorithm is being used. See ticket \ticket{150}\footnote{The ticket references and links are included to allow CRTM developers to easily navigate to the CRTM Source Code Management system from this document}.

\item[Allowed for user profile top level pressures to be less than 0.005hPa in the ODAS algorithm.] This corrected a bug that generated negative absorber amounts for the top layer when a user input a profile where the top level pressure is \emph{less than} 0.005hPa. See ticket \ticket{151}.

\item[Fixed test of \f{SensorData\%Tb} component] The previous test (called within the \f{CRTM\_Surface\_IsValid} procedure) caused a \f{FAILURE} when \emph{any} of the supplied brightness temperatures were less than zero. This test has been changed to fail only when \emph{all} of the input brightness temperatures are less than zero to allow channel subsets of data to be passed. See ticket \ticket{110}.

\item[Corrected error mesage in \f{CRTM\_Atmosphere\_IsValid} function.] The error message for invalid input absorber units was corrected. See ticket \ticket{141}.

\item[Coefficient load message suppression in the \f{CRTM\_Init} function was not occurring correctly] This problem was traced to a logic error in several of the coefficient load procedures when the optional MPI process identifier arguments were passed in. The logic has been corrected in the affected load procedures. See ticket \ticket{143}.

\end{description}


\section*{Refactor for Compiler Defects}
%---------------------------------------
\addcontentsline{toc}{section}{Refactor for Compiler Defects}

\begin{description}

\item[Memory leak in \f{CRTM\_IRSSEM} module fixed] This was a bug caused by apparent compiler bugs (in more than one compiler) where declaring the internals of a local (i.e. not \f{PUBLIC}) structure as \f{PRIVATE} caused a memory leak. Removal of the internal \f{PRIVATE} statement solved the problem. See ticket \ticket{144}.

\item[Modification of \f{Type\_Kinds} module to allow for Intel ifort compilation] This work around was necessary due to an ifort v11.1 compiler bug that surfaced due to the CRTM build switches for this compiler promote compiler warnings to errors. Rather than require users to modify their compilation setup to avoid this error, the \f{Type\_Kinds} module was modified to avoind it entirely. See ticket \ticket{112}.

\item[Modification of \f{CRTM\_Atmosphere\_AddLayerCopy} procedure to allow for PGI pgf95 compilation] The REL-2.0 version of the \f{CRTM\_Atmosphere\_AddLayerCopy} procedure was identified as a problem for the PGI pgf95 v10.2-1 compiler. A bug report was submitted to PGI Support and filed as TPR 16814. The bug is fixed in the v10.4 release of the pgf95 compiler, which does not have a problem with the original CRTM code. See ticket \ticket{114}.

\end{description}


\section*{Reorganistion of Test/Example Programs}
%------------------------------------------------
\addcontentsline{toc}{section}{Reorganistion of Test/Example Programs}

This update is probably the biggest change in REL-2.0.1. The CRTM tarball structure was updated to include the test/example codes -- as opposed to supplying a separate tarball just for the example programs. The reasoning here was to establish the typical ``\f{make}, \f{make test}'' procedures for building packages, but be aware that the setup is still rather unsophisticated; we are still investigating ways to more easily configure the CRTM library and test/example programs (e.g. \href{http://www.gnu.org/software/autoconf/}{\f{autoconf}}).

For a full description of the necessary steps to build the CRTM library and test/example programs, refer to the \f{README} file supplied with the CRTM release tarball.

