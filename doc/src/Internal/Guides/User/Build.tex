\chapter{How to build the CRTM library}
%======================================
\label{sec:build}

\section{Build Files}
%====================
The build system for the CRTM is currently quite unsophisticated. It consists of a number of make and include files in the CRTM tarball hierarchy:

\begin{tabular}{l@{ : }p{4.75in}}
  \,\texttt{makefile} & The main makefile\\
  \,\texttt{make.macros} & The include file containing all the defined macros, including the compiler and linker flags.\\
  \,\texttt{make.rules} & The include file containing the suffix rules for compiling Fortran95 source code.\\
\end{tabular}

\section{Predefined Build Targets}
%=================================
Several targets are available for specific compilers. The compilers and makefile target names are shown in table \ref{tab:predefined_build_targets}. Both ``production'' and debug target names are supplied, with the former using compiler switches to produce fast code and the latter using compiler switches to turn on all the available debugging capabilities. Note that the debug targets will produce executables much slower than the production builds. 
\begin{table}[htp]
  \centering
  \begin{tabular}{clcc}
    \hline
    \sffamily\textbf{Platform} & \sffamily\textbf{Compiler} & \sffamily\textbf{Production Target} & \sffamily\textbf{Debug Target} \\
    \hline\hline
    \multirow{6}{*}{Linux} & GNU gfortran          & \texttt{gfortran} & \texttt{gfortran\_debug}\\
                           & Lahey lf95$^\dagger$  & \texttt{lahey}    & \texttt{lahey\_debug}   \\
                           & Intel ifort           & \texttt{intel}    & \texttt{intel\_debug}   \\
                           & PGI pgf95             & \texttt{pgi}      & \texttt{pgi\_debug}     \\
                           & g95                   & \texttt{g95}      & \texttt{g95\_debug}     \\
                           & Absoft$^\dagger$      & \texttt{absoft}   & \texttt{absoft\_debug}  \\[0.2cm]
    IBM                    & AIX xlf95             & \texttt{ibm}      & \texttt{ibm\_debug}     \\[0.2cm]
    HP                     & HP-UX f90$^\dagger$   & \texttt{hp}       & \texttt{hp\_debug}      \\[0.2cm]
    SGI                    & IRIX f90$^\dagger$    & \texttt{sgi}      & \texttt{sgi\_debug}     \\[0.2cm]
    Sun                    & Solaris f90$^\dagger$ & \texttt{sun}      & \texttt{sun\_debug}     \\
    \hline
  \end{tabular}
  \caption{Predefined makefile targets for CRTM library build. ($^\dagger$ Untested.)}
  \label{tab:predefined_build_targets}
\end{table}

The actual compiler and switch definitions are listed in the \texttt{make.macros} file. Also note that not all of the above compilers have been tested for the current release.

\section{Building the library}
%=============================
The default build is performed by simply typing,

\begin{verbatim}     make\end{verbatim}

and is operating system sensitive in that the compiler is selected based on the platform. For example, if you build on an IBM/AIX system, the xlf95 compiler is invoked. Doing the same on a linux or MacOSX system invokes the default linux compiler; currently this is the GNU gfortran\footnote{Note that for gfortran a version of 4.4 or later is required to build the CRTM library. Earlier versions (4.1,4.2,4.3) did not implement all the Fortran95+ features used in the CRTM source code.} compiler.

So, if you are using a linux or MacOSX system and want to modify the \emph{default} CRTM build (different compiler, different switches), you will need to edit the \texttt{make.macros} file. At the bottom of this file, you will see the following macro definitions,
\begin{verbatim}
     LINUX_FLAGS = $(LINUX_FLAGS_GFORTRAN)
     #LINUX_FLAGS = $(LINUX_FLAGS_LAHEY)
     #LINUX_FLAGS = $(LINUX_FLAGS_PGI)
     #LINUX_FLAGS = $(LINUX_FLAGS_INTEL)
     #LINUX_FLAGS = $(LINUX_FLAGS_G95)\end{verbatim}
where you can see the generic \texttt{LINUX\_FLAGS} macros is set to \texttt{LINUX\_FLAGS\_GFORTRAN}. To use any of the other defined compilers (Lahey, PGI, Intel, or g95), simply uncomment the required line. For example, if your compiler of choice is Intel \texttt{ifort} and you want that to be the default build, the change would simply be
\begin{verbatim}
     #LINUX_FLAGS = $(LINUX_FLAGS_GFORTRAN)
     #LINUX_FLAGS = $(LINUX_FLAGS_LAHEY)
     #LINUX_FLAGS = $(LINUX_FLAGS_PGI)
     LINUX_FLAGS = $(LINUX_FLAGS_INTEL)
     #LINUX_FLAGS = $(LINUX_FLAGS_G95)\end{verbatim}
Note, however, than an Intel \texttt{ifort} build can also be obtained by using the predefined makefile targets as shown in table \ref{tab:predefined_build_targets}.
     
     
\section{Installing the library}
%===============================
A very simple install target is specified in the supplied makefile to put all the necessary include files (the generated \texttt{*.mod} files containing all the procedure interface information) in an \texttt{/include} subdirectory and the library itself (the generated \texttt{libCRTM.a} file) in a \texttt{/lib} subdirectory. The make command is
\begin{verbatim}     make install\end{verbatim}
The \texttt{/include} and \texttt{/lib} subdirectories can then be moved to a more suitable location on your system, for example: \texttt{\$HOME/local/CRTM}


\section{Linking to the library}
%===============================
Let's assume you've built the CRTM library and placed the \texttt{/include} and \texttt{/lib} subdirectories in your own local area, \texttt{\$HOME/local/CRTM}. In the makefile for your application that uses the CRTM, you will need to add
\begin{verbatim}     -I$HOME/local/CRTM/include\end{verbatim}
to your list of compilation switches, and the following to your list of link switches,
\begin{verbatim}     -L$HOME/local/CRTM/lib -lCRTM\end{verbatim}
