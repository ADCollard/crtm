\chapter{Structure and procedure interface definitions}
%======================================================

\clearpage
\section{\ChannelInfo{} Structure}
%=================================
\label{sec:channelinfo_structure}

\begin{figure}[htp]
  \centering
  \doublebox{
  \begin{minipage}[b]{6.5in}
    \begin{alltt}
  TYPE :: CRTM_ChannelInfo_type
    INTEGER :: n_Allocates = 0
    ! Dimensions
    INTEGER :: n_Channels = 0  ! L dimension
    ! Scalar data
    CHARACTER(STRLEN) :: Sensor_ID        = ' '
    INTEGER           :: WMO_Satellite_ID = INVALID_WMO_SATELLITE_ID
    INTEGER           :: WMO_Sensor_ID    = INVALID_WMO_SENSOR_ID
    INTEGER           :: Sensor_Index     = 0
    ! Array data
    INTEGER, POINTER :: Sensor_Channel(:) => NULL()  ! L
    INTEGER, POINTER :: Channel_Index(:)  => NULL()  ! L
  END TYPE CRTM_ChannelInfo_type
    \end{alltt}
  \end{minipage}
  }
  \caption{CRTM\_ChannelInfo\_type structure definition.}
  \label{fig:CRTM_ChannelInfo_type_structure}
\end{figure}


% ChannelInfo structure methods
%-----------------------------
\subsection{\texttt{CRTM\_Associated\_ChannelInfo} interface}
  \label{sec:CRTM_Associated_ChannelInfo_interface}
  \begin{alltt}
 
  NAME:
        CRTM_Associated_ChannelInfo
 
  PURPOSE:
        Function to test the association status of the pointer members of a
        CRTM_ChannelInfo structure.
 
  CALLING SEQUENCE:
        Association_Status = CRTM_Associated_ChannelInfo( ChannelInfo      , &
                                                          ANY_Test=Any_Test  )
 
  INPUT ARGUMENTS:
        ChannelInfo: ChannelInfo structure which is to have its pointer
                     member's association status tested.
                     UNITS:      N/A
                     TYPE:       CRTM_ChannelInfo_type
                     DIMENSION:  Scalar OR Rank-1 array
                     ATTRIBUTES: INTENT(IN)
 
  OPTIONAL INPUT ARGUMENTS:
        ANY_Test:    Set this argument to test if ANY of the
                     ChannelInfo structure pointer members are associated.
                     The default is to test if ALL the pointer members
                     are associated.
                     If ANY_Test = 0, test if ALL the pointer members
                                      are associated.  (DEFAULT)
                        ANY_Test = 1, test if ANY of the pointer members
                                      are associated.
 
  FUNCTION RESULT:
        Association_Status:  The return value is a logical value indicating the
                             association status of the ChannelInfo pointer
                             members.
                             .TRUE.  - if ALL the ChannelInfo pointer members
                                       are associated, or if the ANY_Test argument
                                       is set and ANY of the ChannelInfo
                                       pointer members are associated.
                             .FALSE. - some or all of the ChannelInfo pointer
                                       members are NOT associated.
                             UNITS:      N/A
                             TYPE:       LOGICAL
                             DIMENSION:  Same as input ChannelInfo argument
 
  \end{alltt}

\subsection{\texttt{CRTM\_Allocate\_ChannelInfo} interface}
  \label{sec:CRTM_Allocate_ChannelInfo_interface}
  \begin{alltt}
 
  NAME:
        CRTM_Allocate_ChannelInfo
  
  PURPOSE:
        Function to allocate the pointer members of a CRTM ChannelInfo
        data structure.
 
  CALLING SEQUENCE:
        Error_Status = CRTM_Allocate_ChannelInfo( n_Channels             , &
                                                  ChannelInfo            , &
                                                  Message_Log=Message_Log  )
 
  INPUT ARGUMENTS:
        n_Channels:   The number of channels in the ChannelInfo structure.
                      Must be > 0.
                      UNITS:      N/A
                      TYPE:       INTEGER
                      DIMENSION:  Scalar or Rank-1 array
                      ATTRIBUTES: INTENT(IN)
 
  OUTPUT ARGUMENTS:
        ChannelInfo:  ChannelInfo structure with allocated pointer members
                      UNITS:      N/A
                      TYPE:       CRTM_ChannelInfo_type
                      DIMENSION:  Same as input n_Channels argument
                      ATTRIBUTES: INTENT(IN OUT)
 
  OPTIONAL INPUT ARGUMENTS:
        Message_Log:  Character string specifying a filename in which any
                      Messages will be logged. If not specified, or if an
                      error occurs opening the log file, the default action
                      is to output Messages to standard output.
                      UNITS:      N/A
                      TYPE:       CHARACTER(*)
                      DIMENSION:  Scalar
                      ATTRIBUTES: INTENT(IN), OPTIONAL
 
  FUNCTION RESULT:
        Error_Status: The return value is an integer defining the error status.
                      The error codes are defined in the ERROR_HANDLER module.
                      If == SUCCESS the structure pointer allocations were
                                    successful
                         == FAILURE - an error occurred, or
                                    - the structure internal allocation counter
                                      is not equal to one (1) upon exiting this
                                      function. This value is incremented and
                                      decremented for every structure allocation
                                      and deallocation respectively.
                      UNITS:      N/A
                      TYPE:       INTEGER
                      DIMENSION:  Scalar
 
  COMMENTS:
        Note the INTENT on the output ChannelInfo argument is IN OUT rather than
        just OUT. This is necessary because the argument may be defined upon
        input. To prevent memory leaks, the IN OUT INTENT is a must.
 
  \end{alltt}

\subsection{\texttt{CRTM\_Destroy\_ChannelInfo} interface}
  \label{sec:CRTM_Destroy_ChannelInfo_interface}
  \begin{alltt}
 
  NAME:
        CRTM_Destroy_ChannelInfo
  
  PURPOSE:
        Function to re-initialize the scalar and pointer members of a CRTM
        ChannelInfo data structures.
 
  CALLING SEQUENCE:
        Error_Status = CRTM_Destroy_ChannelInfo( ChannelInfo            , &
                                                 Message_Log=Message_Log  )
  
  OUTPUT ARGUMENTS:
        ChannelInfo:  Re-initialized ChannelInfo structure.
                      UNITS:      N/A
                      TYPE:       CRTM_ChannelInfo_type
                      DIMENSION:  Scalar
                      ATTRIBUTES: INTENT(IN OUT)
 
  OPTIONAL INPUT ARGUMENTS:
        Message_Log:  Character string specifying a filename in which any
                      Messages will be logged. If not specified, or if an
                      error occurs opening the log file, the default action
                      is to output Messages to standard output.
                      UNITS:      N/A
                      TYPE:       CHARACTER(*)
                      DIMENSION:  Scalar
                      ATTRIBUTES: INTENT(IN), OPTIONAL
 
  FUNCTION RESULT:
        Error_Status: The return value is an integer defining the error status.
                      The error codes are defined in the ERROR_HANDLER module.
                      If == SUCCESS the structure re-initialisation was successful
                         == FAILURE - an error occurred, or
                                    - the structure internal allocation counter
                                      is not equal to zero (0) upon exiting this
                                      function. This value is incremented and
                                      decremented for every structure allocation
                                      and deallocation respectively.
                      UNITS:      N/A
                      TYPE:       INTEGER
                      DIMENSION:  Scalar
 
  COMMENTS:
        Note the INTENT on the output ChannelInfo argument is IN OUT rather than
        just OUT. This is necessary because the argument may be defined upon
        input. To prevent memory leaks, the IN OUT INTENT is a must.
 
  \end{alltt}

\subsection{\texttt{CRTM\_Assign\_ChannelInfo} interface}
  \label{sec:CRTM_Assign_ChannelInfo_interface}
  \begin{alltt}
 
  NAME:
        CRTM_Assign_ChannelInfo
 
  PURPOSE:
        Function to copy valid CRTM ChannelInfo structures.
 
  CALLING SEQUENCE:
        Error_Status = CRTM_Assign_ChannelInfo( ChannelInfo_in         , &
                                                ChannelInfo_out        , &
                                                Message_Log=Message_Log  )
 
  INPUT ARGUMENTS:
        ChannelInfo_in:  ChannelInfo structure which is to be copied.
                         UNITS:      N/A
                         TYPE:       CRTM_ChannelInfo_type
                         DIMENSION:  Scalar OR Rank-1 array
                         ATTRIBUTES: INTENT(IN)
 
  OUTPUT ARGUMENTS:
        ChannelInfo_out: Copy of the input structure, ChannelInfo_in.
                         UNITS:      N/A
                         TYPE:       CRTM_ChannelInfo_type
                         DIMENSION:  Same as ChannelInfo_in argument
                         ATTRIBUTES: INTENT(IN OUT)
 
  OPTIONAL INPUT ARGUMENTS:
        Message_Log:     Character string specifying a filename in which any
                         Messages will be logged. If not specified, or if an
                         error occurs opening the log file, the default action
                         is to output Messages to standard output.
                         UNITS:      N/A
                         TYPE:       CHARACTER(*)
                         DIMENSION:  Scalar
                         ATTRIBUTES: INTENT(IN), OPTIONAL
 
  FUNCTION RESULT:
        Error_Status:    The return value is an integer defining the error status.
                         The error codes are defined in the ERROR_HANDLER module.
                         If == SUCCESS the structure assignment was successful
                            == FAILURE an error occurred
                         UNITS:      N/A
                         TYPE:       INTEGER
                         DIMENSION:  Scalar
 
  COMMENTS:
        Note the INTENT on the output ChannelInfo argument is IN OUT rather than
        just OUT. This is necessary because the argument may be defined upon
        input. To prevent memory leaks, the IN OUT INTENT is a must.
 
  \end{alltt}

\subsection{\texttt{CRTM\_Equal\_ChannelInfo} interface}
  \label{sec:CRTM_Equal_ChannelInfo_interface}
  \begin{alltt}
 
  NAME:
        CRTM_Equal_ChannelInfo
 
  PURPOSE:
        Function to test if two ChannelInfo structures are equal.
 
  CALLING SEQUENCE:
        Error_Status = CRTM_Equal_ChannelInfo( ChannelInfo_LHS        , &
                                               ChannelInfo_RHS        , &
                                               ULP_Scale  =ULP_Scale  , &
                                               Check_All  =Check_All  , &
                                               Message_Log=Message_Log  )
 
 
  INPUT ARGUMENTS:
        ChannelInfo_LHS: ChannelInfo structure to be compared; equivalent
                         to the left-hand side of a lexical comparison, e.g.
                           IF ( ChannelInfo_LHS == ChannelInfo_RHS ).
                         UNITS:      N/A
                         TYPE:       CRTM_ChannelInfo_type
                         DIMENSION:  Scalar OR Rank-1 array
                         ATTRIBUTES: INTENT(IN)
 
        ChannelInfo_RHS: ChannelInfo structure to be compared to; equivalent
                         to the right-hand side of a lexical comparison, e.g.
                           IF ( ChannelInfo_LHS == ChannelInfo_RHS ).
                         UNITS:      N/A
                         TYPE:       CRTM_ChannelInfo_type
                         DIMENSION:  Same as ChannelInfo_LHS argument
                         ATTRIBUTES: INTENT(IN)
 
  OPTIONAL INPUT ARGUMENTS:
        ULP_Scale:      Unit of data precision used to scale the floating
                        point comparison. ULP stands for "Unit in the Last Place,"
                        the smallest possible increment or decrement that can be
                        made using a machine's floating point arithmetic.
                        Value must be positive - if a negative value is supplied,
                        the absolute value is used. If not specified, the default
                        value is 1.
                        ** NOTE: This is a hook for future changes and is not used.
                        UNITS:      N/A
                        TYPE:       INTEGER
                        DIMENSION:  Scalar
                        ATTRIBUTES: INTENT(IN), OPTIONAL
 
        Check_All:      Set this argument to check ALL the floating point
                        channel data of the ChannelInfo structures. The default
                        action is return with a FAILURE status as soon as
                        any difference is found. This optional argument can
                        be used to get a listing of ALL the differences
                        between data in ChannelInfo structures.
                        If == 0, Return with FAILURE status as soon as
                                 ANY difference is found  *DEFAULT*
                           == 1, Set FAILURE status if ANY difference is
                                 found, but continue to check ALL data.
                        UNITS:      N/A
                        TYPE:       INTEGER
                        DIMENSION:  Scalar
                        ATTRIBUTES: INTENT(IN), OPTIONAL
 
        Message_Log:    Character string specifying a filename in which any
                        messages will be logged. If not specified, or if an
                        error occurs opening the log file, the default action
                        is to output messages to standard output.
                        UNITS:      None
                        TYPE:       CHARACTER(*)
                        DIMENSION:  Scalar
                        ATTRIBUTES: INTENT(IN), OPTIONAL
 
  FUNCTION RESULT:
        Error_Status:   The return value is an integer defining the error status.
                        The error codes are defined in the Message_Handler module.
                        If == SUCCESS the structures were equal
                           == FAILURE - an error occurred, or
                                      - the structures were different.
                        UNITS:      N/A
                        TYPE:       INTEGER
                        DIMENSION:  Scalar
 
  \end{alltt}

\subsection{\texttt{CRTM\_nChannels\_ChannelInfo} interface}
  \label{sec:CRTM_nChannels_ChannelInfo_interface}
  \begin{alltt}
 
  NAME:
        CRTM_nChannels_ChannelInfo
 
  PURPOSE:
        Function to return the number of channels defined in a ChannelInfo
        structure or structure array
 
  CALLING SEQUENCE:
        nChannels = CRTM_nChannels_ChannelInfo( ChannelInfo )
 
  INPUT ARGUMENTS:
        ChannelInfo: ChannelInfo structure or structure which is to have its
                     channels counted.
                     UNITS:      N/A
                     TYPE:       CRTM_ChannelInfo_type
                     DIMENSION:  Scalar
                                   or
                                 Rank-1
                     ATTRIBUTES: INTENT(IN)
 
  FUNCTION RESULT:
        nChannels:   The number of defined channels in the input argument.
                     UNITS:      N/A
                     TYPE:       INTEGER
                     DIMENSION:  Scalar
 
  \end{alltt}

\subsection{\texttt{CRTM\_RCS\_ID\_ChannelInfo} interface}
  \label{sec:CRTM_RCS_ID_ChannelInfo_interface}
  \begin{alltt}
 
  NAME:
        CRTM_RCS_ID_ChannelInfo
 
  PURPOSE:
        Subroutine to return the module RCS Id information.
 
  CALLING SEQUENCE:
        CALL CRTM_RCS_Id_ChannelInfo( RCS_Id )
 
  OUTPUT ARGUMENTS:
        RCS_Id:        Character string containing the Revision Control
                       System Id field for the module.
                       UNITS:      N/A
                       TYPE:       CHARACTER(*)
                       DIMENSION:  Scalar
                       ATTRIBUTES: INTENT(OUT)
 
  \end{alltt}




\clearpage
\section{\Atmosphere{} Structure}
%================================
\label{sec:atmosphere_structure}

\begin{figure}[htp]
  \centering
  \doublebox{
  \begin{minipage}[b]{6.5in}
    \begin{alltt}
  TYPE :: CRTM_Atmosphere_type
    ! Allocation indicator
    LOGICAL :: Is_Allocated = .FALSE.
    ! Dimension values
    INTEGER :: Max_Layers   = 0  ! K dimension
    INTEGER :: n_Layers     = 0  ! Kuse dimension
    INTEGER :: n_Absorbers  = 0  ! J dimension
    INTEGER :: Max_Clouds   = 0  ! Nc dimension
    INTEGER :: n_Clouds     = 0  ! NcUse dimension
    INTEGER :: Max_Aerosols = 0  ! Na dimension
    INTEGER :: n_Aerosols   = 0  ! NaUse dimension
    ! Number of added layers
    INTEGER :: n_Added_Layers = 0
    ! Climatology model associated with the profile
    INTEGER :: Climatology = US_STANDARD_ATMOSPHERE
    ! Absorber ID and units
    INTEGER, ALLOCATABLE :: Absorber_ID(:)    ! J
    INTEGER, ALLOCATABLE :: Absorber_Units(:) ! J
    ! Profile LEVEL and LAYER quantities
    REAL(fp), ALLOCATABLE :: Level_Pressure(:)  ! 0:K
    REAL(fp), ALLOCATABLE :: Pressure(:)        ! K
    REAL(fp), ALLOCATABLE :: Temperature(:)     ! K
    REAL(fp), ALLOCATABLE :: Absorber(:,:)      ! K x J
    ! Clouds associated with each profile
    TYPE(CRTM_Cloud_type),   ALLOCATABLE :: Cloud(:)    ! Nc
    ! Aerosols associated with each profile
    TYPE(CRTM_Aerosol_type), ALLOCATABLE :: Aerosol(:)  ! Na
  END TYPE CRTM_Atmosphere_type
    \end{alltt}
  \end{minipage}
  }
  \caption{CRTM\_Atmosphere\_type structure definition.}
  \label{fig:CRTM_Atmosphere_type_structure}
\end{figure}


% Climatology table
\begin{table}[htp]
  \centering
  \begin{tabular}{c l}
    \hline
    \sffamily\textbf{Climatology Type} & \sffamily\textbf{Parameter} \\
    \hline\hline
             Tropical          &  \texttt{TROPICAL}\\              
        Midlatitude summer     &  \texttt{MIDLATITUDE\_SUMMER}\\
        Midlatitude winter     &  \texttt{MIDLATITUDE\_WINTER}\\
         Subarctic summer      &  \texttt{SUBARCTIC\_SUMMER}\\
         Subarctic winter      &  \texttt{SUBARCTIC\_WINTER}\\
     U.S. Standard Atmosphere  &  \texttt{US\_STANDARD\_ATMOSPHERE}\\
    \hline 
  \end{tabular}
  \caption{CRTM \Atmosphere{} structure valid \texttt{Climatology} definitions. The same set as defined for LBLRTM is used.}
  \label{tab:climatology}
\end{table}

% Absorber id table
\begin{table}[htp]
  \centering
  \begin{tabular}{ c l c c l }
    \hline
    \sffamily\textbf{Molecule} & \sffamily\textbf{Parameter} & \hspace{0.5cm} & \sffamily\textbf{Molecule} & \sffamily\textbf{Parameter}\\
    \hline\hline
     H\subscript{2}O  & \texttt{H2O\_ID}  & \hspace{0.5cm} & HI                           & \texttt{HI\_ID}    \\
     CO\subscript{2}  & \texttt{CO2\_ID}  & \hspace{0.5cm} & ClO                          & \texttt{ClO\_ID}   \\
     O\subscript{3}   & \texttt{O3\_ID}   & \hspace{0.5cm} & OCS                          & \texttt{OCS\_ID}   \\
     N\subscript{2}O  & \texttt{N2O\_ID}  & \hspace{0.5cm} & H\subscript{2}CO             & \texttt{H2CO\_ID}  \\
     CO               & \texttt{CO\_ID}   & \hspace{0.5cm} & HOCl                         & \texttt{HOCl\_ID}  \\
     CH\subscript{4}  & \texttt{CH4\_ID}  & \hspace{0.5cm} & N\subscript{2}               & \texttt{N2\_ID}    \\
     O\subscript{2}   & \texttt{O2\_ID}   & \hspace{0.5cm} & HCN                          & \texttt{HCN\_ID}   \\
     NO               & \texttt{NO\_ID}   & \hspace{0.5cm} & CH\subscript{3}l             & \texttt{CH3l\_ID}  \\
     SO\subscript{2}  & \texttt{SO2\_ID}  & \hspace{0.5cm} & H\subscript{2}O\subscript{2} & \texttt{H2O2\_ID}  \\
     NO\subscript{2}  & \texttt{NO2\_ID}  & \hspace{0.5cm} & C\subscript{2}H\subscript{2} & \texttt{C2H2\_ID}  \\
     NH\subscript{3}  & \texttt{NH3\_ID}  & \hspace{0.5cm} & C\subscript{2}H\subscript{6} & \texttt{C2H6\_ID}  \\
     HNO\subscript{3} & \texttt{HNO3\_ID} & \hspace{0.5cm} & PH\subscript{3}              & \texttt{PH3\_ID}   \\
     OH               & \texttt{OH\_ID}   & \hspace{0.5cm} & COF\subscript{2}             & \texttt{COF2\_ID}  \\
     HF               & \texttt{HF\_ID}   & \hspace{0.5cm} & SF\subscript{6}              & \texttt{SF6\_ID}   \\
     HCl              & \texttt{HCl\_ID}  & \hspace{0.5cm} & H\subscript{2}S              & \texttt{H2S\_ID}   \\
     HBr              & \texttt{HBr\_ID}  & \hspace{0.5cm} & HCOOH                        & \texttt{HCOOH\_ID} \\
    \hline
  \end{tabular}
  \caption{CRTM \Atmosphere{} structure valid \texttt{Absorber\_ID} definitions. The same molecule set as defined for HITRAN is used.}
  \label{tab:absorber_id}
\end{table}

% Absorber units table
\begin{table}[htp]
  \centering
  \begin{tabular}{c l}
    \hline
    \sffamily\textbf{Units} & \sffamily\textbf{Parameter} \\
    \hline\hline
     Volume mixing ratio, ppmv                       & \texttt{VOLUME\_MIXING\_RATIO\_UNITS} \\
     Number density, cm$^{-3}$                       & \texttt{NUMBER\_DENSITY\_UNITS} \\
     Mass mixing ratio, g/kg                         & \texttt{MASS\_MIXING\_RATIO\_UNITS} \\
     Mass density, g.m$^{-3}$                        & \texttt{MASS\_DENSITY\_UNITS} \\
     Partial pressure, hPa                           & \texttt{PARTIAL\_PRESSURE\_UNITS} \\
     Dewpoint temperature, K  \textbf{(H$\mathbf{_2}$O ONLY)} & \texttt{DEWPOINT\_TEMPERATURE\_K\_UNITS} \\
     Dewpoint temperature, C  \textbf{(H$\mathbf{_2}$O ONLY)} & \texttt{DEWPOINT\_TEMPERATURE\_C\_UNITS} \\
     Relative humidity, \%    \textbf{(H$\mathbf{_2}$O ONLY)} & \texttt{RELATIVE\_HUMIDITY\_UNITS} \\
     Specific amount, g/g                            & \texttt{SPECIFIC\_AMOUNT\_UNITS} \\
     Integrated path, mm                             & \texttt{INTEGRATED\_PATH\_UNITS} \\
    \hline
  \end{tabular}
  \caption{CRTM \Atmosphere{} structure valid \texttt{Absorber\_Units} definitions. The same set as defined for LBLRTM is used.}
  \label{tab:absorber_units}
\end{table}

% Atmosphere structure methods
%-----------------------------
\subsection{\texttt{CRTM\_Associated\_Atmosphere} interface}
  \label{sec:CRTM_Associated_Atmosphere_interface}
  \begin{alltt}
 
  NAME:
        CRTM_Associated_Atmosphere
 
  PURPOSE:
        Function to test the association status of the components of a
        CRTM_Atmosphere structure.
 
  CALLING SEQUENCE:
        Association_Status = CRTM_Associated_Atmosphere( Atmosphere               , &
                                                         ANY_Test    =Any_Test    , &
                                                         Skip_Cloud  =Skip_Cloud  , &
                                                         Skip_Aerosol=Skip_Aerosol  )
 
  INPUT ARGUMENTS:
        Atmosphere:          Structure which is to have its pointer
                             member's association status tested.
                             UNITS:      N/A
                             TYPE:       CRTM_Atmosphere_type
                             DIMENSION:  Scalar, Rank-1, OR Rank-2 array
                             ATTRIBUTES: INTENT(IN)
 
  OPTIONAL INPUT ARGUMENTS:
        ANY_Test:            Set this argument to test if ANY of the
                             Atmosphere structure components are associated.
                             The default is to test if ALL the components
                             are associated.
                             If ANY_Test = 0, test if ALL the components
                                              are associated.  (DEFAULT)
                                ANY_Test = 1, test if ANY of the components
                                              are associated.
                             UNITS:      N/A
                             TYPE:       INTEGER
                             DIMENSION:  Scalar
                             ATTRIBUTES: INTENT(IN), OPTIONAL
 
        Skip_Cloud:          Set this argument to not include the Cloud
                             member in the association test. This is required
                             because a valid Atmosphere structure can be
                             cloud-free.
                             If Skip_Cloud = 0, the Cloud member association
                                                status is tested.  (DEFAULT)
                                Skip_Cloud = 1, the Cloud member association
                                                status is NOT tested.
                             UNITS:      N/A
                             TYPE:       INTEGER
                             DIMENSION:  Scalar
                             ATTRIBUTES: INTENT(IN), OPTIONAL
 
        Skip_Aerosol:        Set this argument to not include the Aerosol
                             member in the association test. This is required
                             because a valid Atmosphere structure can be
                             aerosol-free.
                             If Skip_Aerosol = 0, the Aerosol member association
                                                  status is tested.  (DEFAULT)
                                Skip_Aerosol = 1, the Aerosol member association
                                                  status is NOT tested.
                             UNITS:      N/A
                             TYPE:       INTEGER
                             DIMENSION:  Scalar
                             ATTRIBUTES: INTENT(IN), OPTIONAL
 
  FUNCTION RESULT:
        Association_Status:  The return value is a logical value indicating the
                             association status of the Atmosphere components.
                             .TRUE.  - if ALL the Atmosphere components are
                                       associated, or if the ANY_Test argument
                                       is set and ANY of the Atmosphere pointer
                                       members are associated.
                             .FALSE. - some or all of the Atmosphere pointer
                                       members are NOT associated.
                             UNITS:      N/A
                             TYPE:       LOGICAL
                             DIMENSION:  Same as input Atmosphere argument
 
  \end{alltt}

\subsection{\texttt{CRTM\_Allocate\_Atmosphere} interface}
  \label{sec:CRTM_Allocate_Atmosphere_interface}
  \begin{alltt}
 
  NAME:
        CRTM_Allocate_Atmosphere
  
  PURPOSE:
        Function to allocate CRTM_Atmosphere data structures.
 
  CALLING SEQUENCE:
        Error_Status = CRTM_Allocate_Atmosphere( n_Layers               , &
                                                 n_Absorbers            , &
                                                 n_Clouds               , &
                                                 n_Aerosols             , &
                                                 Atmosphere             , &
                                                 Message_Log=Message_Log  )
 
  INPUT ARGUMENTS:
        n_Layers:     Number of layers dimension.
                      Must be > 0.
                      UNITS:      N/A
                      TYPE:       INTEGER
                      DIMENSION:  Scalar OR Rank-1
                                  See output Atmosphere dimensionality chart
                      ATTRIBUTES: INTENT(IN)
 
        n_Absorbers:  Number of absorbers dimension. This will be the same for
                      all elements if the Atmosphere argument is an array.
                      Must be > 0.
                      UNITS:      N/A
                      TYPE:       INTEGER
                      DIMENSION:  Scalar
                      ATTRIBUTES: INTENT(IN)
 
        n_Clouds:     Number of clouds dimension of Atmosphere data.
                      ** Note: Can be = 0 (i.e. clear sky). **
                      UNITS:      N/A
                      TYPE:       INTEGER
                      DIMENSION:  Scalar OR Rank-1
                                  See output Atmosphere dimensionality chart
                      ATTRIBUTES: INTENT(IN)
 
        n_Aerosols:   Number of aerosol types dimension of Atmosphere data.
                      ** Note: Can be = 0 (i.e. no aerosols). **
                      UNITS:      N/A
                      TYPE:       INTEGER
                      DIMENSION:  Scalar OR Rank-1
                                  See output Atmosphere dimensionality chart
                      ATTRIBUTES: INTENT(IN)
 
  OPTIONAL INPUT ARGUMENTS:
        Message_Log:  Character string specifying a filename in which any
                      messages will be logged. If not specified, or if an
                      error occurs opening the log file, the default action
                      is to output messages to standard output.
                      UNITS:      N/A
                      TYPE:       CHARACTER(*)
                      DIMENSION:  Scalar
                      ATTRIBUTES: INTENT(IN), OPTIONAL
 
  OUTPUT ARGUMENTS:
        Atmosphere:   Atmosphere structure with allocated components. The
                      following table shows the allowable dimension combinations
                      for the calling routine, where M == number of profiles:
 
                         Input       Input       Input      Input        Output
                        n_Layers   n_Absorbers  n_Clouds  n_Aerosols    Atmosphere
                        dimension   dimension   dimension  dimension    dimension
                      --------------------------------------------------------------
                         scalar      scalar      scalar     scalar       scalar
                         scalar      scalar      scalar     scalar         M
                         scalar      scalar      scalar       M            M
                         scalar      scalar        M        scalar         M
                         scalar      scalar        M          M            M
                           M         scalar      scalar     scalar         M
                           M         scalar      scalar       M            M
                           M         scalar        M        scalar         M
                           M         scalar        M          M            M
 
                      Note the number of absorbers cannot vary with the profile.
 
                      These multiple interfaces are supplied purely for ease of
                      use depending on what data is available.
                      
                      UNITS:      N/A
                      TYPE:       CRTM_Atmosphere_type
                      DIMENSION:  Scalar or Rank-1
                                  See chart above.
                      ATTRIBUTES: INTENT(IN OUT)
 
 
  FUNCTION RESULT:
        Error_Status: The return value is an integer defining the error status.
                      The error codes are defined in the Message_Handler module.
                      If == SUCCESS the structure re-initialisation was successful
                         == FAILURE - an error occurred, or
                                    - the structure internal allocation counter
                                      is not equal to one (1) upon exiting this
                                      function. This value is incremented and
                                      decremented for every structure allocation
                                      and deallocation respectively.
                      UNITS:      N/A
                      TYPE:       INTEGER
                      DIMENSION:  Scalar
 
  COMMENTS:
        Note the INTENT on the output Atmosphere argument is IN OUT rather than
        just OUT. This is necessary because the argument may be defined upon
        input. To prevent memory leaks, the IN OUT INTENT is a must.
 
  \end{alltt}

\subsection{\texttt{CRTM\_Destroy\_Atmosphere} interface}
  \label{sec:CRTM_Destroy_Atmosphere_interface}
  \begin{alltt}
 
  NAME:
        CRTM_Destroy_Atmosphere
  
  PURPOSE:
        Function to re-initialize CRTM Atmosphere data structures.
 
  CALLING SEQUENCE:
        Error_Status = CRTM_Destroy_Atmosphere( Atmosphere             , &
                                                Message_Log=Message_Log  )
 
  OUTPUT ARGUMENTS:
        Atmosphere:   Re-initialized Atmosphere structure. In the context of
                      the CRTM, rank-1 corresponds to an vector of profiles,
                      and rank-2 corresponds to an array of channels x profiles.
                      The latter is used in the K-matrix model.
                      UNITS:      N/A
                      TYPE:       CRTM_Atmosphere_type
                      DIMENSION:  Scalar, Rank-1, OR Rank-2 array
                      ATTRIBUTES: INTENT(IN OUT)
 
  OPTIONAL INPUT ARGUMENTS:
        Message_Log:  Character string specifying a filename in which any
                      messages will be logged. If not specified, or if an
                      error occurs opening the log file, the default action
                      is to output messages to standard output.
                      UNITS:      N/A
                      TYPE:       CHARACTER(*)
                      DIMENSION:  Scalar
                      ATTRIBUTES: INTENT(IN), OPTIONAL
 
  FUNCTION RESULT:
        Error_Status: The return value is an integer defining the error status.
                      The error codes are defined in the Message_Handler module.
                      If == SUCCESS the structure re-initialisation was successful
                         == FAILURE - an error occurred, or
                                    - the structure internal allocation counter
                                      is not equal to zero (0) upon exiting this
                                      function. This value is incremented and
                                      decremented for every structure allocation
                                      and deallocation respectively.
                      UNITS:      N/A
                      TYPE:       INTEGER
                      DIMENSION:  Scalar
 
  COMMENTS:
        Note the INTENT on the output Atmosphere argument is IN OUT rather than
        just OUT. This is necessary because the argument may be defined upon
        input. To prevent memory leaks, the IN OUT INTENT is a must.
 
  \end{alltt}

\subsection{\texttt{CRTM\_Assign\_Atmosphere} interface}
  \label{sec:CRTM_Assign_Atmosphere_interface}
  \begin{alltt}
 
  NAME:
        CRTM_Assign_Atmosphere
 
  PURPOSE:
        Function to copy valid CRTM_Atmosphere structures.
 
  CALLING SEQUENCE:
        Error_Status = CRTM_Assign_Atmosphere( Atmosphere_in          , &
                                               Atmosphere_out         , &
                                               Message_Log=Message_Log  )
 
  INPUT ARGUMENTS:
        Atmosphere_in:   Atmosphere structure which is to be copied. In the
                         context of the CRTM, rank-1 corresponds to an vector
                         of profiles, and rank-2 corresponds to an array of
                         channels x profiles. The latter is used in the K-matrix
                         model.
                         UNITS:      N/A
                         TYPE:       CRTM_Atmosphere_type
                         DIMENSION:  Scalar, Rank-1, or Rank-2 array
                         ATTRIBUTES: INTENT(IN)
 
  OUTPUT ARGUMENTS:
        Atmosphere_out:  Copy of the input structure, Atmosphere_in.
                         UNITS:      N/A
                         TYPE:       CRTM_Atmosphere_type
                         DIMENSION:  Same as Atmosphere_in
                         ATTRIBUTES: INTENT(IN OUT)
 
  OPTIONAL INPUT ARGUMENTS:
        Message_Log:     Character string specifying a filename in which any
                         messages will be logged. If not specified, or if an
                         error occurs opening the log file, the default action
                         is to output messages to standard output.
                         UNITS:      N/A
                         TYPE:       CHARACTER(*)
                         DIMENSION:  Scalar
                         ATTRIBUTES: INTENT(IN), OPTIONAL
 
  FUNCTION RESULT:
        Error_Status:    The return value is an integer defining the error status.
                         The error codes are defined in the Message_Handler module.
                         If == SUCCESS the structure assignment was successful
                            == FAILURE an error occurred
                         UNITS:      N/A
                         TYPE:       INTEGER
                         DIMENSION:  Scalar
 
  COMMENTS:
        Note the INTENT on the output Atmosphere argument is IN OUT rather than
        just OUT. This is necessary because the argument may be defined upon
        input. To prevent memory leaks, the IN OUT INTENT is a must.
 
  \end{alltt}

\subsection{\texttt{CRTM\_Equal\_Atmosphere} interface}
  \label{sec:CRTM_Equal_Atmosphere_interface}
  \begin{alltt}
 
  NAME:
        CRTM_Equal_Atmosphere
 
  PURPOSE:
        Function to test if two Atmosphere structures are equal.
 
  CALLING SEQUENCE:
        Error_Status = CRTM_Equal_Atmosphere( Atmosphere_LHS                       , &
                                              Atmosphere_RHS                       , &
                                              ULP_Scale         =ULP_Scale         , &
                                              Percent_Difference=Percent_Difference, &
                                              Check_All         =Check_All         , &
                                              Message_Log       =Message_Log         )
 
 
  INPUT ARGUMENTS:
        Atmosphere_LHS:     Atmosphere structure to be compared; equivalent to the
                            left-hand side of a lexical comparison, e.g.
                              IF ( Atmosphere_LHS == Atmosphere_RHS ).
                            In the context of the CRTM, rank-1 corresponds to an
                            vector of profiles, and rank-2 corresponds to an array
                            of channels x profiles. The latter is used in the
                            K-matrix model.
                            UNITS:      N/A
                            TYPE:       CRTM_Atmosphere_type
                            DIMENSION:  Scalar, Rank-1, or Rank-2 array
                            ATTRIBUTES: INTENT(IN)
 
        Atmosphere_RHS:     Atmosphere structure to be compared to; equivalent to
                            right-hand side of a lexical comparison, e.g.
                              IF ( Atmosphere_LHS == Atmosphere_RHS ).
                            UNITS:      N/A
                            TYPE:       CRTM_Atmosphere_type
                            DIMENSION:  Same as Atmosphere_LHS
                            ATTRIBUTES: INTENT(IN)
 
  OPTIONAL INPUT ARGUMENTS:
        ULP_Scale:          Unit of data precision used to scale the floating
                            point comparison. ULP stands for "Unit in the Last Place,"
                            the smallest possible increment or decrement that can be
                            made using a machine's floating point arithmetic.
                            Value must be positive - if a negative value is supplied,
                            the absolute value is used. If not specified, the default
                            value is 1.
                            UNITS:      N/A
                            TYPE:       INTEGER
                            DIMENSION:  Scalar
                            ATTRIBUTES: INTENT(IN), OPTIONAL
 
        Percent_Differnece: Percentage difference value to use in comparing
                            the numbers rather than testing within some numerical
                            limit. The ULP_Scale argument is ignored if this argument is
                            specified.
                            UNITS:      N/A
                            TYPE:       REAL(fp)
                            DIMENSION:  Scalar
                            ATTRIBUTES: OPTIONAL, INTENT(IN)
 
        Check_All:          Set this argument to check ALL the floating point
                            channel data of the Atmosphere structures. The default
                            action is return with a FAILURE status as soon as
                            any difference is found. This optional argument can
                            be used to get a listing of ALL the differences
                            between data in Atmosphere structures.
                            If == 0, Return with FAILURE status as soon as
                                     ANY difference is found  *DEFAULT*
                               == 1, Set FAILURE status if ANY difference is
                                     found, but continue to check ALL data.
                            UNITS:      N/A
                            TYPE:       INTEGER
                            DIMENSION:  Scalar
                            ATTRIBUTES: INTENT(IN), OPTIONAL
 
        Message_Log:        Character string specifying a filename in which any
                            messages will be logged. If not specified, or if an
                            error occurs opening the log file, the default action
                            is to output messages to standard output.
                            UNITS:      None
                            TYPE:       CHARACTER(*)
                            DIMENSION:  Scalar
                            ATTRIBUTES: INTENT(IN), OPTIONAL
 
  FUNCTION RESULT:
        Error_Status:       The return value is an integer defining the error status.
                            The error codes are defined in the Message_Handler module.
                            If == SUCCESS the structures were equal
                               == FAILURE - an error occurred, or
                                          - the structures were different.
                            UNITS:      N/A
                            TYPE:       INTEGER
                            DIMENSION:  Scalar
 
  \end{alltt}

\subsection{\texttt{CRTM\_SetLayers\_Atmosphere} interface}
  \label{sec:CRTM_SetLayers_Atmosphere_interface}
  \begin{alltt}
 
  NAME:
        CRTM_SetLayers_Atmosphere
  
  PURPOSE:
        Function to set the number of layers to use in a CRTM Atmosphere
        structure.
 
  CALLING SEQUENCE:
        Error_Status = CRTM_SetLayers_Atmosphere( n_Layers  , &
                                                  Atmosphere  )
 
  INPUTS:
        n_Layers:     The value to set the n_Layers component of the 
                      Atmosphere structure, as well as any of its
                      Cloud or Aerosol structure components.
                      UNITS:      N/A
                      TYPE:       CRTM_Atmosphere_type
                      DIMENSION:  Scalar
                      ATTRIBUTES: INTENT(IN)
 
        Atmosphere:   Atmosphere structure in which the n_Layers dimension
                      is to be updated.
                      UNITS:      N/A
                      TYPE:       CRTM_Atmosphere_type
                      DIMENSION:  Scalar, Rank-1, or Rank-2 array
                      ATTRIBUTES: INTENT(IN OUT)
  OUTPUTS:
        Atmosphere:   On output, the atmosphere structure with the updated
                      n_Layers dimension.
                      UNITS:      N/A
                      TYPE:       CRTM_Atmosphere_type
                      DIMENSION:  Scalar, Rank-1, or Rank-2 array
                      ATTRIBUTES: INTENT(IN OUT)
 
  FUNCTION RESULT:
        Error_Status: The return value is an integer defining the error status.
                      The error codes are defined in the Message_Handler module.
                      If == SUCCESS the layer reset was successful
                         == FAILURE an error occurred
                      UNITS:      N/A
                      TYPE:       INTEGER
                      DIMENSION:  Scalar
 
  SIDE EFFECTS:
        The argument Atmosphere is INTENT(IN OUT) and is modified upon output.
 
  COMMENTS:
        - Note that the n_Layers input is *ALWAYS* scalar. Thus, all Atmosphere
          elements will be set to the same number of layers.
 
        - If n_Layers <= Atmosphere%Max_Layers, then only the dimension value
          of the structure and any sub-structures are changed.
 
        - If n_Layers > Atmosphere%Max_Layers, then the entire structure is
          reallocated to the required number of layers. No other dimensions
          of the structure or substructures are altered.
 
  \end{alltt}

\subsection{\texttt{CRTM\_Sum\_Atmosphere} interface}
  \label{sec:CRTM_Sum_Atmosphere_interface}
  \begin{alltt}
 
  NAME:
        CRTM_Sum_Atmosphere
 
  PURPOSE:
        Function to perform a sum of two valid CRTM Atmosphere structures. The
        summation performed is:
          A = A + Scale_Factor*B + Offset
        where A and B are the CRTM Atmosphere structures, and Scale_Factor and
        Offset are optional weighting factors.
 
  CALLING SEQUENCE:
        Error_Status = CRTM_Sum_Atmosphere( A                        , &
                                            B                        , &
                                            Scale_Factor=Scale_Factor, &
                                            Offset      =Offset      , &
                                            Message_Log =Message_Log   )
 
  INPUT ARGUMENTS:
        A:               Atmosphere structure that is to be added to.
                         In the context of the CRTM, rank-1 corresponds to an
                         vector of profiles, and rank-2 corresponds to an array
                         of channels x profiles. The latter is used in the
                         K-matrix model.
                         UNITS:      N/A
                         TYPE:       CRTM_Atmosphere_type
                         DIMENSION:  Scalar, Rank-1, or Rank-2 array
                         ATTRIBUTES: INTENT(IN OUT)
 
        B:               Atmosphere structure that is to be weighted and
                         added to structure A.
                         UNITS:      N/A
                         TYPE:       CRTM_Atmosphere_type
                         DIMENSION:  Same as A
                         ATTRIBUTES: INTENT(IN)
 
  OPTIONAL INPUT ARGUMENTS:
        Scale_Factor:    The first weighting factor used to scale the
                         contents of the input structure, B.
                         If not specified, Scale_Factor = 1.0.
                         UNITS:      N/A
                         TYPE:       REAL(fp)
                         DIMENSION:  Scalar
                         ATTRIBUTES: INTENT(IN), OPTIONAL
 
        Offset:          The second weighting factor used to offset the
                         sum of the input structures.
                         If not specified, Offset = 0.0.
                         UNITS:      N/A
                         TYPE:       REAL(fp)
                         DIMENSION:  Scalar
                         ATTRIBUTES: INTENT(IN), OPTIONAL
 
        Message_Log:     Character string specifying a filename in which any
                         messages will be logged. If not specified, or if an
                         error occurs opening the log file, the default action
                         is to output messages to standard output.
                         UNITS:      N/A
                         TYPE:       CHARACTER(*)
                         DIMENSION:  Scalar
                         ATTRIBUTES: INTENT(IN), OPTIONAL
 
  OUTPUT ARGUMENTS:
        A:               Structure containing the summation result,
                         A = A + Scale_Factor*B + Offset
                         UNITS:      N/A
                         TYPE:       CRTM_Atmosphere_type
                         DIMENSION:  Same as B
                         ATTRIBUTES: INTENT(IN OUT)
 
  FUNCTION RESULT:
        Error_Status:    The return value is an integer defining the error status.
                         The error codes are defined in the Message_Handler module.
                         If == SUCCESS the structure assignment was successful
                            == FAILURE an error occurred
                         UNITS:      N/A
                         TYPE:       INTEGER
                         DIMENSION:  Scalar
 
  SIDE EFFECTS:
        The argument A is INTENT(IN OUT) and is modified upon output.
 
  \end{alltt}

\subsection{\texttt{CRTM\_Zero\_Atmosphere} interface}
  \label{sec:CRTM_Zero_Atmosphere_interface}
  \begin{alltt}
 
  NAME:
        CRTM_Zero_Atmosphere
  
  PURPOSE:
        Subroutine to zero-out various members of a CRTM Atmosphere structure -
        both scalar and pointer.
 
  CALLING SEQUENCE:
        CALL CRTM_Zero_Atmosphere( Atmosphere )
 
  OUTPUT ARGUMENTS:
        Atmosphere:   Zeroed out Atmosphere structure.
                      In the context of the CRTM, rank-1 corresponds to an
                      vector of profiles, and rank-2 corresponds to an array
                      of channels x profiles. The latter is used in the
                      K-matrix model.
                      UNITS:      N/A
                      TYPE:       CRTM_Atmosphere_type
                      DIMENSION:  Scalar, Rank-1, or Rank-2 array
                      ATTRIBUTES: INTENT(IN OUT)
 
  COMMENTS:
        - No checking of the input structure is performed, so there are no
          tests for component association status. This means the Atmosphere
          structure must have allocated components upon entry to this
          routine.
 
        - The dimension components of the structure are *NOT* set to zero.
 
        - The Absorber_ID and Absorber_Units components are *NOT* zeroed out
          in this routine.
 
        - Note the INTENT on the output Atmosphere argument is IN OUT rather than
          just OUT. This is necessary because the argument must be defined upon
          input.
 
  \end{alltt}

\subsection{\texttt{CRTM\_RCS\_ID\_Atmosphere} interface}
  \label{sec:CRTM_RCS_ID_Atmosphere_interface}
  \begin{alltt}
 
  NAME:
        CRTM_RCS_ID_Atmosphere
 
  PURPOSE:
        Subroutine to return the module RCS Id information.
 
  CALLING SEQUENCE:
        CALL CRTM_RCS_Id_Atmosphere( RCS_Id )
 
  OUTPUT ARGUMENTS:
        RCS_Id:        Character string containing the Revision Control
                       System Id field for the module.
                       UNITS:      N/A
                       TYPE:       CHARACTER(*)
                       DIMENSION:  Scalar
                       ATTRIBUTES: INTENT(OUT)
 
  \end{alltt}

\subsection{\texttt{CRTM\_Inquire\_Atmosphere\_Binary} interface}
  \label{sec:CRTM_Inquire_Atmosphere_Binary_interface}
  \begin{alltt}
 
  NAME:
        CRTM_Inquire_Atmosphere_Binary
 
  PURPOSE:
        Function to inquire Binary format CRTM Atmosphere structure files.
 
  CALLING SEQUENCE:
        Error_Status = CRTM_Inquire_Atmosphere_Binary( Filename               , &
                                                       n_Channels =n_Channels , &
                                                       n_Profiles =n_Profiles , &
                                                       RCS_Id     =RCS_Id     , &
                                                       Message_Log=Message_Log  )
 
  INPUT ARGUMENTS:
        Filename:     Character string specifying the name of an
                      Atmosphere format data file to read.
                      UNITS:      N/A
                      TYPE:       CHARACTER(*)
                      DIMENSION:  Scalar
                      ATTRIBUTES: INTENT(IN)
 
  OPTIONAL INPUT ARGUMENTS:
        Message_Log:  Character string specifying a filename in which any
                      messages will be logged. If not specified, or if an
                      error occurs opening the log file, the default action
                      is to output messages to standard output.
                      UNITS:      N/A
                      TYPE:       CHARACTER(*)
                      DIMENSION:  Scalar
                      ATTRIBUTES: INTENT(IN), OPTIONAL
 
  OPTIONAL OUTPUT ARGUMENTS:
        n_Channels:   The number of spectral channels for which there is
                      data in the file. Note that this value will always
                      be 0 for a profile-only dataset-- it only has meaning
                      for K-matrix data.
                      UNITS:      N/A
                      TYPE:       INTEGER
                      DIMENSION:  Scalar
                      ATTRIBUTES: OPTIONAL, INTENT(OUT)
 
        n_Profiles:   The number of profiles in the data file.
                      UNITS:      N/A
                      TYPE:       INTEGER
                      DIMENSION:  Scalar
                      ATTRIBUTES: OPTIONAL, INTENT(OUT)
 
        RCS_Id:       Character string containing the version control Id
                      field for the module.
                      UNITS:      N/A
                      TYPE:       CHARACTER(*)
                      DIMENSION:  Scalar
                      ATTRIBUTES: OPTIONAL, INTENT(OUT)
 
  FUNCTION RESULT:
        Error_Status: The return value is an integer defining the error status.
                      The error codes are defined in the Message_Handler module.
                      If == SUCCESS the Binary file inquire was successful
                         == FAILURE an unrecoverable error occurred.
                      UNITS:      N/A
                      TYPE:       INTEGER
                      DIMENSION:  Scalar
 
  \end{alltt}

\subsection{\texttt{CRTM\_Read\_Atmosphere\_Binary} interface}
  \label{sec:CRTM_Read_Atmosphere_Binary_interface}
  \begin{alltt}
 
  NAME:
        CRTM_Read_Atmosphere_Binary
 
  PURPOSE:
        Function to read Binary format CRTM Atmosphere structure files.
 
  CALLING SEQUENCE:
        Error_Status = CRTM_Read_Atmosphere_Binary( Filename               , &
                                                    Atmosphere             , &
                                                    Quiet      =Quiet      , &
                                                    n_Channels =n_Channels , &
                                                    n_Profiles =n_Profiles , &
                                                    RCS_Id     =RCS_Id     , &
                                                    Message_Log=Message_Log  )
 
  INPUT ARGUMENTS:
        Filename:     Character string specifying the name of an
                      Atmosphere format data file to read.
                      UNITS:      N/A
                      TYPE:       CHARACTER(*)
                      DIMENSION:  Scalar
                      ATTRIBUTES: INTENT(IN)
 
  OUTPUT ARGUMENTS:
        Atmosphere:   Structure containing the Atmosphere data. Note the
                      following meanings attributed to the dimensions of
                      the structure array:
                      Rank-1: M profiles.
                              Only profile data are to be read in. The file
                              does not contain channel information. The
                              dimension of the structure is understood to
                              be the PROFILE dimension.
                      Rank-2: L channels  x  M profiles
                              Channel and profile data are to be read in.
                              The file contains both channel and profile
                              information. The first dimension of the 
                              structure is the CHANNEL dimension, the second
                              is the PROFILE dimension. This is to allow
                              K-matrix structures to be read in with the
                              same function.
                      UNITS:      N/A
                      TYPE:       CRTM_Atmosphere_type
                      DIMENSION:  Rank-1 (M) or Rank-2 (L x M)
                      ATTRIBUTES: INTENT(IN OUT)
 
  OPTIONAL INPUT ARGUMENTS:
        Quiet:        Set this argument to suppress INFORMATION messages
                      being printed to standard output (or the message
                      log file if the Message_Log optional argument is
                      used.) By default, INFORMATION messages are printed.
                      If QUIET = 0, INFORMATION messages are OUTPUT.
                         QUIET = 1, INFORMATION messages are SUPPRESSED.
                      UNITS:      N/A
                      TYPE:       INTEGER
                      DIMENSION:  Scalar
                      ATTRIBUTES: INTENT(IN), OPTIONAL
 
        Message_Log:  Character string specifying a filename in which any
                      messages will be logged. If not specified, or if an
                      error occurs opening the log file, the default action
                      is to output messages to standard output.
                      UNITS:      N/A
                      TYPE:       CHARACTER(*)
                      DIMENSION:  Scalar
                      ATTRIBUTES: INTENT(IN), OPTIONAL
 
  OPTIONAL OUTPUT ARGUMENTS:
        n_Channels:   The number of channels for which data was read. Note that
                      this value will always be 0 for a profile-only dataset--
                      it only has meaning for K-matrix data.
                      UNITS:      N/A
                      TYPE:       INTEGER
                      DIMENSION:  Scalar
                      ATTRIBUTES: OPTIONAL, INTENT(OUT)
 
        n_Profiles:   The number of profiles for which data was read.
                      UNITS:      N/A
                      TYPE:       INTEGER
                      DIMENSION:  Scalar
                      ATTRIBUTES: OPTIONAL, INTENT(OUT)
 
        RCS_Id:       Character string containing the version control Id
                      field for the module.
                      UNITS:      N/A
                      TYPE:       CHARACTER(*)
                      DIMENSION:  Scalar
                      ATTRIBUTES: OPTIONAL, INTENT(OUT)
 
  FUNCTION RESULT:
        Error_Status: The return value is an integer defining the error status.
                      The error codes are defined in the Message_Handler module.
                      If == SUCCESS the Binary file read was successful
                         == FAILURE an unrecoverable error occurred.
                      UNITS:      N/A
                      TYPE:       INTEGER
                      DIMENSION:  Scalar
 
  COMMENTS:
        Note the INTENT on the output Atmosphere argument is IN OUT rather
        than just OUT. This is necessary because the argument may be defined on
        input. To prevent memory leaks, the IN OUT INTENT is a must.
 
  \end{alltt}

\subsection{\texttt{CRTM\_Write\_Atmosphere\_Binary} interface}
  \label{sec:CRTM_Write_Atmosphere_Binary_interface}
  \begin{alltt}
 
  NAME:
        CRTM_Write_Atmosphere_Binary
 
  PURPOSE:
        Function to write Binary format Atmosphere files.
 
  CALLING SEQUENCE:
        Error_Status = CRTM_Write_Atmosphere_Binary( Filename               , &
                                                     Atmosphere             , &
                                                     Quiet      =Quiet      , &
                                                     RCS_Id     =RCS_Id     , &
                                                     Message_Log=Message_Log  )
 
  INPUT ARGUMENTS:
        Filename:     Character string specifying the name of an output
                      Atmosphere format data file.
                      UNITS:      N/A
                      TYPE:       CHARACTER(*)
                      DIMENSION:  Scalar
                      ATTRIBUTES: INTENT(IN)
 
        Atmosphere:   Structure containing the Atmosphere data to write.
                      Note the following meanings attributed to the
                      dimensions of the structure array:
                      Rank-1: M profiles.
                              Only profile data are to be read in. The file
                              does not contain channel information. The
                              dimension of the structure is understood to
                              be the PROFILE dimension.
                      Rank-2: L channels  x  M profiles
                              Channel and profile data are to be read in.
                              The file contains both channel and profile
                              information. The first dimension of the 
                              structure is the CHANNEL dimension, the second
                              is the PROFILE dimension. This is to allow
                              K-matrix structures to be read in with the
                              same function.
                      UNITS:      N/A
                      TYPE:       CRTM_Atmosphere_type
                      DIMENSION:  Rank-1 (M) or Rank-2 (L x M)
                      ATTRIBUTES: INTENT(IN)
 
  OPTIONAL INPUT ARGUMENTS:
        Quiet:        Set this argument to suppress INFORMATION messages
                      being printed to standard output (or the message
                      log file if the Message_Log optional argument is
                      used.) By default, INFORMATION messages are printed.
                      If QUIET = 0, INFORMATION messages are OUTPUT.
                         QUIET = 1, INFORMATION messages are SUPPRESSED.
                      UNITS:      N/A
                      TYPE:       INTEGER
                      DIMENSION:  Scalar
                      ATTRIBUTES: INTENT(IN), OPTIONAL
 
        Message_Log:  Character string specifying a filename in which any
                      messages will be logged. If not specified, or if an
                      error occurs opening the log file, the default action
                      is to output messages to standard output.
                      UNITS:      N/A
                      TYPE:       CHARACTER(*)
                      DIMENSION:  Scalar
                      ATTRIBUTES: INTENT(IN), OPTIONAL
 
  OPTIONAL OUTPUT ARGUMENTS:
        RCS_Id:       Character string containing the version control Id
                      field for the module.
                      UNITS:      N/A
                      TYPE:       CHARACTER(*)
                      DIMENSION:  Scalar
                      ATTRIBUTES: OPTIONAL, INTENT(OUT)
 
  FUNCTION RESULT:
        Error_Status: The return value is an integer defining the error status.
                      The error codes are defined in the Message_Handler module.
                      If == SUCCESS the Binary file write was successful
                         == FAILURE an unrecoverable error occurred.
                      UNITS:      N/A
                      TYPE:       INTEGER
                      DIMENSION:  Scalar
 
  SIDE EFFECTS:
        - If the output file already exists, it is overwritten.
        - If an error occurs *during* the write phase, the output file is deleted
          before returning to the calling routine.
 
  \end{alltt}



\newpage
\section{\Cloud{} Structure}
%================================
\label{sec:cloud_structure}

\begin{figure}[htp]
  \centering
  \doublebox{
  \begin{minipage}[b]{6.5in}
    \begin{alltt}
  TYPE :: CRTM_Cloud_type
    INTEGER :: n_Allocates = 0
    ! Dimension values
    INTEGER :: Max_Layers = 0  ! K dimension.
    INTEGER :: n_Layers   = 0  ! Kuse dimension.
    ! Number of added layers
    INTEGER :: n_Added_Layers = 0
    ! Cloud type
    INTEGER :: Type = NO_CLOUD
    ! Cloud state variables
    REAL(fp), POINTER :: Effective_Radius(:)   => NULL() ! K. Units are microns
    REAL(fp), POINTER :: Effective_Variance(:) => NULL() ! K. Units are microns^2
    REAL(fp), POINTER :: Water_Content(:) => NULL()      ! K. Units are kg/m^2
  END TYPE CRTM_Cloud_type
    \end{alltt}
  \end{minipage}
  }
  \caption{CRTM\_Cloud\_type structure definition.}
  \label{fig:CRTM_Cloud_type_structure}
\end{figure}


% Cloud type table
\begin{table}[htp]
  \centering
  \begin{tabular}{cc l}
    \hline
    \sffamily\textbf{Cloud Type} & \hspace{0.5cm} & \sffamily\textbf{Parameter} \\
    \hline\hline
     Water   & \hspace{0.5cm} & \texttt{WATER\_CLOUD}\\
     Ice     & \hspace{0.5cm} & \texttt{ICE\_CLOUD}\\
     Rain    & \hspace{0.5cm} & \texttt{RAIN\_CLOUD}\\
     Snow    & \hspace{0.5cm} & \texttt{SNOW\_CLOUD}\\
     Graupel & \hspace{0.5cm} & \texttt{GRAUPEL\_CLOUD}\\
     Hail    & \hspace{0.5cm} & \texttt{HAIL\_CLOUD}\\
    \hline
  \end{tabular}
  \caption{CRTM \Cloud{} structure valid \texttt{Type} definitions.}
  \label{tab:cloud_type}
\end{table}

% Cloud structure methods
%------------------------
\subsection{\texttt{CRTM\_Associated\_Cloud} interface}
  \label{sec:CRTM_Associated_Cloud_interface}
  \begin{alltt}
 
  NAME:
        CRTM_Associated_Cloud
 
  PURPOSE:
        Function to test the association status of a CRTM Cloud structure.
 
  CALLING SEQUENCE:
        Association_Status = CRTM_Associated_Cloud( Cloud            , &
                                                    ANY_Test=Any_Test  )
 
  INPUT ARGUMENTS:
        Cloud:               Cloud structure which is to have its pointer
                             member's association status tested.
                             UNITS:      N/A
                             TYPE:       CRTM_Cloud_type
                             DIMENSION:  Scalar OR Rank-1 array
                             ATTRIBUTES: INTENT(IN)
 
  OPTIONAL INPUT ARGUMENTS:
        ANY_Test:            Set this argument to test if ANY of the
                             Cloud structure pointer members are associated.
                             The default is to test if ALL the pointer members
                             are associated.
                             If ANY_Test = 0, test if ALL the pointer members
                                              are associated.  (DEFAULT)
                                ANY_Test = 1, test if ANY of the pointer members
                                              are associated.
                             UNITS:      N/A
                             TYPE:       INTEGER
                             DIMENSION:  Scalar
                             ATTRIBUTES: INTENT(IN), OPTIONAL
 
  FUNCTION RESULT:
        Association_Status:  The return value is a logical value indicating the
                             association status of the Cloud pointer members.
                             .TRUE.  - if ALL the Cloud pointer members are
                                       associated, or if the ANY_Test argument
                                       is set and ANY of the Cloud pointer
                                       members are associated.
                             .FALSE. - some or all of the Cloud pointer
                                       members are NOT associated.
                             UNITS:      N/A
                             TYPE:       LOGICAL
                             DIMENSION:  Same as input Cloud argument
 
  \end{alltt}

\subsection{\texttt{CRTM\_Allocate\_Cloud} interface}
  \label{sec:CRTM_Allocate_Cloud_interface}
  \begin{alltt}
 
  NAME:
        CRTM_Allocate_Cloud
  
  PURPOSE:
        Function to allocate CRTM_Cloud structures.
 
  CALLING SEQUENCE:
        Error_Status = CRTM_Allocate_Cloud( n_Layers               , &
                                            Cloud                  , &
                                            Message_Log=Message_Log  )
 
  INPUT ARGUMENTS:
        n_Layers:     Number of layers for which there is cloud data.
                      Must be > 0.
                      UNITS:      N/A
                      TYPE:       INTEGER
                      DIMENSION:  Scalar OR Rank-1 array
                      ATTRIBUTES: INTENT(IN)
 
  OUTPUT ARGUMENTS:
        Cloud:        Cloud structure with allocated pointer members.
                      UNITS:      N/A
                      TYPE:       CRTM_Cloud_type
                      DIMENSION:  Same as input n_Layers argument
                      ATTRIBUTES: INTENT(IN OUT)
 
  OPTIONAL INPUT ARGUMENTS:
        Message_Log:  Character string specifying a filename in which any
                      messages will be logged. If not specified, or if an
                      error occurs opening the log file, the default action
                      is to output messages to standard output.
                      UNITS:      N/A
                      TYPE:       CHARACTER(*)
                      DIMENSION:  Scalar
                      ATTRIBUTES: INTENT(IN), OPTIONAL
 
  FUNCTION RESULT:
        Error_Status: The return value is an integer defining the error status.
                      The error codes are defined in the Message_Handler module.
                      If == SUCCESS the structure re-initialisation was successful
                         == FAILURE - an error occurred, or
                                    - the structure internal allocation counter
                                      is not equal to one (1) upon exiting this
                                      function. This value is incremented and
                                      decremented for every structure allocation
                                      and deallocation respectively.
                      UNITS:      N/A
                      TYPE:       INTEGER
                      DIMENSION:  Scalar
 
  COMMENTS:
        Note the INTENT on the output Cloud argument is IN OUT rather than
        just OUT. This is necessary because the argument may be defined upon
        input. To prevent memory leaks, the IN OUT INTENT is a must.
 
  \end{alltt}

\subsection{\texttt{CRTM\_Destroy\_Cloud} interface}
  \label{sec:CRTM_Destroy_Cloud_interface}
  \begin{alltt}
 
  NAME:
        CRTM_Destroy_Cloud
  
  PURPOSE:
        Function to re-initialize CRTM_Cloud structures.
 
  CALLING SEQUENCE:
        Error_Status = CRTM_Destroy_Cloud( Cloud                  , &
                                           Message_Log=Message_Log  )
 
  OUTPUT ARGUMENTS:
        Cloud:        Re-initialized Cloud structure.
                      UNITS:      N/A
                      TYPE:       CRTM_Cloud_type
                      DIMENSION:  Scalar OR Rank-1 array
                      ATTRIBUTES: INTENT(IN OUT)
 
  OPTIONAL INPUT ARGUMENTS:
        Message_Log:  Character string specifying a filename in which any
                      messages will be logged. If not specified, or if an
                      error occurs opening the log file, the default action
                      is to output messages to standard output.
                      UNITS:      N/A
                      TYPE:       CHARACTER(*)
                      DIMENSION:  Scalar
                      ATTRIBUTES: INTENT(IN), OPTIONAL
 
  FUNCTION RESULT:
        Error_Status: The return value is an integer defining the error status.
                      The error codes are defined in the Message_Handler module.
                      If == SUCCESS the structure re-initialisation was successful
                         == FAILURE - an error occurred, or
                                    - the structure internal allocation counter
                                      is not equal to zero (0) upon exiting this
                                      function. This value is incremented and
                                      decremented for every structure allocation
                                      and deallocation respectively.
                      UNITS:      N/A
                      TYPE:       INTEGER
                      DIMENSION:  Scalar
 
  COMMENTS:
        Note the INTENT on the output Cloud argument is IN OUT rather than
        just OUT. This is necessary because the argument may be defined upon
        input. To prevent memory leaks, the IN OUT INTENT is a must.
 
  \end{alltt}

\subsection{\texttt{CRTM\_Assign\_Cloud} interface}
  \label{sec:CRTM_Assign_Cloud_interface}
  \begin{alltt}
 
  NAME:
        CRTM_Assign_Cloud
 
  PURPOSE:
        Function to copy valid CRTM_Cloud structures.
 
  CALLING SEQUENCE:
        Error_Status = CRTM_Assign_Cloud( Cloud_in               , &
                                          Cloud_out              , &
                                          Message_Log=Message_Log  )
 
  INPUT ARGUMENTS:
        Cloud_in:      Cloud structure which is to be copied.
                       UNITS:      N/A
                       TYPE:       CRTM_Cloud_type
                       DIMENSION:  Scalar or Rank-1 array
                       ATTRIBUTES: INTENT(IN)
 
  OUTPUT ARGUMENTS:
        Cloud_out:     Copy of the input structure, Cloud_in.
                       UNITS:      N/A
                       TYPE:       CRTM_Cloud_type
                       DIMENSION:  Same as Cloud_in argument
                       ATTRIBUTES: INTENT(IN OUT)
 
  OPTIONAL INPUT ARGUMENTS:
        Message_Log:   Character string specifying a filename in which any
                       messages will be logged. If not specified, or if an
                       error occurs opening the log file, the default action
                       is to output messages to standard output.
                       UNITS:      N/A
                       TYPE:       CHARACTER(*)
                       DIMENSION:  Scalar
                       ATTRIBUTES: INTENT(IN), OPTIONAL
 
  FUNCTION RESULT:
        Error_Status:  The return value is an integer defining the error status.
                       The error codes are defined in the Message_Handler module.
                       If == SUCCESS the structure assignment was successful
                          == FAILURE an error occurred
                       UNITS:      N/A
                       TYPE:       INTEGER
                       DIMENSION:  Scalar
 
  COMMENTS:
        Note the INTENT on the output Cloud argument is IN OUT rather than
        just OUT. This is necessary because the argument may be defined upon
        input. To prevent memory leaks, the IN OUT INTENT is a must.
 
  \end{alltt}

\subsection{\texttt{CRTM\_Equal\_Cloud} interface}
  \label{sec:CRTM_Equal_Cloud_interface}
  \begin{alltt}
 
  NAME:
        CRTM_Equal_Cloud
 
  PURPOSE:
        Function to test if two Cloud structures are equal.
 
  CALLING SEQUENCE:
        Error_Status = CRTM_Equal_Cloud( Cloud_LHS              , &
                                         Cloud_RHS              , &
                                         ULP_Scale  =ULP_Scale  , &
                                         Check_All  =Check_All  , &
                                         Message_Log=Message_Log  )
 
 
  INPUT ARGUMENTS:
        Cloud_LHS:         Cloud structure to be compared; equivalent to the
                           left-hand side of a lexical comparison, e.g.
                             IF ( Cloud_LHS == Cloud_RHS ).
                           UNITS:      N/A
                           TYPE:       CRTM_Cloud_type
                           DIMENSION:  Scalar OR Rank-1 array
                           ATTRIBUTES: INTENT(IN)
 
        Cloud_RHS:         Cloud structure to be compared to; equivalent to
                           right-hand side of a lexical comparison, e.g.
                             IF ( Cloud_LHS == Cloud_RHS ).
                           UNITS:      N/A
                           TYPE:       CRTM_Cloud_type
                           DIMENSION:  Same as Cloud_RHS argument
                           ATTRIBUTES: INTENT(IN)
 
  OPTIONAL INPUT ARGUMENTS:
        ULP_Scale:         Unit of data precision used to scale the floating
                           point comparison. ULP stands for "Unit in the Last Place,"
                           the smallest possible increment or decrement that can be
                           made using a machine's floating point arithmetic.
                           Value must be positive - if a negative value is supplied,
                           the absolute value is used. If not specified, the default
                           value is 1.
                           UNITS:      N/A
                           TYPE:       INTEGER
                           DIMENSION:  Scalar
                           ATTRIBUTES: INTENT(IN), OPTIONAL
 
        Check_All:         Set this argument to check ALL the floating point
                           channel data of the Cloud structures. The default
                           action is return with a FAILURE status as soon as
                           any difference is found. This optional argument can
                           be used to get a listing of ALL the differences
                           between data in Cloud structures.
                           If == 0, Return with FAILURE status as soon as
                                    ANY difference is found  *DEFAULT*
                              == 1, Set FAILURE status if ANY difference is
                                    found, but continue to check ALL data.
                           UNITS:      N/A
                           TYPE:       INTEGER
                           DIMENSION:  Scalar
                           ATTRIBUTES: INTENT(IN), OPTIONAL
 
        Message_Log:       Character string specifying a filename in which any
                           messages will be logged. If not specified, or if an
                           error occurs opening the log file, the default action
                           is to output messages to standard output.
                           UNITS:      None
                           TYPE:       CHARACTER(*)
                           DIMENSION:  Scalar
                           ATTRIBUTES: INTENT(IN), OPTIONAL
 
  FUNCTION RESULT:
        Error_Status:      The return value is an integer defining the error status.
                           The error codes are defined in the Message_Handler module.
                           If == SUCCESS the structures were equal
                              == FAILURE - an error occurred, or
                                         - the structures were different.
                           UNITS:      N/A
                           TYPE:       INTEGER
                           DIMENSION:  Scalar
 
  \end{alltt}

\subsection{\texttt{CRTM\_SetLayers\_Cloud} interface}
  \label{sec:CRTM_SetLayers_Cloud_interface}
  \begin{alltt}
 
  NAME:
        CRTM_SetLayers_Cloud
  
  PURPOSE:
        Function to set the number of layers to use in a CRTM_Cloud
        structure.
 
  CALLING SEQUENCE:
        Error_Status = CRTM_SetLayers_Cloud( n_Layers               , &
                                             Cloud                  , &
                                             Message_Log=Message_Log  )
 
  INPUT ARGUMENTS:
        n_Layers:     The value to set the n_Layers component of the 
                      Cloud structure, as well as those of any of its
                      structure components.
                      UNITS:      N/A
                      TYPE:       CRTM_Cloud_type
                      DIMENSION:  Scalar
                      ATTRIBUTES: INTENT(IN)
 
        Cloud:        Cloud structure in which the n_Layers dimension
                      is to be updated.
                      UNITS:      N/A
                      TYPE:       CRTM_Cloud_type
                      DIMENSION:  Scalar OR Rank-1 array
                      ATTRIBUTES: INTENT(IN OUT)
  OUTPUT ARGUMENTS:
        Cloud:        On output, the Cloud structure with the updated
                      n_Layers dimension.
                      UNITS:      N/A
                      TYPE:       CRTM_Cloud_type
                      DIMENSION:  Scalar or Rank-1 array
                      ATTRIBUTES: INTENT(IN OUT)
 
  OPTIONAL INPUT ARGUMENTS:
        Message_Log:  Character string specifying a filename in which any
                      messages will be logged. If not specified, or if an
                      error occurs opening the log file, the default action
                      is to output messages to standard output.
                      UNITS:      None
                      TYPE:       CHARACTER(*)
                      DIMENSION:  Scalar
                      ATTRIBUTES: INTENT(IN), OPTIONAL
 
  FUNCTION RESULT:
        Error_Status: The return value is an integer defining the error status.
                      The error codes are defined in the Message_Handler module.
                      If == SUCCESS the layer reset was successful
                         == FAILURE an error occurred
                      UNITS:      N/A
                      TYPE:       INTEGER
                      DIMENSION:  Scalar
 
  SIDE EFFECTS:
        The argument Cloud is INTENT(IN OUT) and is modified upon output. The
        elements of the structure are reinitialised
 
  COMMENTS:
        - Note that the n_Layers input is *ALWAYS* scalar. Thus, all Cloud
          elements will be set to the same number of layers.
 
        - If n_Layers <= Cloud%Max_Layers, then only the dimension value
          of the structure and any sub-structures are changed.
 
        - If n_Layers > Cloud%Max_Layers, then the entire structure is
          reallocated to the required number of layers.
 
  \end{alltt}

\subsection{\texttt{CRTM\_Sum\_Cloud} interface}
  \label{sec:CRTM_Sum_Cloud_interface}
  \begin{alltt}
 
  NAME:
        CRTM_Sum_Cloud
 
  PURPOSE:
        Function to perform a sum of two valid CRTM Cloud structures. The
        summation performed is:
          A = A + Scale_Factor*B + Offset
        where A and B are the CRTM Cloud structures, and Scale_Factor and Offset
        are optional weighting factors.
 
  CALLING SEQUENCE:
        Error_Status = CRTM_Sum_Cloud( A                        , &
                                       B                        , &
                                       Scale_Factor=Scale_Factor, &
                                       Offset      =Offset      , &
                                       Message_Log =Message_Log   )
 
  INPUT ARGUMENTS:
        A:             Cloud structure that is to be added to.
                       UNITS:      N/A
                       TYPE:       CRTM_Cloud_type
                       DIMENSION:  Scalar OR Rank-1 array
                       ATTRIBUTES: INTENT(IN OUT)
 
        B:             Cloud structure that is to be weighted and
                       added to structure A.
                       UNITS:      N/A
                       TYPE:       CRTM_Cloud_type
                       DIMENSION:  Same as A
                       ATTRIBUTES: INTENT(IN)
 
  OUTPUT ARGUMENTS:
        A:             Structure containing the summation result,
                         A = A + Scale_Factor*B + Offset
                       UNITS:      N/A
                       TYPE:       CRTM_Cloud_type
                       DIMENSION:  Same as B
                       ATTRIBUTES: INTENT(IN OUT)
 
 
  OPTIONAL INPUT ARGUMENTS:
        Scale_Factor:  The first weighting factor used to scale the
                       contents of the input structure, B.
                       If not specified, Scale_Factor = 1.0.
                       UNITS:      N/A
                       TYPE:       REAL(fp)
                       DIMENSION:  Scalar
                       ATTRIBUTES: INTENT(IN), OPTIONAL
 
        Offset:        The second weighting factor used to offset the
                       sum of the input structures.
                       If not specified, Offset = 0.0.
                       UNITS:      N/A
                       TYPE:       REAL(fp)
                       DIMENSION:  Scalar
                       ATTRIBUTES: INTENT(IN), OPTIONAL
 
        Message_Log:   Character string specifying a filename in which any
                       messages will be logged. If not specified, or if an
                       error occurs opening the log file, the default action
                       is to output messages to standard output.
                       UNITS:      N/A
                       TYPE:       CHARACTER(*)
                       DIMENSION:  Scalar
                       ATTRIBUTES: INTENT(IN), OPTIONAL
 
  FUNCTION RESULT:
        Error_Status:  The return value is an integer defining the error status.
                       The error codes are defined in the Message_Handler module.
                       If == SUCCESS the structure summation was successful
                          == FAILURE an error occurred
                       UNITS:      N/A
                       TYPE:       INTEGER
                       DIMENSION:  Scalar
 
  SIDE EFFECTS:
        The argument A is INTENT(IN OUT) and is modified upon output.
 
  \end{alltt}

\subsection{\texttt{CRTM\_Zero\_Cloud} interface}
  \label{sec:CRTM_Zero_Cloud_interface}
  \begin{alltt}
 
  NAME:
        CRTM_Zero_Cloud
  
  PURPOSE:
        Subroutine to zero-out various members of a CRTM_Cloud structure - both
        scalar and pointer.
 
  CALLING SEQUENCE:
        CALL CRTM_Zero_Cloud( Cloud )
 
  OUTPUT ARGUMENTS:
        Cloud:        Zeroed out Cloud structure.
                      UNITS:      N/A
                      TYPE:       CRTM_Cloud_type
                      DIMENSION:  Scalar or Rank-1 array
                      ATTRIBUTES: INTENT(IN OUT)
 
  COMMENTS:
        - No checking of the input structure is performed, so there are no
          tests for pointer member association status. This means the Cloud
          structure must have allocated pointer members upon entry to this
          routine.
 
        - The dimension components of the structure are *NOT* set to zero.
 
        - The cloud type component is *NOT* reset.
 
        - Note the INTENT on the output Cloud argument is IN OUT rather than
          just OUT. This is necessary because the argument must be defined upon
          input.
 
  \end{alltt}

\subsection{\texttt{CRTM\_RCS\_ID\_Cloud} interface}
  \label{sec:CRTM_RCS_ID_Cloud_interface}
  \begin{alltt}
 
  NAME:
        CRTM_RCS_ID_Cloud
 
  PURPOSE:
        Subroutine to return the module RCS Id information.
 
  CALLING SEQUENCE:
        CALL CRTM_RCS_Id_Cloud( RCS_Id )
 
  OUTPUT ARGUMENTS:
        RCS_Id:        Character string containing the Revision Control
                       System Id field for the module.
                       UNITS:      N/A
                       TYPE:       CHARACTER(*)
                       DIMENSION:  Scalar
                       ATTRIBUTES: INTENT(OUT)
 
  \end{alltt}

\subsection{\texttt{CRTM\_Inquire\_Cloud\_Binary} interface}
  \label{sec:CRTM_Inquire_Cloud_Binary_interface}
  \begin{alltt}
 
  NAME:
        CRTM_Inquire_Cloud_Binary
 
  PURPOSE:
        Function to inquire Binary format CRTM Cloud structure files.
 
  CALLING SEQUENCE:
        Error_Status = CRTM_Inquire_Cloud_Binary( Filename              , &
                                                  n_Clouds   =n_Clouds  , &
                                                  RCS_Id     =RCS_Id    , &
                                                  Message_Log=Message_Log )
 
  INPUT ARGUMENTS:
        Filename:       Character string specifying the name of a
                        Cloud format data file to read.
                        UNITS:      N/A
                        TYPE:       CHARACTER(*)
                        DIMENSION:  Scalar
                        ATTRIBUTES: INTENT(IN)
 
  OPTIONAL INPUT ARGUMENTS:
        Message_Log:    Character string specifying a filename in which any
                        messages will be logged. If not specified, or if an
                        error occurs opening the log file, the default action
                        is to output messages to standard output.
                        UNITS:      N/A
                        TYPE:       CHARACTER(*)
                        DIMENSION:  Scalar
                        ATTRIBUTES: INTENT(IN), OPTIONAL
 
  OPTIONAL OUTPUT ARGUMENTS:
        n_Clouds:       The number of Cloud profiles in the data file.
                        UNITS:      N/A
                        TYPE:       INTEGER
                        DIMENSION:  Scalar
                        ATTRIBUTES: OPTIONAL, INTENT(OUT)
 
        RCS_Id:         Character string containing the Revision Control
                        System Id field for the module.
                        UNITS:      N/A
                        TYPE:       CHARACTER(*)
                        DIMENSION:  Scalar
                        ATTRIBUTES: OPTIONAL, INTENT(OUT)
 
  FUNCTION RESULT:
        Error_Status:   The return value is an integer defining the error status.
                        The error codes are defined in the ERROR_HANDLER module.
                        If == SUCCESS the Binary file inquire was successful
                           == FAILURE an unrecoverable error occurred.
                        UNITS:      N/A
                        TYPE:       INTEGER
                        DIMENSION:  Scalar
 
  \end{alltt}

\subsection{\texttt{CRTM\_Read\_Cloud\_Binary} interface}
  \label{sec:CRTM_Read_Cloud_Binary_interface}
  \begin{alltt}
 
  NAME:
        CRTM_Read_Cloud_Binary
 
  PURPOSE:
        Function to read Binary format CRTM Cloud structure files.
 
  CALLING SEQUENCE:
        Error_Status = CRTM_Read_Cloud_Binary( Filename                   , &
                                               Cloud                      , &
                                               Quiet        =Quiet        , &
                                               No_File_Close=No_File_Close, &
                                               No_Allocate  =No_Allocate  , &
                                               n_Clouds     =n_Clouds     , &
                                               RCS_Id       =RCS_Id       , &
                                               Message_Log  =Message_Log    )
 
  INPUT ARGUMENTS:
        Filename:       Character string specifying the name of a
                        Cloud format data file to read.
                        UNITS:      N/A
                        TYPE:       CHARACTER(*)
                        DIMENSION:  Scalar
                        ATTRIBUTES: INTENT(IN)
 
  OUTPUT ARGUMENTS:
        Cloud:          Structure containing the Cloud data.
                        UNITS:      N/A
                        TYPE:       CRTM_Cloud_type
                        DIMENSION:  Rank-1
                        ATTRIBUTES: INTENT(IN OUT)
 
  OPTIONAL INPUT ARGUMENTS:
        Quiet:          Set this argument to suppress INFORMATION messages
                        being printed to standard output (or the message
                        log file if the Message_Log optional argument is
                        used.)
                        If == 0, INFORMATION messages are OUTPUT [DEFAULT].
                           == 1, INFORMATION messages are SUPPRESSED.
                        If not specified, information messages are output.
                        UNITS:      N/A
                        TYPE:       INTEGER
                        DIMENSION:  Scalar
                        ATTRIBUTES: INTENT(IN), OPTIONAL
 
        No_File_Close:  Set this argument to NOT close the file upon exit.
                        If == 0, the input file is closed upon exit [DEFAULT]
                           == 1, the input file is NOT closed upon exit. 
                        If not specified, the default action is to close the
                        input file upon exit.
                        the 
                        UNITS:      N/A
                        TYPE:       INTEGER
                        DIMENSION:  Scalar
                        ATTRIBUTES: OPTIONAL, INTENT(IN)
 
        No_Allocate:    Set this argument to NOT allocate the output Cloud
                        structure in this routine based on the data dimensions
                        read from the input data file. This assumes that the
                        structure has already been allocated prior to calling 
                        this function.
                        If == 0, the output Cloud structure is allocated [DEFAULT]
                           == 1, the output Cloud structure is NOT allocated
                        If not specified, the default action is to allocate 
                        the output Cloud structure to the dimensions specified
                        in the input data file.
                        the 
                        UNITS:      N/A
                        TYPE:       INTEGER
                        DIMENSION:  Scalar
                        ATTRIBUTES: OPTIONAL, INTENT(IN)
 
        Message_Log:    Character string specifying a filename in which any
                        messages will be logged. If not specified, or if an
                        error occurs opening the log file, the default action
                        is to output messages to standard output.
                        UNITS:      N/A
                        TYPE:       CHARACTER(*)
                        DIMENSION:  Scalar
                        ATTRIBUTES: INTENT(IN), OPTIONAL
 
 
  OPTIONAL OUTPUT ARGUMENTS:
        n_Clouds:       The actual number of cloud profiles read in.
                        UNITS:      N/A
                        TYPE:       INTEGER
                        DIMENSION:  Scalar
                        ATTRIBUTES: OPTIONAL, INTENT(OUT)
 
        RCS_Id:         Character string containing the Revision Control
                        System Id field for the module.
                        UNITS:      N/A
                        TYPE:       CHARACTER(*)
                        DIMENSION:  Scalar
                        ATTRIBUTES: OPTIONAL, INTENT(OUT)
 
  FUNCTION RESULT:
        Error_Status:   The return value is an integer defining the error status.
                        The error codes are defined in the ERROR_HANDLER module.
                        If == SUCCESS the Binary file read was successful
                           == FAILURE an unrecoverable error occurred.
                        UNITS:      N/A
                        TYPE:       INTEGER
                        DIMENSION:  Scalar
 
  SIDE EFFECTS:
        If an error occurs:
        - the input file is closed and,
        - the output Cloud structure is deallocated.
 
  COMMENTS:
        Note the INTENT on the output Cloud argument is IN OUT rather
        than just OUT. This is necessary because the argument may be defined on
        input. To prevent memory leaks, the IN OUT INTENT is a must.
 
  \end{alltt}

\subsection{\texttt{CRTM\_Write\_Cloud\_Binary} interface}
  \label{sec:CRTM_Write_Cloud_Binary_interface}
  \begin{alltt}
 
  NAME:
        CRTM_Write_Cloud_Binary
 
  PURPOSE:
        Function to write Binary format Cloud files.
 
  CALLING SEQUENCE:
        Error_Status = CRTM_Write_Cloud_Binary( Filename                   , &
                                                Cloud                      , &
                                                Quiet        =Quiet        , &
                                                No_File_Close=No_File_Close, &
                                                RCS_Id       =RCS_Id       , &
                                                Message_Log  =Message_Log    )
 
  INPUT ARGUMENTS:
        Filename:       Character string specifying the name of an output
                        Cloud format data file.
                        UNITS:      N/A
                        TYPE:       CHARACTER(*)
                        DIMENSION:  Scalar
                        ATTRIBUTES: INTENT(IN)
 
        Cloud:          Structure containing the Cloud data.
                        UNITS:      N/A
                        TYPE:       CRTM_Cloud_type
                        DIMENSION:  Rank-1
                        ATTRIBUTES: INTENT(IN)
 
  OPTIONAL INPUT ARGUMENTS:
        Quiet:          Set this argument to suppress INFORMATION messages
                        being printed to standard output (or the message
                        log file if the Message_Log optional argument is
                        used.)
                        If == 0, INFORMATION messages are OUTPUT [DEFAULT].
                           == 1, INFORMATION messages are SUPPRESSED.
                        If not specified, information messages are output.
                        UNITS:      N/A
                        TYPE:       INTEGER
                        DIMENSION:  Scalar
                        ATTRIBUTES: INTENT(IN), OPTIONAL
 
        No_File_Close:  Set this argument to NOT close the file upon exit.
                        If == 0, the input file is closed upon exit [DEFAULT]
                           == 1, the input file is NOT closed upon exit. 
                        If not specified, the default action is to close the
                        input file upon exit.
                        the 
                        UNITS:      N/A
                        TYPE:       INTEGER
                        DIMENSION:  Scalar
                        ATTRIBUTES: OPTIONAL, INTENT(IN)
 
        Message_Log:    Character string specifying a filename in which any
                        messages will be logged. If not specified, or if an
                        error occurs opening the log file, the default action
                        is to output messages to standard output.
                        UNITS:      N/A
                        TYPE:       CHARACTER(*)
                        DIMENSION:  Scalar
                        ATTRIBUTES: INTENT(IN), OPTIONAL
 
  OPTIONAL OUTPUT ARGUMENTS:
        RCS_Id:         Character string containing the Revision Control
                        System Id field for the module.
                        UNITS:      N/A
                        TYPE:       CHARACTER(*)
                        DIMENSION:  Scalar
                        ATTRIBUTES: OPTIONAL, INTENT(OUT)
 
  FUNCTION RESULT:
        Error_Status:   The return value is an integer defining the error status.
                        The error codes are defined in the ERROR_HANDLER module.
                        If == SUCCESS the Binary file write was successful
                           == FAILURE an unrecoverable error occurred.
                        UNITS:      N/A
                        TYPE:       INTEGER
                        DIMENSION:  Scalar
 
  SIDE EFFECTS:
        - If the output file already exists, it is overwritten.
        - If an error occurs during *writing*, the output file is deleted before
          returning to the calling routine.
 
  \end{alltt}



\clearpage
\section{\Aerosol{} Structure}
%=============================
\label{sec:aerosol_structure}

\begin{figure}[htp]
  \centering
  \doublebox{
  \begin{minipage}[b]{6.5in}
    \begin{alltt}
  TYPE :: CRTM_Aerosol_type
    ! Allocation indicator
    LOGICAL :: Is_Allocated = .FALSE.
    ! Dimension values
    INTEGER :: Max_Layers = 0  ! K dimension.
    INTEGER :: n_Layers   = 0  ! Kuse dimension
    ! Number of added layers
    INTEGER :: n_Added_Layers = 0
    ! Aerosol type
    INTEGER :: Type = INVALID_AEROSOL
    ! Aerosol state variables
    REAL(fp), ALLOCATABLE :: Effective_Radius(:)  ! K. Units are microns
    REAL(fp), ALLOCATABLE :: Concentration(:)     ! K. Units are kg/m^2  
  END TYPE CRTM_Aerosol_type
    \end{alltt}
  \end{minipage}
  }
  \caption{CRTM\_Aerosol\_type structure definition.}
  \label{fig:CRTM_Aerosol_type_structure}
\end{figure}


% Aerosol type table
\begin{table}[htp]
  \centering
  \begin{tabular}{c l}
    \hline
    \sffamily\textbf{Aerosol Type} & \sffamily\textbf{Parameter} \\
    \hline\hline
    Dust               &  \texttt{DUST\_AEROSOL}\\
    Sea salt SSAM\footnote{SSAM $\equiv$ sea salt accumulation mode, $r_{eff}\sim$0.5-5.0\micron.} &  \texttt{SEASALT\_SSAM\_AEROSOL}\\
    Sea salt SSCM\footnote{SSCM $\equiv$ sea salt coarse mode, $r_{eff}\sim$5.0-30\micron} &  \texttt{SEASALT\_SSCM\_AEROSOL}\\
    Dry organic carbon &  \texttt{DRY\_ORGANIC\_CARBON\_AEROSOL}\\
    Wet organic carbon &  \texttt{WET\_ORGANIC\_CARBON\_AEROSOL}\\
    Dry black carbon   &  \texttt{DRY\_BLACK\_CARBON\_AEROSOL}\\
    Wet black carbon   &  \texttt{WET\_BLACK\_CARBON\_AEROSOL}\\
    Sulfate            &  \texttt{SULFATE\_AEROSOL}\\
    \hline
  \end{tabular}
  \caption{CRTM \Aerosol{} structure valid \texttt{Type} definitions.}
  \label{tab:aerosol_type}
\end{table}

% Aerosol structure methods
%--------------------------
\subsection{\texttt{CRTM\_Associated\_Aerosol} interface}
  \label{sec:CRTM_Associated_Aerosol_interface}
  \begin{alltt}
 
  NAME:
        CRTM_Associated_Aerosol
 
  PURPOSE:
        Function to test the association status of the pointer members of a
        CRTM Aerosol structure.
 
  CALLING SEQUENCE:
        Association_Status = CRTM_Associated_Aerosol( Aerosol          , &
                                                      ANY_Test=Any_Test  )
 
  INPUT ARGUMENTS:
        Aerosol:             Aerosol structure which is to have its pointer
                             member's association status tested.
                             UNITS:      N/A
                             TYPE:       CRTM_Aerosol_type
                             DIMENSION:  Scalar OR Rank-1 array
                             ATTRIBUTES: INTENT(IN)
 
  OPTIONAL INPUT ARGUMENTS:
        ANY_Test:            Set this argument to test if ANY of the
                             Aerosol structure pointer members are associated.
                             The default is to test if ALL the pointer members
                             are associated.
                             If ANY_Test = 0, test if ALL the pointer members
                                              are associated.  (DEFAULT)
                                ANY_Test = 1, test if ANY of the pointer members
                                              are associated.
                             UNITS:      N/A
                             TYPE:       INTEGER
                             DIMENSION:  Scalar
                             ATTRIBUTES: INTENT(IN), OPTIONAL
 
  FUNCTION RESULT:
        Association_Status:  The return value is a logical value indicating the
                             association status of the Aerosol pointer members.
                             .TRUE.  - if ALL the Aerosol pointer members are
                                       associated, or if the ANY_Test argument
                                       is set and ANY of the Aerosol pointer
                                       members are associated.
                             .FALSE. - some or all of the Aerosol pointer
                                       members are NOT associated.
                             UNITS:      N/A
                             TYPE:       LOGICAL
                             DIMENSION:  Same as input Aerosol argument
 
  \end{alltt}

\subsection{\texttt{CRTM\_Allocate\_Aerosol} interface}
  \label{sec:CRTM_Allocate_Aerosol_interface}
  \begin{alltt}
 
  NAME:
        CRTM_Allocate_Aerosol
  
  PURPOSE:
        Function to allocate the pointer members of the CRTM_Aerosol
        data structure.
 
  CALLING SEQUENCE:
        Error_Status = CRTM_Allocate_Aerosol( n_Layers               , &
                                              Aerosol                , &
                                              Message_Log=Message_Log  )
 
  INPUT ARGUMENTS:
          n_Layers:   Number of atmospheric layers dimension.
                      Must be > 0
                      UNITS:      N/A
                      TYPE:       INTEGER
                      DIMENSION:  Scalar OR Rank-1 array
                      ATTRIBUTES: INTENT(IN)
 
  OPTIONAL INPUT ARGUMENTS:
        Message_Log:  Character string specifying a filename in which any
                      messages will be logged. If not specified, or if an
                      error occurs opening the log file, the default action
                      is to output messages to standard output.
                      UNITS:      N/A
                      TYPE:       CHARACTER(*)
                      DIMENSION:  Scalar
                      ATTRIBUTES: INTENT(IN), OPTIONAL
 
  OUTPUT ARGUMENTS:
        Aerosol:      CRTM_Aerosol structure with allocated pointer members.
                      UNITS:      N/A
                      TYPE:       CRTM_Aerosol_type
                      DIMENSION:  Same as input n_Layers argument
                      ATTRIBUTES: INTENT(IN OUT)
 
 
  FUNCTION RESULT:
        Error_Status: The return value is an integer defining the error status.
                      The error codes are defined in the Message_Handler module.
                      If == SUCCESS the structure re-initialisation was successful
                         == FAILURE - an error occurred, or
                                    - the structure internal allocation counter
                                      is not equal to one (1) upon exiting this
                                      function. This value is incremented and
                                      decremented for every structure allocation
                                      and deallocation respectively.
                      UNITS:      N/A
                      TYPE:       INTEGER
                      DIMENSION:  Scalar
 
  COMMENTS:
        Note the INTENT on the output Aerosol argument is IN OUT rather than
        just OUT. This is necessary because the argument may be defined upon
        input. To prevent memory leaks, the IN OUT INTENT is a must.
 
  \end{alltt}

\subsection{\texttt{CRTM\_Destroy\_Aerosol} interface}
  \label{sec:CRTM_Destroy_Aerosol_interface}
  \begin{alltt}
 
  NAME:
        CRTM_Destroy_Aerosol
  
  PURPOSE:
        Function to re-initialize the scalar and pointer members of
        a CRTM_Aerosol data structure.
 
  CALLING SEQUENCE:
        Error_Status = CRTM_Destroy_Aerosol( Aerosol                , &
                                             Message_Log=Message_Log  )
 
  OPTIONAL INPUT ARGUMENTS:
        Message_Log:  Character string specifying a filename in which any
                      messages will be logged. If not specified, or if an
                      error occurs opening the log file, the default action
                      is to output messages to standard output.
                      UNITS:      N/A
                      TYPE:       CHARACTER(*)
                      DIMENSION:  Scalar
                      ATTRIBUTES: INTENT(IN), OPTIONAL
 
  OUTPUT ARGUMENTS:
        Aerosol:      Re-initialized CRTM_Aerosol structure.
                      UNITS:      N/A
                      TYPE:       CRTM_Aerosol_type
                      DIMENSION:  Scalar OR Rank-1 array
                      ATTRIBUTES: INTENT(IN OUT)
 
  FUNCTION RESULT:
        Error_Status: The return value is an integer defining the error status.
                      The error codes are defined in the Message_Handler module.
                      If == SUCCESS the structure re-initialisation was successful
                         == FAILURE - an error occurred, or
                                    - the structure internal allocation counter
                                      is not equal to zero (0) upon exiting this
                                      function. This value is incremented and
                                      decremented for every structure allocation
                                      and deallocation respectively.
                      UNITS:      N/A
                      TYPE:       INTEGER
                      DIMENSION:  Scalar
 
  COMMENTS:
        Note the INTENT on the output Aerosol argument is IN OUT rather than
        just OUT. This is necessary because the argument may be defined upon
        input. To prevent memory leaks, the IN OUT INTENT is a must.
 
  \end{alltt}

\subsection{\texttt{CRTM\_Assign\_Aerosol} interface}
  \label{sec:CRTM_Assign_Aerosol_interface}
  \begin{alltt}
 
  NAME:
        CRTM_Assign_Aerosol
 
  PURPOSE:
        Function to copy valid CRTM Aerosol structures.
 
  CALLING SEQUENCE:
        Error_Status = CRTM_Assign_Aerosol( Aerosol_in             , &
                                            Aerosol_out            , &
                                            Message_Log=Message_Log  )
 
  INPUT ARGUMENTS:
        Aerosol_in:      Aerosol structure which is to be copied.
                         UNITS:      N/A
                         TYPE:       CRTM_Aerosol_type
                         DIMENSION:  Scalar OR Rank-1 array
                         ATTRIBUTES: INTENT(IN)
 
  OUTPUT ARGUMENTS:
        Aerosol_out:     Copy of the input structure, Aerosol_in.
                         UNITS:      N/A
                         TYPE:       CRTM_Aerosol_type
                         DIMENSION:  Same as Aerosol_in argument
                         ATTRIBUTES: INTENT(IN OUT)
 
 
  OPTIONAL INPUT ARGUMENTS:
        Message_Log:     Character string specifying a filename in which any
                         messages will be logged. If not specified, or if an
                         error occurs opening the log file, the default action
                         is to output messages to standard output.
                         UNITS:      N/A
                         TYPE:       CHARACTER(*)
                         DIMENSION:  Scalar
                         ATTRIBUTES: INTENT(IN), OPTIONAL
 
  FUNCTION RESULT:
        Error_Status:    The return value is an integer defining the error status.
                         The error codes are defined in the Message_Handler module.
                         If == SUCCESS the structure assignment was successful
                            == FAILURE an error occurred
                         UNITS:      N/A
                         TYPE:       INTEGER
                         DIMENSION:  Scalar
 
  COMMENTS:
        Note the INTENT on the output Aerosol argument is IN OUT rather than
        just OUT. This is necessary because the argument may be defined upon
        input. To prevent memory leaks, the IN OUT INTENT is a must.
 
  \end{alltt}

\subsection{\texttt{CRTM\_Equal\_Aerosol} interface}
  \label{sec:CRTM_Equal_Aerosol_interface}
  \begin{alltt}
 
  NAME:
        CRTM_Equal_Aerosol
 
  PURPOSE:
        Function to test if two Aerosol structures are equal.
 
  CALLING SEQUENCE:
        Error_Status = CRTM_Equal_Aerosol( Aerosol_LHS            , &
                                           Aerosol_RHS            , &
                                           ULP_Scale  =ULP_Scale  , &
                                           Check_All  =Check_All  , &
                                           Message_Log=Message_Log  )
 
 
  INPUT ARGUMENTS:
        Aerosol_LHS:       Aerosol structure to be compared; equivalent to the
                           left-hand side of a lexical comparison, e.g.
                             IF ( Aerosol_LHS == Aerosol_RHS ).
                           UNITS:      N/A
                           TYPE:       CRTM_Aerosol_type
                           DIMENSION:  Scalar OR Rank-1 array
                           ATTRIBUTES: INTENT(IN)
 
        Aerosol_RHS:       Aerosol structure to be compared to; equivalent to
                           right-hand side of a lexical comparison, e.g.
                             IF ( Aerosol_LHS == Aerosol_RHS ).
                           UNITS:      N/A
                           TYPE:       CRTM_Aerosol_type
                           DIMENSION:  Same as Aerosol_LHS
                           ATTRIBUTES: INTENT(IN)
 
  OPTIONAL INPUT ARGUMENTS:
        ULP_Scale:         Unit of data precision used to scale the floating
                           point comparison. ULP stands for "Unit in the Last Place,"
                           the smallest possible increment or decrement that can be
                           made using a machine's floating point arithmetic.
                           Value must be positive - if a negative value is supplied,
                           the absolute value is used. If not specified, the default
                           value is 1.
                           UNITS:      N/A
                           TYPE:       INTEGER
                           DIMENSION:  Scalar
                           ATTRIBUTES: INTENT(IN), OPTIONAL
 
        Check_All:         Set this argument to check ALL the floating point
                           channel data of the Aerosol structures. The default
                           action is return with a FAILURE status as soon as
                           any difference is found. This optional argument can
                           be used to get a listing of ALL the differences
                           between data in Aerosol structures.
                           If == 0, Return with FAILURE status as soon as
                                    ANY difference is found  *DEFAULT*
                              == 1, Set FAILURE status if ANY difference is
                                    found, but continue to check ALL data.
                           UNITS:      N/A
                           TYPE:       INTEGER
                           DIMENSION:  Scalar
                           ATTRIBUTES: INTENT(IN), OPTIONAL
 
        Message_Log:       Character string specifying a filename in which any
                           messages will be logged. If not specified, or if an
                           error occurs opening the log file, the default action
                           is to output messages to standard output.
                           UNITS:      None
                           TYPE:       CHARACTER(*)
                           DIMENSION:  Scalar
                           ATTRIBUTES: INTENT(IN), OPTIONAL
 
  FUNCTION RESULT:
        Error_Status:      The return value is an integer defining the error status.
                           The error codes are defined in the Message_Handler module.
                           If == SUCCESS the structures were equal
                              == FAILURE - an error occurred, or
                                         - the structures were different.
                           UNITS:      N/A
                           TYPE:       INTEGER
                           DIMENSION:  Scalar
 
  \end{alltt}

\subsection{\texttt{CRTM\_SetLayers\_Aerosol} interface}
  \label{sec:CRTM_SetLayers_Aerosol_interface}
  \begin{alltt}
 
  NAME:
        CRTM_SetLayers_Aerosol
  
  PURPOSE:
        Function to set the number of layers to use in a CRTM Aerosol
        structure.
 
  CALLING SEQUENCE:
        Error_Status = CRTM_SetLayers_Aerosol( n_Layers, &
                                               Aerosol   )
 
  INPUTS:
        n_Layers:     The value to set the n_Layers component of the 
                      Aerosol structure, as well as those of any of its
                      structure components.
                      UNITS:      N/A
                      TYPE:       CRTM_Aerosol_type
                      DIMENSION:  Scalar
                      ATTRIBUTES: INTENT(IN)
 
        Aerosol:      Aerosol structure in which the n_Layers dimension
                      is to be updated.
                      UNITS:      N/A
                      TYPE:       CRTM_Aerosol_type
                      DIMENSION:  Scalar or Rank-1 array
                      ATTRIBUTES: INTENT(IN OUT)
  OUTPUTS:
        Aerosol:      On output, the Aerosol structure with the updated
                      n_Layers dimension.
                      UNITS:      N/A
                      TYPE:       CRTM_Aerosol_type
                      DIMENSION:  Scalar or Rank-1 array
                      ATTRIBUTES: INTENT(IN OUT)
 
  FUNCTION RESULT:
        Error_Status: The return value is an integer defining the error status.
                      The error codes are defined in the Message_Handler module.
                      If == SUCCESS the layer reset was successful
                         == FAILURE an error occurred
                      UNITS:      N/A
                      TYPE:       INTEGER
                      DIMENSION:  Scalar
 
  SIDE EFFECTS:
        The argument Aerosol is INTENT(IN OUT) and is modified upon output. The
        elements of the structureare reinitialised
 
  COMMENTS:
        - Note that the n_Layers input is *ALWAYS* scalar. Thus, all Aerosol
          elements will be set to the same number of layers.
 
        - If n_Layers <= Aerosol%Max_Layers, then only the dimension value
          of the structure and any sub-structures are changed.
 
        - If n_Layers > Aerosol%Max_Layers, then the entire structure is
          reallocated to the required number of layers.
 
  \end{alltt}

\subsection{\texttt{CRTM\_Sum\_Aerosol} interface}
  \label{sec:CRTM_Sum_Aerosol_interface}
  \begin{alltt}
 
  NAME:
        CRTM_Sum_Aerosol
 
  PURPOSE:
        Function to perform a sum of two valid CRTM Aerosol structures. The
        summation performed is:
          A = A + Scale_Factor*B + Offset
        where A and B are the CRTM Aerosol structures, and Scale_Factor and Offset
        are optional weighting factors.
 
  CALLING SEQUENCE:
        Error_Status = CRTM_Sum_Aerosol( A                        , &
                                         B                        , &
                                         Scale_Factor=Scale_Factor, &
                                         Offset      =Offset      , &
                                         Message_Log =Message_Log   )
 
  INPUT ARGUMENTS:
        A:             Aerosol structure that is to be added to.
                       UNITS:      N/A
                       TYPE:       CRTM_Aerosol_type
                       DIMENSION:  Scalar OR Rank-1
                       ATTRIBUTES: INTENT(IN OUT)
 
        B:             Aerosol structure that is to be weighted and
                       added to structure A.
                       UNITS:      N/A
                       TYPE:       CRTM_Aerosol_type
                       DIMENSION:  Same as A
                       ATTRIBUTES: INTENT(IN)
 
  OPTIONAL INPUT ARGUMENTS:
        Scale_Factor:  The first weighting factor used to scale the
                       contents of the input structure, B.
                       If not specified, Scale_Factor = 1.0.
                       UNITS:      N/A
                       TYPE:       REAL(fp)
                       DIMENSION:  Scalar
                       ATTRIBUTES: INTENT(IN), OPTIONAL
 
        Offset:        The second weighting factor used to offset the
                       sum of the input structures.
                       If not specified, Offset = 0.0.
                       UNITS:      N/A
                       TYPE:       REAL(fp)
                       DIMENSION:  Scalar
                       ATTRIBUTES: INTENT(IN), OPTIONAL
 
        Message_Log:   Character string specifying a filename in which any
                       messages will be logged. If not specified, or if an
                       error occurs opening the log file, the default action
                       is to output messages to standard output.
                       UNITS:      N/A
                       TYPE:       CHARACTER(*)
                       DIMENSION:  Scalar
                       ATTRIBUTES: INTENT(IN), OPTIONAL
 
  OUTPUT ARGUMENTS:
        A:             Structure containing the summation result,
                         A = A + Scale_Factor*B + Offset
                       UNITS:      N/A
                       TYPE:       CRTM_Aerosol_type
                       DIMENSION:  Same as B
                       ATTRIBUTES: INTENT(IN OUT)
 
 
  FUNCTION RESULT:
        Error_Status:  The return value is an integer defining the error status.
                       The error codes are defined in the Message_Handler module.
                       If == SUCCESS the structure summation was successful
                          == FAILURE an error occurred
                       UNITS:      N/A
                       TYPE:       INTEGER
                       DIMENSION:  Scalar
 
  SIDE EFFECTS:
        The argument A is INTENT(IN OUT) and is modified upon output.
 
  \end{alltt}

\subsection{\texttt{CRTM\_Zero\_Aerosol} interface}
  \label{sec:CRTM_Zero_Aerosol_interface}
  \begin{alltt}
 
  NAME:
        CRTM_Zero_Aerosol
  
  PURPOSE:
        Subroutine to zero-out all members of a CRTM_Aerosol structure - both
        scalar and pointer.
 
  CALLING SEQUENCE:
        CALL CRTM_Zero_Aerosol( Aerosol )
 
  OUTPUT ARGUMENTS:
        Aerosol: Zeroed out Aerosol structure.
                 UNITS:      N/A
                 TYPE:       CRTM_Aerosol_type
                 DIMENSION:  Scalar or Rank-1 array
                 ATTRIBUTES: INTENT(IN OUT)
 
  COMMENTS:
        - No checking of the input structure is performed, so there are no
          tests for pointer member association status. This means the Aerosol
          structure must have allocated pointer members upon entry to this
          routine.
 
        - The dimension components of the structure are *NOT*
          set to zero.
 
        - The aerosol type component is *NOT* reset.
 
        - Note the INTENT on the output Aerosol argument is IN OUT rather than
          just OUT. This is necessary because the argument must be defined upon
          input.
 
  \end{alltt}

\subsection{\texttt{CRTM\_RCS\_ID\_Aerosol} interface}
  \label{sec:CRTM_RCS_ID_Aerosol_interface}
  \begin{alltt}
 
  NAME:
        CRTM_RCS_ID_Aerosol
 
  PURPOSE:
        Subroutine to return the module RCS Id information.
 
  CALLING SEQUENCE:
        CALL CRTM_RCS_Id_Aerosol( RCS_Id )
 
  OUTPUT ARGUMENTS:
        RCS_Id:        Character string containing the Revision Control
                       System Id field for the module.
                       UNITS:      N/A
                       TYPE:       CHARACTER(*)
                       DIMENSION:  Scalar
                       ATTRIBUTES: INTENT(OUT)
 
  \end{alltt}

\subsection{\texttt{CRTM\_Inquire\_Aerosol\_Binary} interface}
  \label{sec:CRTM_Inquire_Aerosol_Binary_interface}
  \begin{alltt}
 
  NAME:
        CRTM_Inquire_Aerosol_Binary
 
  PURPOSE:
        Function to inquire Binary format CRTM Aerosol structure files.
 
  CALLING SEQUENCE:
        Error_Status = CRTM_Inquire_Aerosol_Binary( Filename              , &
                                                    n_Aerosols =n_Aerosols, &
                                                    RCS_Id     =RCS_Id    , &
                                                    Message_Log=Message_Log )
 
  INPUT ARGUMENTS:
        Filename:       Character string specifying the name of a
                        Aerosol format data file to read.
                        UNITS:      N/A
                        TYPE:       CHARACTER(*)
                        DIMENSION:  Scalar
                        ATTRIBUTES: INTENT(IN)
 
  OPTIONAL INPUT ARGUMENTS:
        Message_Log:    Character string specifying a filename in which any
                        messages will be logged. If not specified, or if an
                        error occurs opening the log file, the default action
                        is to output messages to standard output.
                        UNITS:      N/A
                        TYPE:       CHARACTER(*)
                        DIMENSION:  Scalar
                        ATTRIBUTES: INTENT(IN), OPTIONAL
 
  OPTIONAL OUTPUT ARGUMENTS:
        n_Aerosols:     The number of Aerosol profiles in the data file.
                        UNITS:      N/A
                        TYPE:       INTEGER
                        DIMENSION:  Scalar
                        ATTRIBUTES: OPTIONAL, INTENT(OUT)
 
        RCS_Id:         Character string containing the Revision Control
                        System Id field for the module.
                        UNITS:      N/A
                        TYPE:       CHARACTER(*)
                        DIMENSION:  Scalar
                        ATTRIBUTES: OPTIONAL, INTENT(OUT)
 
  FUNCTION RESULT:
        Error_Status:   The return value is an integer defining the error status.
                        The error codes are defined in the ERROR_HANDLER module.
                        If == SUCCESS the Binary file inquire was successful
                           == FAILURE an unrecoverable error occurred.
                        UNITS:      N/A
                        TYPE:       INTEGER
                        DIMENSION:  Scalar
 
  \end{alltt}

\subsection{\texttt{CRTM\_Read\_Aerosol\_Binary} interface}
  \label{sec:CRTM_Read_Aerosol_Binary_interface}
  \begin{alltt}
 
  NAME:
        CRTM_Read_Aerosol_Binary
 
  PURPOSE:
        Function to read Binary format CRTM Aerosol structure files.
 
  CALLING SEQUENCE:
        Error_Status = CRTM_Read_Aerosol_Binary( Filename                   , &
                                                 Aerosol                    , &
                                                 Quiet        =Quiet        , &
                                                 No_File_Close=No_File_Close, &
                                                 No_Allocate  =No_Allocate  , &
                                                 n_Aerosols   =n_Aerosols   , &
                                                 RCS_Id       =RCS_Id       , &
                                                 Message_Log  =Message_Log    )
 
  INPUT ARGUMENTS:
        Filename:       Character string specifying the name of a
                        Aerosol format data file to read.
                        UNITS:      N/A
                        TYPE:       CHARACTER(*)
                        DIMENSION:  Scalar
                        ATTRIBUTES: INTENT(IN)
 
  OUTPUT ARGUMENTS:
        Aerosol:        Structure containing the Aerosol data.
                        UNITS:      N/A
                        TYPE:       CRTM_Aerosol_type
                        DIMENSION:  Rank-1
                        ATTRIBUTES: INTENT(IN OUT)
 
  OPTIONAL INPUT ARGUMENTS:
        Quiet:          Set this argument to suppress INFORMATION messages
                        being printed to standard output (or the message
                        log file if the Message_Log optional argument is
                        used.)
                        If == 0, INFORMATION messages are OUTPUT [DEFAULT].
                           == 1, INFORMATION messages are SUPPRESSED.
                        If not specified, information messages are output.
                        UNITS:      N/A
                        TYPE:       INTEGER
                        DIMENSION:  Scalar
                        ATTRIBUTES: INTENT(IN), OPTIONAL
 
        No_File_Close:  Set this argument to NOT close the file upon exit.
                        If == 0, the input file is closed upon exit [DEFAULT]
                           == 1, the input file is NOT closed upon exit. 
                        If not specified, the default action is to close the
                        input file upon exit.
                        the 
                        UNITS:      N/A
                        TYPE:       INTEGER
                        DIMENSION:  Scalar
                        ATTRIBUTES: OPTIONAL, INTENT(IN)
 
        No_Allocate:    Set this argument to NOT allocate the output Aerosol
                        structure in this routine based on the data dimensions
                        read from the input data file. This assumes that the
                        structure has already been allocated prior to calling 
                        this function.
                        If == 0, the output Aerosol structure is allocated [DEFAULT]
                           == 1, the output Aerosol structure is NOT allocated
                        If not specified, the default action is to allocate 
                        the output Aerosol structure to the dimensions specified
                        in the input data file.
                        the 
                        UNITS:      N/A
                        TYPE:       INTEGER
                        DIMENSION:  Scalar
                        ATTRIBUTES: OPTIONAL, INTENT(IN)
 
        Message_Log:    Character string specifying a filename in which any
                        messages will be logged. If not specified, or if an
                        error occurs opening the log file, the default action
                        is to output messages to standard output.
                        UNITS:      N/A
                        TYPE:       CHARACTER(*)
                        DIMENSION:  Scalar
                        ATTRIBUTES: INTENT(IN), OPTIONAL
 
 
  OPTIONAL OUTPUT ARGUMENTS:
        n_Aerosols:     The actual number of aerosol profiles read in.
                        UNITS:      N/A
                        TYPE:       INTEGER
                        DIMENSION:  Scalar
                        ATTRIBUTES: OPTIONAL, INTENT(OUT)
 
        RCS_Id:         Character string containing the Revision Control
                        System Id field for the module.
                        UNITS:      N/A
                        TYPE:       CHARACTER(*)
                        DIMENSION:  Scalar
                        ATTRIBUTES: OPTIONAL, INTENT(OUT)
 
  FUNCTION RESULT:
        Error_Status:   The return value is an integer defining the error status.
                        The error codes are defined in the ERROR_HANDLER module.
                        If == SUCCESS the Binary file read was successful
                           == FAILURE an unrecoverable error occurred.
                        UNITS:      N/A
                        TYPE:       INTEGER
                        DIMENSION:  Scalar
 
  SIDE EFFECTS:
        If an error occurs:
        - the input file is closed and,
        - the output Aerosol structure is deallocated.
 
  COMMENTS:
        Note the INTENT on the output Aerosol argument is IN OUT rather
        than just OUT. This is necessary because the argument may be defined on
        input. To prevent memory leaks, the IN OUT INTENT is a must.
 
  \end{alltt}

\subsection{\texttt{CRTM\_Write\_Aerosol\_Binary} interface}
  \label{sec:CRTM_Write_Aerosol_Binary_interface}
  \begin{alltt}
 
  NAME:
        CRTM_Write_Aerosol_Binary
 
  PURPOSE:
        Function to write Binary format Aerosol files.
 
  CALLING SEQUENCE:
        Error_Status = CRTM_Write_Aerosol_Binary( Filename                   , &
                                                  Aerosol                    , &
                                                  Quiet        =Quiet        , &
                                                  No_File_Close=No_File_Close, &
                                                  RCS_Id       =RCS_Id       , &
                                                  Message_Log  =Message_Log    )
 
  INPUT ARGUMENTS:
        Filename:       Character string specifying the name of an output
                        Aerosol format data file.
                        UNITS:      N/A
                        TYPE:       CHARACTER(*)
                        DIMENSION:  Scalar
                        ATTRIBUTES: INTENT(IN)
 
        Aerosol:        Structure containing the Aerosol data.
                        UNITS:      N/A
                        TYPE:       CRTM_Aerosol_type
                        DIMENSION:  Rank-1
                        ATTRIBUTES: INTENT(IN)
 
  OPTIONAL INPUT ARGUMENTS:
        Quiet:          Set this argument to suppress INFORMATION messages
                        being printed to standard output (or the message
                        log file if the Message_Log optional argument is
                        used.)
                        If == 0, INFORMATION messages are OUTPUT [DEFAULT].
                           == 1, INFORMATION messages are SUPPRESSED.
                        If not specified, information messages are output.
                        UNITS:      N/A
                        TYPE:       INTEGER
                        DIMENSION:  Scalar
                        ATTRIBUTES: INTENT(IN), OPTIONAL
 
        No_File_Close:  Set this argument to NOT close the file upon exit.
                        If == 0, the input file is closed upon exit [DEFAULT]
                           == 1, the input file is NOT closed upon exit. 
                        If not specified, the default action is to close the
                        input file upon exit.
                        the 
                        UNITS:      N/A
                        TYPE:       INTEGER
                        DIMENSION:  Scalar
                        ATTRIBUTES: OPTIONAL, INTENT(IN)
 
        Message_Log:    Character string specifying a filename in which any
                        messages will be logged. If not specified, or if an
                        error occurs opening the log file, the default action
                        is to output messages to standard output.
                        UNITS:      N/A
                        TYPE:       CHARACTER(*)
                        DIMENSION:  Scalar
                        ATTRIBUTES: INTENT(IN), OPTIONAL
 
  OPTIONAL OUTPUT ARGUMENTS:
        RCS_Id:         Character string containing the Revision Control
                        System Id field for the module.
                        UNITS:      N/A
                        TYPE:       CHARACTER(*)
                        DIMENSION:  Scalar
                        ATTRIBUTES: OPTIONAL, INTENT(OUT)
 
  FUNCTION RESULT:
        Error_Status:   The return value is an integer defining the error status.
                        The error codes are defined in the ERROR_HANDLER module.
                        If == SUCCESS the Binary file write was successful
                           == FAILURE an unrecoverable error occurred.
                        UNITS:      N/A
                        TYPE:       INTEGER
                        DIMENSION:  Scalar
 
  SIDE EFFECTS:
        - If the output file already exists, it is overwritten.
        - If an error occurs during *writing*, the output file is deleted before
          returning to the calling routine.
 
  \end{alltt}




\clearpage
\section{\Surface{} Structure}
%=============================
\label{sec:surface_structure}

\begin{figure}[htp]
  \centering
  \doublebox{
  \begin{minipage}[b]{6.5in}
    \begin{alltt}
  TYPE :: CRTM_Surface_type
    INTEGER :: n_Allocates = 0
    ! Dimension values
    INTEGER :: Max_Sensors  = 0  ! N dimension
    INTEGER :: n_Sensors    = 0  ! Nuse dimension
    ! Gross type of surface determined by coverage
    REAL(fp) :: Land_Coverage  = ZERO
    REAL(fp) :: Water_Coverage = ZERO
    REAL(fp) :: Snow_Coverage  = ZERO
    REAL(fp) :: Ice_Coverage   = ZERO
    ! Surface type independent data
    REAL(fp) :: Wind_Speed     = DEFAULT_WIND_SPEED
    REAL(fp) :: Wind_Direction = DEFAULT_WIND_DIRECTION
    ! Land surface type data
    INTEGER  :: Land_Type             = DEFAULT_LAND_TYPE
    REAL(fp) :: Land_Temperature      = DEFAULT_LAND_TEMPERATURE
    REAL(fp) :: Soil_Moisture_Content = DEFAULT_SOIL_MOISTURE_CONTENT
    REAL(fp) :: Canopy_Water_Content  = DEFAULT_CANOPY_WATER_CONTENT
    REAL(fp) :: Vegetation_Fraction   = DEFAULT_VEGETATION_FRACTION
    REAL(fp) :: Soil_Temperature      = DEFAULT_SOIL_TEMPERATURE
    ! Water type data
    INTEGER  :: Water_Type        = DEFAULT_WATER_TYPE
    REAL(fp) :: Water_Temperature = DEFAULT_WATER_TEMPERATURE
    REAL(fp) :: Salinity          = DEFAULT_SALINITY
    ! Snow surface type data
    INTEGER  :: Snow_Type        = DEFAULT_SNOW_TYPE
    REAL(fp) :: Snow_Temperature = DEFAULT_SNOW_TEMPERATURE
    REAL(fp) :: Snow_Depth       = DEFAULT_SNOW_DEPTH
    REAL(fp) :: Snow_Density     = DEFAULT_SNOW_DENSITY
    REAL(fp) :: Snow_Grain_Size  = DEFAULT_SNOW_GRAIN_SIZE
    ! Ice surface type data
    INTEGER  :: Ice_Type        = DEFAULT_ICE_TYPE
    REAL(fp) :: Ice_Temperature = DEFAULT_ICE_TEMPERATURE
    REAL(fp) :: Ice_Thickness   = DEFAULT_ICE_THICKNESS
    REAL(fp) :: Ice_Density     = DEFAULT_ICE_DENSITY
    REAL(fp) :: Ice_Roughness   = DEFAULT_ICE_ROUGHNESS
    ! SensorData containing channel brightness temperatures
    TYPE(CRTM_SensorData_type) :: SensorData  ! N
  END TYPE CRTM_Surface_type
    \end{alltt}
  \end{minipage}
  }
  \caption{CRTM\_Surface\_type structure definition.}
  \label{fig:CRTM_Surface_type_structure}
\end{figure}


\begin{table}[htp]
  \centering
  \begin{tabular}{l p{7cm} c c}
    \hline
    \sffamily\textbf{Component} & \sffamily\textbf{Description} & \sffamily\textbf{Units} & \sffamily\textbf{Dimensions} \\
    \hline\hline
    \texttt{n\_Sensors} & The number of sensors for which data is provided inside the SensorData structure & N/A & Scalar \\
    \hline
    \texttt{Land\_Coverage}  & Fraction of the FOV that is land surface & N/A & Scalar \\
    \texttt{Water\_Coverage} & Fraction of the FOV that is water surface & N/A & Scalar \\
    \texttt{Snow\_Coverage}  & Fraction of the FOV that is snow surface & N/A & Scalar \\
    \texttt{Ice\_Coverage}   & Fraction of the FOV that is ice surface & N/A & Scalar \\
    \hline
    \texttt{Wind\_Speed}     & Surface wind speed & m.s$^{-1}$ & Scalar \\
    \texttt{Wind\_Direction} & Surface wind direction & deg. E from N & Scalar \\
    \hline
    \texttt{Land\_Type}              & Land surface type & N/A & Scalar \\
    \texttt{Land\_Temperature}       & Land surface temperature & Kelvin & Scalar \\
    \texttt{Soil\_Moisture\_Content} & Volumetric water content of the soil & g.cm$^{-3}$ & Scalar \\
    \texttt{Canopy\_Water\_Content}  & Gravimetric water content of the canopy & g.cm$^{-3}$ & Scalar \\
    \texttt{Vegetation\_Fraction}    & Vegetation fraction of the surface & \% & Scalar \\
    \texttt{Soil\_Temperature}       & Soil temperature & Kelvin & Scalar \\
    \hline
    \texttt{Water\_Type}        & Water surface type & N/A & Scalar \\
    \texttt{Water\_Temperature} & Water surface temperature & Kelvin & Scalar \\
    \texttt{Salinity}           & Water salinity & \textperthousand & Scalar \\
    \hline
    \texttt{Snow\_Type}        & Snow surface type & N/A & Scalar \\ 
    \texttt{Snow\_Temperature} & Snow surface temperature & Kelvin & Scalar \\ 
    \texttt{Snow\_Depth}       & Snow depth & mm & Scalar \\ 
    \texttt{Snow\_Density}     & Snow density & g.m$^{-3}$ & Scalar \\ 
    \texttt{Snow\_Grain\_Size} & Snow grain size & mm & Scalar \\ 
    \hline
    \texttt{Ice\_Type}        & Ice surface type & N/A & Scalar \\ 
    \texttt{Ice\_Temperature} & Ice surface temperature & Kelvin & Scalar \\ 
    \texttt{Ice\_Thickness}   & Thickness of ice & mm & Scalar \\ 
    \texttt{Ice\_Density}     & Density of ice & g.m$^{-3}$ & Scalar \\ 
    \texttt{Ice\_Roughness}   & Measure of the surface roughness of the ice & N/A & Scalar \\ 
    \hline
    \texttt{SensorData} & Satellite sensor data required for some surface emissivity algorithms & N/A & Scalar \\ 
    \hline
  \end{tabular}
  \caption{CRTM \Surface{} structure component description.}
  \label{tab:surface_structure}
\end{table}

% Default surface value table
\begin{table}[htp]
  \centering
  \begin{tabular}{l c c}
    \hline
    \sffamily\textbf{Parameter} & \sffamily\textbf{Value}  & \sffamily\textbf{Units} \\
    \hline\hline
    \multicolumn{3}{c}{\textsf{Surface type independent data}}\\
    \hline
    \texttt{DEFAULT\_WIND\_SPEED}             & 5.0        & m.s$^{-1}$\\
    \texttt{DEFAULT\_WIND\_DIRECTION}         & 0.0        & deg. E from N\\[0.2cm]
    \multicolumn{3}{c}{\textsf{Land surface type data}}\\
    \hline
    \texttt{DEFAULT\_LAND\_TYPE}              & \texttt{GRASS\_SOIL}& N/A \\
    \texttt{DEFAULT\_LAND\_TEMPERATURE}       & 283.0      & K\\
    \texttt{DEFAULT\_SOIL\_MOISTURE\_CONTENT} & 0.05       & g.cm$^{-3}$\\
    \texttt{DEFAULT\_CANOPY\_WATER\_CONTENT}  & 0.05       & g.cm$^{-3}$\\
    \texttt{DEFAULT\_VEGETATION\_FRACTION}    & 0.3        & 30\%\\
    \texttt{DEFAULT\_SOIL\_TEMPERATURE}       & 283.0      & K\\[0.2cm]
    \multicolumn{3}{c}{\textsf{Water type data}}\\
    \hline
    \texttt{DEFAULT\_WATER\_TYPE}             & \texttt{SEA\_WATER} & N/A\\
    \texttt{DEFAULT\_WATER\_TEMPERATURE}      & 283.0      & K\\
    \texttt{DEFAULT\_SALINITY}                & 33.0       & ppmv\\[0.2cm]
    \multicolumn{3}{c}{\textsf{Snow surface type data}}\\
    \hline
    \texttt{DEFAULT\_SNOW\_TYPE}              & \texttt{NEW\_SNOW}  & N/A\\
    \texttt{DEFAULT\_SNOW\_TEMPERATURE}       & 263.0      & K\\
    \texttt{DEFAULT\_SNOW\_DEPTH}             & 50.0       & mm\\
    \texttt{DEFAULT\_SNOW\_DENSITY}           & 0.2        & g.cm$^{-3}$\\
    \texttt{DEFAULT\_SNOW\_GRAIN\_SIZE}       & 2.0        & mm\\[0.2cm]
    \multicolumn{3}{c}{\textsf{Ice surface type data}}\\
    \hline
    \texttt{DEFAULT\_ICE\_TYPE}               & \texttt{FRESH\_ICE} & N/A\\
    \texttt{DEFAULT\_ICE\_TEMPERATURE}        & 263.0      & K\\
    \texttt{DEFAULT\_ICE\_THICKNESS}          & 10.0       & mm\\
    \texttt{DEFAULT\_ICE\_DENSITY}            & 0.9        & g.cm$^{-3}$\\
    \texttt{DEFAULT\_ICE\_ROUGHNESS}          & 0.0        & N/A\\
    \hline
  \end{tabular}
  \caption{CRTM \Surface{} structure default values.}
  \label{tab:surface_default}
\end{table}

% Land surface type table
\begin{table}[htp]
  \centering
  \begin{tabular}{c l}
    \hline
    \sffamily\textbf{Land Type} & \sffamily\textbf{Parameter} \\
    \hline\hline
          Compacted soil      & \texttt{COMPACTED\_SOIL} \\
            Tilled soil       & \texttt{TILLED\_SOIL} \\
              Sand            & \texttt{SAND} \\
              Rock            & \texttt{ROCK} \\
     Irrigated low vegetation & \texttt{IRRIGATED\_LOW\_VEGETATION} \\
           Meadow grass       & \texttt{MEADOW\_GRASS} \\
              Scrub           & \texttt{SCRUB} \\
         Broadleaf forest     & \texttt{BROADLEAF\_FOREST} \\
           Pine forest        & \texttt{PINE\_FOREST} \\
             Tundra           & \texttt{TUNDRA} \\
           Grass soil         & \texttt{GRASS\_SOIL} \\
       Broadleaf-pine forest  & \texttt{BROADLEAF\_PINE\_FOREST} \\
           Grass scrub        & \texttt{GRASS\_SCRUB} \\
            Oil grass         & \texttt{OIL\_GRASS} \\
          Urban concrete      & \texttt{URBAN\_CONCRETE} \\
            Pine brush        & \texttt{PINE\_BRUSH} \\
          Broadleaf brush     & \texttt{BROADLEAF\_BRUSH} \\
             Wet soil         & \texttt{WET\_SOIL} \\
            Scrub soil        & \texttt{SCRUB\_SOIL} \\
      Broadleaf(70)-Pine(30)  & \texttt{BROADLEAF70\_PINE30} \\
    \hline
  \end{tabular}
  \caption{CRTM \Surface{} structure valid \texttt{Land\_Type} definitions.}
  \label{tab:surface_land_type}
\end{table}

% Water surface type table
\begin{table}[htp]
  \centering
  \begin{tabular}{cc l}
    \hline
    \sffamily\textbf{Water Type} & \hspace{0.5cm} & \sffamily\textbf{Parameter} \\
    \hline\hline
      Sea water  & \hspace{0.5cm} &  \texttt{SEA\_WATER} \\     
     Fresh water & \hspace{0.5cm} &  \texttt{FRESH\_WATER} \\   
    \hline
  \end{tabular}
  \caption{CRTM \Surface{} structure valid \texttt{Water\_Type} definitions.}
  \label{tab:surface_water_type}
\end{table}

% Snow surface type table
\begin{table}[htp]
  \centering
  \begin{tabular}{c l}
    \hline
    \sffamily\textbf{Snow Type} & \sffamily\textbf{Parameter} \\
    \hline\hline
         Wet snow          &   \texttt{WET\_SNOW} \\           
      Grass after snow     &   \texttt{GRASS\_AFTER\_SNOW} \\   
        Powder snow        &   \texttt{POWDER\_SNOW} \\        
         RS snow(A)        &   \texttt{RS\_SNOW\_A} \\          
         RS snow(B)        &   \texttt{RS\_SNOW\_B} \\          
         RS snow(C)        &   \texttt{RS\_SNOW\_C} \\          
         RS snow(D)        &   \texttt{RS\_SNOW\_D} \\          
         RS snow(E)        &   \texttt{RS\_SNOW\_E} \\          
      Thin Crust snow      &   \texttt{THIN\_CRUST\_SNOW} \\    
      Thick crust snow     &   \texttt{THICK\_CRUST\_SNOW } \\  
        Shallow snow       &   \texttt{SHALLOW\_SNOW} \\       
         Deep snow         &   \texttt{DEEP\_SNOW} \\          
        Crust snow         &   \texttt{CRUST\_SNOW} \\         
        Medium snow        &   \texttt{MEDIUM\_SNOW} \\        
     Bottom crust snow(A)  &   \texttt{BOTTOM\_CRUST\_SNOW\_A} \\
     Bottom crust snow(B)  &   \texttt{BOTTOM\_CRUST\_SNOW\_B} \\
    \hline
  \end{tabular}
  \caption{CRTM \Surface{} structure valid \texttt{Snow\_Type} definitions.}
  \label{tab:surface_snow_type}
\end{table}

% Ice surface type table
\begin{table}[htp]
  \centering
  \begin{tabular}{c l}
    \hline
    \sffamily\textbf{Ice Type} & \sffamily\textbf{Parameter} \\
    \hline\hline
            Fresh ice        &   \texttt{FRESH\_ICE} \\       
        First year sea ice   &   \texttt{FIRST\_YEAR\_SEA\_ICE} \\
      Multiple year sea ice  &   \texttt{MULTI\_YEAR\_SEA\_ICE} \\
            Ice floe         &   \texttt{ICE\_FLOE} \\            
            Ice ridge        &   \texttt{ICE\_RIDGE} \\           
    \hline
  \end{tabular}
  \caption{CRTM \Surface{} structure valid \texttt{Ice\_Type} definitions.}
  \label{tab:surface_ice_type}
\end{table}

% Surface structure methods
%--------------------------
\clearpage
\subsection{\texttt{CRTM\_Allocate\_Surface} interface}
  \label{sec:CRTM_Allocate_Surface_interface}
  \begin{alltt}
 
  NAME:
        CRTM_Allocate_Surface
  
  PURPOSE:
        Function to allocate CRTM_Surface data structures.
 
        NOTE: This function is a wrapper for the CRTM_SensorData allocation 
              routine to provide the functionality and convenience for
              allocation of both scalar and rank-1 Surface structures in
              the same manner as for CRTM_Atmosphere_type structures.
 
  CALLING SEQUENCE:
        Error_Status = CRTM_Allocate_Surface( n_Channels             , &
                                              Surface                , &
                                              Message_Log=Message_Log  )
 
  INPUT ARGUMENTS:
        n_Channels:   Number of channels dimension of Surface%SensorData
                      structure
                      ** Note: Can be = 0 (i.e. no sensor data). **
                      UNITS:      N/A
                      TYPE:       INTEGER
                      DIMENSION:  Scalar OR Rank-1
                                  See output Surface dimensionality table
                      ATTRIBUTES: INTENT(IN)
 
  OUTPUT ARGUMENTS:
        Surface:      Surface structure with allocated SensorData pointer
                      members. The following table shows the allowable dimension
                      combinations for the calling routine, where M == number of
                      profiles/surface locations:
 
                         Input       Output
                       n_Channels   Surface
                        dimension   dimension
                      --------------------------
                         scalar      scalar
                         scalar        M
                           M           M
 
                      These multiple interfaces are supplied purely for ease of
                      use depending on what data is available.
                      
                      UNITS:      N/A
                      TYPE:       CRTM_Surface_type
                      DIMENSION:  Scalar or Rank-1
                                  See chart above.
                      ATTRIBUTES: INTENT(IN OUT)
 
 
  OPTIONAL INPUT ARGUMENTS:
        Message_Log:  Character string specifying a filename in which any
                      messages will be logged. If not specified, or if an
                      error occurs opening the log file, the default action
                      is to output messages to standard output.
                      UNITS:      N/A
                      TYPE:       CHARACTER(*)
                      DIMENSION:  Scalar
                      ATTRIBUTES: INTENT(IN), OPTIONAL
 
  FUNCTION RESULT:
        Error_Status: The return value is an integer defining the error status.
                      The error codes are defined in the Message_Handler module.
                      If == SUCCESS the structure re-initialisation was successful
                         == FAILURE - an error occurred, or
                                    - the structure internal allocation counter
                                      is not equal to one (1) upon exiting this
                                      function. This value is incremented and
                                      decremented for every structure allocation
                                      and deallocation respectively.
                      UNITS:      N/A
                      TYPE:       INTEGER
                      DIMENSION:  Scalar
 
  COMMENTS:
        Note the INTENT on the output Surface argument is IN OUT rather than
        just OUT. This is necessary because the argument may be defined upon
        input. To prevent memory leaks, the IN OUT INTENT is a must.
 
  \end{alltt}

\subsection{\texttt{CRTM\_Destroy\_Surface} interface}
  \label{sec:CRTM_Destroy_Surface_interface}
  \begin{alltt}
 
  NAME:
        CRTM_Destroy_Surface
  
  PURPOSE:
        Function to re-initialize the scalar and pointer members of Surface
        data structures.
 
        NOTE: This function is mostly a wrapper for the CRTM_SensorData
              destruction routine to provide the functionality and convenience
              of allocation of both scalar and rank-1 Surface structures in
              the same manner as for CRTM_Atmosphere_type structures.
 
  CALLING SEQUENCE:
        Error_Status = CRTM_Destroy_Surface( Surface                , &
                                             Message_Log=Message_Log  )
 
  OUTPUT ARGUMENTS:
        Surface:      Re-initialized Surface structure. In the context of
                      the CRTM, rank-1 corresponds to an vector of profiles,
                      and rank-2 corresponds to an array of channels x profiles.
                      The latter is used in the K-matrix model.
                      UNITS:      N/A
                      TYPE:       CRTM_Surface_type
                      DIMENSION:  Scalar, Rank-1, OR Rank-2 array
                      ATTRIBUTES: INTENT(IN OUT)
 
  OPTIONAL INPUT ARGUMENTS:
        Message_Log:  Character string specifying a filename in which any
                      messages will be logged. If not specified, or if an
                      error occurs opening the log file, the default action
                      is to output messages to standard output.
                      UNITS:      N/A
                      TYPE:       CHARACTER(*)
                      DIMENSION:  Scalar
                      ATTRIBUTES: INTENT(IN), OPTIONAL
 
  FUNCTION RESULT:
        Error_Status: The return value is an integer defining the error status.
                      The error codes are defined in the Message_Handler module.
                      If == SUCCESS the structure re-initialisation was successful
                         == FAILURE - an error occurred, or
                                    - the structure internal allocation counter
                                      is not equal to zero (0) upon exiting this
                                      function. This value is incremented and
                                      decremented for every structure allocation
                                      and deallocation respectively.
                      UNITS:      N/A
                      TYPE:       INTEGER
                      DIMENSION:  Scalar
 
  COMMENTS:
        Note the INTENT on the output Surface argument is IN OUT rather than
        just OUT. This is necessary because the argument may be defined upon
        input. To prevent memory leaks, the IN OUT INTENT is a must.
 
  \end{alltt}

\subsection{\texttt{CRTM\_Assign\_Surface} interface}
  \label{sec:CRTM_Assign_Surface_interface}
  \begin{alltt}
 
  NAME:
        CRTM_Assign_Surface
 
  PURPOSE:
        Function to copy valid Surface structures.
 
  CALLING SEQUENCE:
        Error_Status = CRTM_Assign_Surface( Surface_in             , &
                                            Surface_out            , &
                                            Message_Log=Message_Log  )
 
  INPUT ARGUMENTS:
        Surface_in:   Surface structure which is to be copied. In the context of
                      the CRTM, rank-1 corresponds to an vector of profiles,
                      and rank-2 corresponds to an array of channels x profiles.
                      The latter is used in the K-matrix model.
                      UNITS:      N/A
                      TYPE:       CRTM_Surface_type
                      DIMENSION:  Scalar, Rank-1, OR Rank-2 array
                      ATTRIBUTES: INTENT(IN)
 
  OUTPUT ARGUMENTS:
        Surface_out:  Copy of the input structure, Surface_in.
                      UNITS:      N/A
                      TYPE:       CRTM_Surface_type
                      DIMENSION:  Same as Surface_in
                      ATTRIBUTES: INTENT(IN OUT)
 
  OPTIONAL INPUT ARGUMENTS:
        Message_Log:  Character string specifying a filename in which any
                      messages will be logged. If not specified, or if an
                      error occurs opening the log file, the default action
                      is to output messages to standard output.
                      UNITS:      N/A
                      TYPE:       CHARACTER(*)
                      DIMENSION:  Scalar
                      ATTRIBUTES: INTENT(IN), OPTIONAL
 
  FUNCTION RESULT:
        Error_Status: The return value is an integer defining the error status.
                      The error codes are defined in the Message_Handler module.
                      If == SUCCESS the structure assignment was successful
                         == FAILURE an error occurred
                      UNITS:      N/A
                      TYPE:       INTEGER
                      DIMENSION:  Scalar
 
  COMMENTS:
        Note the INTENT on the output Surface argument is IN OUT rather than
        just OUT. This is necessary because the argument may be defined upon
        input. To prevent memory leaks, the IN OUT INTENT is a must.
 
  \end{alltt}

\subsection{\texttt{CRTM\_Equal\_Surface} interface}
  \label{sec:CRTM_Equal_Surface_interface}
  \begin{alltt}
 
  NAME:
        CRTM_Equal_Surface
 
  PURPOSE:
        Function to test if two Surface structures are equal.
 
  CALLING SEQUENCE:
        Error_Status = CRTM_Equal_Surface( Surface_LHS                          , &
                                           Surface_RHS                          , &
                                           ULP_Scale         =ULP_Scale         , &
                                           Percent_Difference=Percent_Difference, &
                                           Check_All         =Check_All         , &
                                           Message_Log       =Message_Log         )
 
 
  INPUT ARGUMENTS:
        Surface_LHS:        Surface structure to be compared; equivalent to the
                            left-hand side of a lexical comparison, e.g.
                              IF ( Surface_LHS == Surface_RHS ).
                            UNITS:      N/A
                            TYPE:       CRTM_Surface_type
                            DIMENSION:  Scalar
                            ATTRIBUTES: INTENT(IN)
 
        Surface_RHS:        Surface structure to be compared to; equivalent to
                            right-hand side of a lexical comparison, e.g.
                              IF ( Surface_LHS == Surface_RHS ).
                            UNITS:      N/A
                            TYPE:       CRTM_Surface_type
                            DIMENSION:  Scalar
                            ATTRIBUTES: INTENT(IN)
 
  OPTIONAL INPUT ARGUMENTS:
        ULP_Scale:          Unit of data precision used to scale the floating
                            point comparison. ULP stands for "Unit in the Last Place,"
                            the smallest possible increment or decrement that can be
                            made using a machine's floating point arithmetic.
                            Value must be positive - if a negative value is supplied,
                            the absolute value is used. If not specified, the default
                            value is 1.
                            UNITS:      N/A
                            TYPE:       INTEGER
                            DIMENSION:  Scalar
                            ATTRIBUTES: INTENT(IN), OPTIONAL
 
        Percent_Differnece: Percentage difference value to use in comparing
                            the numbers rather than testing within some numerical
                            limit. The ULP_Scale argument is ignored if this argument is
                            specified.
                            UNITS:      N/A
                            TYPE:       REAL(fp)
                            DIMENSION:  Scalar
                            ATTRIBUTES: OPTIONAL, INTENT(IN)
 
        Check_All:          Set this argument to check ALL the floating point
                            channel data of the Surface structures. The default
                            action is return with a FAILURE status as soon as
                            any difference is found. This optional argument can
                            be used to get a listing of ALL the differences
                            between data in Surface structures.
                            If == 0, Return with FAILURE status as soon as
                                     ANY difference is found  *DEFAULT*
                               == 1, Set FAILURE status if ANY difference is
                                     found, but continue to check ALL data.
                            UNITS:      N/A
                            TYPE:       INTEGER
                            DIMENSION:  Scalar
                            ATTRIBUTES: INTENT(IN), OPTIONAL
 
        Message_Log:        Character string specifying a filename in which any
                            messages will be logged. If not specified, or if an
                            error occurs opening the log file, the default action
                            is to output messages to standard output.
                            UNITS:      None
                            TYPE:       CHARACTER(*)
                            DIMENSION:  Scalar
                            ATTRIBUTES: INTENT(IN), OPTIONAL
 
  FUNCTION RESULT:
        Error_Status:       The return value is an integer defining the error status.
                            The error codes are defined in the Message_Handler module.
                            If == SUCCESS the structures were equal
                               == FAILURE - an error occurred, or
                                          - the structures were different.
                            UNITS:      N/A
                            TYPE:       INTEGER
                            DIMENSION:  Scalar
 
  \end{alltt}

\subsection{\texttt{CRTM\_Sum\_Surface} interface}
  \label{sec:CRTM_Sum_Surface_interface}
  \begin{alltt}
 
  NAME:
        CRTM_Sum_Surface
 
  PURPOSE:
        Function to perform a sum of two valid CRTM Surface structures. The
        summation performed is:
          A = A + Scale_Factor*B + Offset
        where A and B are the CRTM Surface structures, and Scale_Factor and
        Offset are optional weighting factors.
 
  CALLING SEQUENCE:
        Error_Status = CRTM_Sum_Surface( A                        , &
                                         B                        , &
                                         Scale_Factor=Scale_Factor, &
                                         Offset      =Offset      , &
                                         Message_Log =Message_Log   )
 
  INPUT ARGUMENTS:
        A:               Surface structure that is to be added to.
                         In the context of the CRTM, rank-1 corresponds to an
                         vector of profiles, and rank-2 corresponds to an array
                         of channels x profiles. The latter is used in the
                         K-matrix model.
                         UNITS:      N/A
                         TYPE:       CRTM_Surface_type
                         DIMENSION:  Scalar, Rank-1, or Rank-2 array
                         ATTRIBUTES: INTENT(IN OUT)
 
        B:               Surface structure that is to be weighted and
                         added to structure A.
                         UNITS:      N/A
                         TYPE:       CRTM_Surface_type
                         DIMENSION:  Same as A
                         ATTRIBUTES: INTENT(IN)
 
  OUTPUT ARGUMENTS:
        A:               Structure containing the weight sum result,
                           A = A + Scale_Factor*B + Offset
                         UNITS:      N/A
                         TYPE:       CRTM_Surface_type
                         DIMENSION:  Same as B
                         ATTRIBUTES: INTENT(IN OUT)
 
 
  OPTIONAL INPUT ARGUMENTS:
        Scale_Factor:    The first weighting factor used to scale the
                         contents of the input structure, B.
                         If not specified, Scale_Factor = 1.0.
                         UNITS:      N/A
                         TYPE:       REAL(fp)
                         DIMENSION:  Scalar
                         ATTRIBUTES: INTENT(IN), OPTIONAL
 
        Offset:          The second weighting factor used to offset the
                         sum of the input structures.
                         If not specified, Offset = 0.0.
                         UNITS:      N/A
                         TYPE:       REAL(fp)
                         DIMENSION:  Scalar
                         ATTRIBUTES: INTENT(IN), OPTIONAL
 
        Message_Log:     Character string specifying a filename in which any
                         messages will be logged. If not specified, or if an
                         error occurs opening the log file, the default action
                         is to output messages to standard output.
                         UNITS:      N/A
                         TYPE:       CHARACTER(*)
                         DIMENSION:  Scalar
                         ATTRIBUTES: INTENT(IN), OPTIONAL
 
  FUNCTION RESULT:
        Error_Status:    The return value is an integer defining the error status.
                         The error codes are defined in the Message_Handler module.
                         If == SUCCESS the structure assignment was successful
                            == FAILURE an error occurred
                         UNITS:      N/A
                         TYPE:       INTEGER
                         DIMENSION:  Scalar
 
  SIDE EFFECTS:
        The argument A is INTENT(IN OUT) and is modified upon output.
 
  COMMENTS:
        The SensorData component of the Surface structures are not operated on.
 
  \end{alltt}

\subsection{\texttt{CRTM\_Zero\_Surface} interface}
  \label{sec:CRTM_Zero_Surface_interface}
  \begin{alltt}
 
  NAME:
        CRTM_Zero_Surface
  
  PURPOSE:
        Subroutine to zero-out members of a CRTM_Surface structure.
 
  CALLING SEQUENCE:
        CALL CRTM_Zero_Surface( Surface )
 
  OUTPUT ARGUMENTS:
        Surface:      Zeroed out Surface structure.
                      In the context of the CRTM, rank-1 corresponds to an
                      vector of profiles, and rank-2 corresponds to an array
                      of channels x profiles. The latter is used in the
                      K-matrix model.
                      UNITS:      N/A
                      TYPE:       CRTM_Surface_type
                      DIMENSION:  Scalar, Rank-1, or Rank-2 array
                      ATTRIBUTES: INTENT(IN OUT)
 
  COMMENTS:
        - No checking of the input structure is performed.
 
        - The SensorData dimension and structure components are *NOT*
          reset.
 
        - Note the INTENT on the output Surface argument is IN OUT rather than
          just OUT. This is necessary because the argument must be defined upon
          input.
 
  \end{alltt}

\subsection{\texttt{CRTM\_RCS\_ID\_Surface} interface}
  \label{sec:CRTM_RCS_ID_Surface_interface}
  \begin{alltt}
 
  NAME:
        CRTM_RCS_ID_Surface
 
  PURPOSE:
        Subroutine to return the module RCS Id information.
 
  CALLING SEQUENCE:
        CALL CRTM_RCS_Id_Surface( RCS_Id )
 
  OUTPUT ARGUMENTS:
        RCS_Id:        Character string containing the Revision Control
                       System Id field for the module.
                       UNITS:      N/A
                       TYPE:       CHARACTER(*)
                       DIMENSION:  Scalar
                       ATTRIBUTES: INTENT(OUT)
 
  \end{alltt}

\subsection{\texttt{CRTM\_Inquire\_Surface\_Binary} interface}
  \label{sec:CRTM_Inquire_Surface_Binary_interface}
  \begin{alltt}
 
  NAME:
        CRTM_Inquire_Surface_Binary
 
  PURPOSE:
        Function to inquire Binary format CRTM Surface structure files.
 
  CALLING SEQUENCE:
        Error_Status = CRTM_Inquire_Surface_Binary( Filename               , &
                                                    n_Channels =n_Channels , &
                                                    n_Profiles =n_Profiles , &
                                                    RCS_Id     =RCS_Id     , &
                                                    Message_Log=Message_Log  )
 
  INPUT ARGUMENTS:
        Filename:     Character string specifying the name of an
                      Surface format data file to read.
                      UNITS:      N/A
                      TYPE:       CHARACTER(*)
                      DIMENSION:  Scalar
                      ATTRIBUTES: INTENT(IN)
 
  OPTIONAL INPUT ARGUMENTS:
        Message_Log:  Character string specifying a filename in which any
                      messages will be logged. If not specified, or if an
                      error occurs opening the log file, the default action
                      is to output messages to standard output.
                      UNITS:      N/A
                      TYPE:       CHARACTER(*)
                      DIMENSION:  Scalar
                      ATTRIBUTES: INTENT(IN), OPTIONAL
 
  OPTIONAL OUTPUT ARGUMENTS:
        n_Channels:   The number of spectral channels for which there is
                      data in the file. Note that this value will always
                      be 0 for a profile-only dataset-- it only has meaning
                      for K-matrix data.
                      UNITS:      N/A
                      TYPE:       INTEGER
                      DIMENSION:  Scalar
                      ATTRIBUTES: OPTIONAL, INTENT(OUT)
 
        n_Profiles:   The number of profiles in the data file.
                      UNITS:      N/A
                      TYPE:       INTEGER
                      DIMENSION:  Scalar
                      ATTRIBUTES: OPTIONAL, INTENT(OUT)
 
        RCS_Id:       Character string containing the version control Id
                      field for the module.
                      UNITS:      N/A
                      TYPE:       CHARACTER(*)
                      DIMENSION:  Scalar
                      ATTRIBUTES: OPTIONAL, INTENT(OUT)
 
  FUNCTION RESULT:
        Error_Status: The return value is an integer defining the error status.
                      The error codes are defined in the Message_Handler module.
                      If == SUCCESS the Binary inquiry was successful
                         == FAILURE an unrecoverable error occurred.
                      UNITS:      N/A
                      TYPE:       INTEGER
                      DIMENSION:  Scalar
 
  \end{alltt}

\subsection{\texttt{CRTM\_Read\_Surface\_Binary} interface}
  \label{sec:CRTM_Read_Surface_Binary_interface}
  \begin{alltt}
 
  NAME:
        CRTM_Read_Surface_Binary
 
  PURPOSE:
        Function to read Binary format CRTM Surface structure files.
 
  CALLING SEQUENCE:
        Error_Status = CRTM_Read_Surface_Binary( Filename               , &
                                                 Surface                , &
                                                 Quiet      =Quiet      , &
                                                 n_Channels =n_Channels , &
                                                 n_Profiles =n_Profiles , &
                                                 RCS_Id     =RCS_Id     , &
                                                 Message_Log=Message_Log  )
 
  INPUT ARGUMENTS:
        Filename:     Character string specifying the name of an
                      Surface format data file to read.
                      UNITS:      N/A
                      TYPE:       CHARACTER(*)
                      DIMENSION:  Scalar
                      ATTRIBUTES: INTENT(IN)
 
  OUTPUT ARGUMENTS:
        Surface:      Structure containing the Surface data. Note the
                      following meanings attributed to the dimensions of
                      the structure array:
                      Rank-1: M profiles.
                              Only profile data are to be read in. The file
                              does not contain channel information. The
                              dimension of the structure is understood to
                              be the PROFILE dimension.
                      Rank-2: L channels  x  M profiles
                              Channel and profile data are to be read in.
                              The file contains both channel and profile
                              information. The first dimension of the 
                              structure is the CHANNEL dimension, the second
                              is the PROFILE dimension. This is to allow
                              K-matrix structures to be read in with the
                              same function.
                      UNITS:      N/A
                      TYPE:       CRTM_Surface_type
                      DIMENSION:  Rank-1 (M) or Rank-2 (L x M)
                      ATTRIBUTES: INTENT(IN OUT)
 
  OPTIONAL INPUT ARGUMENTS:
        Quiet:        Set this argument to suppress INFORMATION messages
                      being printed to standard output (or the message
                      log file if the Message_Log optional argument is
                      used.) By default, INFORMATION messages are printed.
                      If QUIET = 0, INFORMATION messages are OUTPUT.
                         QUIET = 1, INFORMATION messages are SUPPRESSED.
                      UNITS:      N/A
                      TYPE:       INTEGER
                      DIMENSION:  Scalar
                      ATTRIBUTES: INTENT(IN), OPTIONAL
 
        Message_Log:  Character string specifying a filename in which any
                      messages will be logged. If not specified, or if an
                      error occurs opening the log file, the default action
                      is to output messages to standard output.
                      UNITS:      N/A
                      TYPE:       CHARACTER(*)
                      DIMENSION:  Scalar
                      ATTRIBUTES: INTENT(IN), OPTIONAL
 
  OPTIONAL OUTPUT ARGUMENTS:
        n_Channels:   The number of channels for which data was read. Note that
                      this value will always be 0 for a profile-only dataset--
                      it only has meaning for K-matrix data.
                      UNITS:      N/A
                      TYPE:       INTEGER
                      DIMENSION:  Scalar
                      ATTRIBUTES: OPTIONAL, INTENT(OUT)
 
        n_Profiles:   The number of profiles for which data was read.
                      UNITS:      N/A
                      TYPE:       INTEGER
                      DIMENSION:  Scalar
                      ATTRIBUTES: OPTIONAL, INTENT(OUT)
 
        RCS_Id:       Character string containing the version control Id
                      field for the module.
                      UNITS:      N/A
                      TYPE:       CHARACTER(*)
                      DIMENSION:  Scalar
                      ATTRIBUTES: OPTIONAL, INTENT(OUT)
 
  FUNCTION RESULT:
        Error_Status: The return value is an integer defining the error status.
                      The error codes are defined in the Message_Handler module.
                      If == SUCCESS the Binary file read was successful
                         == FAILURE an unrecoverable error occurred.
                      UNITS:      N/A
                      TYPE:       INTEGER
                      DIMENSION:  Scalar
 
  COMMENTS:
        Note the INTENT on the output Surface argument is IN OUT rather than
        just OUT. This is necessary because the argument may be defined upon
        input. To prevent memory leaks, the IN OUT INTENT is a must.
 
  \end{alltt}

\subsection{\texttt{CRTM\_Write\_Surface\_Binary} interface}
  \label{sec:CRTM_Write_Surface_Binary_interface}
  \begin{alltt}
 
  NAME:
        CRTM_Write_Surface_Binary
 
  PURPOSE:
        Function to write Binary format Surface files.
 
  CALLING SEQUENCE:
        Error_Status = CRTM_Write_Surface_Binary( Filename               , &
                                                  Surface                , &
                                                  Quiet      =Quiet      , &
                                                  RCS_Id     =RCS_Id     , &
                                                  Message_Log=Message_Log  )
 
  INPUT ARGUMENTS:
        Filename:     Character string specifying the name of an output
                      Surface format data file.
                      UNITS:      N/A
                      TYPE:       CHARACTER(*)
                      DIMENSION:  Scalar
                      ATTRIBUTES: INTENT(IN)
 
        Surface:      Structure containing the Surface data to write.
                      Note the following meanings attributed to the
                      dimensions of the structure array:
                      Rank-1: M profiles.
                              Only profile data are to be read in. The file
                              does not contain channel information. The
                              dimension of the structure is understood to
                              be the PROFILE dimension.
                      Rank-2: L channels  x  M profiles
                              Channel and profile data are to be read in.
                              The file contains both channel and profile
                              information. The first dimension of the 
                              structure is the CHANNEL dimension, the second
                              is the PROFILE dimension. This is to allow
                              K-matrix structures to be read in with the
                              same function.
                      UNITS:      N/A
                      TYPE:       CRTM_Surface_type
                      DIMENSION:  Rank-1 (M) or Rank-2 (L x M)
                      ATTRIBUTES: INTENT(IN)
 
  OPTIONAL INPUT ARGUMENTS:
        Quiet:        Set this argument to suppress INFORMATION messages
                      being printed to standard output (or the message
                      log file if the Message_Log optional argument is
                      used.) By default, INFORMATION messages are printed.
                      If QUIET = 0, INFORMATION messages are OUTPUT.
                         QUIET = 1, INFORMATION messages are SUPPRESSED.
                      UNITS:      N/A
                      TYPE:       INTEGER
                      DIMENSION:  Scalar
                      ATTRIBUTES: INTENT(IN), OPTIONAL
 
        Message_Log:  Character string specifying a filename in which any
                      messages will be logged. If not specified, or if an
                      error occurs opening the log file, the default action
                      is to output messages to standard output.
                      UNITS:      N/A
                      TYPE:       CHARACTER(*)
                      DIMENSION:  Scalar
                      ATTRIBUTES: INTENT(IN), OPTIONAL
 
  OPTIONAL OUTPUT ARGUMENTS:
        RCS_Id:       Character string containing the version control Id
                      field for the module.
                      UNITS:      N/A
                      TYPE:       CHARACTER(*)
                      DIMENSION:  Scalar
                      ATTRIBUTES: OPTIONAL, INTENT(OUT)
 
  FUNCTION RESULT:
        Error_Status: The return value is an integer defining the error status.
                      The error codes are defined in the Message_Handler module.
                      If == SUCCESS the Binary file write was successful
                         == FAILURE an unrecoverable error occurred.
                      UNITS:      N/A
                      TYPE:       INTEGER
                      DIMENSION:  Scalar
 
  SIDE EFFECTS:
        - If the output file already exists, it is overwritten.
        - If an error occurs *during* the write phase, the output file is deleted
          before returning to the calling routine.
 
  \end{alltt}




\clearpage
\subsection{\SensorData{} Structure}
%-----------------------------------
\label{sec:sensordata_structure}

\begin{figure}[htp]
  \centering
  \doublebox{
  \begin{minipage}[b]{6.5in}
    \begin{alltt}
  TYPE :: CRTM_SensorData_type
    ! Allocation indicator
    LOGICAL :: Is_Allocated = .FALSE.
    ! Dimension values
    INTEGER :: n_Channels = 0  ! L
    ! The data sensor IDs
    CHARACTER(STRLEN) :: Sensor_Id        = ' '
    INTEGER           :: WMO_Satellite_ID = INVALID_WMO_SATELLITE_ID
    INTEGER           :: WMO_Sensor_ID    = INVALID_WMO_SENSOR_ID
    ! The sensor channels and brightness temperatures
    INTEGER , ALLOCATABLE :: Sensor_Channel(:)   ! L
    REAL(fp), ALLOCATABLE :: Tb(:)               ! L
  END TYPE CRTM_SensorData_type
    \end{alltt}
  \end{minipage}
  }
  \caption{CRTM\_SensorData\_type structure definition.}
  \label{fig:CRTM_SensorData_type_structure}
\end{figure}


% SensorData component description table
\begin{table}[htp]
  \centering
  \begin{tabular}{l p{7cm} c c}
    \hline
    \sffamily\textbf{Component} & \sffamily\textbf{Description} & \sffamily\textbf{Units} & \sffamily\textbf{Dimensions} \\
    \hline\hline
    \texttt{n\_Channels} & Number of channels to use in SfcOptics emissivty algorithms (\texttt{L}) & N/A & Scalar \\
    \texttt{Select\_WMO\_Sensor\_Id} & The WMO Sensor Id value of the sensor for which the data is to be used. & N/A & Scalar \\
    \texttt{Sensor\_Id} & The sensor id character string for each channel of data & N/A & \texttt{L} \\
    \texttt{WMO\_Satellite\_Id} & The WMO satellite Id for each channel of data & N/A & \texttt{L} \\
    \texttt{WMO\_Sensor\_Id} & The WMO sensor Id for each channel of data & N/A & \texttt{L} \\
    \texttt{Sensor\_Channel} & The channel number for each channel of data & N/A & \texttt{L} \\
    \texttt{Tb} & The brightness temperature measurements for each channel & Kelvin & \texttt{L} \\
    \hline
  \end{tabular}
  \caption{CRTM \SensorData{} structure component description.}
  \label{tab:sensordata_structure}
\end{table}

% Surface structure methods
%--------------------------
\clearpage
\subsection{\texttt{CRTM\_Associated\_SensorData} interface}
  \label{sec:CRTM_Associated_SensorData_interface}
  \begin{alltt}
 
  NAME:
        CRTM_Associated_SensorData
 
  PURPOSE:
        Function to test the association status of the pointer members of a
        CRTM_SensorData structure.
 
  CALLING SEQUENCE:
        Association_Status = CRTM_Associated_SensorData( SensorData       , &
                                                         ANY_Test=Any_Test  )
 
  INPUT ARGUMENTS:
        SensorData:  SensorData structure which is to have its pointer
                     member's association status tested.
                     UNITS:      N/A
                     TYPE:       CRTM_SensorData_type
                     DIMENSION:  Scalar
                     ATTRIBUTES: INTENT(IN)
 
  OPTIONAL INPUT ARGUMENTS:
        ANY_Test:    Set this argument to test if ANY of the
                     SensorData structure pointer members are associated.
                     The default is to test if ALL the pointer members
                     are associated.
                     If ANY_Test = 0, test if ALL the pointer members
                                      are associated.  (DEFAULT)
                        ANY_Test = 1, test if ANY of the pointer members
                                      are associated.
 
  FUNCTION RESULT:
        Association_Status:  The return value is a logical value indicating the
                             association status of the SensorData pointer
                             members.
                             .TRUE.  - if ALL the SensorData pointer members
                                       are associated, or if the ANY_Test argument
                                       is set and ANY of the SensorData
                                       pointer members are associated.
                             .FALSE. - some or all of the SensorData pointer
                                       members are NOT associated.
                             UNITS:      N/A
                             TYPE:       LOGICAL
                             DIMENSION:  Scalar
 
  \end{alltt}

\subsection{\texttt{CRTM\_Allocate\_SensorData} interface}
  \label{sec:CRTM_Allocate_SensorData_interface}
  \begin{alltt}
 
  NAME:
        CRTM_Allocate_SensorData
  
  PURPOSE:
        Function to allocate the pointer members of a CRTM SensorData
        data structure.
 
  CALLING SEQUENCE:
        Error_Status = CRTM_Allocate_SensorData( n_Channels             , &
                                                 SensorData             , &
                                                 Message_Log=Message_Log  )
 
  INPUT ARGUMENTS:
        n_Channels:   The number of channels in the SensorData structure.
                      Must be > 0.
                      UNITS:      N/A
                      TYPE:       INTEGER
                      DIMENSION:  Scalar
                      ATTRIBUTES: INTENT(IN)
 
  OUTPUT ARGUMENTS:
        SensorData:   SensorData structure with allocated pointer members
                      UNITS:      N/A
                      TYPE:       CRTM_SensorData_type
                      DIMENSION:  Scalar
                      ATTRIBUTES: INTENT(IN OUT)
 
 
  OPTIONAL INPUT ARGUMENTS:
        Message_Log:  Character string specifying a filename in which any
                      Messages will be logged. If not specified, or if an
                      error occurs opening the log file, the default action
                      is to output Messages to standard output.
                      UNITS:      N/A
                      TYPE:       CHARACTER(*)
                      DIMENSION:  Scalar
                      ATTRIBUTES: INTENT(IN), OPTIONAL
 
  FUNCTION RESULT:
        Error_Status: The return value is an integer defining the error status.
                      The error codes are defined in the ERROR_HANDLER module.
                      If == SUCCESS the structure pointer allocations were
                                    successful
                         == FAILURE - an error occurred, or
                                    - the structure internal allocation counter
                                      is not equal to one (1) upon exiting this
                                      function. This value is incremented and
                                      decremented for every structure allocation
                                      and deallocation respectively.
                      UNITS:      N/A
                      TYPE:       INTEGER
                      DIMENSION:  Scalar
 
  COMMENTS:
        Note the INTENT on the output SensorData argument is IN OUT rather than
        just OUT. This is necessary because the argument may be defined upon
        input. To prevent memory leaks, the IN OUT INTENT is a must.
 
  \end{alltt}

\subsection{\texttt{CRTM\_Destroy\_SensorData} interface}
  \label{sec:CRTM_Destroy_SensorData_interface}
  \begin{alltt}
 
  NAME:
        CRTM_Destroy_SensorData
  
  PURPOSE:
        Function to re-initialize the scalar and pointer members of SensorData
        data structures.
 
  CALLING SEQUENCE:
        Error_Status = CRTM_Destroy_SensorData( SensorData             , &
                                                Message_Log=Message_Log  )
 
  OUTPUT ARGUMENTS:
        SensorData:   Re-initialized SensorData structure.
                      UNITS:      N/A
                      TYPE:       CRTM_SensorData_type
                      DIMENSION:  Scalar
                      ATTRIBUTES: INTENT(IN OUT)
 
  OPTIONAL INPUT ARGUMENTS:
        Message_Log:  Character string specifying a filename in which any
                      messages will be logged. If not specified, or if an
                      error occurs opening the log file, the default action
                      is to output messages to standard output.
                      UNITS:      N/A
                      TYPE:       CHARACTER(*)
                      DIMENSION:  Scalar
                      ATTRIBUTES: INTENT(IN), OPTIONAL
 
  FUNCTION RESULT:
        Error_Status: The return value is an integer defining the error status.
                      The error codes are defined in the ERROR_HANDLER module.
                      If == SUCCESS the structure re-initialisation was successful
                         == FAILURE - an error occurred, or
                                    - the structure internal allocation counter
                                      is not equal to zero (0) upon exiting this
                                      function. This value is incremented and
                                      decremented for every structure allocation
                                      and deallocation respectively.
                      UNITS:      N/A
                      TYPE:       INTEGER
                      DIMENSION:  Scalar
 
  COMMENTS:
        Note the INTENT on the output SensorData argument is IN OUT rather than
        just OUT. This is necessary because the argument may be defined upon
        input. To prevent memory leaks, the IN OUT INTENT is a must.
 
  \end{alltt}

\subsection{\texttt{CRTM\_Assign\_SensorData} interface}
  \label{sec:CRTM_Assign_SensorData_interface}
  \begin{alltt}
 
  NAME:
        CRTM_Assign_SensorData
 
  PURPOSE:
        Function to copy valid SensorData structures.
 
  CALLING SEQUENCE:
        Error_Status = CRTM_Assign_SensorData( SensorData_in          , &
                                               SensorData_out         , &
                                               Message_Log=Message_Log  )
 
  INPUT ARGUMENTS:
        SensorData_in:   SensorData structure which is to be copied.
                         UNITS:      N/A
                         TYPE:       CRTM_SensorData_type
                         DIMENSION:  Scalar OR Rank-1
                         ATTRIBUTES: INTENT(IN)
 
  OUTPUT ARGUMENTS:
        SensorData_out:  Copy of the input structure, SensorData_in.
                         UNITS:      N/A
                         TYPE:       CRTM_SensorData_type
                         DIMENSION:  Same as SensorData_in
                         ATTRIBUTES: INTENT(IN OUT)
 
  OPTIONAL INPUT ARGUMENTS:
        Message_Log:     Character string specifying a filename in which any
                         messages will be logged. If not specified, or if an
                         error occurs opening the log file, the default action
                         is to output messages to standard output.
                         UNITS:      N/A
                         TYPE:       CHARACTER(*)
                         DIMENSION:  Scalar
                         ATTRIBUTES: INTENT(IN), OPTIONAL
 
  FUNCTION RESULT:
        Error_Status:    The return value is an integer defining the error status.
                         The error codes are defined in the ERROR_HANDLER module.
                         If == SUCCESS the structure assignment was successful
                            == FAILURE an error occurred
                         UNITS:      N/A
                         TYPE:       INTEGER
                         DIMENSION:  Scalar
 
  COMMENTS:
        Note the INTENT on the output SensorData argument is IN OUT rather than
        just OUT. This is necessary because the argument may be defined upon
        input. To prevent memory leaks, the IN OUT INTENT is a must.
 
  \end{alltt}

\subsection{\texttt{CRTM\_Equal\_SensorData} interface}
  \label{sec:CRTM_Equal_SensorData_interface}
  \begin{alltt}
 
  NAME:
        CRTM_Equal_SensorData
 
  PURPOSE:
        Function to test if two SensorData structures are equal.
 
  CALLING SEQUENCE:
        Error_Status = CRTM_Equal_SensorData( SensorData_LHS         , &
                                              SensorData_RHS         , &
                                              ULP_Scale  =ULP_Scale  , &
                                              Check_All  =Check_All  , &
                                              Message_Log=Message_Log  )
 
 
  INPUT ARGUMENTS:
        SensorData_LHS:    SensorData structure to be compared; equivalent to the
                           left-hand side of a lexical comparison, e.g.
                             IF ( SensorData_LHS == SensorData_RHS ).
                           UNITS:      N/A
                           TYPE:       CRTM_SensorData_type
                           DIMENSION:  Scalar
                           ATTRIBUTES: INTENT(IN)
 
        SensorData_RHS:    SensorData structure to be compared to; equivalent to
                           right-hand side of a lexical comparison, e.g.
                             IF ( SensorData_LHS == SensorData_RHS ).
                           UNITS:      N/A
                           TYPE:       CRTM_SensorData_type
                           DIMENSION:  Scalar
                           ATTRIBUTES: INTENT(IN)
 
  OPTIONAL INPUT ARGUMENTS:
        ULP_Scale:         Unit of data precision used to scale the floating
                           point comparison. ULP stands for "Unit in the Last Place,"
                           the smallest possible increment or decrement that can be
                           made using a machine's floating point arithmetic.
                           Value must be positive - if a negative value is supplied,
                           the absolute value is used. If not specified, the default
                           value is 1.
                           UNITS:      N/A
                           TYPE:       INTEGER
                           DIMENSION:  Scalar
                           ATTRIBUTES: INTENT(IN), OPTIONAL
 
        Check_All:         Set this argument to check ALL the floating point
                           channel data of the SensorData structures. The default
                           action is return with a FAILURE status as soon as
                           any difference is found. This optional argument can
                           be used to get a listing of ALL the differences
                           between data in SensorData structures.
                           If == 0, Return with FAILURE status as soon as
                                    ANY difference is found  *DEFAULT*
                              == 1, Set FAILURE status if ANY difference is
                                    found, but continue to check ALL data.
                           UNITS:      N/A
                           TYPE:       INTEGER
                           DIMENSION:  Scalar
                           ATTRIBUTES: INTENT(IN), OPTIONAL
 
        Message_Log:       Character string specifying a filename in which any
                           messages will be logged. If not specified, or if an
                           error occurs opening the log file, the default action
                           is to output messages to standard output.
                           UNITS:      None
                           TYPE:       CHARACTER(*)
                           DIMENSION:  Scalar
                           ATTRIBUTES: INTENT(IN), OPTIONAL
 
  FUNCTION RESULT:
        Error_Status:      The return value is an integer defining the error status.
                           The error codes are defined in the Message_Handler module.
                           If == SUCCESS the structures were equal
                              == FAILURE - an error occurred, or
                                         - the structures were different.
                           UNITS:      N/A
                           TYPE:       INTEGER
                           DIMENSION:  Scalar
 
  \end{alltt}

\subsection{\texttt{CRTM\_RCS\_ID\_SensorData} interface}
  \label{sec:CRTM_RCS_ID_SensorData_interface}
  \begin{alltt}
 
  NAME:
        CRTM_RCS_ID_SensorData
 
  PURPOSE:
        Subroutine to return the module RCS Id information.
 
  CALLING SEQUENCE:
        CALL CRTM_RCS_Id_SensorData( RCS_Id )
 
  OUTPUT ARGUMENTS:
        RCS_Id:        Character string containing the Revision Control
                       System Id field for the module.
                       UNITS:      N/A
                       TYPE:       CHARACTER(*)
                       DIMENSION:  Scalar
                       ATTRIBUTES: INTENT(OUT)
 
  \end{alltt}



\clearpage
\section{\GeometryInfo{} Structure}
%==================================
\label{sec:geometryinfo_structure}

\begin{figure}[htp]
  \centering
  \doublebox{
  \begin{minipage}[b]{6.5in}
    \begin{alltt}
  TYPE :: CRTM_GeometryInfo_type
    ! User Input
    ! ----------
    ! Earth location
    REAL(fp) :: Longitude        = ZERO
    REAL(fp) :: Latitude         = ZERO
    REAL(fp) :: Surface_Altitude = ZERO
    ! Field of view index (1-nFOV)
    INTEGER  :: iFOV = 0
    ! Sensor angle information
    REAL(fp) :: Sensor_Scan_Angle    = ZERO
    REAL(fp) :: Sensor_Zenith_Angle  = ZERO
    REAL(fp) :: Sensor_Azimuth_Angle = ZERO 
    ! Source angle information
    REAL(fp) :: Source_Zenith_Angle  = 100.0_fp  ! Below horizon
    REAL(fp) :: Source_Azimuth_Angle = ZERO
    ! Flux angle information
    REAL(fp) :: Flux_Zenith_Angle = DIFFUSIVITY_ANGLE
    ! Derived from User Input
    ! -----------------------
    ! Default distance ratio
    REAL(fp) :: Distance_Ratio = EARTH_RADIUS/(EARTH_RADIUS + SATELLITE_HEIGHT)
    ! Sensor angle information
    REAL(fp) :: Sensor_Scan_Radian    = ZERO
    REAL(fp) :: Sensor_Zenith_Radian  = ZERO
    REAL(fp) :: Sensor_Azimuth_Radian = ZERO
    REAL(fp) :: Secant_Sensor_Zenith  = ZERO
    ! Source angle information
    REAL(fp) :: Source_Zenith_Radian  = ZERO
    REAL(fp) :: Source_Azimuth_Radian = ZERO
    REAL(fp) :: Secant_Source_Zenith  = ZERO
    ! Flux angle information
    REAL(fp) :: Flux_Zenith_Radian = DIFFUSIVITY_RADIAN
    REAL(fp) :: Secant_Flux_Zenith = SECANT_DIFFUSIVITY
  END TYPE CRTM_GeometryInfo_type
    \end{alltt}
  \end{minipage}
  }
  \caption{CRTM\_GeometryInfo\_type structure definition.}
  \label{fig:CRTM_GeometryInfo_type_structure}
\end{figure}


% GeometryInfo component description table
\begin{table}[htp]
  \centering
  \begin{tabular}{l p{7cm} c c}
    \hline
    \sffamily\textbf{Component} & \sffamily\textbf{Description} & \sffamily\textbf{Units} & \sffamily\textbf{Dimensions} \\
    \hline\hline
    \texttt{Longitude}               & Earth longitude & deg. E (0$\rightarrow$360) & Scalar \\
    \texttt{Latitude}                & Earth latitude  & deg. N (-90$\rightarrow$+90) & Scalar \\
    \texttt{Surface\_Altitude}       & Altitude of the Earth's surface at the specified lon/lat location & metres (m) & Scalar \\
    \texttt{iFOV}                    & The scan line FOV index & N/A & Scalar \\
    \texttt{Sensor\_Scan\_Angle}     & The sensor scan angle from nadir. See fig.\ref{fig:gInfo_sensor_scan_angle} & degrees & Scalar \\
    \texttt{Sensor\_Zenith\_Angle}   & The sensor zenith angle of the FOV. See fig.\ref{fig:gInfo_sensor_zenith_angle} & degrees & Scalar \\
    \texttt{Sensor\_Azimuth\_Angle}  & The sensor azimuth angle is the angle subtended by the horizontal projection of a direct line from the satellite to the FOV and the North-South axis measured clockwise from North. See fig.\ref{fig:gInfo_sensor_azimuth_angle} & deg. from N & Scalar \\
    \texttt{Source\_Zenith\_Angle}   & The source zenith angle. The source is typically the Sun (IR/VIS) or Moon (MW/VIS) [only solar source valid in current release] See fig.\ref{fig:gInfo_source_zenith_angle} & degrees & Scalar \\
    \texttt{Source\_Azimuth\_Angle}  & The source azimuth angle is the angle subtended by the horizontal projection of a direct line from the source to the FOV and the North-South axis measured clockwise from North. See fig.\ref{fig:gInfo_source_azimuth_angle} & deg. from N & Scalar \\
    \texttt{Flux\_Zenith\_Angle}     & The zenith angle used to approximate downwelling flux transmissivity. If not set, the default value is that of the diffusivity approximation, such that $\sec(F) = 5/3$. Maximum allowed value is determined from $\sec(F) = 9/4$ & degrees & Scalar \\
    \hline
  \end{tabular}
  \caption{Description of CRTM \GeometryInfo{} structure components defined by the user.}
  \label{tab:user_defined_geometryinfo_structure}
\end{table}

\begin{table}[htp]
  \centering
  \begin{tabular}{l p{7cm} c c}
    \hline
    \sffamily\textbf{Component} & \sffamily\textbf{Description} & \sffamily\textbf{Units} & \sffamily\textbf{Dimensions} \\
    \hline\hline
    \texttt{Distance\_Ratio}         & The ratio of the radius of the earth at the FOV location to the sum of the radius of the earth at nadir, and the satellite altitude;
    
     \mbox{\hspace{1cm}$r = \displaystyle\frac{R_e(FOV)}{R_e(nadir) + h}$}.
     
     Note that this quantity is actually computed using the user input sensor scan and zenith angles; 
     
     \mbox{\hspace{1cm}$r = \displaystyle\frac{\sin(\theta_{scan})}{\sin(\theta_{zenith})}$} & N/A & Scalar \\
    \texttt{Sensor\_Scan\_Radian}    & The sensor scan angle in radians & radians & Scalar \\
    \texttt{Sensor\_Zenith\_Radian}  & The sensor scan angle in radians & radians & Scalar \\
    \texttt{Sensor\_Azimuth\_Radian} & The sensor azimuth angle in radians & radians & Scalar \\
    \texttt{Secant\_Sensor\_Zenith}  & The secant of the sensor zenith angle & N/A & Scalar \\
    \texttt{Source\_Zenith\_Radian}  & The source zenith angle in radians & radians & Scalar \\
    \texttt{Source\_Azimuth\_Radian} & The source azimuth angle in radians & radians & Scalar \\
    \texttt{Secant\_Source\_Zenith}  & The secant of the source zenith angle & N/A & Scalar \\
    \texttt{Flux\_Zenith\_Radian}    & The flux zenith angle in radians & radians & Scalar \\
    \texttt{Secant\_Flux\_Zenith}    & The secant of the flux zenith angle & N/A & Scalar \\
    \hline
  \end{tabular}
  \caption{Description of CRTM \GeometryInfo{} structure components derived from user inputs.}
  \label{tab:derived_geometryinfo_structure}
\end{table}

% GeometryInfo structure methods
%--------------------------
\clearpage
\subsection{\texttt{CRTM\_Compute\_GeometryInfo} interface}
  \label{sec:CRTM_Compute_GeometryInfo_interface}
  \begin{alltt}
 
  NAME:
        CRTM_Compute_GeometryInfo
  
  PURPOSE:
        Function to compute the derived geometry from the user specified
        components of the CRTM GeometryInfo structure.
 
  CALLING SEQUENCE:
        Error_Status = CRTM_Compute_GeometryInfo( GeometryInfo           , &
                                                  Message_Log=Message_Log  )
 
  INPUT ARGUMENTS:
        GeometryInfo:  The GeometryInfo structure containing the user
                       defined inputs, in particular the angles.
                       UNITS:      N/A
                       TYPE:       CRTM_GeometryInfo_type
                       DIMENSION:  Scalar
                       ATTRIBUTES: INTENT(IN OUT)
 
  OUTPUT ARGUMENTS:
        GeometryInfo:  The GeometryInfo structure with the derived
                       angle components filled..
                       UNITS:      N/A
                       TYPE:       CRTM_GeometryInfo_type
                       DIMENSION:  Scalar
                       ATTRIBUTES: INTENT(IN OUT)
 
  OPTIONAL INPUT ARGUMENTS:
        Message_Log:   Character string specifying a filename in which any
                       messages will be logged. If not specified, or if an
                       error occurs opening the log file, the default action
                       is to output messages to the screen.
                       UNITS:      N/A
                       TYPE:       CHARACTER( * )
                       DIMENSION:  Scalar
                       ATTRIBUTES: INTENT( IN ), OPTIONAL
 
  FUNCTION RESULT:
        Error_Status:   The return value is an integer defining the error status.
                        The error codes are defined in the ERROR_HANDLER module.
                        If == SUCCESS the computation was sucessful
                           == WARNING invalid data was found, but altered to default.
                           == FAILURE invalid data was found
                        UNITS:      N/A
                        TYPE:       INTEGER
                        DIMENSION:  Scalar
 
  SIDE EFFECTS:
        This function changes the values of the derived components of the
        GeometryInfo structure argument.
 
  \end{alltt}

\subsection{\texttt{CRTM\_RCS\_ID\_GeometryInfo} interface}
  \label{sec:CRTM_RCS_ID_GeometryInfo_interface}
  \begin{alltt}
 
  NAME:
        CRTM_RCS_ID_GeometryInfo
 
  PURPOSE:
        Subroutine to return the module RCS Id information.
 
  CALLING SEQUENCE:
        CALL CRTM_RCS_Id_GeometryInfo( RCS_Id )
 
  OUTPUT ARGUMENTS:
        RCS_Id:        Character string containing the Revision Control
                       System Id field for the module.
                       UNITS:      N/A
                       TYPE:       CHARACTER(*)
                       DIMENSION:  Scalar
                       ATTRIBUTES: INTENT(OUT)
 
  \end{alltt}



\clearpage
\section{\RTSolution{} Structure}
%==================================
\label{sec:rtsolution_structure}

\begin{figure}[htp]
  \centering
  \doublebox{
  \begin{minipage}[b]{6.5in}
    \begin{alltt}
  TYPE :: CRTM_RTSolution_type
    ! Allocation indicator
    LOGICAL :: Is_Allocated = .FALSE.
    ! Dimensions
    INTEGER :: n_Layers = 0  ! K
    ! Internal variables. Users do not need to worry about these.
    LOGICAL :: Scattering_Flag = .TRUE.
    INTEGER :: n_Full_Streams  = 0
    INTEGER :: n_Stokes        = 0
    ! Forward radiative transfer intermediate results for a single channel
    !    These components are not defined when they are used as TL, AD
    !    and K variables
    REAL(fp) :: Surface_Emissivity      = ZERO
    REAL(fp) :: Up_Radiance             = ZERO
    REAL(fp) :: Down_Radiance           = ZERO
    REAL(fp) :: Down_Solar_Radiance     = ZERO
    REAL(fp) :: Surface_Planck_Radiance = ZERO
    REAL(fp), ALLOCATABLE :: Upwelling_Radiance(:)   ! K
    ! The layer optical depths
    REAL(fp), ALLOCATABLE :: Layer_Optical_Depth(:)  ! K
    ! Radiative transfer results for a single channel/node
    REAL(fp) :: Radiance               = ZERO
    REAL(fp) :: Brightness_Temperature = ZERO
  END TYPE CRTM_RTSolution_type
    \end{alltt}
  \end{minipage}
  }
  \caption{CRTM\_RTSolution\_type structure definition.}
  \label{fig:CRTM_RTSolution_type_structure}
\end{figure}


\begin{table}[htp]
  \centering
  \begin{tabular}{l p{7cm} c c}
    \hline
    \sffamily\textbf{Component} & \sffamily\textbf{Description} & \sffamily\textbf{Units} & \sffamily\textbf{Dimensions} \\
    \hline\hline
    \texttt{n\_Layers}    & Number of atmospheric profile layers (\texttt{K}) & N/A & Scalar \\
    \texttt{Surface\_Emissivity      }  & The computed surface emissivity & N/A & Scalar \\
    \texttt{Up\_Radiance             }  & The atmospheric portion of the upwelling radiance & \radunit & Scalar \\
    \texttt{Down\_Radiance           }  & The atmospheric portion of the downwelling radiance & \radunit & Scalar \\
    \texttt{Down\_Solar\_Radiance    }  & The downwelling direct solar radiance & \radunit & Scalar \\
    \texttt{Surface\_Planck\_Radiance}  & The surface radiance & \radunit & Scalar \\
    \texttt{Upwelling\_Radiance      }  & The upwelling radiance profile, including the reflected downwelling and surface contributions. & \radunit & \texttt{K} \\
    \texttt{Layer\_Optical\_Depth    }  & The layer optical depth profile & N/A & \texttt{K} \\
    \texttt{Radiance                 }  & The sensor radiance & \radunit & Scalar \\
    \texttt{Brightness\_Temperature  }  & The sensor brightness temperature & Kelvin & Scalar \\
    \hline
  \end{tabular}
  \caption{CRTM \RTSolution{} structure component description}
  \label{tab:rtsolution_structure}
\end{table}

% RTSolution structure methods
%--------------------------
\clearpage
\subsection{\texttt{CRTM\_Associated\_RTSolution} interface}
  \label{sec:CRTM_Associated_RTSolution_interface}
  \begin{alltt}
 
  NAME:
        CRTM_Associated_RTSolution
 
  PURPOSE:
        Function to test the association status of the pointer members of a
        CRTM RTSolution structure.
 
  CALLING SEQUENCE:
        Association_Status = CRTM_Associated_RTSolution( RTSolution       , &
                                                         ANY_Test=Any_Test  )
 
  INPUT ARGUMENTS:
        RTSolution:          RTSolution structure which is to have its pointer
                             member's association status tested.
                             UNITS:      N/A
                             TYPE:       CRTM_RTSolution_type
                             DIMENSION:  Scalar, Rank-1, OR Rank-2 array
                             ATTRIBUTES: INTENT(IN)
 
 
  OPTIONAL INPUT ARGUMENTS:
        ANY_Test:            Set this argument to test if ANY of the
                             RTSolution structure pointer members are associated.
                             The default is to test if ALL the pointer members
                             are associated.
                             If ANY_Test = 0, test if ALL the pointer members
                                              are associated.  (DEFAULT)
                                ANY_Test = 1, test if ANY of the pointer members
                                              are associated.
                             UNITS:      N/A
                             TYPE:       INTEGER
                             DIMENSION:  Scalar
                             ATTRIBUTES: INTENT(IN), OPTIONAL
 
  FUNCTION RESULT:
        Association_Status:  The return value is a logical value indicating the
                             association status of the RTSolution pointer
                             members.
                             .TRUE.  - if ALL the RTSolution pointer members
                                       are associated, or if the ANY_Test argument
                                       is set and ANY of the RTSolution
                                       pointer members are associated.
                             .FALSE. - some or all of the RTSolution pointer
                                       members are NOT associated.
                             UNITS:      N/A
                             TYPE:       LOGICAL
                             DIMENSION:  Same as input RTSolution argument
 
  \end{alltt}

\subsection{\texttt{CRTM\_Allocate\_RTSolution} interface}
  \label{sec:CRTM_Allocate_RTSolution_interface}
  \begin{alltt}
 
  NAME:
        CRTM_Allocate_RTSolution
  
  PURPOSE:
        Function to allocate the pointer members of the CRTM_RTSolution
        data structure.
 
  CALLING SEQUENCE:
        Error_Status = CRTM_Allocate_RTSolution( n_Layers               , &
                                                 RTSolution             , &
                                                 Message_Log=Message_Log  )
 
  INPUT ARGUMENTS:
        n_Layers:     Number of atmospheric layers 
                      Must be > 0
                      UNITS:      N/A
                      TYPE:       INTEGER
                      DIMENSION:  Scalar
                      ATTRIBUTES: INTENT(IN)
 
  OUTPUT ARGUMENTS:
        RTSolution:   CRTM_RTSolution structure with allocated pointer members.
                      Upon allocation, all pointer members are initialized to
                      a value of zero.
 
                      The following chart shows the allowable dimension
                      combinations for the calling routine, where
                        L == number of channels
                        M == number of profiles
 
                         Input           Output
                        n_Layers       RTSolution
                        dimension      dimension
                      -------------------------------
                         scalar         scalar
                         scalar      Rank-1 (L or M)
                         scalar      Rank-2 (L x M)
 
                      These multiple interfaces are supplied purely for ease of
                      use depending on how it's used.
                      UNITS:      N/A
                      TYPE:       CRTM_RTSolution_type
                      DIMENSION:  Scalar, Rank-1, or Rank-2
                      ATTRIBUTES: INTENT(IN OUT)
 
  OPTIONAL INPUT ARGUMENTS:
        Message_Log:  Character string specifying a filename in which any
                      messages will be logged. If not specified, or if an
                      error occurs opening the log file, the default action
                      is to output messages to standard output.
                      UNITS:      N/A
                      TYPE:       CHARACTER(*)
                      DIMENSION:  Scalar
                      ATTRIBUTES: INTENT(IN), OPTIONAL
 
  FUNCTION RESULT:
        Error_Status: The return value is an integer defining the error status.
                      The error codes are defined in the Message_Handler module.
                      If == SUCCESS the structure re-initialisation was successful
                         == FAILURE - an error occurred, or
                                    - the structure internal allocation counter
                                      is not equal to one (1) upon exiting this
                                      function. This value is incremented and
                                      decremented for every structure allocation
                                      and deallocation respectively.
                      UNITS:      N/A
                      TYPE:       INTEGER
                      DIMENSION:  Scalar
 
  COMMENTS:
        Note the INTENT on the output RTSolution argument is IN OUT rather than
        just OUT. This is necessary because the argument may be defined upon
        input. To prevent memory leaks, the IN OUT INTENT is a must.
 
  \end{alltt}

\subsection{\texttt{CRTM\_Destroy\_RTSolution} interface}
  \label{sec:CRTM_Destroy_RTSolution_interface}
  \begin{alltt}
 
  NAME:
        CRTM_Destroy_RTSolution
  
  PURPOSE:
        Function to re-initialize the scalar and pointer members of a CRTM
        RTSolution data structures.
 
  CALLING SEQUENCE:
        Error_Status = CRTM_Destroy_RTSolution( RTSolution             , &
                                                Message_Log=Message_Log  )
  
  OUTPUT ARGUMENTS:
        RTSolution:   Re-initialized RTSolution structure.
                      UNITS:      N/A
                      TYPE:       CRTM_RTSolution_type
                      DIMENSION:  Scalar, Rank-1, or Rank-2
                      ATTRIBUTES: INTENT(IN OUT)
 
  OPTIONAL INPUT ARGUMENTS:
        Message_Log:  Character string specifying a filename in which any
                      Messages will be logged. If not specified, or if an
                      error occurs opening the log file, the default action
                      is to output Messages to standard output.
                      UNITS:      N/A
                      TYPE:       CHARACTER(*)
                      DIMENSION:  Scalar
                      ATTRIBUTES: INTENT(IN), OPTIONAL
 
  FUNCTION RESULT:
        Error_Status: The return value is an integer defining the error status.
                      The error codes are defined in the Message_Handler module.
                      If == SUCCESS the structure re-initialisation was successful
                         == FAILURE - an error occurred, or
                                    - the structure internal allocation counter
                                      is not equal to zero (0) upon exiting this
                                      function. This value is incremented and
                                      decremented for every structure allocation
                                      and deallocation respectively.
                      UNITS:      N/A
                      TYPE:       INTEGER
                      DIMENSION:  Scalar
 
  COMMENTS:
        Note the INTENT on the output RTSolution argument is IN OUT rather than
        just OUT. This is necessary because the argument may be defined upon
        input. To prevent memory leaks, the IN OUT INTENT is a must.
 
  \end{alltt}

\subsection{\texttt{CRTM\_Assign\_RTSolution} interface}
  \label{sec:CRTM_Assign_RTSolution_interface}
  \begin{alltt}
 
  NAME:
        CRTM_Assign_RTSolution
 
  PURPOSE:
        Function to copy valid CRTM_RTSolution structures.
 
  CALLING SEQUENCE:
        Error_Status = CRTM_Assign_RTSolution( RTSolution_in          , &
                                               RTSolution_out         , &
                                               Message_Log=Message_Log  )
 
  INPUT ARGUMENTS:
        RTSolution_in:   CRTM_RTSolution structure which is to be copied.
                         UNITS:      N/A
                         TYPE:       CRTM_RTSolution_type
                         DIMENSION:  Scalar, Rank-1, or Rank-2
                         ATTRIBUTES: INTENT(IN)
 
  OUTPUT ARGUMENTS:
        RTSolution_out:  Copy of the input structure, CRTM_RTSolution_in.
                         UNITS:      N/A
                         TYPE:       CRTM_RTSolution_type
                         DIMENSION:  Same as input RTSolution_in
                         ATTRIBUTES: INTENT(IN OUT)
 
 
  OPTIONAL INPUT ARGUMENTS:
        Message_Log:     Character string specifying a filename in which any
                         messages will be logged. If not specified, or if an
                         error occurs opening the log file, the default action
                         is to output messages to standard output.
                         UNITS:      N/A
                         TYPE:       CHARACTER(*)
                         DIMENSION:  Scalar
                         ATTRIBUTES: INTENT(IN), OPTIONAL
 
  FUNCTION RESULT:
        Error_Status:    The return value is an integer defining the error status.
                         The error codes are defined in the Message_Handler module.
                         If == SUCCESS the structure assignment was successful
                            == FAILURE an error occurred
                         UNITS:      N/A
                         TYPE:       INTEGER
                         DIMENSION:  Scalar
 
  COMMENTS:
        Note the INTENT on the output RTSolution argument is IN OUT rather than
        just OUT. This is necessary because the argument may be defined upon
        input. To prevent memory leaks, the IN OUT INTENT is a must.
 
  \end{alltt}

\subsection{\texttt{CRTM\_Equal\_RTSolution} interface}
  \label{sec:CRTM_Equal_RTSolution_interface}
  \begin{alltt}
 
  NAME:
        CRTM_Equal_RTSolution
 
  PURPOSE:
        Function to test if two RTSolution structures are equal.
 
  CALLING SEQUENCE:
        Error_Status = CRTM_Equal_RTSolution( RTSolution_LHS                       , 
                                              RTSolution_RHS                       , 
                                              ULP_Scale         =ULP_Scale         , 
                                              Percent_Difference=Percent_Difference, 
                                              Check_All         =Check_All         , 
                                              Check_Intermediate=Check_Intermediate, 
                                              Message_Log       =Message_Log         
 
 
  INPUT ARGUMENTS:
        RTSolution_LHS:     RTSolution structure to be compared; equivalent to the
                            left-hand side of a lexical comparison, e.g.
                              IF ( RTSolution_LHS == RTSolution_RHS ).
                            UNITS:      N/A
                            TYPE:       CRTM_RTSolution_type
                            DIMENSION:  Scalar
                            ATTRIBUTES: INTENT(IN)
 
        RTSolution_RHS:     RTSolution structure to be compared to; equivalent to
                            right-hand side of a lexical comparison, e.g.
                              IF ( RTSolution_LHS == RTSolution_RHS ).
                            UNITS:      N/A
                            TYPE:       CRTM_RTSolution_type
                            DIMENSION:  Scalar
                            ATTRIBUTES: INTENT(IN)
 
  OPTIONAL INPUT ARGUMENTS:
        ULP_Scale:          Unit of data precision used to scale the floating
                            point comparison. ULP stands for "Unit in the Last Place,"
                            the smallest possible increment or decrement that can be
                            made using a machine's floating point arithmetic.
                            Value must be positive - if a negative value is supplied,
                            the absolute value is used. If not specified, the default
                            value is 1.
                            UNITS:      N/A
                            TYPE:       INTEGER
                            DIMENSION:  Scalar
                            ATTRIBUTES: INTENT(IN), OPTIONAL
 
        Percent_Differnece: Percentage difference value to use in comparing
                            the numbers rather than testing within some numerical
                            limit. The ULP_Scale argument is ignored if this argument is
                            specified.
                            UNITS:      N/A
                            TYPE:       REAL(fp)
                            DIMENSION:  Scalar
                            ATTRIBUTES: OPTIONAL, INTENT(IN)
 
        Check_All:          Set this argument to check ALL the floating point
                            channel data of the RTSolution structures. The default
                            action is return with a FAILURE status as soon as
                            any difference is found. This optional argument can
                            be used to get a listing of ALL the differences
                            between data in RTSolution structures.
                            If == 0, Return with FAILURE status as soon as
                                     ANY difference is found  *DEFAULT*
                               == 1, Set FAILURE status if ANY difference is
                                     found, but continue to check ALL data.
                            UNITS:      N/A
                            TYPE:       INTEGER
                            DIMENSION:  Scalar
                            ATTRIBUTES: INTENT(IN), OPTIONAL
 
        Check_Intermediate: Set this argument to check the intermediate results held
                            in the RTSolution structure. The default action does NOT
                            check these components.
                            If == 0, Intermediate result components not checked
                                     for equality.  *DEFAULT*
                               == 1, Intermediate result components ARE checked
                                     for equality. Note that this could generate
                                     a lot of comparison failures.
                            UNITS:      N/A
                            TYPE:       INTEGER
                            DIMENSION:  Scalar
                            ATTRIBUTES: INTENT(IN), OPTIONAL
 
        Message_Log:        Character string specifying a filename in which any
                            messages will be logged. If not specified, or if an
                            error occurs opening the log file, the default action
                            is to output messages to standard output.
                            UNITS:      None
                            TYPE:       CHARACTER(*)
                            DIMENSION:  Scalar
                            ATTRIBUTES: INTENT(IN), OPTIONAL
 
  FUNCTION RESULT:
        Error_Status:       The return value is an integer defining the error status.
                            The error codes are defined in the Message_Handler module.
                            If == SUCCESS the structures were equal
                               == FAILURE - an error occurred, or
                                          - the structures were different.
                            UNITS:      N/A
                            TYPE:       INTEGER
                            DIMENSION:  Scalar
 
  \end{alltt}

\subsection{\texttt{CRTM\_RCS\_ID\_RTSolution} interface}
  \label{sec:CRTM_RCS_ID_RTSolution_interface}
  \begin{alltt}
 
  NAME:
        CRTM_RCS_ID_RTSolution
 
  PURPOSE:
        Subroutine to return the module RCS Id information.
 
  CALLING SEQUENCE:
        CALL CRTM_RCS_Id_RTSolution( RCS_Id )
 
  OUTPUT ARGUMENTS:
        RCS_Id:        Character string containing the Revision Control
                       System Id field for the module.
                       UNITS:      N/A
                       TYPE:       CHARACTER(*)
                       DIMENSION:  Scalar
                       ATTRIBUTES: INTENT(OUT)
 
  \end{alltt}

\subsection{\texttt{CRTM\_Inquire\_RTSolution\_Binary} interface}
  \label{sec:CRTM_Inquire_RTSolution_Binary_interface}
  \begin{alltt}
 
  NAME:
        CRTM_Inquire_RTSolution_Binary
 
  PURPOSE:
        Function to inquire Binary format CRTM RTSolution structure files.
 
  CALLING SEQUENCE:
        Error_Status = CRTM_Inquire_RTSolution_Binary( Filename               , &
                                                       n_Channels =n_Channels , &
                                                       n_Profiles =n_Profiles , &
                                                       RCS_Id     =RCS_Id     , &
                                                       Message_Log=Message_Log  )
 
  INPUT ARGUMENTS:
        Filename:     Character string specifying the name of an
                      RTSolution format data file to read.
                      UNITS:      N/A
                      TYPE:       CHARACTER(*)
                      DIMENSION:  Scalar
                      ATTRIBUTES: INTENT(IN)
 
  OPTIONAL INPUT ARGUMENTS:
        Message_Log:  Character string specifying a filename in which any
                      messages will be logged. If not specified, or if an
                      error occurs opening the log file, the default action
                      is to output messages to standard output.
                      UNITS:      N/A
                      TYPE:       CHARACTER(*)
                      DIMENSION:  Scalar
                      ATTRIBUTES: INTENT(IN), OPTIONAL
 
  OPTIONAL OUTPUT ARGUMENTS:
        n_Channels:   The number of spectral channels for which there is
                      data in the file. Note that this value will always
                      be 0 for a profile-only RTSolution dataset-- it only
                      has meaning for K-matrix RTSolution data.
                      UNITS:      N/A
                      TYPE:       INTEGER
                      DIMENSION:  Scalar
                      ATTRIBUTES: OPTIONAL, INTENT(OUT)
 
        n_Profiles:   The number of atmospheric profiles in the data file.
                      UNITS:      N/A
                      TYPE:       INTEGER
                      DIMENSION:  Scalar
                      ATTRIBUTES: OPTIONAL, INTENT(OUT)
 
        RCS_Id:       Character string containing the version control Id
                      field for the module.
                      UNITS:      N/A
                      TYPE:       CHARACTER(*)
                      DIMENSION:  Scalar
                      ATTRIBUTES: OPTIONAL, INTENT(OUT)
 
  FUNCTION RESULT:
        Error_Status: The return value is an integer defining the error status.
                      The error codes are defined in the Message_Handler module.
                      If == SUCCESS the Binary file inquire was successful
                         == FAILURE an unrecoverable error occurred.
                      UNITS:      N/A
                      TYPE:       INTEGER
                      DIMENSION:  Scalar
 
  \end{alltt}

\subsection{\texttt{CRTM\_Read\_RTSolution\_Binary} interface}
  \label{sec:CRTM_Read_RTSolution_Binary_interface}
  \begin{alltt}
 
  NAME:
        CRTM_Read_RTSolution_Binary
 
  PURPOSE:
        Function to read Binary format CRTM RTSolution structure files.
 
  CALLING SEQUENCE:
        Error_Status = CRTM_Read_RTSolution_Binary( Filename               , &
                                                    RTSolution             , &
                                                    Quiet      =Quiet      , &
                                                    n_Channels =n_Channels , &
                                                    n_Profiles =n_Profiles , &
                                                    RCS_Id     =RCS_Id     , &
                                                    Message_Log=Message_Log  )
 
  INPUT ARGUMENTS:
        Filename:     Character string specifying the name of an
                      RTSolution format data file to read.
                      UNITS:      N/A
                      TYPE:       CHARACTER(*)
                      DIMENSION:  Scalar
                      ATTRIBUTES: INTENT(IN)
 
  OUTPUT ARGUMENTS:
        RTSolution:   Structure array containing the RTSolution data. Note 
                      the rank is CHANNELS x PROFILES.
                      UNITS:      N/A
                      TYPE:       CRTM_RTSolution_type
                      DIMENSION:  Rank-2 (L x M)
                      ATTRIBUTES: INTENT(IN OUT)
 
 
  OPTIONAL INPUT ARGUMENTS:
        Quiet:        Set this argument to suppress INFORMATION messages
                      being printed to standard output (or the message
                      log file if the Message_Log optional argument is
                      used.) By default, INFORMATION messages are printed.
                      If QUIET = 0, INFORMATION messages are OUTPUT.
                         QUIET = 1, INFORMATION messages are SUPPRESSED.
                      UNITS:      N/A
                      TYPE:       INTEGER
                      DIMENSION:  Scalar
                      ATTRIBUTES: INTENT(IN), OPTIONAL
 
        Message_Log:  Character string specifying a filename in which any
                      messages will be logged. If not specified, or if an
                      error occurs opening the log file, the default action
                      is to output messages to standard output.
                      UNITS:      N/A
                      TYPE:       CHARACTER(*)
                      DIMENSION:  Scalar
                      ATTRIBUTES: INTENT(IN), OPTIONAL
 
  OPTIONAL OUTPUT ARGUMENTS:
        n_Channels:   The number of channels for which data was read.
                      UNITS:      N/A
                      TYPE:       INTEGER
                      DIMENSION:  Scalar
                      ATTRIBUTES: OPTIONAL, INTENT(OUT)
 
        n_Profiles:   The number of profiles for which data was read.
                      UNITS:      N/A
                      TYPE:       INTEGER
                      DIMENSION:  Scalar
                      ATTRIBUTES: OPTIONAL, INTENT(OUT)
 
        RCS_Id:       Character string containing the version control Id
                      field for the module.
                      UNITS:      N/A
                      TYPE:       CHARACTER(*)
                      DIMENSION:  Scalar
                      ATTRIBUTES: OPTIONAL, INTENT(OUT)
 
  FUNCTION RESULT:
        Error_Status: The return value is an integer defining the error status.
                      The error codes are defined in the Message_Handler module.
                      If == SUCCESS the Binary file read was successful
                         == FAILURE an unrecoverable error occurred.
                      UNITS:      N/A
                      TYPE:       INTEGER
                      DIMENSION:  Scalar
 
  COMMENTS:
        Note the INTENT on the output RTSolution argument is IN OUT rather
        than just OUT. This is necessary because the argument may be defined on
        input. To prevent memory leaks, the IN OUT INTENT is a must.
 
  \end{alltt}

\subsection{\texttt{CRTM\_Write\_RTSolution\_Binary} interface}
  \label{sec:CRTM_Write_RTSolution_Binary_interface}
  \begin{alltt}
 
  NAME:
        CRTM_Write_RTSolution_Binary
 
  PURPOSE:
        Function to write Binary format RTSolution files.
 
  CALLING SEQUENCE:
        Error_Status = CRTM_Write_RTSolution_Binary( Filename               , &
                                                     RTSolution             , &
                                                     Quiet      =Quiet      , &
                                                     RCS_Id     =RCS_Id     , &
                                                     Message_Log=Message_Log  )
 
  INPUT ARGUMENTS:
        Filename:     Character string specifying the name of an output
                      RTSolution format data file.
                      UNITS:      N/A
                      TYPE:       CHARACTER(*)
                      DIMENSION:  Scalar
                      ATTRIBUTES: INTENT(IN)
 
        RTSolution:   Structure containing the RTSolution data to write.
                      Note the rank is CHANNELS x PROFILES.
                      UNITS:      N/A
                      TYPE:       CRTM_RTSolution_type
                      DIMENSION:  Rank-2 (L x M)
                      ATTRIBUTES: INTENT(IN)
 
  OPTIONAL INPUT ARGUMENTS:
        Quiet:        Set this argument to suppress INFORMATION messages
                      being printed to standard output (or the message
                      log file if the Message_Log optional argument is
                      used.) By default, INFORMATION messages are printed.
                      If QUIET = 0, INFORMATION messages are OUTPUT.
                         QUIET = 1, INFORMATION messages are SUPPRESSED.
                      UNITS:      N/A
                      TYPE:       INTEGER
                      DIMENSION:  Scalar
                      ATTRIBUTES: INTENT(IN), OPTIONAL
 
        Message_Log:  Character string specifying a filename in which any
                      messages will be logged. If not specified, or if an
                      error occurs opening the log file, the default action
                      is to output messages to standard output.
                      UNITS:      N/A
                      TYPE:       CHARACTER(*)
                      DIMENSION:  Scalar
                      ATTRIBUTES: INTENT(IN), OPTIONAL
 
  OPTIONAL OUTPUT ARGUMENTS:
        RCS_Id:       Character string containing the version control Id
                      field for the module.
                      UNITS:      N/A
                      TYPE:       CHARACTER(*)
                      DIMENSION:  Scalar
                      ATTRIBUTES: OPTIONAL, INTENT(OUT)
 
  FUNCTION RESULT:
        Error_Status: The return value is an integer defining the error status.
                      The error codes are defined in the Message_Handler module.
                      If == SUCCESS the Binary file write was successful
                         == FAILURE an unrecoverable error occurred.
                      UNITS:      N/A
                      TYPE:       INTEGER
                      DIMENSION:  Scalar
 
  SIDE EFFECTS:
        - If the output file already exists, it is overwritten.
        - If an error occurs *during* the write phase, the output file is deleted
          before returning to the calling routine.
 
  \end{alltt}



\clearpage
\section{\Options{} Structure}
%=============================
\label{sec:options_structure}

\begin{figure}[htp]
  \centering
  \doublebox{
  \begin{minipage}[b]{6.5in}
    \begin{alltt}
  TYPE :: CRTM_Options_type
    ! Allocation indicator
    LOGICAL :: Is_Allocated = .FALSE.

    ! Input checking on by default
    LOGICAL :: Check_Input = .TRUE.

    ! User defined MW water emissivity algorithm
    LOGICAL :: Use_Old_MWSSEM = .FALSE.

    ! Antenna correction application
    LOGICAL :: Use_Antenna_Correction = .FALSE.

    ! NLTE radiance correction is ON by default
    LOGICAL :: Apply_NLTE_Correction = .TRUE.

    ! RT Algorithm is set to ADA by default
    INTEGER(Long) :: RT_Algorithm_Id = RT_ADA

    ! Aircraft flight level pressure
    ! Value > 0 turns "on" the aircraft option
    REAL(Double) :: Aircraft_Pressure = -ONE

    ! User defined number of RT solver streams (streams up + streams down)
    LOGICAL       :: Use_n_Streams = .FALSE.
    INTEGER(Long) :: n_Streams = 0

    ! Scattering switch. Default is for
    ! Cloud/Aerosol scattering to be included.
    LOGICAL :: Include_Scattering = .TRUE.

    ! Cloud cover overlap id is set to averaging type by default
    INTEGER(Long) :: Overlap_Id = DEFAULT_OVERLAP_ID

    ! User defined emissivity/reflectivity
    ! ...Dimensions
    INTEGER(Long) :: n_Channels = 0  ! L dimension
    ! ...Index into channel-specific components
    INTEGER(Long) :: Channel = 0
    ! ...Emissivity optional arguments
    LOGICAL :: Use_Emissivity = .FALSE.
    REAL(Double), ALLOCATABLE :: Emissivity(:)  ! L
    ! ...Direct reflectivity optional arguments
    LOGICAL :: Use_Direct_Reflectivity = .FALSE.
    REAL(Double), ALLOCATABLE :: Direct_Reflectivity(:) ! L

    ! SSU instrument input
    TYPE(SSU_Input_type) :: SSU

    ! Zeeman-splitting input
    TYPE(Zeeman_Input_type) :: Zeeman

  END TYPE CRTM_Options_type
    \end{alltt}
  \end{minipage}
  }
  \caption{CRTM\_Options\_type structure definition.}
  \label{fig:CRTM_Options_type_structure}
\end{figure}


\begin{table}[htp]
  \centering
  \begin{tabular}{l p{7cm} c c}
    \hline
    \sffamily\textbf{Component} & \sffamily\textbf{Description} & \sffamily\textbf{Units} & \sffamily\textbf{Dimensions} \\
    \hline\hline
    \texttt{n\_Channels}                  & Number of sensor channels (\texttt{L}). & N/A & Scalar \\
    \texttt{Channel}                      & Index into channel-specific components. & N/A & Scalar \\
    \texttt{Emissivity\_Switch}           & Switch to apply user-defined surface emissivity. Valid values:

    \parbox{7cm}{\hspace{0.5cm}\texttt{NOT\_SET}: Calculate emissivity (default).
    
                 \hspace{0.5cm}\texttt{SET}: Use user-defined emissivity}
     & N/A & Scalar \\
    \texttt{Emissivity}                   & User-defined surface emissivity for each sensor channel. & N/A & \texttt{L} \\
    \texttt{Direct\_Reflectivity\_Switch} & Switch to apply user-defined reflectivity for downwelling source (e.g. solar). This switch is ignored unless the \texttt{Emissivity\_Switch} is also set. Valid values:

    \parbox{7cm}{\hspace{0.5cm}\texttt{NOT\_SET}: Calculate reflectivity (default).
    
                 \hspace{0.5cm}\texttt{SET}: Use user-defined reflectivity}
     & N/A & Scalar \\
    \texttt{Direct\_Reflectivity}         & User-defined direct reflectivity for downwelling source for each sensor channel. & N/A & \texttt{L} \\
    \texttt{Antenna\_Correction}          & Switch to apply antenna correction for select microwave instruments. & N/A & Scalar \\
    \hline
  \end{tabular}
  \caption{CRTM \Options{} structure component description}
  \label{tab:options_structure}
\end{table}

% Options structure methods
%--------------------------
\clearpage
\subsection{\texttt{CRTM\_Associated\_Options} interface}
  \label{sec:CRTM_Associated_Options_interface}
  \begin{alltt}
 
  NAME:
        CRTM_Associated_Options
 
  PURPOSE:
        Function to test the association status of the pointer members of a
        CRTM_Options structure.
 
  CALLING SEQUENCE:
        Association_Status = CRTM_Associated_Options( Options          , &
                                                      ANY_Test=Any_Test  )
 
  INPUT ARGUMENTS:
        Options:             Options structure which is to have its pointer
                             member's association status tested.
                             UNITS:      N/A
                             TYPE:       CRTM_Options_type
                             DIMENSION:  Scalar or Rank-1 array
                             ATTRIBUTES: INTENT(IN)
 
  OPTIONAL INPUT ARGUMENTS:
        ANY_Test:            Set this argument to test if ANY of the
                             Options structure pointer members are associated.
                             The default is to test if ALL the pointer members
                             are associated.
                             If ANY_Test = 0, test if ALL the pointer members
                                              are associated.  (DEFAULT)
                                ANY_Test = 1, test if ANY of the pointer members
                                              are associated.
                             UNITS:      N/A
                             TYPE:       INTEGER
                             DIMENSION:  Scalar
                             ATTRIBUTES: INTENT(IN), OPTIONAL
 
  FUNCTION RESULT:
        Association_Status:  The return value is a logical value indicating the
                             association status of the Options pointer
                             members.
                             .TRUE.  - if ALL the Options pointer members
                                       are associated, or if the ANY_Test argument
                                       is set and ANY of the Options
                                       pointer members are associated.
                             .FALSE. - some or all of the Options pointer
                                       members are NOT associated.
                             UNITS:      N/A
                             TYPE:       LOGICAL
                             DIMENSION:  Same as input Options argument
 
  \end{alltt}

\subsection{\texttt{CRTM\_Allocate\_Options} interface}
  \label{sec:CRTM_Allocate_Options_interface}
  \begin{alltt}
 
  NAME:
        CRTM_Allocate_Options
  
  PURPOSE:
        Function to allocate the pointer members of a CRTM Options
        data structure.
 
  CALLING SEQUENCE:
        Error_Status = CRTM_Allocate_Options( n_Channels             , &
                                              Options                , &
                                              Message_Log=Message_Log  )
 
  INPUT ARGUMENTS:
        n_Channels:   Number of sensor channels
                      Must be > 0
                      UNITS:      N/A
                      TYPE:       INTEGER
                      DIMENSION:  Scalar
                      ATTRIBUTES: INTENT(IN)
 
  OUTPUT ARGUMENTS:
        Options:      Options structure with allocated pointer members.
                      Upon allocation, all pointer members are initialized to
                      a value of zero.
                      UNITS:      N/A
                      TYPE:       CRTM_Options_type
                      DIMENSION:  Scalar or Rank-1
                      ATTRIBUTES: INTENT(IN OUT)
 
  OPTIONAL INPUT ARGUMENTS:
        Message_Log:  Character string specifying a filename in which any
                      Messages will be logged. If not specified, or if an
                      error occurs opening the log file, the default action
                      is to output Messages to standard output.
                      UNITS:      N/A
                      TYPE:       CHARACTER(*)
                      DIMENSION:  Scalar
                      ATTRIBUTES: INTENT(IN), OPTIONAL
 
  FUNCTION RESULT:
        Error_Status: The return value is an integer defining the error status.
                      The error codes are defined in the Message_Handler module.
                      If == SUCCESS the structure pointer allocations were
                                    successful
                         == FAILURE - an error occurred, or
                                    - the structure internal allocation counter
                                      is not equal to one (1) upon exiting this
                                      function. This value is incremented and
                                      decremented for every structure allocation
                                      and deallocation respectively.
                      UNITS:      N/A
                      TYPE:       INTEGER
                      DIMENSION:  Scalar
 
  COMMENTS:
        Note the INTENT on the output Options argument is IN OUT rather than
        just OUT. This is necessary because the argument may be defined upon
        input. To prevent memory leaks, the IN OUT INTENT is a must.
 
  \end{alltt}

\subsection{\texttt{CRTM\_Destroy\_Options} interface}
  \label{sec:CRTM_Destroy_Options_interface}
  \begin{alltt}
 
  NAME:
        CRTM_Destroy_Options
  
  PURPOSE:
        Function to re-initialize the scalar and pointer members of a CRTM
        Options data structure.
 
  CALLING SEQUENCE:
        Error_Status = CRTM_Destroy_Options( Options                , &
                                             Message_Log=Message_Log  )
  
  OUTPUT ARGUMENTS:
        Options:      Re-initialized Options structure.
                      UNITS:      N/A
                      TYPE:       CRTM_Options_type
                      DIMENSION:  Scalar or Rank-1
                      ATTRIBUTES: INTENT(IN OUT)
 
  OPTIONAL INPUT ARGUMENTS:
        Message_Log:  Character string specifying a filename in which any
                      Messages will be logged. If not specified, or if an
                      error occurs opening the log file, the default action
                      is to output Messages to standard output.
                      UNITS:      N/A
                      TYPE:       CHARACTER(*)
                      DIMENSION:  Scalar
                      ATTRIBUTES: INTENT(IN), OPTIONAL
 
  FUNCTION RESULT:
        Error_Status: The return value is an integer defining the error status.
                      The error codes are defined in the Message_Handler module.
                      If == SUCCESS the structure re-initialisation was successful
                         == FAILURE - an error occurred, or
                                    - the structure internal allocation counter
                                      is not equal to zero (0) upon exiting this
                                      function. This value is incremented and
                                      decremented for every structure allocation
                                      and deallocation respectively.
                      UNITS:      N/A
                      TYPE:       INTEGER
                      DIMENSION:  Scalar
 
  COMMENTS:
        Note the INTENT on the output Options argument is IN OUT rather than
        just OUT. This is necessary because the argument may be defined upon
        input. To prevent memory leaks, the IN OUT INTENT is a must.
 
  \end{alltt}

\subsection{\texttt{CRTM\_Assign\_Options} interface}
  \label{sec:CRTM_Assign_Options_interface}
  \begin{alltt}
 
  NAME:
        CRTM_Assign_Options
 
  PURPOSE:
        Function to copy valid CRTM Options structures.
 
  CALLING SEQUENCE:
        Error_Status = CRTM_Assign_Options( Options_in             , &
                                            Options_out            , &
                                            Message_Log=Message_Log  )
 
  INPUT ARGUMENTS:
        Options_in:      Options structure which is to be copied.
                         UNITS:      N/A
                         TYPE:       CRTM_Options_type
                         DIMENSION:  Scalar or Rank-1
                         ATTRIBUTES: INTENT(IN)
 
  OUTPUT ARGUMENTS:
        Options_out:     Copy of the input structure, Options_in.
                         UNITS:      N/A
                         TYPE:       CRTM_Options_type
                         DIMENSION:  Same as Options_in
                         ATTRIBUTES: INTENT(IN OUT)
 
  OPTIONAL INPUT ARGUMENTS:
        Message_Log:     Character string specifying a filename in which any
                         Messages will be logged. If not specified, or if an
                         error occurs opening the log file, the default action
                         is to output Messages to standard output.
                         UNITS:      N/A
                         TYPE:       CHARACTER(*)
                         DIMENSION:  Scalar
                         ATTRIBUTES: INTENT(IN), OPTIONAL
 
  FUNCTION RESULT:
        Error_Status:    The return value is an integer defining the error status.
                         The error codes are defined in the Message_Handler module.
                         If == SUCCESS the structure assignment was successful
                            == FAILURE an error occurred
                         UNITS:      N/A
                         TYPE:       INTEGER
                         DIMENSION:  Scalar
 
  COMMENTS:
        Note the INTENT on the output Options argument is IN OUT rather than
        just OUT. This is necessary because the argument may be defined upon
        input. To prevent memory leaks, the IN OUT INTENT is a must.
 
  \end{alltt}

\subsection{\texttt{CRTM\_Equal\_Options} interface}
  \label{sec:CRTM_Equal_Options_interface}
  \begin{alltt}
 
  NAME:
        CRTM_Equal_Options
 
  PURPOSE:
        Function to test if two CRTM Options structures are equal.
 
  CALLING SEQUENCE:
        Error_Status = CRTM_Equal_Options( Options_LHS            , &
                                           Options_RHS            , &
                                           ULP_Scale  =ULP_Scale  , &
                                           Check_All  =Check_All  , &
                                           Message_Log=Message_Log  )
 
 
  INPUT ARGUMENTS:
        Options_LHS:        Options structure to be compared; equivalent to the
                            left-hand side of a lexical comparison, e.g.
                              IF ( Options_LHS == Options_RHS ).
                            In the context of the CRTM, rank-1 corresponds to an
                            vector of profiles.
                            UNITS:      N/A
                            TYPE:       CRTM_Options_type
                            DIMENSION:  Scalar or Rank-1 array
                            ATTRIBUTES: INTENT(IN)
 
        Options_RHS:        Options structure to be compared to; equivalent to
                            right-hand side of a lexical comparison, e.g.
                              IF ( Options_LHS == Options_RHS ).
                            UNITS:      N/A
                            TYPE:       CRTM_Options_type
                            DIMENSION:  Same as Options_LHS
                            ATTRIBUTES: INTENT(IN)
 
  OPTIONAL INPUT ARGUMENTS:
        ULP_Scale:          Unit of data precision used to scale the floating
                            point comparison. ULP stands for "Unit in the Last Place,"
                            the smallest possible increment or decrement that can be
                            made using a machine's floating point arithmetic.
                            Value must be positive - if a negative value is supplied,
                            the absolute value is used. If not specified, the default
                            value is 1.
                            UNITS:      N/A
                            TYPE:       INTEGER
                            DIMENSION:  Scalar
                            ATTRIBUTES: INTENT(IN), OPTIONAL
 
        Check_All:          Set this argument to check ALL the floating point
                            channel data of the Options structures. The default
                            action is return with a FAILURE status as soon as
                            any difference is found. This optional argument can
                            be used to get a listing of ALL the differences
                            between data in Options structures.
                            If == 0, Return with FAILURE status as soon as
                                     ANY difference is found  *DEFAULT*
                               == 1, Set FAILURE status if ANY difference is
                                     found, but continue to check ALL data.
                            UNITS:      N/A
                            TYPE:       INTEGER
                            DIMENSION:  Scalar
                            ATTRIBUTES: INTENT(IN), OPTIONAL
 
        Message_Log:        Character string specifying a filename in which any
                            messages will be logged. If not specified, or if an
                            error occurs opening the log file, the default action
                            is to output messages to standard output.
                            UNITS:      None
                            TYPE:       CHARACTER(*)
                            DIMENSION:  Scalar
                            ATTRIBUTES: INTENT(IN), OPTIONAL
 
  FUNCTION RESULT:
        Error_Status:       The return value is an integer defining the error status.
                            The error codes are defined in the Message_Handler module.
                            If == SUCCESS the structures were equal
                               == FAILURE - an error occurred, or
                                          - the structures were different.
                            UNITS:      N/A
                            TYPE:       INTEGER
                            DIMENSION:  Scalar
 
  \end{alltt}

\subsection{\texttt{CRTM\_RCS\_ID\_Options} interface}
  \label{sec:CRTM_RCS_ID_Options_interface}
  \begin{alltt}
 
  NAME:
        CRTM_RCS_ID_Options
 
  PURPOSE:
        Subroutine to return the module RCS Id information.
 
  CALLING SEQUENCE:
        CALL CRTM_RCS_Id_Options( RCS_Id )
 
  OUTPUT ARGUMENTS:
        RCS_Id:        Character string containing the Revision Control
                       System Id field for the module.
                       UNITS:      N/A
                       TYPE:       CHARACTER(*)
                       DIMENSION:  Scalar
                       ATTRIBUTES: INTENT(OUT)
 
  \end{alltt}

