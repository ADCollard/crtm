\chapter{Structure and procedure interface definitions}
%======================================================
\label{sec:structure_and_interface_definition}

\clearpage
\section{\ChannelInfo{} Structure}
%=================================
\label{sec:channelinfo_structure}

\begin{figure}[htp]
  \centering
  \doublebox{
  \begin{minipage}[b]{6.5in}
    \begin{alltt}
  TYPE :: CRTM_ChannelInfo_type
    INTEGER :: n_Allocates = 0
    ! Dimensions
    INTEGER :: n_Channels = 0  ! L dimension
    ! Scalar data
    CHARACTER(STRLEN) :: Sensor_ID        = ' '
    INTEGER           :: WMO_Satellite_ID = INVALID_WMO_SATELLITE_ID
    INTEGER           :: WMO_Sensor_ID    = INVALID_WMO_SENSOR_ID
    INTEGER           :: Sensor_Index     = 0
    ! Array data
    INTEGER, POINTER :: Sensor_Channel(:) => NULL()  ! L
    INTEGER, POINTER :: Channel_Index(:)  => NULL()  ! L
  END TYPE CRTM_ChannelInfo_type
    \end{alltt}
  \end{minipage}
  }
  \caption{CRTM\_ChannelInfo\_type structure definition.}
  \label{fig:CRTM_ChannelInfo_type_structure}
\end{figure}


% ChannelInfo structure methods
%-----------------------------
\subsection{\texttt{CRTM\_ChannelInfo\_Associated} interface}
  \label{sec:CRTM_ChannelInfo_Associated_interface}
  \begin{alltt}
 
  NAME:
        CRTM_ChannelInfo_Associated
 
  PURPOSE:
        Elemental function to test the status of the allocatable components
        of a CRTM ChannelInfo object.
 
  CALLING SEQUENCE:
        Status = CRTM_ChannelInfo_Associated( ChannelInfo )
 
  OBJECTS:
        ChannelInfo:   ChannelInfo object which is to have its member's
                       status tested.
                       UNITS:      N/A
                       TYPE:       TYPE(CRTM_ChannelInfo_type)
                       DIMENSION:  Scalar or any rank
                       ATTRIBUTES: INTENT(IN)
 
  FUNCTION RESULT:
        Status:        The return value is a logical value indicating the
                       status of the ChannelInfo members.
                         .TRUE.  - if the array components are allocated.
                         .FALSE. - if the array components are not allocated.
                       UNITS:      N/A
                       TYPE:       LOGICAL
                       DIMENSION:  Same as input ChannelInfo argument
 
  \end{alltt}

\subsection{\texttt{CRTM\_ChannelInfo\_DefineVersion} interface}
  \label{sec:CRTM_ChannelInfo_DefineVersion_interface}
  \begin{alltt}
 
  NAME:
        CRTM_ChannelInfo_DefineVersion
 
  PURPOSE:
        Subroutine to return the module version information.
 
  CALLING SEQUENCE:
        CALL CRTM_ChannelInfo_DefineVersion( Id )
 
  OUTPUTS:
        Id:    Character string containing the version Id information
               for the module.
               UNITS:      N/A
               TYPE:       CHARACTER(*)
               DIMENSION:  Scalar
               ATTRIBUTES: INTENT(OUT)
 
  \end{alltt}

\subsection{\texttt{CRTM\_ChannelInfo\_Destroy} interface}
  \label{sec:CRTM_ChannelInfo_Destroy_interface}
  \begin{alltt}
 
  NAME:
        CRTM_ChannelInfo_Destroy
 
  PURPOSE:
        Elemental subroutine to re-initialize CRTM ChannelInfo objects.
 
  CALLING SEQUENCE:
        CALL CRTM_ChannelInfo_Destroy( ChannelInfo )
 
  OBJECTS:
        ChannelInfo:    Re-initialized ChannelInfo object.
                        UNITS:      N/A
                        TYPE:       TYPE(CRTM_ChannelInfo_type)
                        DIMENSION:  Scalar OR any rank
                        ATTRIBUTES: INTENT(OUT)
 
  \end{alltt}

\subsection{\texttt{CRTM\_ChannelInfo\_Inspect} interface}
  \label{sec:CRTM_ChannelInfo_Inspect_interface}
  \begin{alltt}
 
  NAME:
        CRTM_ChannelInfo_Inspect
 
  PURPOSE:
        Subroutine to print the contents of a CRTM ChannelInfo object
        to stdout.
 
  CALLING SEQUENCE:
        CALL CRTM_ChannelInfo_Inspect( ChannelInfo )
 
  INPUTS:
        ChannelInfo:   ChannelInfo object to display.
                       UNITS:      N/A
                       TYPE:       TYPE(CRTM_ChannelInfo_type)
                       DIMENSION:  Scalar
                       ATTRIBUTES: INTENT(IN)
 
  \end{alltt}

\subsection{\texttt{CRTM\_ChannelInfo\_n\_Channels} interface}
  \label{sec:CRTM_ChannelInfo_n_Channels_interface}
  \begin{alltt}
 
  NAME:
        CRTM_ChannelInfo_n_Channels
 
  PURPOSE:
        Function to return the number of channels defined in a ChannelInfo
        structure or structure array
 
  CALLING SEQUENCE:
        n_Channels = CRTM_ChannelInfo_n_Channels( ChannelInfo )
 
  INPUTS:
        ChannelInfo: ChannelInfo structure or structure which is to have its
                     channels counted.
                     UNITS:      N/A
                     TYPE:       TYPE(CRTM_ChannelInfo_type)
                     DIMENSION:  Scalar
                                   or
                                 Rank-1
                     ATTRIBUTES: INTENT(IN)
 
  FUNCTION RESULT:
        n_Channels:   The number of defined channels in the input argument.
                     UNITS:      N/A
                     TYPE:       INTEGER
                     DIMENSION:  Scalar
 
  \end{alltt}




\clearpage
\section{\Atmosphere{} Structure}
%================================
\label{sec:atmosphere_structure}

\begin{figure}[htp]
  \centering
  \doublebox{
  \begin{minipage}[b]{6.5in}
    \begin{alltt}
  TYPE :: CRTM_Atmosphere_type
    ! Allocation indicator
    LOGICAL :: Is_Allocated = .FALSE.
    ! Dimension values
    INTEGER :: Max_Layers   = 0  ! K dimension
    INTEGER :: n_Layers     = 0  ! Kuse dimension
    INTEGER :: n_Absorbers  = 0  ! J dimension
    INTEGER :: Max_Clouds   = 0  ! Nc dimension
    INTEGER :: n_Clouds     = 0  ! NcUse dimension
    INTEGER :: Max_Aerosols = 0  ! Na dimension
    INTEGER :: n_Aerosols   = 0  ! NaUse dimension
    ! Number of added layers
    INTEGER :: n_Added_Layers = 0
    ! Climatology model associated with the profile
    INTEGER :: Climatology = US_STANDARD_ATMOSPHERE
    ! Absorber ID and units
    INTEGER, ALLOCATABLE :: Absorber_ID(:)    ! J
    INTEGER, ALLOCATABLE :: Absorber_Units(:) ! J
    ! Profile LEVEL and LAYER quantities
    REAL(fp), ALLOCATABLE :: Level_Pressure(:)  ! 0:K
    REAL(fp), ALLOCATABLE :: Pressure(:)        ! K
    REAL(fp), ALLOCATABLE :: Temperature(:)     ! K
    REAL(fp), ALLOCATABLE :: Absorber(:,:)      ! K x J
    ! Clouds associated with each profile
    TYPE(CRTM_Cloud_type),   ALLOCATABLE :: Cloud(:)    ! Nc
    ! Aerosols associated with each profile
    TYPE(CRTM_Aerosol_type), ALLOCATABLE :: Aerosol(:)  ! Na
  END TYPE CRTM_Atmosphere_type
    \end{alltt}
  \end{minipage}
  }
  \caption{CRTM\_Atmosphere\_type structure definition.}
  \label{fig:CRTM_Atmosphere_type_structure}
\end{figure}


% Climatology table
\begin{table}[htp]
  \centering
  \begin{tabular}{c l}
    \hline
    \sffamily\textbf{Climatology Type} & \sffamily\textbf{Parameter} \\
    \hline\hline
             Tropical          &  \texttt{TROPICAL}\\              
        Midlatitude summer     &  \texttt{MIDLATITUDE\_SUMMER}\\
        Midlatitude winter     &  \texttt{MIDLATITUDE\_WINTER}\\
         Subarctic summer      &  \texttt{SUBARCTIC\_SUMMER}\\
         Subarctic winter      &  \texttt{SUBARCTIC\_WINTER}\\
     U.S. Standard Atmosphere  &  \texttt{US\_STANDARD\_ATMOSPHERE}\\
    \hline 
  \end{tabular}
  \caption{CRTM \Atmosphere{} structure valid \texttt{Climatology} definitions. The same set as defined for LBLRTM is used.}
  \label{tab:climatology}
\end{table}

% Absorber id table
\begin{table}[htp]
  \centering
  \begin{tabular}{ c l c c l }
    \hline
    \sffamily\textbf{Molecule} & \sffamily\textbf{Parameter} & \hspace{0.5cm} & \sffamily\textbf{Molecule} & \sffamily\textbf{Parameter}\\
    \hline\hline
     H\subscript{2}O  & \texttt{H2O\_ID}  & \hspace{0.5cm} & HI                           & \texttt{HI\_ID}    \\
     CO\subscript{2}  & \texttt{CO2\_ID}  & \hspace{0.5cm} & ClO                          & \texttt{ClO\_ID}   \\
     O\subscript{3}   & \texttt{O3\_ID}   & \hspace{0.5cm} & OCS                          & \texttt{OCS\_ID}   \\
     N\subscript{2}O  & \texttt{N2O\_ID}  & \hspace{0.5cm} & H\subscript{2}CO             & \texttt{H2CO\_ID}  \\
     CO               & \texttt{CO\_ID}   & \hspace{0.5cm} & HOCl                         & \texttt{HOCl\_ID}  \\
     CH\subscript{4}  & \texttt{CH4\_ID}  & \hspace{0.5cm} & N\subscript{2}               & \texttt{N2\_ID}    \\
     O\subscript{2}   & \texttt{O2\_ID}   & \hspace{0.5cm} & HCN                          & \texttt{HCN\_ID}   \\
     NO               & \texttt{NO\_ID}   & \hspace{0.5cm} & CH\subscript{3}l             & \texttt{CH3l\_ID}  \\
     SO\subscript{2}  & \texttt{SO2\_ID}  & \hspace{0.5cm} & H\subscript{2}O\subscript{2} & \texttt{H2O2\_ID}  \\
     NO\subscript{2}  & \texttt{NO2\_ID}  & \hspace{0.5cm} & C\subscript{2}H\subscript{2} & \texttt{C2H2\_ID}  \\
     NH\subscript{3}  & \texttt{NH3\_ID}  & \hspace{0.5cm} & C\subscript{2}H\subscript{6} & \texttt{C2H6\_ID}  \\
     HNO\subscript{3} & \texttt{HNO3\_ID} & \hspace{0.5cm} & PH\subscript{3}              & \texttt{PH3\_ID}   \\
     OH               & \texttt{OH\_ID}   & \hspace{0.5cm} & COF\subscript{2}             & \texttt{COF2\_ID}  \\
     HF               & \texttt{HF\_ID}   & \hspace{0.5cm} & SF\subscript{6}              & \texttt{SF6\_ID}   \\
     HCl              & \texttt{HCl\_ID}  & \hspace{0.5cm} & H\subscript{2}S              & \texttt{H2S\_ID}   \\
     HBr              & \texttt{HBr\_ID}  & \hspace{0.5cm} & HCOOH                        & \texttt{HCOOH\_ID} \\
    \hline
  \end{tabular}
  \caption{CRTM \Atmosphere{} structure valid \texttt{Absorber\_ID} definitions. The same molecule set as defined for HITRAN is used.}
  \label{tab:absorber_id}
\end{table}

% Absorber units table
\begin{table}[htp]
  \centering
  \begin{tabular}{c l}
    \hline
    \sffamily\textbf{Units} & \sffamily\textbf{Parameter} \\
    \hline\hline
     Volume mixing ratio, ppmv                       & \texttt{VOLUME\_MIXING\_RATIO\_UNITS} \\
     Number density, cm$^{-3}$                       & \texttt{NUMBER\_DENSITY\_UNITS} \\
     Mass mixing ratio, g/kg                         & \texttt{MASS\_MIXING\_RATIO\_UNITS} \\
     Mass density, g.m$^{-3}$                        & \texttt{MASS\_DENSITY\_UNITS} \\
     Partial pressure, hPa                           & \texttt{PARTIAL\_PRESSURE\_UNITS} \\
     Dewpoint temperature, K  \textbf{(H$\mathbf{_2}$O ONLY)} & \texttt{DEWPOINT\_TEMPERATURE\_K\_UNITS} \\
     Dewpoint temperature, C  \textbf{(H$\mathbf{_2}$O ONLY)} & \texttt{DEWPOINT\_TEMPERATURE\_C\_UNITS} \\
     Relative humidity, \%    \textbf{(H$\mathbf{_2}$O ONLY)} & \texttt{RELATIVE\_HUMIDITY\_UNITS} \\
     Specific amount, g/g                            & \texttt{SPECIFIC\_AMOUNT\_UNITS} \\
     Integrated path, mm                             & \texttt{INTEGRATED\_PATH\_UNITS} \\
    \hline
  \end{tabular}
  \caption{CRTM \Atmosphere{} structure valid \texttt{Absorber\_Units} definitions. The same set as defined for LBLRTM is used.}
  \label{tab:absorber_units}
\end{table}

% Atmosphere structure methods
%-----------------------------
\clearpage
\subsection{\texttt{CRTM\_Atmosphere\_AddLayerCopy} interface}
  \label{sec:CRTM_Atmosphere_AddLayerCopy_interface}
  \begin{alltt}
 
  NAME:
        CRTM_Atmosphere_AddLayerCopy
 
  PURPOSE:
        Elemental function to copy an instance of the CRTM Atmosphere object
        with additional layers added to the TOA of the input.
 
  CALLING SEQUENCE:
        Atm_out = CRTM_Atmosphere_AddLayerCopy( Atm, n_Added_Layers )
 
  OBJECTS:
        Atm:             Atmosphere structure to copy.
                         UNITS:      N/A
                         TYPE:       CRTM_Atmosphere_type
                         DIMENSION:  Scalar or any rank
                         ATTRIBUTES: INTENT(OUT)
 
  INPUTS:
        n_Added_Layers:  Number of layers to add to the function result.
                         UNITS:      N/A
                         TYPE:       INTEGER
                         DIMENSION:  Same as atmosphere object
                         ATTRIBUTES: INTENT(IN)
 
  FUNCTION RESULT:
        Atm_out:         Copy of the input atmosphere structure with space for
                         extra layers added to TOA.
                         UNITS:      N/A
                         TYPE:       CRTM_Atmosphere_type
                         DIMENSION:  Same as input.
                         ATTRIBUTES: INTENT(OUT)
 
 
  \end{alltt}

\subsection{\texttt{CRTM\_Atmosphere\_Associated} interface}
  \label{sec:CRTM_Atmosphere_Associated_interface}
  \begin{alltt}
 
  NAME:
        CRTM_Atmosphere_Associated
 
  PURPOSE:
        Elemental function to test the status of the allocatable components
        of a CRTM Atmosphere object.
 
  CALLING SEQUENCE:
        Status = CRTM_Atmosphere_Associated( Atm )
 
  OBJECTS:
        Atm:       Atmosphere structure which is to have its member's
                   status tested.
                   UNITS:      N/A
                   TYPE:       CRTM_Atmosphere_type
                   DIMENSION:  Scalar or any rank
                   ATTRIBUTES: INTENT(IN)
 
  FUNCTION RESULT:
        Status:    The return value is a logical value indicating the
                   status of the Atmosphere members.
                     .TRUE.  - if the array components are allocated.
                     .FALSE. - if the array components are not allocated.
                   UNITS:      N/A
                   TYPE:       LOGICAL
                   DIMENSION:  Same as input
 
  \end{alltt}

\subsection{\texttt{CRTM\_Atmosphere\_Compare} interface}
  \label{sec:CRTM_Atmosphere_Compare_interface}
  \begin{alltt}
  NAME:
        CRTM_Atmosphere_Compare
 
  PURPOSE:
        Elemental function to compare two CRTM_Atmosphere objects to within
        a user specified number of significant figures.
 
  CALLING SEQUENCE:
        is_comparable = CRTM_Atmosphere_Compare( x, y, n_SigFig=n_SigFig )
 
  OBJECTS:
        x, y:          Two CRTM Atmosphere objects to be compared.
                       UNITS:      N/A
                       TYPE:       CRTM_Atmosphere_type
                       DIMENSION:  Scalar or any rank
                       ATTRIBUTES: INTENT(IN)
 
  OPTIONAL INPUTS:
        n_SigFig:      Number of significant figure to compare floating point
                       components.
                       UNITS:      N/A
                       TYPE:       INTEGER
                       DIMENSION:  Scalar or same as input
                       ATTRIBUTES: INTENT(IN), OPTIONAL
 
  FUNCTION RESULT:
        is_equal:      Logical value indicating whether the inputs are equal.
                       UNITS:      N/A
                       TYPE:       LOGICAL
                       DIMENSION:  Same as inputs.
  \end{alltt}

\subsection{\texttt{CRTM\_Atmosphere\_Create} interface}
  \label{sec:CRTM_Atmosphere_Create_interface}
  \begin{alltt}
 
  NAME:
        CRTM_Atmosphere_Create
  
  PURPOSE:
        Elemental subroutine to create an instance of the CRTM Atmosphere object.
 
  CALLING SEQUENCE:
        CALL CRTM_Atmosphere_Create( Atm        , &
                                     n_Layers   , &
                                     n_Absorbers, &
                                     n_Clouds   , &
                                     n_Aerosols   )
 
  OBJECTS:
        Atm:          Atmosphere structure.
                      UNITS:      N/A
                      TYPE:       CRTM_Atmosphere_type
                      DIMENSION:  Scalar or any rank
                      ATTRIBUTES: INTENT(OUT)
 
  INPUTS:
        n_Layers:     Number of layers dimension.
                      Must be > 0.
                      UNITS:      N/A
                      TYPE:       INTEGER
                      DIMENSION:  Same as atmosphere object
                      ATTRIBUTES: INTENT(IN)
 
        n_Absorbers:  Number of absorbers dimension.
                      Must be > 0.
                      UNITS:      N/A
                      TYPE:       INTEGER
                      DIMENSION:  Same as atmosphere object
                      ATTRIBUTES: INTENT(IN)
 
        n_Clouds:     Number of clouds dimension.
                      Can be = 0 (i.e. clear sky).
                      UNITS:      N/A
                      TYPE:       INTEGER
                      DIMENSION:  Same as atmosphere object
                      ATTRIBUTES: INTENT(IN)
 
        n_Aerosols:   Number of aerosols dimension.
                      Can be = 0 (i.e. no aerosols).
                      UNITS:      N/A
                      TYPE:       INTEGER
                      DIMENSION:  Same as atmosphere object
                      ATTRIBUTES: INTENT(IN)
 
  \end{alltt}

\subsection{\texttt{CRTM\_Atmosphere\_DefineVersion} interface}
  \label{sec:CRTM_Atmosphere_DefineVersion_interface}
  \begin{alltt}
 
  NAME:
        CRTM_Atmosphere_DefineVersion
 
  PURPOSE:
        Subroutine to return the module version information.
 
  CALLING SEQUENCE:
        CALL CRTM_Atmosphere_DefineVersion( Id )
 
  OUTPUTS:
        Id:            Character string containing the version Id information
                       for the module.
                       UNITS:      N/A
                       TYPE:       CHARACTER(*)
                       DIMENSION:  Scalar
                       ATTRIBUTES: INTENT(OUT)
 
  \end{alltt}

\subsection{\texttt{CRTM\_Atmosphere\_Destroy} interface}
  \label{sec:CRTM_Atmosphere_Destroy_interface}
  \begin{alltt}
 
  NAME:
        CRTM_Atmosphere_Destroy
 
  PURPOSE:
        Elemental subroutine to re-initialize CRTM Atmosphere objects.
 
  CALLING SEQUENCE:
        CALL CRTM_Atmosphere_Destroy( Atm )
 
  OBJECTS:
        Atm:          Re-initialized Atmosphere structure.
                      UNITS:      N/A
                      TYPE:       CRTM_Atmosphere_type
                      DIMENSION:  Scalar or any rank
                      ATTRIBUTES: INTENT(OUT)
 
  \end{alltt}

\subsection{\texttt{CRTM\_Atmosphere\_Inspect} interface}
  \label{sec:CRTM_Atmosphere_Inspect_interface}
  \begin{alltt}
 
  NAME:
        CRTM_Atmosphere_Inspect
 
  PURPOSE:
        Subroutine to print the contents of a CRTM Atmosphere object to stdout.
 
  CALLING SEQUENCE:
        CALL CRTM_Atmosphere_Inspect( Atm )
 
  INPUTS:
        Atm:  CRTM Atmosphere object to display.
              UNITS:      N/A
              TYPE:       CRTM_Atmosphere_type
              DIMENSION:  Scalar, Rank-1, or Rank-2 array
              ATTRIBUTES: INTENT(IN)
 
  \end{alltt}

\subsection{\texttt{CRTM\_Atmosphere\_IsValid} interface}
  \label{sec:CRTM_Atmosphere_IsValid_interface}
  \begin{alltt}
 
  NAME:
        CRTM_Atmosphere_IsValid
 
  PURPOSE:
        Non-pure function to perform some simple validity checks on a
        CRTM Atmosphere object. 
 
        If invalid data is found, a message is printed to stdout.
 
  CALLING SEQUENCE:
        result = CRTM_Atmosphere_IsValid( Atm )
 
          or
 
        IF ( CRTM_Atmosphere_IsValid( Atm ) ) THEN....
 
  OBJECTS:
        Atm:       CRTM Atmosphere object which is to have its
                   contents checked.
                   UNITS:      N/A
                   TYPE:       CRTM_Atmosphere_type
                   DIMENSION:  Scalar
                   ATTRIBUTES: INTENT(IN)
 
  FUNCTION RESULT:
        result:    Logical variable indicating whether or not the input
                   passed the check.
                   If == .FALSE., Atmosphere object is unused or contains
                                  invalid data.
                      == .TRUE.,  Atmosphere object can be used in CRTM.
                   UNITS:      N/A
                   TYPE:       LOGICAL
                   DIMENSION:  Scalar
 
  \end{alltt}

\subsection{\texttt{CRTM\_Atmosphere\_Zero} interface}
  \label{sec:CRTM_Atmosphere_Zero_interface}
  \begin{alltt}
 
  NAME:
        CRTM_Atmosphere_Zero
 
  PURPOSE:
        Elemental subroutine to zero out the data arrays
        in a CRTM Atmosphere object.
 
  CALLING SEQUENCE:
        CALL CRTM_Atmosphere_Zero( Atm )
 
  OUTPUTS:
        Atm:          CRTM Atmosphere structure in which the data arrays
                      are to be zeroed out.
                      UNITS:      N/A
                      TYPE:       CRTM_Atmosphere_type
                      DIMENSION:  Scalar or any rank
                      ATTRIBUTES: INTENT(IN OUT)
 
  COMMENTS:
        - The dimension components of the structure are *NOT* set to zero.
        - The Climatology, Absorber_ID, and Absorber_Units components are
          *NOT* reset in this routine.
 
  \end{alltt}

\subsection{\texttt{CRTM\_Atmosphere\_IOVersion} interface}
  \label{sec:CRTM_Atmosphere_IOVersion_interface}
  \begin{alltt}
 
  NAME:
        CRTM_Atmosphere_IOVersion
 
  PURPOSE:
        Subroutine to return the module version information.
 
  CALLING SEQUENCE:
        CALL CRTM_Atmosphere_IOVersion( Id )
 
  OUTPUTS:
        Id:            Character string containing the version Id information
                       for the module.
                       UNITS:      N/A
                       TYPE:       CHARACTER(*)
                       DIMENSION:  Scalar
                       ATTRIBUTES: INTENT(OUT)
 
  \end{alltt}

\subsection{\texttt{CRTM\_Atmosphere\_InquireFile} interface}
  \label{sec:CRTM_Atmosphere_InquireFile_interface}
  \begin{alltt}
 
  NAME:
        CRTM_Atmosphere_InquireFile
 
  PURPOSE:
        Function to inquire CRTM Atmosphere object files.
 
  CALLING SEQUENCE:
        Error_Status = CRTM_Atmosphere_InquireFile( Filename               , &
                                                    n_Channels = n_Channels, &
                                                    n_Profiles = n_Profiles  )
 
  INPUTS:
        Filename:       Character string specifying the name of a
                        CRTM Atmosphere data file to read.
                        UNITS:      N/A
                        TYPE:       CHARACTER(*)
                        DIMENSION:  Scalar
                        ATTRIBUTES: INTENT(IN)
 
  OPTIONAL OUTPUTS:
        n_Channels:     The number of spectral channels for which there is
                        data in the file. Note that this value will always
                        be 0 for a profile-only dataset-- it only has meaning
                        for K-matrix data.
                        UNITS:      N/A
                        TYPE:       INTEGER
                        DIMENSION:  Scalar
                        ATTRIBUTES: OPTIONAL, INTENT(OUT)
 
        n_Profiles:     The number of profiles in the data file.
                        UNITS:      N/A
                        TYPE:       INTEGER
                        DIMENSION:  Scalar
                        ATTRIBUTES: OPTIONAL, INTENT(OUT)
 
  FUNCTION RESULT:
        Error_Status:   The return value is an integer defining the error status.
                        The error codes are defined in the Message_Handler module.
                        If == SUCCESS, the file inquire was successful
                           == FAILURE, an unrecoverable error occurred.
                        UNITS:      N/A
                        TYPE:       INTEGER
                        DIMENSION:  Scalar
 
  \end{alltt}

\subsection{\texttt{CRTM\_Atmosphere\_ReadFile} interface}
  \label{sec:CRTM_Atmosphere_ReadFile_interface}
  \begin{alltt}
 
  NAME:
        CRTM_Atmosphere_ReadFile
 
  PURPOSE:
        Function to read CRTM Atmosphere object files.
 
  CALLING SEQUENCE:
        Error_Status = CRTM_Atmosphere_ReadFile( Filename                , &
                                                 Atmosphere              , &
                                                 Quiet      = Quiet      , &
                                                 n_Channels = n_Channels , &
                                                 n_Profiles = n_Profiles , &
 
  INPUTS:
        Filename:     Character string specifying the name of an
                      Atmosphere format data file to read.
                      UNITS:      N/A
                      TYPE:       CHARACTER(*)
                      DIMENSION:  Scalar
                      ATTRIBUTES: INTENT(IN)
 
  OUTPUTS:
        Atmosphere:   CRTM Atmosphere object array containing the Atmosphere
                      data. Note the following meanings attributed to the
                      dimensions of the object array:
                      Rank-1: M profiles.
                              Only profile data are to be read in. The file
                              does not contain channel information. The
                              dimension of the structure is understood to
                              be the PROFILE dimension.
                      Rank-2: L channels  x  M profiles
                              Channel and profile data are to be read in.
                              The file contains both channel and profile
                              information. The first dimension of the 
                              structure is the CHANNEL dimension, the second
                              is the PROFILE dimension. This is to allow
                              K-matrix structures to be read in with the
                              same function.
                      UNITS:      N/A
                      TYPE:       CRTM_Atmosphere_type
                      DIMENSION:  Rank-1 (M) or Rank-2 (L x M)
                      ATTRIBUTES: INTENT(OUT)
 
  OPTIONAL INPUTS:
        Quiet:        Set this logical argument to suppress INFORMATION
                      messages being printed to stdout
                      If == .FALSE., INFORMATION messages are OUTPUT [DEFAULT].
                         == .TRUE.,  INFORMATION messages are SUPPRESSED.
                      If not specified, default is .FALSE.
                      UNITS:      N/A
                      TYPE:       LOGICAL
                      DIMENSION:  Scalar
                      ATTRIBUTES: INTENT(IN), OPTIONAL
 
  OPTIONAL OUTPUTS:
        n_Channels:   The number of channels for which data was read. Note that
                      this value will always be 0 for a profile-only dataset--
                      it only has meaning for K-matrix data.
                      UNITS:      N/A
                      TYPE:       INTEGER
                      DIMENSION:  Scalar
                      ATTRIBUTES: OPTIONAL, INTENT(OUT)
 
        n_Profiles:   The number of profiles for which data was read.
                      UNITS:      N/A
                      TYPE:       INTEGER
                      DIMENSION:  Scalar
                      ATTRIBUTES: OPTIONAL, INTENT(OUT)
 
 
  FUNCTION RESULT:
        Error_Status: The return value is an integer defining the error status.
                      The error codes are defined in the Message_Handler module.
                      If == SUCCESS, the file read was successful
                         == FAILURE, an unrecoverable error occurred.
                      UNITS:      N/A
                      TYPE:       INTEGER
                      DIMENSION:  Scalar
 
  \end{alltt}

\subsection{\texttt{CRTM\_Atmosphere\_WriteFile} interface}
  \label{sec:CRTM_Atmosphere_WriteFile_interface}
  \begin{alltt}
 
  NAME:
        CRTM_Atmosphere_WriteFile
 
  PURPOSE:
        Function to write CRTM Atmosphere object files.
 
  CALLING SEQUENCE:
        Error_Status = CRTM_Atmosphere_WriteFile( Filename     , &
                                                  Atmosphere   , &
                                                  Quiet = Quiet  )
 
  INPUTS:
        Filename:     Character string specifying the name of the
                      Atmosphere format data file to write.
                      UNITS:      N/A
                      TYPE:       CHARACTER(*)
                      DIMENSION:  Scalar
                      ATTRIBUTES: INTENT(IN)
 
        Atmosphere:   CRTM Atmosphere object array containing the Atmosphere
                      data. Note the following meanings attributed to the
                      dimensions of the Atmosphere array:
                      Rank-1: M profiles.
                              Only profile data are to be read in. The file
                              does not contain channel information. The
                              dimension of the array is understood to
                              be the PROFILE dimension.
                      Rank-2: L channels  x  M profiles
                              Channel and profile data are to be read in.
                              The file contains both channel and profile
                              information. The first dimension of the 
                              array is the CHANNEL dimension, the second
                              is the PROFILE dimension. This is to allow
                              K-matrix structures to be read in with the
                              same function.
                      UNITS:      N/A
                      TYPE:       CRTM_Atmosphere_type
                      DIMENSION:  Rank-1 (M) or Rank-2 (L x M)
                      ATTRIBUTES: INTENT(IN)
 
  OPTIONAL INPUTS:
        Quiet:        Set this logical argument to suppress INFORMATION
                      messages being printed to stdout
                      If == .FALSE., INFORMATION messages are OUTPUT [DEFAULT].
                         == .TRUE.,  INFORMATION messages are SUPPRESSED.
                      If not specified, default is .FALSE.
                      UNITS:      N/A
                      TYPE:       LOGICAL
                      DIMENSION:  Scalar
                      ATTRIBUTES: INTENT(IN), OPTIONAL
 
  FUNCTION RESULT:
        Error_Status: The return value is an integer defining the error status.
                      The error codes are defined in the Message_Handler module.
                      If == SUCCESS, the file write was successful
                         == FAILURE, an unrecoverable error occurred.
                      UNITS:      N/A
                      TYPE:       INTEGER
                      DIMENSION:  Scalar
 
  SIDE EFFECTS:
        - If the output file already exists, it is overwritten.
        - If an error occurs during *writing*, the output file is deleted before
          returning to the calling routine.
 
  \end{alltt}



\newpage
\section{\Cloud{} Structure}
%================================
\label{sec:cloud_structure}

\begin{figure}[htp]
  \centering
  \doublebox{
  \begin{minipage}[b]{6.5in}
    \begin{alltt}
  TYPE :: CRTM_Cloud_type
    INTEGER :: n_Allocates = 0
    ! Dimension values
    INTEGER :: Max_Layers = 0  ! K dimension.
    INTEGER :: n_Layers   = 0  ! Kuse dimension.
    ! Number of added layers
    INTEGER :: n_Added_Layers = 0
    ! Cloud type
    INTEGER :: Type = NO_CLOUD
    ! Cloud state variables
    REAL(fp), POINTER :: Effective_Radius(:)   => NULL() ! K. Units are microns
    REAL(fp), POINTER :: Effective_Variance(:) => NULL() ! K. Units are microns^2
    REAL(fp), POINTER :: Water_Content(:) => NULL()      ! K. Units are kg/m^2
  END TYPE CRTM_Cloud_type
    \end{alltt}
  \end{minipage}
  }
  \caption{CRTM\_Cloud\_type structure definition.}
  \label{fig:CRTM_Cloud_type_structure}
\end{figure}


% Cloud type table
\begin{table}[htp]
  \centering
  \begin{tabular}{cc l}
    \hline
    \sffamily\textbf{Cloud Type} & \hspace{0.5cm} & \sffamily\textbf{Parameter} \\
    \hline\hline
     Water   & \hspace{0.5cm} & \texttt{WATER\_CLOUD}\\
     Ice     & \hspace{0.5cm} & \texttt{ICE\_CLOUD}\\
     Rain    & \hspace{0.5cm} & \texttt{RAIN\_CLOUD}\\
     Snow    & \hspace{0.5cm} & \texttt{SNOW\_CLOUD}\\
     Graupel & \hspace{0.5cm} & \texttt{GRAUPEL\_CLOUD}\\
     Hail    & \hspace{0.5cm} & \texttt{HAIL\_CLOUD}\\
    \hline
  \end{tabular}
  \caption{CRTM \Cloud{} structure valid \texttt{Type} definitions.}
  \label{tab:cloud_type}
\end{table}

% Cloud structure methods
%------------------------
\clearpage
\subsection{\texttt{CRTM\_Cloud\_AddLayerCopy} interface}
  \label{sec:CRTM_Cloud_AddLayerCopy_interface}
  \begin{alltt}
 
  NAME:
        CRTM_Cloud_AddLayerCopy
 
  PURPOSE:
        Elemental function to copy an instance of the CRTM Cloud object
        with additional layers added to the TOA of the input.
 
  CALLING SEQUENCE:
        cld_out = CRTM_Cloud_AddLayerCopy( cld, n_Added_Layers )
 
  OBJECTS:
        cld:             Cloud structure to copy.
                         UNITS:      N/A
                         TYPE:       CRTM_Cloud_type
                         DIMENSION:  Scalar or any rank
                         ATTRIBUTES: INTENT(OUT)
 
  INPUTS:
        n_Added_Layers:  Number of layers to add to the function result.
                         UNITS:      N/A
                         TYPE:       INTEGER
                         DIMENSION:  Same as Cloud object
                         ATTRIBUTES: INTENT(IN)
 
  FUNCTION RESULT:
        cld_out:         Copy of the input Cloud structure with space for
                         extra layers added to TOA.
                         UNITS:      N/A
                         TYPE:       CRTM_Cloud_type
                         DIMENSION:  Same as input.
                         ATTRIBUTES: INTENT(OUT)
 
 
  \end{alltt}

\subsection{\texttt{CRTM\_Cloud\_Associated} interface}
  \label{sec:CRTM_Cloud_Associated_interface}
  \begin{alltt}
 
  NAME:
        CRTM_Cloud_Associated
 
  PURPOSE:
        Elemental function to test the status of the allocatable components
        of a CRTM Cloud object.
 
  CALLING SEQUENCE:
        Status = CRTM_Cloud_Associated( Cloud )
 
  OBJECTS:
        Cloud:   Cloud structure which is to have its member's
                 status tested.
                 UNITS:      N/A
                 TYPE:       CRTM_Cloud_type
                 DIMENSION:  Scalar or any rank
                 ATTRIBUTES: INTENT(IN)
 
  FUNCTION RESULT:
        Status:  The return value is a logical value indicating the
                 status of the Cloud members.
                   .TRUE.  - if the array components are allocated.
                   .FALSE. - if the array components are not allocated.
                 UNITS:      N/A
                 TYPE:       LOGICAL
                 DIMENSION:  Same as input Cloud argument
 
  \end{alltt}

\subsection{\texttt{CRTM\_Cloud\_Compare} interface}
  \label{sec:CRTM_Cloud_Compare_interface}
  \begin{alltt}
  NAME:
        CRTM_Cloud_Compare
 
  PURPOSE:
        Elemental function to compare two CRTM_Cloud objects to within
        a user specified number of significant figures.
 
  CALLING SEQUENCE:
        is_comparable = CRTM_Cloud_Compare( x, y, n_SigFig=n_SigFig )
 
  OBJECTS:
        x, y:          Two CRTM Cloud objects to be compared.
                       UNITS:      N/A
                       TYPE:       CRTM_Cloud_type
                       DIMENSION:  Scalar or any rank
                       ATTRIBUTES: INTENT(IN)
 
  OPTIONAL INPUTS:
        n_SigFig:      Number of significant figure to compare floating point
                       components.
                       UNITS:      N/A
                       TYPE:       INTEGER
                       DIMENSION:  Scalar or same as input
                       ATTRIBUTES: INTENT(IN), OPTIONAL
 
  FUNCTION RESULT:
        is_equal:      Logical value indicating whether the inputs are equal.
                       UNITS:      N/A
                       TYPE:       LOGICAL
                       DIMENSION:  Same as inputs.
  \end{alltt}

\subsection{\texttt{CRTM\_Cloud\_Create} interface}
  \label{sec:CRTM_Cloud_Create_interface}
  \begin{alltt}
 
  NAME:
        CRTM_Cloud_Create
  
  PURPOSE:
        Elemental subroutine to create an instance of the CRTM Cloud object.
 
  CALLING SEQUENCE:
        CALL CRTM_Cloud_Create( Cloud, n_Layers )
 
  OBJECTS:
        Cloud:        Cloud structure.
                      UNITS:      N/A
                      TYPE:       CRTM_Cloud_type
                      DIMENSION:  Scalar or any rank
                      ATTRIBUTES: INTENT(OUT)
 
  INPUTS:
        n_Layers:     Number of layers for which there is cloud data.
                      Must be > 0.
                      UNITS:      N/A
                      TYPE:       INTEGER
                      DIMENSION:  Same as Cloud object
                      ATTRIBUTES: INTENT(IN)
 
  \end{alltt}

\subsection{\texttt{CRTM\_Cloud\_DefineVersion} interface}
  \label{sec:CRTM_Cloud_DefineVersion_interface}
  \begin{alltt}
 
  NAME:
        CRTM_Cloud_DefineVersion
 
  PURPOSE:
        Subroutine to return the module version information.
 
  CALLING SEQUENCE:
        CALL CRTM_Cloud_DefineVersion( Id )
 
  OUTPUTS:
        Id:            Character string containing the version Id information
                       for the module.
                       UNITS:      N/A
                       TYPE:       CHARACTER(*)
                       DIMENSION:  Scalar
                       ATTRIBUTES: INTENT(OUT)
 
  \end{alltt}

\subsection{\texttt{CRTM\_Cloud\_Destroy} interface}
  \label{sec:CRTM_Cloud_Destroy_interface}
  \begin{alltt}
 
  NAME:
        CRTM_Cloud_Destroy
 
  PURPOSE:
        Elemental subroutine to re-initialize CRTM Cloud objects.
 
  CALLING SEQUENCE:
        CALL CRTM_Cloud_Destroy( Cloud )
 
  OBJECTS:
        Cloud:        Re-initialized Cloud structure.
                      UNITS:      N/A
                      TYPE:       CRTM_Cloud_type
                      DIMENSION:  Scalar OR any rank
                      ATTRIBUTES: INTENT(OUT)
 
  \end{alltt}

\subsection{\texttt{CRTM\_Cloud\_Inspect} interface}
  \label{sec:CRTM_Cloud_Inspect_interface}
  \begin{alltt}
 
  NAME:
        CRTM_Cloud_Inspect
 
  PURPOSE:
        Subroutine to print the contents of a CRTM Cloud object to stdout.
 
  CALLING SEQUENCE:
        CALL CRTM_Cloud_Inspect( Cloud )
 
  INPUTS:
        Cloud:         CRTM Cloud object to display.
                       UNITS:      N/A
                       TYPE:       CRTM_Cloud_type
                       DIMENSION:  Scalar, Rank-1, or Rank-2 array
                       ATTRIBUTES: INTENT(IN)
 
  \end{alltt}

\subsection{\texttt{CRTM\_Cloud\_IsValid} interface}
  \label{sec:CRTM_Cloud_IsValid_interface}
  \begin{alltt}
 
  NAME:
        CRTM_Cloud_IsValid
 
  PURPOSE:
        Non-pure function to perform some simple validity checks on a
        CRTM Cloud object.
 
        If invalid data is found, a message is printed to stdout.
 
  CALLING SEQUENCE:
        result = CRTM_Cloud_IsValid( cloud )
 
          or
 
        IF ( CRTM_Cloud_IsValid( cloud ) ) THEN....
 
  OBJECTS:
        cloud:         CRTM Cloud object which is to have its
                       contents checked.
                       UNITS:      N/A
                       TYPE:       CRTM_Cloud_type
                       DIMENSION:  Scalar
                       ATTRIBUTES: INTENT(IN)
 
  FUNCTION RESULT:
        result:        Logical variable indicating whether or not the input
                       passed the check.
                       If == .FALSE., Cloud object is unused or contains
                                      invalid data.
                          == .TRUE.,  Cloud object can be used in CRTM.
                       UNITS:      N/A
                       TYPE:       LOGICAL
                       DIMENSION:  Scalar
 
  \end{alltt}

\subsection{\texttt{CRTM\_Cloud\_Zero} interface}
  \label{sec:CRTM_Cloud_Zero_interface}
  \begin{alltt}
 
  NAME:
        CRTM_Cloud_Zero
 
  PURPOSE:
        Elemental subroutine to zero out the data arrays in a CRTM Cloud object.
 
  CALLING SEQUENCE:
        CALL CRTM_Cloud_Zero( Cloud )
 
  OBJECTS:
        Cloud:         CRTM Cloud structure in which the data arrays are
                       to be zeroed out.
                       UNITS:      N/A
                       TYPE:       CRTM_Cloud_type
                       DIMENSION:  Scalar or any rank
                       ATTRIBUTES: INTENT(IN OUT)
 
  COMMENTS:
        - The dimension components of the structure are *NOT* set to zero.
        - The cloud type component is *NOT* reset.
 
  \end{alltt}

\subsection{\texttt{CRTM\_Cloud\_IOVersion} interface}
  \label{sec:CRTM_Cloud_IOVersion_interface}
  \begin{alltt}
 
  NAME:
        CRTM_Cloud_IOVersion
 
  PURPOSE:
        Subroutine to return the module version information.
 
  CALLING SEQUENCE:
        CALL CRTM_Cloud_IOVersion( Id )
 
  OUTPUT ARGUMENTS:
        Id:            Character string containing the version Id information
                       for the module.
                       UNITS:      N/A
                       TYPE:       CHARACTER(*)
                       DIMENSION:  Scalar
                       ATTRIBUTES: INTENT(OUT)
 
  \end{alltt}

\subsection{\texttt{CRTM\_Cloud\_InquireFile} interface}
  \label{sec:CRTM_Cloud_InquireFile_interface}
  \begin{alltt}
 
  NAME:
        CRTM_Cloud_InquireFile
 
  PURPOSE:
        Function to inquire CRTM Cloud object files.
 
  CALLING SEQUENCE:
        Error_Status = CRTM_Cloud_InquireFile( Filename           , &
                                               n_Clouds = n_Clouds  )
 
  INPUTS:
        Filename:       Character string specifying the name of a
                        CRTM Cloud data file to read.
                        UNITS:      N/A
                        TYPE:       CHARACTER(*)
                        DIMENSION:  Scalar
                        ATTRIBUTES: INTENT(IN)
 
  OPTIONAL OUTPUTS:
        n_Clouds:       The number of Cloud profiles in the data file.
                        UNITS:      N/A
                        TYPE:       INTEGER
                        DIMENSION:  Scalar
                        ATTRIBUTES: OPTIONAL, INTENT(OUT)
 
  FUNCTION RESULT:
        Error_Status:   The return value is an integer defining the error status.
                        The error codes are defined in the Message_Handler module.
                        If == SUCCESS, the file inquire was successful
                           == FAILURE, an unrecoverable error occurred.
                        UNITS:      N/A
                        TYPE:       INTEGER
                        DIMENSION:  Scalar
 
  \end{alltt}

\subsection{\texttt{CRTM\_Cloud\_ReadFile} interface}
  \label{sec:CRTM_Cloud_ReadFile_interface}
  \begin{alltt}
 
  NAME:
        CRTM_Cloud_ReadFile
 
  PURPOSE:
        Function to read CRTM Cloud object files.
 
  CALLING SEQUENCE:
        Error_Status = CRTM_Cloud_ReadFile( Filename           , &
                                            Cloud              , &
                                            Quiet    = Quiet   , &
                                            No_Close = No_Close, &
                                            n_Clouds = n_Clouds  )
 
  INPUTS:
        Filename:       Character string specifying the name of a
                        Cloud format data file to read.
                        UNITS:      N/A
                        TYPE:       CHARACTER(*)
                        DIMENSION:  Scalar
                        ATTRIBUTES: INTENT(IN)
 
  OUTPUTS:
        Cloud:          CRTM Cloud object array containing the Cloud data.
                        UNITS:      N/A
                        TYPE:       CRTM_Cloud_type
                        DIMENSION:  Rank-1
                        ATTRIBUTES: INTENT(OUT)
 
  OPTIONAL INPUTS:
        Quiet:          Set this logical argument to suppress INFORMATION
                        messages being printed to stdout
                        If == .FALSE., INFORMATION messages are OUTPUT [DEFAULT].
                           == .TRUE.,  INFORMATION messages are SUPPRESSED.
                        If not specified, default is .FALSE.
                        UNITS:      N/A
                        TYPE:       LOGICAL
                        DIMENSION:  Scalar
                        ATTRIBUTES: INTENT(IN), OPTIONAL
 
        No_Close:       Set this logical argument to NOT close the file upon exit.
                        If == .FALSE., the input file is closed upon exit [DEFAULT]
                           == .TRUE.,  the input file is NOT closed upon exit.
                        If not specified, default is .FALSE.
                        UNITS:      N/A
                        TYPE:       LOGICAL
                        DIMENSION:  Scalar
                        ATTRIBUTES: INTENT(IN), OPTIONAL
 
  OPTIONAL OUTPUTS:
        n_Clouds:       The actual number of cloud profiles read in.
                        UNITS:      N/A
                        TYPE:       INTEGER
                        DIMENSION:  Scalar
                        ATTRIBUTES: OPTIONAL, INTENT(OUT)
 
  FUNCTION RESULT:
        Error_Status:   The return value is an integer defining the error status.
                        The error codes are defined in the Message_Handler module.
                        If == SUCCESS, the file read was successful
                           == FAILURE, an unrecoverable error occurred.
                        UNITS:      N/A
                        TYPE:       INTEGER
                        DIMENSION:  Scalar
 
  \end{alltt}

\subsection{\texttt{CRTM\_Cloud\_WriteFile} interface}
  \label{sec:CRTM_Cloud_WriteFile_interface}
  \begin{alltt}
 
  NAME:
        CRTM_Cloud_WriteFile
 
  PURPOSE:
        Function to write CRTM Cloud object files.
 
  CALLING SEQUENCE:
        Error_Status = CRTM_Cloud_WriteFile( Filename           , &
                                             Cloud              , &
                                             Quiet    = Quiet   , &
                                             No_Close = No_Close  )
 
  INPUTS:
        Filename:       Character string specifying the name of the
                        Cloud format data file to write.
                        UNITS:      N/A
                        TYPE:       CHARACTER(*)
                        DIMENSION:  Scalar
                        ATTRIBUTES: INTENT(IN)
 
        Cloud:          CRTM Cloud object array containing the Cloud data.
                        UNITS:      N/A
                        TYPE:       CRTM_Cloud_type
                        DIMENSION:  Rank-1
                        ATTRIBUTES: INTENT(IN)
 
  OPTIONAL INPUTS:
        Quiet:          Set this logical argument to suppress INFORMATION
                        messages being printed to stdout
                        If == .FALSE., INFORMATION messages are OUTPUT [DEFAULT].
                           == .TRUE.,  INFORMATION messages are SUPPRESSED.
                        If not specified, default is .FALSE.
                        UNITS:      N/A
                        TYPE:       LOGICAL
                        DIMENSION:  Scalar
                        ATTRIBUTES: INTENT(IN), OPTIONAL
 
        No_Close:       Set this logical argument to NOT close the file upon exit.
                        If == .FALSE., the input file is closed upon exit [DEFAULT]
                           == .TRUE.,  the input file is NOT closed upon exit.
                        If not specified, default is .FALSE.
                        UNITS:      N/A
                        TYPE:       LOGICAL
                        DIMENSION:  Scalar
                        ATTRIBUTES: INTENT(IN), OPTIONAL
 
  FUNCTION RESULT:
        Error_Status:   The return value is an integer defining the error status.
                        The error codes are defined in the Message_Handler module.
                        If == SUCCESS, the file write was successful
                           == FAILURE, an unrecoverable error occurred.
                        UNITS:      N/A
                        TYPE:       INTEGER
                        DIMENSION:  Scalar
 
  SIDE EFFECTS:
        - If the output file already exists, it is overwritten.
        - If an error occurs during *writing*, the output file is deleted before
          returning to the calling routine.
 
  \end{alltt}



\clearpage
\section{\Aerosol{} Structure}
%=============================
\label{sec:aerosol_structure}

\begin{figure}[htp]
  \centering
  \doublebox{
  \begin{minipage}[b]{6.5in}
    \begin{alltt}
  TYPE :: CRTM_Aerosol_type
    ! Allocation indicator
    LOGICAL :: Is_Allocated = .FALSE.
    ! Dimension values
    INTEGER :: Max_Layers = 0  ! K dimension.
    INTEGER :: n_Layers   = 0  ! Kuse dimension
    ! Number of added layers
    INTEGER :: n_Added_Layers = 0
    ! Aerosol type
    INTEGER :: Type = INVALID_AEROSOL
    ! Aerosol state variables
    REAL(fp), ALLOCATABLE :: Effective_Radius(:)  ! K. Units are microns
    REAL(fp), ALLOCATABLE :: Concentration(:)     ! K. Units are kg/m^2  
  END TYPE CRTM_Aerosol_type
    \end{alltt}
  \end{minipage}
  }
  \caption{CRTM\_Aerosol\_type structure definition.}
  \label{fig:CRTM_Aerosol_type_structure}
\end{figure}


% Aerosol type table
\begin{table}[htp]
  \centering
  \begin{tabular}{c l r@{ - }l}
    \hline
    \sffamily\textbf{Aerosol Type} & \sffamily\textbf{Parameter}  & \multicolumn{2}{c}{$r_{eff}$ \sffamily\textbf{Range} (\micron)} \\
    \hline\hline
    Dust           & \texttt{DUST\_AEROSOL}            & 0.01  & 8 \\
    Sea salt SSAM  & \texttt{SEASALT\_SSAM\_AEROSOL}   & 0.3   & 1.45 \\
    Sea salt SSCM1 & \texttt{SEASALT\_SSCM1\_AEROSOL}  & 1.0   & 4.8  \\
    Sea salt SSCM2 & \texttt{SEASALT\_SSCM2\_AEROSOL}  & 3.25  & 17.3 \\
    Sea salt SSCM3 & \texttt{SEASALT\_SSCM3\_AEROSOL}  & 7.5   & 89\\
    Organic carbon & \texttt{ORGANIC\_CARBON\_AEROSOL} & 0.09  & 0.21 \\
    Black carbon   & \texttt{BLACK\_CARBON\_AEROSOL}   & 0.036 & 0.074 \\
    Sulfate        & \texttt{SULFATE\_AEROSOL}         & 0.24  & 0.8 \\
    \hline
  \end{tabular}
  \caption{CRTM \Aerosol{} structure valid \texttt{Type} definitions and effective radii. SSAM $\equiv$ Sea Salt Accumulation Mode, SSCM $\equiv$ Sea Salt Coarse Mode.}
  \label{tab:aerosol_type}
\end{table}

% Aerosol structure methods
%--------------------------
\clearpage
\subsection{\texttt{CRTM\_Aerosol\_AddLayerCopy} interface}
  \label{sec:CRTM_Aerosol_AddLayerCopy_interface}
  \begin{alltt}
 
  NAME:
        CRTM_Aerosol_AddLayerCopy
  
  PURPOSE:
        Elemental function to copy an instance of the CRTM Aerosol object
        with additional layers added to the TOA of the input.
 
  CALLING SEQUENCE:
        aer_out = CRTM_Aerosol_AddLayerCopy( aer, n_Added_Layers )
 
  OBJECTS:
        aer:             Aerosol structure to copy.
                         UNITS:      N/A
                         TYPE:       CRTM_Aerosol_type
                         DIMENSION:  Scalar or any rank
                         ATTRIBUTES: INTENT(OUT)
 
  INPUTS:
        n_Added_Layers:  Number of layers to add to the function result.
                         UNITS:      N/A
                         TYPE:       INTEGER
                         DIMENSION:  Same as Aerosol object
                         ATTRIBUTES: INTENT(IN)
 
  FUNCTION RESULT:
        aer_out:         Copy of the input Aerosol structure with space for
                         extra layers added to TOA.
                         UNITS:      N/A
                         TYPE:       CRTM_Aerosol_type
                         DIMENSION:  Same as input.
                         ATTRIBUTES: INTENT(OUT)
 
 
  \end{alltt}

\subsection{\texttt{CRTM\_Aerosol\_Associated} interface}
  \label{sec:CRTM_Aerosol_Associated_interface}
  \begin{alltt}
 
  NAME:
        CRTM_Aerosol_Associated
 
  PURPOSE:
        Elemental function to test the status of the allocatable components
        of a CRTM Aerosol object.
 
  CALLING SEQUENCE:
        Status = CRTM_Aerosol_Associated( Aerosol )
 
  OBJECTS:
        Aerosol: Aerosol structure which is to have its member's
                 status tested.
                 UNITS:      N/A
                 TYPE:       CRTM_Aerosol_type
                 DIMENSION:  Scalar or any rank
                 ATTRIBUTES: INTENT(IN)
 
  FUNCTION RESULT:
        Status:  The return value is a logical value indicating the
                 status of the Aerosol members.
                   .TRUE.  - if the array components are allocated.
                   .FALSE. - if the array components are not allocated.
                 UNITS:      N/A
                 TYPE:       LOGICAL
                 DIMENSION:  Same as input Aerosol argument
 
  \end{alltt}

\subsection{\texttt{CRTM\_Aerosol\_Compare} interface}
  \label{sec:CRTM_Aerosol_Compare_interface}
  \begin{alltt}
  NAME:
        CRTM_Aerosol_Compare
 
  PURPOSE:
        Elemental function to compare two CRTM_Aerosol objects to within
        a user specified number of significant figures.
 
  CALLING SEQUENCE:
        is_comparable = CRTM_Aerosol_Compare( x, y, n_SigFig=n_SigFig )
 
  OBJECTS:
        x, y:          Two CRTM Aerosol objects to be compared.
                       UNITS:      N/A
                       TYPE:       CRTM_Aerosol_type
                       DIMENSION:  Scalar or any rank
                       ATTRIBUTES: INTENT(IN)
 
  OPTIONAL INPUTS:
        n_SigFig:      Number of significant figure to compare floating point
                       components.
                       UNITS:      N/A
                       TYPE:       INTEGER
                       DIMENSION:  Scalar or same as input
                       ATTRIBUTES: INTENT(IN), OPTIONAL
 
  FUNCTION RESULT:
        is_equal:      Logical value indicating whether the inputs are equal.
                       UNITS:      N/A
                       TYPE:       LOGICAL
                       DIMENSION:  Same as inputs.
  \end{alltt}

\subsection{\texttt{CRTM\_Aerosol\_Create} interface}
  \label{sec:CRTM_Aerosol_Create_interface}
  \begin{alltt}
 
  NAME:
        CRTM_Aerosol_Create
  
  PURPOSE:
        Elemental subroutine to create an instance of the CRTM Aerosol object.
 
  CALLING SEQUENCE:
        CALL CRTM_Aerosol_Create( Aerosol, n_Layers )
 
  OBJECTS:
        Aerosol:      Aerosol structure.
                      UNITS:      N/A
                      TYPE:       CRTM_Aerosol_type
                      DIMENSION:  Scalar or any rank
                      ATTRIBUTES: INTENT(OUT)
 
  INPUTS:
        n_Layers:     Number of layers for which there is Aerosol data.
                      Must be > 0.
                      UNITS:      N/A
                      TYPE:       INTEGER
                      DIMENSION:  Same as Aerosol object
                      ATTRIBUTES: INTENT(IN)
 
  \end{alltt}

\subsection{\texttt{CRTM\_Aerosol\_DefineVersion} interface}
  \label{sec:CRTM_Aerosol_DefineVersion_interface}
  \begin{alltt}
 
  NAME:
        CRTM_Aerosol_DefineVersion
 
  PURPOSE:
        Subroutine to return the module version information.
 
  CALLING SEQUENCE:
        CALL CRTM_Aerosol_DefineVersion( Id )
 
  OUTPUTS:
        Id:   Character string containing the version Id information
              for the module.
              UNITS:      N/A
              TYPE:       CHARACTER(*)
              DIMENSION:  Scalar
              ATTRIBUTES: INTENT(OUT)
 
  \end{alltt}

\subsection{\texttt{CRTM\_Aerosol\_Destroy} interface}
  \label{sec:CRTM_Aerosol_Destroy_interface}
  \begin{alltt}
 
  NAME:
        CRTM_Aerosol_Destroy
 
  PURPOSE:
        Elemental subroutine to re-initialize CRTM Aerosol objects.
 
  CALLING SEQUENCE:
        CALL CRTM_Aerosol_Destroy( Aerosol )
 
  OBJECTS:
        Aerosol:      Re-initialized Aerosol structure.
                      UNITS:      N/A
                      TYPE:       CRTM_Aerosol_type
                      DIMENSION:  Scalar OR any rank
                      ATTRIBUTES: INTENT(OUT)
 
  \end{alltt}

\subsection{\texttt{CRTM\_Aerosol\_Inspect} interface}
  \label{sec:CRTM_Aerosol_Inspect_interface}
  \begin{alltt}
 
  NAME:
        CRTM_Aerosol_Inspect
 
  PURPOSE:
        Subroutine to print the contents of a CRTM Aerosol object to stdout.
 
  CALLING SEQUENCE:
        CALL CRTM_Aerosol_Inspect( Aerosol )
 
  INPUTS:
        Aerosol:       CRTM Aerosol object to display.
                       UNITS:      N/A
                       TYPE:       CRTM_Aerosol_type
                       DIMENSION:  Scalar, Rank-1, or Rank-2 array
                       ATTRIBUTES: INTENT(IN)
 
  \end{alltt}

\subsection{\texttt{CRTM\_Aerosol\_IsValid} interface}
  \label{sec:CRTM_Aerosol_IsValid_interface}
  \begin{alltt}
 
  NAME:
        CRTM_Aerosol_IsValid
 
  PURPOSE:
        Non-pure function to perform some simple validity checks on a
        CRTM Aerosol object.
 
        If invalid data is found, a message is printed to stdout.
 
  CALLING SEQUENCE:
        result = CRTM_Aerosol_IsValid( Aerosol )
 
          or
 
        IF ( CRTM_Aerosol_IsValid( Aerosol ) ) THEN....
 
  OBJECTS:
        Aerosol:       CRTM Aerosol object which is to have its
                       contents checked.
                       UNITS:      N/A
                       TYPE:       CRTM_Aerosol_type
                       DIMENSION:  Scalar
                       ATTRIBUTES: INTENT(IN)
 
  FUNCTION RESULT:
        result:        Logical variable indicating whether or not the input
                       passed the check.
                       If == .FALSE., Aerosol object is unused or contains
                                      invalid data.
                          == .TRUE.,  Aerosol object can be used in CRTM.
                       UNITS:      N/A
                       TYPE:       LOGICAL
                       DIMENSION:  Scalar
 
  \end{alltt}

\subsection{\texttt{CRTM\_Aerosol\_Zero} interface}
  \label{sec:CRTM_Aerosol_Zero_interface}
  \begin{alltt}
 
  NAME:
        CRTM_Aerosol_Zero
 
  PURPOSE:
        Elemental subroutine to zero out the data arrays in a CRTM Aerosol object.
 
  CALLING SEQUENCE:
        CALL CRTM_Aerosol_Zero( Aerosol )
 
  OBJECTS:
        Aerosol:       CRTM Aerosol object in which the data arrays are
                       to be zeroed out.
                       UNITS:      N/A
                       TYPE:       CRTM_Aerosol_type
                       DIMENSION:  Scalar or any rank
                       ATTRIBUTES: INTENT(IN OUT)
 
  COMMENTS:
        - The dimension components of the structure are *NOT* set to zero.
        - The Aerosol type component is *NOT* reset.
 
  \end{alltt}

\subsection{\texttt{CRTM\_Aerosol\_IOVersion} interface}
  \label{sec:CRTM_Aerosol_IOVersion_interface}
  \begin{alltt}
 
  NAME:
        CRTM_Aerosol_IOVersion
 
  PURPOSE:
        Subroutine to return the module version information.
 
  CALLING SEQUENCE:
        CALL CRTM_Aerosol_IOVersion( Id )
 
  OUTPUT ARGUMENTS:
        Id:   Character string containing the version Id information
              for the module.
              UNITS:      N/A
              TYPE:       CHARACTER(*)
              DIMENSION:  Scalar
              ATTRIBUTES: INTENT(OUT)
 
  \end{alltt}

\subsection{\texttt{CRTM\_Aerosol\_InquireFile} interface}
  \label{sec:CRTM_Aerosol_InquireFile_interface}
  \begin{alltt}
 
  NAME:
        CRTM_Aerosol_InquireFile
 
  PURPOSE:
        Function to inquire CRTM Aerosol object files.
 
  CALLING SEQUENCE:
        Error_Status = CRTM_Aerosol_InquireFile( Filename               , &
                                                 n_Aerosols = n_Aerosols  )
 
  INPUTS:
        Filename:       Character string specifying the name of a
                        CRTM Aerosol data file to read.
                        UNITS:      N/A
                        TYPE:       CHARACTER(*)
                        DIMENSION:  Scalar
                        ATTRIBUTES: INTENT(IN)
 
  OPTIONAL OUTPUTS:
        n_Aerosols:     The number of Aerosol profiles in the data file.
                        UNITS:      N/A
                        TYPE:       INTEGER
                        DIMENSION:  Scalar
                        ATTRIBUTES: OPTIONAL, INTENT(OUT)
 
  FUNCTION RESULT:
        Error_Status:   The return value is an integer defining the error status.
                        The error codes are defined in the Message_Handler module.
                        If == SUCCESS, the file inquire was successful
                           == FAILURE, an unrecoverable error occurred.
                        UNITS:      N/A
                        TYPE:       INTEGER
                        DIMENSION:  Scalar
 
  \end{alltt}

\subsection{\texttt{CRTM\_Aerosol\_ReadFile} interface}
  \label{sec:CRTM_Aerosol_ReadFile_interface}
  \begin{alltt}
 
  NAME:
        CRTM_Aerosol_ReadFile
 
  PURPOSE:
        Function to read CRTM Aerosol object files.
 
  CALLING SEQUENCE:
        Error_Status = CRTM_Aerosol_ReadFile( Filename               , &
                                              Aerosol                , &
                                              Quiet      = Quiet     , &
                                              No_Close   = No_Close  , &
                                              n_Aerosols = n_Aerosols  )
 
  INPUTS:
        Filename:       Character string specifying the name of a
                        Aerosol format data file to read.
                        UNITS:      N/A
                        TYPE:       CHARACTER(*)
                        DIMENSION:  Scalar
                        ATTRIBUTES: INTENT(IN)
 
  OUTPUTS:
        Aerosol:        CRTM Aerosol object array containing the Aerosol data.
                        UNITS:      N/A
                        TYPE:       CRTM_Aerosol_type
                        DIMENSION:  Rank-1
                        ATTRIBUTES: INTENT(OUT)
 
  OPTIONAL INPUTS:
        Quiet:          Set this logical argument to suppress INFORMATION
                        messages being printed to stdout
                        If == .FALSE., INFORMATION messages are OUTPUT [DEFAULT].
                           == .TRUE.,  INFORMATION messages are SUPPRESSED.
                        If not specified, default is .FALSE.
                        UNITS:      N/A
                        TYPE:       LOGICAL
                        DIMENSION:  Scalar
                        ATTRIBUTES: INTENT(IN), OPTIONAL
 
        No_Close:       Set this logical argument to NOT close the file upon exit.
                        If == .FALSE., the input file is closed upon exit [DEFAULT]
                           == .TRUE.,  the input file is NOT closed upon exit.
                        If not specified, default is .FALSE.
                        UNITS:      N/A
                        TYPE:       LOGICAL
                        DIMENSION:  Scalar
                        ATTRIBUTES: INTENT(IN), OPTIONAL
 
  OPTIONAL OUTPUTS:
        n_Aerosols:     The actual number of aerosol profiles read in.
                        UNITS:      N/A
                        TYPE:       INTEGER
                        DIMENSION:  Scalar
                        ATTRIBUTES: OPTIONAL, INTENT(OUT)
 
  FUNCTION RESULT:
        Error_Status:   The return value is an integer defining the error status.
                        The error codes are defined in the Message_Handler module.
                        If == SUCCESS, the file read was successful
                           == FAILURE, an unrecoverable error occurred.
                        UNITS:      N/A
                        TYPE:       INTEGER
                        DIMENSION:  Scalar
 
  \end{alltt}

\subsection{\texttt{CRTM\_Aerosol\_WriteFile} interface}
  \label{sec:CRTM_Aerosol_WriteFile_interface}
  \begin{alltt}
 
  NAME:
        CRTM_Aerosol_WriteFile
 
  PURPOSE:
        Function to write CRTM Aerosol object files.
 
  CALLING SEQUENCE:
        Error_Status = CRTM_Aerosol_WriteFile( Filename           , &
                                               Aerosol            , &
                                               Quiet    = Quiet   , &
                                               No_Close = No_Close  )
 
  INPUTS:
        Filename:       Character string specifying the name of the
                        Aerosol format data file to write.
                        UNITS:      N/A
                        TYPE:       CHARACTER(*)
                        DIMENSION:  Scalar
                        ATTRIBUTES: INTENT(IN)
 
        Aerosol:        CRTM Aerosol object array containing the Aerosol data.
                        UNITS:      N/A
                        TYPE:       CRTM_Aerosol_type
                        DIMENSION:  Rank-1
                        ATTRIBUTES: INTENT(IN)
 
  OPTIONAL INPUTS:
        Quiet:          Set this logical argument to suppress INFORMATION
                        messages being printed to stdout
                        If == .FALSE., INFORMATION messages are OUTPUT [DEFAULT].
                           == .TRUE.,  INFORMATION messages are SUPPRESSED.
                        If not specified, default is .FALSE.
                        UNITS:      N/A
                        TYPE:       LOGICAL
                        DIMENSION:  Scalar
                        ATTRIBUTES: INTENT(IN), OPTIONAL
 
        No_Close:       Set this logical argument to NOT close the file upon exit.
                        If == .FALSE., the input file is closed upon exit [DEFAULT]
                           == .TRUE.,  the input file is NOT closed upon exit.
                        If not specified, default is .FALSE.
                        UNITS:      N/A
                        TYPE:       LOGICAL
                        DIMENSION:  Scalar
                        ATTRIBUTES: INTENT(IN), OPTIONAL
 
  FUNCTION RESULT:
        Error_Status:   The return value is an integer defining the error status.
                        The error codes are defined in the Message_Handler module.
                        If == SUCCESS, the file write was successful
                           == FAILURE, an unrecoverable error occurred.
                        UNITS:      N/A
                        TYPE:       INTEGER
                        DIMENSION:  Scalar
 
  SIDE EFFECTS:
        - If the output file already exists, it is overwritten.
        - If an error occurs during *writing*, the output file is deleted before
          returning to the calling routine.
 
  \end{alltt}




\clearpage
\section{\Surface{} Structure}
%=============================
\label{sec:surface_structure}

\begin{figure}[htp]
  \centering
  \doublebox{
  \begin{minipage}[b]{6.5in}
    \begin{alltt}
  TYPE :: CRTM_Surface_type
    INTEGER :: n_Allocates = 0
    ! Dimension values
    INTEGER :: Max_Sensors  = 0  ! N dimension
    INTEGER :: n_Sensors    = 0  ! Nuse dimension
    ! Gross type of surface determined by coverage
    REAL(fp) :: Land_Coverage  = ZERO
    REAL(fp) :: Water_Coverage = ZERO
    REAL(fp) :: Snow_Coverage  = ZERO
    REAL(fp) :: Ice_Coverage   = ZERO
    ! Surface type independent data
    REAL(fp) :: Wind_Speed     = DEFAULT_WIND_SPEED
    REAL(fp) :: Wind_Direction = DEFAULT_WIND_DIRECTION
    ! Land surface type data
    INTEGER  :: Land_Type             = DEFAULT_LAND_TYPE
    REAL(fp) :: Land_Temperature      = DEFAULT_LAND_TEMPERATURE
    REAL(fp) :: Soil_Moisture_Content = DEFAULT_SOIL_MOISTURE_CONTENT
    REAL(fp) :: Canopy_Water_Content  = DEFAULT_CANOPY_WATER_CONTENT
    REAL(fp) :: Vegetation_Fraction   = DEFAULT_VEGETATION_FRACTION
    REAL(fp) :: Soil_Temperature      = DEFAULT_SOIL_TEMPERATURE
    ! Water type data
    INTEGER  :: Water_Type        = DEFAULT_WATER_TYPE
    REAL(fp) :: Water_Temperature = DEFAULT_WATER_TEMPERATURE
    REAL(fp) :: Salinity          = DEFAULT_SALINITY
    ! Snow surface type data
    INTEGER  :: Snow_Type        = DEFAULT_SNOW_TYPE
    REAL(fp) :: Snow_Temperature = DEFAULT_SNOW_TEMPERATURE
    REAL(fp) :: Snow_Depth       = DEFAULT_SNOW_DEPTH
    REAL(fp) :: Snow_Density     = DEFAULT_SNOW_DENSITY
    REAL(fp) :: Snow_Grain_Size  = DEFAULT_SNOW_GRAIN_SIZE
    ! Ice surface type data
    INTEGER  :: Ice_Type        = DEFAULT_ICE_TYPE
    REAL(fp) :: Ice_Temperature = DEFAULT_ICE_TEMPERATURE
    REAL(fp) :: Ice_Thickness   = DEFAULT_ICE_THICKNESS
    REAL(fp) :: Ice_Density     = DEFAULT_ICE_DENSITY
    REAL(fp) :: Ice_Roughness   = DEFAULT_ICE_ROUGHNESS
    ! SensorData containing channel brightness temperatures
    TYPE(CRTM_SensorData_type) :: SensorData  ! N
  END TYPE CRTM_Surface_type
    \end{alltt}
  \end{minipage}
  }
  \caption{CRTM\_Surface\_type structure definition.}
  \label{fig:CRTM_Surface_type_structure}
\end{figure}


\begin{table}[htp]
  \centering
  \begin{tabular}{l p{7cm} c c}
    \hline
    \sffamily\textbf{Component} & \sffamily\textbf{Description} & \sffamily\textbf{Units} & \sffamily\textbf{Dimensions} \\
    \hline\hline
    \texttt{n\_Sensors} & The number of sensors for which data is provided inside the SensorData structure & N/A & Scalar \\
    \hline
    \texttt{Land\_Coverage}  & Fraction of the FOV that is land surface & N/A & Scalar \\
    \texttt{Water\_Coverage} & Fraction of the FOV that is water surface & N/A & Scalar \\
    \texttt{Snow\_Coverage}  & Fraction of the FOV that is snow surface & N/A & Scalar \\
    \texttt{Ice\_Coverage}   & Fraction of the FOV that is ice surface & N/A & Scalar \\
    \hline
    \texttt{Wind\_Speed}     & Surface wind speed & m.s$^{-1}$ & Scalar \\
    \texttt{Wind\_Direction} & Surface wind direction & deg. E from N & Scalar \\
    \hline
    \texttt{Land\_Type}              & Land surface type & N/A & Scalar \\
    \texttt{Land\_Temperature}       & Land surface temperature & Kelvin & Scalar \\
    \texttt{Soil\_Moisture\_Content} & Volumetric water content of the soil & g.cm$^{-3}$ & Scalar \\
    \texttt{Canopy\_Water\_Content}  & Gravimetric water content of the canopy & g.cm$^{-3}$ & Scalar \\
    \texttt{Vegetation\_Fraction}    & Vegetation fraction of the surface & \% & Scalar \\
    \texttt{Soil\_Temperature}       & Soil temperature & Kelvin & Scalar \\
    \hline
    \texttt{Water\_Type}        & Water surface type & N/A & Scalar \\
    \texttt{Water\_Temperature} & Water surface temperature & Kelvin & Scalar \\
    \texttt{Salinity}           & Water salinity & \textperthousand & Scalar \\
    \hline
    \texttt{Snow\_Type}        & Snow surface type & N/A & Scalar \\ 
    \texttt{Snow\_Temperature} & Snow surface temperature & Kelvin & Scalar \\ 
    \texttt{Snow\_Depth}       & Snow depth & mm & Scalar \\ 
    \texttt{Snow\_Density}     & Snow density & g.m$^{-3}$ & Scalar \\ 
    \texttt{Snow\_Grain\_Size} & Snow grain size & mm & Scalar \\ 
    \hline
    \texttt{Ice\_Type}        & Ice surface type & N/A & Scalar \\ 
    \texttt{Ice\_Temperature} & Ice surface temperature & Kelvin & Scalar \\ 
    \texttt{Ice\_Thickness}   & Thickness of ice & mm & Scalar \\ 
    \texttt{Ice\_Density}     & Density of ice & g.m$^{-3}$ & Scalar \\ 
    \texttt{Ice\_Roughness}   & Measure of the surface roughness of the ice & N/A & Scalar \\ 
    \hline
    \texttt{SensorData} & Satellite sensor data required for some surface emissivity algorithms & N/A & Scalar \\ 
    \hline
  \end{tabular}
  \caption{CRTM \Surface{} structure component description.}
  \label{tab:surface_structure}
\end{table}

% Default surface value table
\begin{table}[htp]
  \centering
  \begin{tabular}{l c c}
    \hline
    \sffamily\textbf{Parameter} & \sffamily\textbf{Value}  & \sffamily\textbf{Units} \\
    \hline\hline
    \multicolumn{3}{c}{\textsf{Surface type independent data}}\\
    \hline
    \texttt{DEFAULT\_WIND\_SPEED}             & 5.0        & m.s$^{-1}$\\
    \texttt{DEFAULT\_WIND\_DIRECTION}         & 0.0        & deg. E from N\\[0.2cm]
    \multicolumn{3}{c}{\textsf{Land surface type data}}\\
    \hline
    \texttt{DEFAULT\_LAND\_TYPE}              & \texttt{GRASS\_SOIL}& N/A \\
    \texttt{DEFAULT\_LAND\_TEMPERATURE}       & 283.0      & K\\
    \texttt{DEFAULT\_SOIL\_MOISTURE\_CONTENT} & 0.05       & g.cm$^{-3}$\\
    \texttt{DEFAULT\_CANOPY\_WATER\_CONTENT}  & 0.05       & g.cm$^{-3}$\\
    \texttt{DEFAULT\_VEGETATION\_FRACTION}    & 0.3        & 30\%\\
    \texttt{DEFAULT\_SOIL\_TEMPERATURE}       & 283.0      & K\\[0.2cm]
    \multicolumn{3}{c}{\textsf{Water type data}}\\
    \hline
    \texttt{DEFAULT\_WATER\_TYPE}             & \texttt{SEA\_WATER} & N/A\\
    \texttt{DEFAULT\_WATER\_TEMPERATURE}      & 283.0      & K\\
    \texttt{DEFAULT\_SALINITY}                & 33.0       & ppmv\\[0.2cm]
    \multicolumn{3}{c}{\textsf{Snow surface type data}}\\
    \hline
    \texttt{DEFAULT\_SNOW\_TYPE}              & \texttt{NEW\_SNOW}  & N/A\\
    \texttt{DEFAULT\_SNOW\_TEMPERATURE}       & 263.0      & K\\
    \texttt{DEFAULT\_SNOW\_DEPTH}             & 50.0       & mm\\
    \texttt{DEFAULT\_SNOW\_DENSITY}           & 0.2        & g.cm$^{-3}$\\
    \texttt{DEFAULT\_SNOW\_GRAIN\_SIZE}       & 2.0        & mm\\[0.2cm]
    \multicolumn{3}{c}{\textsf{Ice surface type data}}\\
    \hline
    \texttt{DEFAULT\_ICE\_TYPE}               & \texttt{FRESH\_ICE} & N/A\\
    \texttt{DEFAULT\_ICE\_TEMPERATURE}        & 263.0      & K\\
    \texttt{DEFAULT\_ICE\_THICKNESS}          & 10.0       & mm\\
    \texttt{DEFAULT\_ICE\_DENSITY}            & 0.9        & g.cm$^{-3}$\\
    \texttt{DEFAULT\_ICE\_ROUGHNESS}          & 0.0        & N/A\\
    \hline
  \end{tabular}
  \caption{CRTM \Surface{} structure default values.}
  \label{tab:surface_default}
\end{table}

% Land surface type table
\begin{table}[htp]
  \centering
  \begin{tabular}{c l}
    \hline
    \sffamily\textbf{Land Type} & \sffamily\textbf{Parameter} \\
    \hline\hline
          Compacted soil      & \texttt{COMPACTED\_SOIL} \\
            Tilled soil       & \texttt{TILLED\_SOIL} \\
              Sand            & \texttt{SAND} \\
              Rock            & \texttt{ROCK} \\
     Irrigated low vegetation & \texttt{IRRIGATED\_LOW\_VEGETATION} \\
           Meadow grass       & \texttt{MEADOW\_GRASS} \\
              Scrub           & \texttt{SCRUB} \\
         Broadleaf forest     & \texttt{BROADLEAF\_FOREST} \\
           Pine forest        & \texttt{PINE\_FOREST} \\
             Tundra           & \texttt{TUNDRA} \\
           Grass-soil         & \texttt{GRASS\_SOIL} \\
       Broadleaf-pine forest  & \texttt{BROADLEAF\_PINE\_FOREST} \\
           Grass scrub        & \texttt{GRASS\_SCRUB} \\
         Soil-grass-scrub     & \texttt{SOIL\_GRASS\_SCRUB} \\
          Urban concrete      & \texttt{URBAN\_CONCRETE} \\
            Pine brush        & \texttt{PINE\_BRUSH} \\
          Broadleaf brush     & \texttt{BROADLEAF\_BRUSH} \\
             Wet soil         & \texttt{WET\_SOIL} \\
            Scrub-soil        & \texttt{SCRUB\_SOIL} \\
      Broadleaf(70)-Pine(30)  & \texttt{BROADLEAF70\_PINE30} \\
    \hline
  \end{tabular}
  \caption{CRTM \Surface{} structure valid \texttt{Land\_Type} definitions.}
  \label{tab:surface_land_type}
\end{table}

% Water surface type table
\begin{table}[htp]
  \centering
  \begin{tabular}{cc l}
    \hline
    \sffamily\textbf{Water Type} & \hspace{0.5cm} & \sffamily\textbf{Parameter} \\
    \hline\hline
      Sea water  & \hspace{0.5cm} &  \texttt{SEA\_WATER} \\     
     Fresh water & \hspace{0.5cm} &  \texttt{FRESH\_WATER} \\   
    \hline
  \end{tabular}
  \caption{CRTM \Surface{} structure valid \texttt{Water\_Type} definitions.}
  \label{tab:surface_water_type}
\end{table}

% Snow surface type table
\begin{table}[htp]
  \centering
  \begin{tabular}{c l}
    \hline
    \sffamily\textbf{Snow Type} & \sffamily\textbf{Parameter} \\
    \hline\hline
         Wet snow          &   \texttt{WET\_SNOW} \\           
      Grass after snow     &   \texttt{GRASS\_AFTER\_SNOW} \\   
        Powder snow        &   \texttt{POWDER\_SNOW} \\        
         RS snow(A)        &   \texttt{RS\_SNOW\_A} \\          
         RS snow(B)        &   \texttt{RS\_SNOW\_B} \\          
         RS snow(C)        &   \texttt{RS\_SNOW\_C} \\          
         RS snow(D)        &   \texttt{RS\_SNOW\_D} \\          
         RS snow(E)        &   \texttt{RS\_SNOW\_E} \\          
      Thin Crust snow      &   \texttt{THIN\_CRUST\_SNOW} \\    
      Thick crust snow     &   \texttt{THICK\_CRUST\_SNOW } \\  
        Shallow snow       &   \texttt{SHALLOW\_SNOW} \\       
         Deep snow         &   \texttt{DEEP\_SNOW} \\          
        Crust snow         &   \texttt{CRUST\_SNOW} \\         
        Medium snow        &   \texttt{MEDIUM\_SNOW} \\        
     Bottom crust snow(A)  &   \texttt{BOTTOM\_CRUST\_SNOW\_A} \\
     Bottom crust snow(B)  &   \texttt{BOTTOM\_CRUST\_SNOW\_B} \\
    \hline
  \end{tabular}
  \caption{CRTM \Surface{} structure valid \texttt{Snow\_Type} definitions.}
  \label{tab:surface_snow_type}
\end{table}

% Ice surface type table
\begin{table}[htp]
  \centering
  \begin{tabular}{c l}
    \hline
    \sffamily\textbf{Ice Type} & \sffamily\textbf{Parameter} \\
    \hline\hline
            Fresh ice        &   \texttt{FRESH\_ICE} \\       
        First year sea ice   &   \texttt{FIRST\_YEAR\_SEA\_ICE} \\
      Multiple year sea ice  &   \texttt{MULTI\_YEAR\_SEA\_ICE} \\
            Ice floe         &   \texttt{ICE\_FLOE} \\            
            Ice ridge        &   \texttt{ICE\_RIDGE} \\           
    \hline
  \end{tabular}
  \caption{CRTM \Surface{} structure valid \texttt{Ice\_Type} definitions.}
  \label{tab:surface_ice_type}
\end{table}

% Surface structure methods
%--------------------------
\clearpage
\subsection{\texttt{CRTM\_Surface\_Associated} interface}
  \label{sec:CRTM_Surface_Associated_interface}
  \begin{alltt}
 
  NAME:
        CRTM_Surface_Associated
 
  PURPOSE:
        Elemental function to test the status of the allocatable components
        of a CRTM Surface object.
 
  CALLING SEQUENCE:
        Status = CRTM_Surface_Associated( Sfc )
 
  OBJECTS:
        Sfc:       Surface structure which is to have its member's
                   status tested.
                   UNITS:      N/A
                   TYPE:       CRTM_Surface_type
                   DIMENSION:  Scalar or any rank
                   ATTRIBUTES: INTENT(IN)
 
  FUNCTION RESULT:
        Status:    The return value is a logical value indicating the
                   status of the Surface members.
                     .TRUE.  - if the array components are allocated.
                     .FALSE. - if the array components are not allocated.
                   UNITS:      N/A
                   TYPE:       LOGICAL
                   DIMENSION:  Same as input
 
  \end{alltt}

\subsection{\texttt{CRTM\_Surface\_Compare} interface}
  \label{sec:CRTM_Surface_Compare_interface}
  \begin{alltt}
  NAME:
        CRTM_Surface_Compare
 
  PURPOSE:
        Elemental function to compare two CRTM_Surface objects to within
        a user specified number of significant figures.
 
  CALLING SEQUENCE:
        is_comparable = CRTM_Surface_Compare( x, y, n_SigFig=n_SigFig )
 
  OBJECTS:
        x, y:          Two CRTM Surface objects to be compared.
                       UNITS:      N/A
                       TYPE:       CRTM_Surface_type
                       DIMENSION:  Scalar or any rank
                       ATTRIBUTES: INTENT(IN)
 
  OPTIONAL INPUTS:
        n_SigFig:      Number of significant figure to compare floating point
                       components.
                       UNITS:      N/A
                       TYPE:       INTEGER
                       DIMENSION:  Scalar or same as input
                       ATTRIBUTES: INTENT(IN), OPTIONAL
 
  FUNCTION RESULT:
        is_equal:      Logical value indicating whether the inputs are equal.
                       UNITS:      N/A
                       TYPE:       LOGICAL
                       DIMENSION:  Same as inputs.
  \end{alltt}

\subsection{\texttt{CRTM\_Surface\_CoverageType} interface}
  \label{sec:CRTM_Surface_CoverageType_interface}
  \begin{alltt}
 
  NAME:
        CRTM_Surface_CoverageType
 
  PURPOSE:
        Elemental function to return the gross surface type based on coverage.
 
  CALLING SEQUENCE:
        type = CRTM_Surface_CoverageType( sfc )
 
  INPUTS:
        Sfc:  CRTM Surface object for which the gross surface type is required.
              UNITS:      N/A
              TYPE:       CRTM_Surface_type
              DIMENSION:  Scalar or any rank
              ATTRIBUTES: INTENT(IN)
 
  FUNCTION:
        type: Surface type indicator for the passed CRTM Surface object.
              UNITS:      N/A
              TYPE:       INTEGER
              DIMENSION:  Same as input
 
  COMMENTS:
        For a scalar Surface object, this function result can be used to
        determine what gross surface types are included by using it to
        index the SURFACE_TYPE_NAME parameter arrays, e.g.
 
          WRITE(*,*) SURFACE_TYPE_NAME(CRTM_Surface_CoverageType(sfc))
  \end{alltt}

\subsection{\texttt{CRTM\_Surface\_Create} interface}
  \label{sec:CRTM_Surface_Create_interface}
  \begin{alltt}
 
  NAME:
        CRTM_Surface_Create
 
  PURPOSE:
        Elemental subroutine to create an instance of the CRTM Surface object.
 
  CALLING SEQUENCE:
        CALL CRTM_Surface_Create( Sfc       , &
                                  n_Channels  )
 
  OBJECTS:
        Sfc:          Surface structure.
                      UNITS:      N/A
                      TYPE:       CRTM_Surface_type
                      DIMENSION:  Scalar or any rank
                      ATTRIBUTES: INTENT(OUT)
 
  INPUT ARGUMENTS:
        n_Channels:   Number of channels dimension of SensorData
                      substructure
                      ** Note: Can be = 0 (i.e. no sensor data). **
                      UNITS:      N/A
                      TYPE:       INTEGER
                      DIMENSION:  Same as Surface object
                      ATTRIBUTES: INTENT(IN)
 
  \end{alltt}

\subsection{\texttt{CRTM\_Surface\_DefineVersion} interface}
  \label{sec:CRTM_Surface_DefineVersion_interface}
  \begin{alltt}
 
  NAME:
        CRTM_Surface_DefineVersion
 
  PURPOSE:
        Subroutine to return the module version information.
 
  CALLING SEQUENCE:
        CALL CRTM_Surface_DefineVersion( Id )
 
  OUTPUT ARGUMENTS:
        Id:            Character string containing the version Id information
                       for the module.
                       UNITS:      N/A
                       TYPE:       CHARACTER(*)
                       DIMENSION:  Scalar
                       ATTRIBUTES: INTENT(OUT)
 
  \end{alltt}

\subsection{\texttt{CRTM\_Surface\_Destroy} interface}
  \label{sec:CRTM_Surface_Destroy_interface}
  \begin{alltt}
 
  NAME:
        CRTM_Surface_Destroy
 
  PURPOSE:
        Elemental subroutine to re-initialize CRTM Surface objects.
 
  CALLING SEQUENCE:
        CALL CRTM_Surface_Destroy( Sfc )
 
  OBJECTS:
        Sfc:          Re-initialized Surface structure.
                      UNITS:      N/A
                      TYPE:       CRTM_Surface_type
                      DIMENSION:  Scalar or any rank
                      ATTRIBUTES: INTENT(OUT)
 
  \end{alltt}

\subsection{\texttt{CRTM\_Surface\_Inspect} interface}
  \label{sec:CRTM_Surface_Inspect_interface}
  \begin{alltt}
 
  NAME:
        CRTM_Surface_Inspect
 
  PURPOSE:
        Subroutine to print the contents of a CRTM Surface object to stdout.
 
  CALLING SEQUENCE:
        CALL CRTM_Surface_Inspect( Sfc )
 
  INPUTS:
        Sfc:  CRTM Surface object to display.
              UNITS:      N/A
              TYPE:       CRTM_Surface_type
              DIMENSION:  Scalar
              ATTRIBUTES: INTENT(IN)
 
  \end{alltt}

\subsection{\texttt{CRTM\_Surface\_IsCoverageValid} interface}
  \label{sec:CRTM_Surface_IsCoverageValid_interface}
  \begin{alltt}
 
  NAME:
        CRTM_Surface_IsCoverageValid
 
  PURPOSE:
        Function to determine if the coverage fractions are valid
        for a CRTM Surface object.
 
  CALLING SEQUENCE:
        result = CRTM_Surface_IsCoverageValid( Sfc )
 
  OBJECTS:
        Sfc:       CRTM Surface object which is to have its
                   coverage fractions checked.
                   UNITS:      N/A
                   TYPE:       CRTM_Surface_type
                   DIMENSION:  Scalar
                   ATTRIBUTES: INTENT(IN)
 
  FUNCTION RESULT:
        result:    Logical variable indicating whether or not the input
                   passed the check.
                   If == .FALSE., Surface object coverage fractions are invalid.
                      == .TRUE.,  Surface object coverage fractions are valid.
                   UNITS:      N/A
                   TYPE:       LOGICAL
                   DIMENSION:  Scalar
 
  \end{alltt}

\subsection{\texttt{CRTM\_Surface\_IsValid} interface}
  \label{sec:CRTM_Surface_IsValid_interface}
  \begin{alltt}
 
  NAME:
        CRTM_Surface_IsValid
 
  PURPOSE:
        Non-pure function to perform some simple validity checks on a
        CRTM Surface object.
 
        If invalid data is found, a message is printed to stdout.
 
  CALLING SEQUENCE:
        result = CRTM_Surface_IsValid( Sfc )
 
          or
 
        IF ( CRTM_Surface_IsValid( Sfc ) ) THEN....
 
  OBJECTS:
        Sfc:       CRTM Surface object which is to have its
                   contents checked.
                   UNITS:      N/A
                   TYPE:       CRTM_Surface_type
                   DIMENSION:  Scalar
                   ATTRIBUTES: INTENT(IN)
 
  FUNCTION RESULT:
        result:    Logical variable indicating whether or not the input
                   passed the check.
                   If == .FALSE., Surface object is unused or contains
                                  invalid data.
                      == .TRUE.,  Surface object can be used in CRTM.
                   UNITS:      N/A
                   TYPE:       LOGICAL
                   DIMENSION:  Scalar
 
  \end{alltt}

\subsection{\texttt{CRTM\_Surface\_Zero} interface}
  \label{sec:CRTM_Surface_Zero_interface}
  \begin{alltt}
 
  NAME:
        CRTM_Surface_Zero
 
  PURPOSE:
        Elemental subroutine to zero out the data arrays
        in a CRTM Surface object.
 
  CALLING SEQUENCE:
        CALL CRTM_Surface_Zero( Sfc )
 
  OUTPUT ARGUMENTS:
        Sfc:          CRTM Surface structure in which the data arrays
                      are to be zeroed out.
                      UNITS:      N/A
                      TYPE:       CRTM_Surface_type
                      DIMENSION:  Scalar or any rank
                      ATTRIBUTES: INTENT(IN OUT)
 
  COMMENTS:
        - The various surface type indicator flags are
          *NOT* reset in this routine.
 
  \end{alltt}

\subsection{\texttt{CRTM\_Surface\_IOVersion} interface}
  \label{sec:CRTM_Surface_IOVersion_interface}
  \begin{alltt}
 
  NAME:
        CRTM_Surface_IOVersion
 
  PURPOSE:
        Subroutine to return the module version information.
 
  CALLING SEQUENCE:
        CALL CRTM_Surface_IOVersion( Id )
 
  OUTPUTS:
        Id:            Character string containing the version Id information
                       for the module.
                       UNITS:      N/A
                       TYPE:       CHARACTER(*)
                       DIMENSION:  Scalar
                       ATTRIBUTES: INTENT(OUT)
 
  \end{alltt}

\subsection{\texttt{CRTM\_Surface\_InquireFile} interface}
  \label{sec:CRTM_Surface_InquireFile_interface}
  \begin{alltt}
 
  NAME:
        CRTM_Surface_InquireFile
 
  PURPOSE:
        Function to inquire CRTM Surface object files.
 
  CALLING SEQUENCE:
        Error_Status = CRTM_Surface_InquireFile( Filename               , &
                                                 n_Channels = n_Channels, &
                                                 n_Profiles = n_Profiles  )
 
  INPUTS:
        Filename:       Character string specifying the name of a
                        CRTM Surface data file to read.
                        UNITS:      N/A
                        TYPE:       CHARACTER(*)
                        DIMENSION:  Scalar
                        ATTRIBUTES: INTENT(IN)
 
  OPTIONAL OUTPUTS:
        n_Channels:     The number of spectral channels for which there is
                        data in the file. Note that this value will always
                        be 0 for a profile-only dataset-- it only has meaning
                        for K-matrix data.
                        UNITS:      N/A
                        TYPE:       INTEGER
                        DIMENSION:  Scalar
                        ATTRIBUTES: OPTIONAL, INTENT(OUT)
 
        n_Profiles:     The number of profiles in the data file.
                        UNITS:      N/A
                        TYPE:       INTEGER
                        DIMENSION:  Scalar
                        ATTRIBUTES: OPTIONAL, INTENT(OUT)
 
  FUNCTION RESULT:
        Error_Status:   The return value is an integer defining the error status.
                        The error codes are defined in the Message_Handler module.
                        If == SUCCESS, the file inquire was successful
                           == FAILURE, an unrecoverable error occurred.
                        UNITS:      N/A
                        TYPE:       INTEGER
                        DIMENSION:  Scalar
 
  \end{alltt}

\subsection{\texttt{CRTM\_Surface\_ReadFile} interface}
  \label{sec:CRTM_Surface_ReadFile_interface}
  \begin{alltt}
 
  NAME:
        CRTM_Surface_ReadFile
 
  PURPOSE:
        Function to read CRTM Surface object files.
 
  CALLING SEQUENCE:
        Error_Status = CRTM_Surface_ReadFile( Filename                , &
                                              Surface                 , &
                                              Quiet      = Quiet      , &
                                              n_Channels = n_Channels , &
                                              n_Profiles = n_Profiles , &
 
  INPUTS:
        Filename:     Character string specifying the name of an
                      Surface format data file to read.
                      UNITS:      N/A
                      TYPE:       CHARACTER(*)
                      DIMENSION:  Scalar
                      ATTRIBUTES: INTENT(IN)
 
  OUTPUTS:
        Surface:      CRTM Surface object array containing the Surface
                      data. Note the following meanings attributed to the
                      dimensions of the object array:
                      Rank-1: M profiles.
                              Only profile data are to be read in. The file
                              does not contain channel information. The
                              dimension of the structure is understood to
                              be the PROFILE dimension.
                      Rank-2: L channels  x  M profiles
                              Channel and profile data are to be read in.
                              The file contains both channel and profile
                              information. The first dimension of the 
                              structure is the CHANNEL dimension, the second
                              is the PROFILE dimension. This is to allow
                              K-matrix structures to be read in with the
                              same function.
                      UNITS:      N/A
                      TYPE:       CRTM_Surface_type
                      DIMENSION:  Rank-1 (M) or Rank-2 (L x M)
                      ATTRIBUTES: INTENT(OUT)
 
  OPTIONAL INPUTS:
        Quiet:        Set this logical argument to suppress INFORMATION
                      messages being printed to stdout
                      If == .FALSE., INFORMATION messages are OUTPUT [DEFAULT].
                         == .TRUE.,  INFORMATION messages are SUPPRESSED.
                      If not specified, default is .FALSE.
                      UNITS:      N/A
                      TYPE:       LOGICAL
                      DIMENSION:  Scalar
                      ATTRIBUTES: INTENT(IN), OPTIONAL
 
  OPTIONAL OUTPUTS:
        n_Channels:   The number of channels for which data was read. Note that
                      this value will always be 0 for a profile-only dataset--
                      it only has meaning for K-matrix data.
                      UNITS:      N/A
                      TYPE:       INTEGER
                      DIMENSION:  Scalar
                      ATTRIBUTES: OPTIONAL, INTENT(OUT)
 
        n_Profiles:   The number of profiles for which data was read.
                      UNITS:      N/A
                      TYPE:       INTEGER
                      DIMENSION:  Scalar
                      ATTRIBUTES: OPTIONAL, INTENT(OUT)
 
 
  FUNCTION RESULT:
        Error_Status: The return value is an integer defining the error status.
                      The error codes are defined in the Message_Handler module.
                      If == SUCCESS, the file read was successful
                         == FAILURE, an unrecoverable error occurred.
                      UNITS:      N/A
                      TYPE:       INTEGER
                      DIMENSION:  Scalar
 
  \end{alltt}

\subsection{\texttt{CRTM\_Surface\_WriteFile} interface}
  \label{sec:CRTM_Surface_WriteFile_interface}
  \begin{alltt}
 
  NAME:
        CRTM_Surface_WriteFile
 
  PURPOSE:
        Function to write CRTM Surface object files.
 
  CALLING SEQUENCE:
        Error_Status = CRTM_Surface_WriteFile( Filename     , &
                                               Surface   , &
                                               Quiet = Quiet  )
 
  INPUTS:
        Filename:     Character string specifying the name of the
                      Surface format data file to write.
                      UNITS:      N/A
                      TYPE:       CHARACTER(*)
                      DIMENSION:  Scalar
                      ATTRIBUTES: INTENT(IN)
 
        Surface:      CRTM Surface object array containing the Surface
                      data. Note the following meanings attributed to the
                      dimensions of the Surface array:
                      Rank-1: M profiles.
                              Only profile data are to be read in. The file
                              does not contain channel information. The
                              dimension of the array is understood to
                              be the PROFILE dimension.
                      Rank-2: L channels  x  M profiles
                              Channel and profile data are to be read in.
                              The file contains both channel and profile
                              information. The first dimension of the 
                              array is the CHANNEL dimension, the second
                              is the PROFILE dimension. This is to allow
                              K-matrix structures to be read in with the
                              same function.
                      UNITS:      N/A
                      TYPE:       CRTM_Surface_type
                      DIMENSION:  Rank-1 (M) or Rank-2 (L x M)
                      ATTRIBUTES: INTENT(IN)
 
  OPTIONAL INPUTS:
        Quiet:        Set this logical argument to suppress INFORMATION
                      messages being printed to stdout
                      If == .FALSE., INFORMATION messages are OUTPUT [DEFAULT].
                         == .TRUE.,  INFORMATION messages are SUPPRESSED.
                      If not specified, default is .FALSE.
                      UNITS:      N/A
                      TYPE:       LOGICAL
                      DIMENSION:  Scalar
                      ATTRIBUTES: INTENT(IN), OPTIONAL
 
  FUNCTION RESULT:
        Error_Status: The return value is an integer defining the error status.
                      The error codes are defined in the Message_Handler module.
                      If == SUCCESS, the file write was successful
                         == FAILURE, an unrecoverable error occurred.
                      UNITS:      N/A
                      TYPE:       INTEGER
                      DIMENSION:  Scalar
 
  SIDE EFFECTS:
        - If the output file already exists, it is overwritten.
        - If an error occurs during *writing*, the output file is deleted before
          returning to the calling routine.
 
  \end{alltt}




\clearpage
\section{\SensorData{} Structure}
%=============================
\label{sec:sensordata_structure}

\begin{figure}[htp]
  \centering
  \doublebox{
  \begin{minipage}[b]{6.5in}
    \begin{alltt}
  TYPE :: CRTM_SensorData_type
    ! Allocation indicator
    LOGICAL :: Is_Allocated = .FALSE.
    ! Dimension values
    INTEGER :: n_Channels = 0  ! L
    ! The data sensor IDs
    CHARACTER(STRLEN) :: Sensor_Id        = ' '
    INTEGER           :: WMO_Satellite_ID = INVALID_WMO_SATELLITE_ID
    INTEGER           :: WMO_Sensor_ID    = INVALID_WMO_SENSOR_ID
    ! The sensor channels and brightness temperatures
    INTEGER , ALLOCATABLE :: Sensor_Channel(:)   ! L
    REAL(fp), ALLOCATABLE :: Tb(:)               ! L
  END TYPE CRTM_SensorData_type
    \end{alltt}
  \end{minipage}
  }
  \caption{CRTM\_SensorData\_type structure definition.}
  \label{fig:CRTM_SensorData_type_structure}
\end{figure}


% SensorData component description table
\begin{table}[htp]
  \centering
  \begin{tabular}{l p{7cm} c c}
    \hline
    \sffamily\textbf{Component} & \sffamily\textbf{Description} & \sffamily\textbf{Units} & \sffamily\textbf{Dimensions} \\
    \hline\hline
    \texttt{n\_Channels}        & Number of channels to use in SfcOptics emissivty algorithms (\texttt{L}) & N/A    & Scalar \\
    \texttt{Sensor\_Id}         & The sensor id                                                            & N/A    & Scalar \\
    \texttt{WMO\_Satellite\_Id} & The WMO satellite Id                                                     & N/A    & Scalar \\
    \texttt{WMO\_Sensor\_Id}    & The WMO sensor Id                                                        & N/A    & Scalar \\
    \texttt{Sensor\_Channel}    & The channel numbers                                                      & N/A    & \texttt{L} \\
    \texttt{Tb}                 & The brightness temperature measurements for each channel                 & Kelvin & \texttt{L} \\
    \hline
  \end{tabular}
  \caption{CRTM \SensorData{} structure component description.}
  \label{tab:sensordata_structure}
\end{table}

% Surface structure methods
%--------------------------
\clearpage
\subsection{\texttt{CRTM\_SensorData\_Associated} interface}
  \label{sec:CRTM_SensorData_Associated_interface}
  \begin{alltt}
 
  NAME:
        CRTM_SensorData_Associated
 
  PURPOSE:
        Elemental function to test the status of the allocatable components
        of a CRTM SensorData object.
 
  CALLING SEQUENCE:
        Status = CRTM_SensorData_Associated( SensorData )
 
  OBJECTS:
        SensorData:  SensorData structure which is to have its member's
                     status tested.
                     UNITS:      N/A
                     TYPE:       CRTM_SensorData_type
                     DIMENSION:  Scalar or any rank
                     ATTRIBUTES: INTENT(IN)
 
  FUNCTION RESULT:
        Status:      The return value is a logical value indicating the
                     status of the SensorData members.
                       .TRUE.  - if the array components are allocated.
                       .FALSE. - if the array components are not allocated.
                     UNITS:      N/A
                     TYPE:       LOGICAL
                     DIMENSION:  Same as input SensorData argument
 
  \end{alltt}

\subsection{\texttt{CRTM\_SensorData\_Compare} interface}
  \label{sec:CRTM_SensorData_Compare_interface}
  \begin{alltt}
  NAME:
        CRTM_SensorData_Compare
 
  PURPOSE:
        Elemental function to compare two CRTM_SensorData objects to within
        a user specified number of significant figures.
 
  CALLING SEQUENCE:
        is_comparable = CRTM_SensorData_Compare( x, y, n_SigFig=n_SigFig )
 
  OBJECTS:
        x, y:          Two CRTM SensorData objects to be compared.
                       UNITS:      N/A
                       TYPE:       CRTM_SensorData_type
                       DIMENSION:  Scalar or any rank
                       ATTRIBUTES: INTENT(IN)
 
  OPTIONAL INPUTS:
        n_SigFig:      Number of significant figure to compare floating point
                       components.
                       UNITS:      N/A
                       TYPE:       INTEGER
                       DIMENSION:  Scalar or same as input
                       ATTRIBUTES: INTENT(IN), OPTIONAL
 
  FUNCTION RESULT:
        is_equal:      Logical value indicating whether the inputs are equal.
                       UNITS:      N/A
                       TYPE:       LOGICAL
                       DIMENSION:  Same as inputs.
  \end{alltt}

\subsection{\texttt{CRTM\_SensorData\_Create} interface}
  \label{sec:CRTM_SensorData_Create_interface}
  \begin{alltt}
 
  NAME:
        CRTM_SensorData_Create
 
  PURPOSE:
        Elemental subroutine to create an instance of the CRTM SensorData object.
 
  CALLING SEQUENCE:
        CALL CRTM_SensorData_Create( SensorData, n_Channels )
 
  OBJECTS:
        SensorData:   SensorData structure.
                      UNITS:      N/A
                      TYPE:       CRTM_SensorData_type
                      DIMENSION:  Scalar or any rank
                      ATTRIBUTES: INTENT(OUT)
 
  INPUTS:
        n_Channels:   Number of sensor channels.
                      Must be > 0.
                      UNITS:      N/A
                      TYPE:       INTEGER
                      DIMENSION:  Same as SensorData object
                      ATTRIBUTES: INTENT(IN)
 
  \end{alltt}

\subsection{\texttt{CRTM\_SensorData\_DefineVersion} interface}
  \label{sec:CRTM_SensorData_DefineVersion_interface}
  \begin{alltt}
 
  NAME:
        CRTM_SensorData_DefineVersion
 
  PURPOSE:
        Subroutine to return the module version information.
 
  CALLING SEQUENCE:
        CALL CRTM_SensorData_DefineVersion( Id )
 
  OUTPUT ARGUMENTS:
        Id:            Character string containing the version Id information
                       for the module.
                       UNITS:      N/A
                       TYPE:       CHARACTER(*)
                       DIMENSION:  Scalar
                       ATTRIBUTES: INTENT(OUT)
 
  \end{alltt}

\subsection{\texttt{CRTM\_SensorData\_Destroy} interface}
  \label{sec:CRTM_SensorData_Destroy_interface}
  \begin{alltt}
 
  NAME:
        CRTM_SensorData_Destroy
  
  PURPOSE:
        Elemental subroutine to re-initialize CRTM SensorData objects.
 
  CALLING SEQUENCE:
        CALL CRTM_SensorData_Destroy( SensorData )
 
  OBJECTS:
        SensorData:   Re-initialized SensorData structure.
                      UNITS:      N/A
                      TYPE:       CRTM_SensorData_type
                      DIMENSION:  Scalar OR any rank
                      ATTRIBUTES: INTENT(OUT)
 
  \end{alltt}

\subsection{\texttt{CRTM\_SensorData\_Inspect} interface}
  \label{sec:CRTM_SensorData_Inspect_interface}
  \begin{alltt}
 
  NAME:
        CRTM_SensorData_Inspect
 
  PURPOSE:
        Subroutine to print the contents of a CRTM SensorData object to stdout.
 
  CALLING SEQUENCE:
        CALL CRTM_SensorData_Inspect( SensorData )
 
  INPUTS:
        SensorData:    CRTM SensorData object to display.
                       UNITS:      N/A
                       TYPE:       CRTM_SensorData_type
                       DIMENSION:  Scalar
                       ATTRIBUTES: INTENT(IN)
 
  \end{alltt}

\subsection{\texttt{CRTM\_SensorData\_IsValid} interface}
  \label{sec:CRTM_SensorData_IsValid_interface}
  \begin{alltt}
 
  NAME:
        CRTM_SensorData_IsValid
 
  PURPOSE:
        Non-pure function to perform some simple validity checks on a
        CRTM SensorData object.
 
        If invalid data is found, a message is printed to stdout.
 
  CALLING SEQUENCE:
        result = CRTM_SensorData_IsValid( SensorData )
 
          or
 
        IF ( CRTM_SensorData_IsValid( SensorData ) ) THEN....
 
  OBJECTS:
        SensorData:    CRTM SensorData object which is to have its
                       contents checked.
                       UNITS:      N/A
                       TYPE:       CRTM_SensorData_type
                       DIMENSION:  Scalar
                       ATTRIBUTES: INTENT(IN)
 
  FUNCTION RESULT:
        result:        Logical variable indicating whether or not the input
                       passed the check.
                       If == .FALSE., SensorData object is unused or contains
                                      invalid data.
                          == .TRUE.,  SensorData object can be used in CRTM.
                       UNITS:      N/A
                       TYPE:       LOGICAL
                       DIMENSION:  Scalar
 
  \end{alltt}

\subsection{\texttt{CRTM\_SensorData\_Zero} interface}
  \label{sec:CRTM_SensorData_Zero_interface}
  \begin{alltt}
 
  NAME:
        CRTM_SensorData_Zero
 
  PURPOSE:
        Elemental subroutine to zero out the data arrays in a
        CRTM SensorData object.
 
  CALLING SEQUENCE:
        CALL CRTM_SensorData_Zero( SensorData )
 
  OBJECTS:
        SensorData:    CRTM SensorData structure in which the data arrays are
                       to be zeroed out.
                       UNITS:      N/A
                       TYPE:       CRTM_SensorData_type
                       DIMENSION:  Scalar or any rank
                       ATTRIBUTES: INTENT(IN OUT)
 
  COMMENTS:
        - The dimension components of the structure are *NOT* set to zero.
        - The SensorData sensor id and channel components are *NOT* reset.
 
  \end{alltt}

\subsection{\texttt{CRTM\_SensorData\_IOVersion} interface}
  \label{sec:CRTM_SensorData_IOVersion_interface}
  \begin{alltt}
 
  NAME:
        CRTM_SensorData_IOVersion
 
  PURPOSE:
        Subroutine to return the module version information.
 
  CALLING SEQUENCE:
        CALL CRTM_SensorData_IOVersion( Id )
 
  OUTPUT ARGUMENTS:
        Id:    Character string containing the version Id information
               for the module.
               UNITS:      N/A
               TYPE:       CHARACTER(*)
               DIMENSION:  Scalar
               ATTRIBUTES: INTENT(OUT)
 
  \end{alltt}

\subsection{\texttt{CRTM\_SensorData\_InquireFile} interface}
  \label{sec:CRTM_SensorData_InquireFile_interface}
  \begin{alltt}
 
  NAME:
        CRTM_SensorData_InquireFile
 
  PURPOSE:
        Function to inquire CRTM SensorData object files.
 
  CALLING SEQUENCE:
        Error_Status = CRTM_SensorData_InquireFile( Filename           , &
                                                    n_DataSets = n_DataSets  )
 
  INPUTS:
        Filename:       Character string specifying the name of a
                        CRTM SensorData data file to read.
                        UNITS:      N/A
                        TYPE:       CHARACTER(*)
                        DIMENSION:  Scalar
                        ATTRIBUTES: INTENT(IN)
 
  OPTIONAL OUTPUTS:
        n_DataSets:     The number of datasets in the file.
                        UNITS:      N/A
                        TYPE:       INTEGER
                        DIMENSION:  Scalar
                        ATTRIBUTES: OPTIONAL, INTENT(OUT)
 
  FUNCTION RESULT:
        Error_Status:   The return value is an integer defining the error status.
                        The error codes are defined in the Message_Handler module.
                        If == SUCCESS, the file inquire was successful
                           == FAILURE, an unrecoverable error occurred.
                        UNITS:      N/A
                        TYPE:       INTEGER
                        DIMENSION:  Scalar
 
  \end{alltt}

\subsection{\texttt{CRTM\_SensorData\_ReadFile} interface}
  \label{sec:CRTM_SensorData_ReadFile_interface}
  \begin{alltt}
 
  NAME:
        CRTM_SensorData_ReadFile
 
  PURPOSE:
        Function to read CRTM SensorData object files.
 
  CALLING SEQUENCE:
        Error_Status = CRTM_SensorData_ReadFile( Filename               , &
                                                 SensorData             , &
                                                 Quiet      = Quiet     , &
                                                 No_Close   = No_Close  , &
                                                 n_DataSets = n_DataSets  )
 
  INPUTS:
        Filename:       Character string specifying the name of a
                        SensorData format data file to read.
                        UNITS:      N/A
                        TYPE:       CHARACTER(*)
                        DIMENSION:  Scalar
                        ATTRIBUTES: INTENT(IN)
 
  OUTPUTS:
        SensorData:     CRTM SensorData object array containing the sensor data.
                        UNITS:      N/A
                        TYPE:       CRTM_SensorData_type
                        DIMENSION:  Rank-1
                        ATTRIBUTES: INTENT(OUT)
 
  OPTIONAL INPUTS:
        Quiet:          Set this logical argument to suppress INFORMATION
                        messages being printed to stdout
                        If == .FALSE., INFORMATION messages are OUTPUT [DEFAULT].
                           == .TRUE.,  INFORMATION messages are SUPPRESSED.
                        If not specified, default is .FALSE.
                        UNITS:      N/A
                        TYPE:       LOGICAL
                        DIMENSION:  Scalar
                        ATTRIBUTES: INTENT(IN), OPTIONAL
 
        No_Close:       Set this logical argument to NOT close the file upon exit.
                        If == .FALSE., the input file is closed upon exit [DEFAULT]
                           == .TRUE.,  the input file is NOT closed upon exit. 
                        If not specified, default is .FALSE.
                        UNITS:      N/A
                        TYPE:       LOGICAL
                        DIMENSION:  Scalar
                        ATTRIBUTES: INTENT(IN), OPTIONAL
 
  OPTIONAL OUTPUTS:
        n_DataSets:     The actual number of datasets read in.
                        UNITS:      N/A
                        TYPE:       INTEGER
                        DIMENSION:  Scalar
                        ATTRIBUTES: OPTIONAL, INTENT(OUT)
 
  FUNCTION RESULT:
        Error_Status:   The return value is an integer defining the error status.
                        The error codes are defined in the Message_Handler module.
                        If == SUCCESS, the file read was successful
                           == FAILURE, an unrecoverable error occurred.
                        UNITS:      N/A
                        TYPE:       INTEGER
                        DIMENSION:  Scalar
 
  \end{alltt}

\subsection{\texttt{CRTM\_SensorData\_WriteFile} interface}
  \label{sec:CRTM_SensorData_WriteFile_interface}
  \begin{alltt}
 
  NAME:
        CRTM_SensorData_WriteFile
 
  PURPOSE:
        Function to write CRTM SensorData object files.
 
  CALLING SEQUENCE:
        Error_Status = CRTM_SensorData_WriteFile( Filename           , &
                                                  SensorData         , &
                                                  Quiet    = Quiet   , &
                                                  No_Close = No_Close  )
 
  INPUTS:
        Filename:       Character string specifying the name of the
                        SensorData format data file to write.
                        UNITS:      N/A
                        TYPE:       CHARACTER(*)
                        DIMENSION:  Scalar
                        ATTRIBUTES: INTENT(IN)
 
        SensorData:     CRTM SensorData object array containing the datasets.
                        UNITS:      N/A
                        TYPE:       CRTM_SensorData_type
                        DIMENSION:  Rank-1
                        ATTRIBUTES: INTENT(IN)
 
  OPTIONAL INPUTS:
        Quiet:          Set this logical argument to suppress INFORMATION
                        messages being printed to stdout
                        If == .FALSE., INFORMATION messages are OUTPUT [DEFAULT].
                           == .TRUE.,  INFORMATION messages are SUPPRESSED.
                        If not specified, default is .FALSE.
                        UNITS:      N/A
                        TYPE:       LOGICAL
                        DIMENSION:  Scalar
                        ATTRIBUTES: INTENT(IN), OPTIONAL
 
        No_Close:       Set this logical argument to NOT close the file upon exit.
                        If == .FALSE., the input file is closed upon exit [DEFAULT]
                           == .TRUE.,  the input file is NOT closed upon exit.
                        If not specified, default is .FALSE.
                        UNITS:      N/A
                        TYPE:       LOGICAL
                        DIMENSION:  Scalar
                        ATTRIBUTES: INTENT(IN), OPTIONAL
 
  FUNCTION RESULT:
        Error_Status:   The return value is an integer defining the error status.
                        The error codes are defined in the Message_Handler module.
                        If == SUCCESS, the file write was successful
                           == FAILURE, an unrecoverable error occurred.
                        UNITS:      N/A
                        TYPE:       INTEGER
                        DIMENSION:  Scalar
 
  SIDE EFFECTS:
        - If the output file already exists, it is overwritten.
        - If an error occurs during *writing*, the output file is deleted before
          returning to the calling routine.
 
  \end{alltt}



\clearpage
\section{\Geometry{} Structure}
%==============================
\label{sec:geometry_structure}

\begin{figure}[htp]
  \centering
  \doublebox{
  \begin{minipage}[b]{6.5in}
    \begin{alltt}
  TYPE :: CRTM_Geometry_type
    ! Allocation indicator
    LOGICAL :: Is_Allocated = .TRUE.  ! Placeholder for future expansion
    ! Field of view index (1-nFOV)
    INTEGER  :: iFOV = 0
    ! Earth location
    REAL(fp) :: Longitude        = ZERO
    REAL(fp) :: Latitude         = ZERO
    REAL(fp) :: Surface_Altitude = ZERO
    ! Sensor angle information
    REAL(fp) :: Sensor_Scan_Angle    = ZERO
    REAL(fp) :: Sensor_Zenith_Angle  = ZERO
    REAL(fp) :: Sensor_Azimuth_Angle = ZERO 
    ! Source angle information
    REAL(fp) :: Source_Zenith_Angle  = 100.0_fp  ! Below horizon
    REAL(fp) :: Source_Azimuth_Angle = ZERO
    ! Flux angle information
    REAL(fp) :: Flux_Zenith_Angle = DIFFUSIVITY_ANGLE
    ! Date for geometry calculations
    INTEGER :: Year  = 2001
    INTEGER :: Month = 1
    INTEGER :: Day   = 1
  END TYPE CRTM_Geometry_type
    \end{alltt}
  \end{minipage}
  }
  \caption{CRTM\_Geometry\_type structure definition.}
  \label{fig:CRTM_Geometry_type_structure}
\end{figure}


% Geometry component description table
\begin{table}[htp]
  \centering
  \begin{tabular}{l p{7cm} c c}
    \hline
    \sffamily\textbf{Component} & \sffamily\textbf{Description} & \sffamily\textbf{Units} & \sffamily\textbf{Dimensions} \\
    \hline\hline
    \texttt{iFOV}                    & The scan line FOV index & N/A & Scalar \\
    \texttt{Longitude}               & Earth longitude & deg. E (0$\rightarrow$360) & Scalar \\
    \texttt{Latitude}                & Earth latitude  & deg. N (-90$\rightarrow$+90) & Scalar \\
    \texttt{Surface\_Altitude}       & Altitude of the Earth's surface at the specified lon/lat location & metres (m) & Scalar \\
    \texttt{Sensor\_Scan\_Angle}     & The sensor scan angle from nadir. See fig.\ref{fig:geo_sensor_scan_angle} & degrees & Scalar \\
    \texttt{Sensor\_Zenith\_Angle}   & The sensor zenith angle of the FOV. See fig.\ref{fig:geo_sensor_zenith_angle} & degrees & Scalar \\
    \texttt{Sensor\_Azimuth\_Angle}  & The sensor azimuth angle is the angle subtended by the horizontal projection of a direct line from the satellite to the FOV and the North-South axis measured clockwise from North. See fig.\ref{fig:geo_sensor_azimuth_angle} & deg. from N & Scalar \\
    \texttt{Source\_Zenith\_Angle}   & The source zenith angle. The source is typically the Sun (IR/VIS) or Moon (MW/VIS) [only solar source valid in current release] See fig.\ref{fig:geo_source_zenith_angle} & degrees & Scalar \\
    \texttt{Source\_Azimuth\_Angle}  & The source azimuth angle is the angle subtended by the horizontal projection of a direct line from the source to the FOV and the North-South axis measured clockwise from North. See fig.\ref{fig:geo_source_azimuth_angle} & deg. from N & Scalar \\
    \texttt{Flux\_Zenith\_Angle}     & The zenith angle used to approximate downwelling flux transmissivity. If not set, the default value is that of the diffusivity approximation, such that $\sec(F) = 5/3$. Maximum allowed value is determined from $\sec(F) = 9/4$ & degrees & Scalar \\
    \texttt{Year}                    & The year in 4-digit format       & N/A & Scalar \\
    \texttt{Month}                   & The month of year (1-12)         & N/A & Scalar \\
    \texttt{Day}                     & The day of month (1-28/29/30/31) & N/A & Scalar \\
    \hline
  \end{tabular}
  \caption{CRTM \Geometry{} structure component description.}
  \label{tab:geometry_structure}
\end{table}

\begin{figure}[htp]
  \centering
  \input{graphics/geo/sensor_scan_angle.pstex_t}
  \caption{Definition of \Geometry{} sensor scan angle component.}
  \label{fig:geo_sensor_scan_angle}
\end{figure}

\begin{figure}[htp]
  \centering
  \input{graphics/geo/sensor_zenith_angle.pstex_t}
  \caption{Definition of \Geometry{} sensor zenith angle component.}
  \label{fig:geo_sensor_zenith_angle}
\end{figure}

\begin{figure}[htp]
  \centering
  \input{graphics/geo/sensor_azimuth_angle.pstex_t}
  \caption{Definition of \Geometry{} sensor azimuth angle component.}
  \label{fig:geo_sensor_azimuth_angle}
\end{figure}

\begin{figure}[htp]
  \centering
  \input{graphics/geo/source_zenith_angle.pstex_t}
  \caption{Definition of \Geometry{} source zenith angle component.}
  \label{fig:geo_source_zenith_angle}
\end{figure}

\begin{figure}[htp]
  \centering
  \input{graphics/geo/source_azimuth_angle.pstex_t}
  \caption{Definition of \Geometry{} source azimuth angle component.}
  \label{fig:geo_source_azimuth_angle}
\end{figure}

% Geometry structure methods
%--------------------------
\clearpage
\subsection{\texttt{CRTM\_Geometry\_DefineVersion} interface}
  \label{sec:CRTM_Geometry_DefineVersion_interface}
  \begin{alltt}
 
  NAME:
        CRTM_Geometry_DefineVersion
 
  PURPOSE:
        Subroutine to return the module version information.
 
  CALLING SEQUENCE:
        CALL CRTM_Geometry_DefineVersion( Id )
 
  OUTPUT ARGUMENTS:
        Id:            Character string containing the version Id information
                       for the module.
                       UNITS:      N/A
                       TYPE:       CHARACTER(*)
                       DIMENSION:  Scalar
                       ATTRIBUTES: INTENT(OUT)
 
  \end{alltt}

\subsection{\texttt{CRTM\_Geometry\_Destroy} interface}
  \label{sec:CRTM_Geometry_Destroy_interface}
  \begin{alltt}
 
  NAME:
        CRTM_Geometry_Destroy
  
  PURPOSE:
        Elemental subroutine to re-initialize CRTM Geometry objects.
 
  CALLING SEQUENCE:
        CALL CRTM_Geometry_Destroy( geo )
 
  OBJECTS:
        geo:          Re-initialized Geometry structure.
                      UNITS:      N/A
                      TYPE:       CRTM_Geometry_type
                      DIMENSION:  Scalar or any rank
                      ATTRIBUTES: INTENT(OUT)
 
  \end{alltt}

\subsection{\texttt{CRTM\_Geometry\_GetValue} interface}
  \label{sec:CRTM_Geometry_GetValue_interface}
  \begin{alltt}
 
  NAME:
        CRTM_Geometry_GetValue
  
  PURPOSE:
        Elemental subroutine to get the values of CRTM Geometry
        object components.
 
  CALLING SEQUENCE:
        CALL CRTM_Geometry_GetValue( geo, &
                                     iFOV                 = iFOV                , &
                                     Longitude            = Longitude           , &
                                     Latitude             = Latitude            , &
                                     Surface_Altitude     = Surface_Altitude    , &
                                     Sensor_Scan_Angle    = Sensor_Scan_Angle   , &
                                     Sensor_Zenith_Angle  = Sensor_Zenith_Angle , &
                                     Sensor_Azimuth_Angle = Sensor_Azimuth_Angle, &
                                     Source_Zenith_Angle  = Source_Zenith_Angle , &
                                     Source_Azimuth_Angle = Source_Azimuth_Angle, &
                                     Flux_Zenith_Angle    = Flux_Zenith_Angle   , &
                                     Year                 = Year                , &
                                     Month                = Month               , &
                                     Day                  = Day                   )
 
  OBJECTS:
        geo:                  Geometry object from which component values
                              are to be retrieved.
                              UNITS:      N/A
                              TYPE:       CRTM_Geometry_type
                              DIMENSION:  Scalar or any rank
                              ATTRIBUTES: INTENT(IN OUT)
 
  OPTIONAL OUTPUTS:
        iFOV:                 Sensor field-of-view index.
                              UNITS:      N/A
                              TYPE:       INTEGER
                              DIMENSION:  Scalar or same as geo input
                              ATTRIBUTES: INTENT(OUT), OPTIONAL
 
        Longitude:            Earth longitude
                              UNITS:      degrees East (0->360) 
                              TYPE:       REAL(fp)
                              DIMENSION:  Scalar or same as geo input
                              ATTRIBUTES: INTENT(OUT), OPTIONAL
 
        Latitude:             Earth latitude.
                              UNITS:      degrees North (-90->+90)
                              TYPE:       REAL(fp)
                              DIMENSION:  Scalar or same as geo input
                              ATTRIBUTES: INTENT(OUT), OPTIONAL
 
        Surface_Altitude:     Altitude of the Earth's surface at the specifed
                              lon/lat location.
                              UNITS:      metres (m)
                              TYPE:       REAL(fp)
                              DIMENSION:  Scalar or same as geo input
                              ATTRIBUTES: INTENT(OUT), OPTIONAL
 
        Sensor_Scan_Angle:    The sensor scan angle from nadir.
                              UNITS:      degrees
                              TYPE:       REAL(fp)
                              DIMENSION:  Scalar or same as geo input
                              ATTRIBUTES: INTENT(OUT), OPTIONAL
 
        Sensor_Zenith_Angle:  The zenith angle from the field-of-view
                              to the sensor.
                              UNITS:      degrees
                              TYPE:       REAL(fp)
                              DIMENSION:  Scalar or same as geo input
                              ATTRIBUTES: INTENT(OUT), OPTIONAL
 
        Sensor_Azimuth_Angle: The azimuth angle subtended by the horizontal
                              projection of a direct line from the satellite
                              to the FOV and the North-South axis measured
                              clockwise from North.
                              UNITS:      degrees from North (0->360) 
                              TYPE:       REAL(fp)
                              DIMENSION:  Scalar or same as geo input
                              ATTRIBUTES: INTENT(OUT), OPTIONAL
 
        Source_Zenith_Angle:  The zenith angle from the field-of-view
                              to a source (sun or moon).
                              UNITS:      degrees
                              TYPE:       REAL(fp)
                              DIMENSION:  Scalar or same as geo input
                              ATTRIBUTES: INTENT(OUT), OPTIONAL
 
        Source_Azimuth_Angle: The azimuth angle subtended by the horizontal
                              projection of a direct line from the source
                              to the FOV and the North-South axis measured
                              clockwise from North.
                              UNITS:      degrees from North (0->360) 
                              TYPE:       REAL(fp)
                              DIMENSION:  Scalar or same as geo input
                              ATTRIBUTES: INTENT(OUT), OPTIONAL
 
        Flux_Zenith_Angle:    The zenith angle used to approximate downwelling
                              flux transmissivity
                              UNITS:      degrees
                              TYPE:       REAL(fp)
                              DIMENSION:  Scalar or same as geo input
                              ATTRIBUTES: INTENT(OUT), OPTIONAL
 
        Year:                 The year in 4-digit format, e.g. 1997.
                              UNITS:      N/A
                              TYPE:       INTEGER
                              DIMENSION:  Scalar or same as geo input
                              ATTRIBUTES: INTENT(OUT), OPTIONAL
 
        Month:                The month of the year (1-12).
                              UNITS:      N/A
                              TYPE:       INTEGER
                              DIMENSION:  Scalar or same as geo input
                              ATTRIBUTES: INTENT(OUT), OPTIONAL
 
        Day:                  The day of the month (1-28/29/30/31).
                              UNITS:      N/A
                              TYPE:       INTEGER
                              DIMENSION:  Scalar or same as geo input
                              ATTRIBUTES: INTENT(OUT), OPTIONAL
 
  \end{alltt}

\subsection{\texttt{CRTM\_Geometry\_Inspect} interface}
  \label{sec:CRTM_Geometry_Inspect_interface}
  \begin{alltt}
 
  NAME:
        CRTM_Geometry_Inspect
 
  PURPOSE:
        Subroutine to print the contents of a CRTM Geometry object to stdout.
 
  CALLING SEQUENCE:
        CALL CRTM_Geometry_Inspect( geo )
 
  INPUTS:
        geo:  CRTM Geometry object to display.
              UNITS:      N/A
              TYPE:       CRTM_Geometry_type
              DIMENSION:  Scalar
              ATTRIBUTES: INTENT(IN)
 
  \end{alltt}

\subsection{\texttt{CRTM\_Geometry\_IsValid} interface}
  \label{sec:CRTM_Geometry_IsValid_interface}
  \begin{alltt}
 
  NAME:
        CRTM_Geometry_IsValid
 
  PURPOSE:
        Non-pure function to perform some simple validity checks on a
        CRTM Geometry object. 
 
        If invalid data is found, a message is printed to stdout.
 
  CALLING SEQUENCE:
        result = CRTM_Geometry_IsValid( geo )
 
          or
 
        IF ( CRTM_Geometry_IsValid( geo ) ) THEN....
 
  OBJECTS:
        geo:       CRTM Geometry object which is to have its
                   contents checked.
                   UNITS:      N/A
                   TYPE:       CRTM_Geometry_type
                   DIMENSION:  Scalar
                   ATTRIBUTES: INTENT(IN)
 
  FUNCTION RESULT:
        result:    Logical variable indicating whether or not the input
                   passed the check.
                   If == .FALSE., Geometry object is unused or contains
                                  invalid data.
                      == .TRUE.,  Geometry object can be used in CRTM.
                   UNITS:      N/A
                   TYPE:       LOGICAL
                   DIMENSION:  Scalar
 
  \end{alltt}

\subsection{\texttt{CRTM\_Geometry\_SetValue} interface}
  \label{sec:CRTM_Geometry_SetValue_interface}
  \begin{alltt}
 
  NAME:
        CRTM_Geometry_SetValue
  
  PURPOSE:
        Elemental subroutine to set the values of CRTM Geometry
        object components.
 
  CALLING SEQUENCE:
        CALL CRTM_Geometry_SetValue( geo, &
                                     iFOV                 = iFOV                , &
                                     Longitude            = Longitude           , &
                                     Latitude             = Latitude            , &
                                     Surface_Altitude     = Surface_Altitude    , &
                                     Sensor_Scan_Angle    = Sensor_Scan_Angle   , &
                                     Sensor_Zenith_Angle  = Sensor_Zenith_Angle , &
                                     Sensor_Azimuth_Angle = Sensor_Azimuth_Angle, &
                                     Source_Zenith_Angle  = Source_Zenith_Angle , &
                                     Source_Azimuth_Angle = Source_Azimuth_Angle, &
                                     Flux_Zenith_Angle    = Flux_Zenith_Angle   , &
                                     Year                 = Year                , &
                                     Month                = Month               , &
                                     Day                  = Day                   )
 
  OBJECTS:
        geo:                  Geometry object for which component values
                              are to be set.
                              UNITS:      N/A
                              TYPE:       CRTM_Geometry_type
                              DIMENSION:  Scalar or any rank
                              ATTRIBUTES: INTENT(IN OUT)
 
  OPTIONAL INPUTS:
        iFOV:                 Sensor field-of-view index.
                              UNITS:      N/A
                              TYPE:       INTEGER
                              DIMENSION:  Scalar or same as geo input
                              ATTRIBUTES: INTENT(IN), OPTIONAL
 
        Longitude:            Earth longitude
                              UNITS:      degrees East (0->360) 
                              TYPE:       REAL(fp)
                              DIMENSION:  Scalar or same as geo input
                              ATTRIBUTES: INTENT(IN), OPTIONAL
 
        Latitude:             Earth latitude.
                              UNITS:      degrees North (-90->+90)
                              TYPE:       REAL(fp)
                              DIMENSION:  Scalar or same as geo input
                              ATTRIBUTES: INTENT(IN), OPTIONAL
 
        Surface_Altitude:     Altitude of the Earth's surface at the specifed
                              lon/lat location.
                              UNITS:      metres (m)
                              TYPE:       REAL(fp)
                              DIMENSION:  Scalar or same as geo input
                              ATTRIBUTES: INTENT(IN), OPTIONAL
 
        Sensor_Scan_Angle:    The sensor scan angle from nadir.
                              UNITS:      degrees
                              TYPE:       REAL(fp)
                              DIMENSION:  Scalar or same as geo input
                              ATTRIBUTES: INTENT(IN), OPTIONAL
 
        Sensor_Zenith_Angle:  The zenith angle from the field-of-view
                              to the sensor.
                              UNITS:      degrees
                              TYPE:       REAL(fp)
                              DIMENSION:  Scalar or same as geo input
                              ATTRIBUTES: INTENT(IN), OPTIONAL
 
        Sensor_Azimuth_Angle: The azimuth angle subtended by the horizontal
                              projection of a direct line from the satellite
                              to the FOV and the North-South axis measured
                              clockwise from North.
                              UNITS:      degrees from North (0->360) 
                              TYPE:       REAL(fp)
                              DIMENSION:  Scalar or same as geo input
                              ATTRIBUTES: INTENT(IN), OPTIONAL
 
        Source_Zenith_Angle:  The zenith angle from the field-of-view
                              to a source (sun or moon).
                              UNITS:      degrees
                              TYPE:       REAL(fp)
                              DIMENSION:  Scalar or same as geo input
                              ATTRIBUTES: INTENT(IN), OPTIONAL
 
        Source_Azimuth_Angle: The azimuth angle subtended by the horizontal
                              projection of a direct line from the source
                              to the FOV and the North-South axis measured
                              clockwise from North.
                              UNITS:      degrees from North (0->360) 
                              TYPE:       REAL(fp)
                              DIMENSION:  Scalar or same as geo input
                              ATTRIBUTES: INTENT(IN), OPTIONAL
 
        Flux_Zenith_Angle:    The zenith angle used to approximate downwelling
                              flux transmissivity
                              UNITS:      degrees
                              TYPE:       REAL(fp)
                              DIMENSION:  Scalar or same as geo input
                              ATTRIBUTES: INTENT(IN), OPTIONAL
 
        Year:                 The year in 4-digit format, e.g. 1997.
                              UNITS:      N/A
                              TYPE:       INTEGER
                              DIMENSION:  Scalar or same as geo input
                              ATTRIBUTES: INTENT(IN), OPTIONAL
 
        Month:                The month of the year (1-12).
                              UNITS:      N/A
                              TYPE:       INTEGER
                              DIMENSION:  Scalar or same as geo input
                              ATTRIBUTES: INTENT(IN), OPTIONAL
 
        Day:                  The day of the month (1-28/29/30/31).
                              UNITS:      N/A
                              TYPE:       INTEGER
                              DIMENSION:  Scalar or same as geo input
                              ATTRIBUTES: INTENT(IN), OPTIONAL
 
  \end{alltt}

\subsection{\texttt{CRTM\_Geometry\_IOVersion} interface}
  \label{sec:CRTM_Geometry_IOVersion_interface}
  \begin{alltt}
 
  NAME:
        CRTM_Geometry_IOVersion
 
  PURPOSE:
        Subroutine to return the module version information.
 
  CALLING SEQUENCE:
        CALL CRTM_Geometry_IOVersion( Id )
 
  OUTPUT ARGUMENTS:
        Id:            Character string containing the version Id information
                       for the module.
                       UNITS:      N/A
                       TYPE:       CHARACTER(*)
                       DIMENSION:  Scalar
                       ATTRIBUTES: INTENT(OUT)
 
  \end{alltt}

\subsection{\texttt{CRTM\_Geometry\_InquireFile} interface}
  \label{sec:CRTM_Geometry_InquireFile_interface}
  \begin{alltt}
 
  NAME:
        CRTM_Geometry_InquireFile
 
  PURPOSE:
        Function to inquire CRTM Geometry object files.
 
  CALLING SEQUENCE:
        Error_Status = CRTM_Geometry_InquireFile( Filename               , &
                                                  n_Profiles = n_Profiles  )
 
  INPUTS:
        Filename:       Character string specifying the name of a
                        CRTM Geometry data file to read.
                        UNITS:      N/A
                        TYPE:       CHARACTER(*)
                        DIMENSION:  Scalar
                        ATTRIBUTES: INTENT(IN)
 
  OPTIONAL OUTPUTS:
        n_Profiles:     The number of profiles for which their is geometry 
                        information in the data file.
                        UNITS:      N/A
                        TYPE:       INTEGER
                        DIMENSION:  Scalar
                        ATTRIBUTES: OPTIONAL, INTENT(OUT)
 
  FUNCTION RESULT:
        Error_Status:   The return value is an integer defining the error status.
                        The error codes are defined in the Message_Handler module.
                        If == SUCCESS, the file inquire was successful
                           == FAILURE, an unrecoverable error occurred.
                        UNITS:      N/A
                        TYPE:       INTEGER
                        DIMENSION:  Scalar
 
  \end{alltt}

\subsection{\texttt{CRTM\_Geometry\_ReadFile} interface}
  \label{sec:CRTM_Geometry_ReadFile_interface}
  \begin{alltt}
 
  NAME:
        CRTM_Geometry_ReadFile
 
  PURPOSE:
        Function to read CRTM Geometry object files.
 
  CALLING SEQUENCE:
        Error_Status = CRTM_Geometry_ReadFile( Filename               , &
                                               Geometry               , &
                                               Quiet      = Quiet     , &
                                               No_Close   = No_Close  , &
                                               n_Profiles = n_Profiles  )
 
  INPUTS:
        Filename:     Character string specifying the name of an
                      a Geometry data file to read.
                      UNITS:      N/A
                      TYPE:       CHARACTER(*)
                      DIMENSION:  Scalar
                      ATTRIBUTES: INTENT(IN)
 
  OUTPUTS:
        Geometry:     CRTM Geometry object array containing the
                      data read from file.
                      UNITS:      N/A
                      TYPE:       CRTM_Geometry_type
                      DIMENSION:  Rank-1
                      ATTRIBUTES: INTENT(OUT)
 
  OPTIONAL INPUTS:
        Quiet:        Set this logical argument to suppress INFORMATION
                      messages being printed to stdout
                      If == .FALSE., INFORMATION messages are OUTPUT [DEFAULT].
                         == .TRUE.,  INFORMATION messages are SUPPRESSED.
                      If not specified, default is .FALSE.
                      UNITS:      N/A
                      TYPE:       LOGICAL
                      DIMENSION:  Scalar
                      ATTRIBUTES: INTENT(IN), OPTIONAL
 
        No_Close:     Set this logical argument to NOT close the file upon exit.
                      If == .FALSE., the input file is closed upon exit [DEFAULT]
                         == .TRUE.,  the input file is NOT closed upon exit.
                      If not specified, default is .FALSE.
                      UNITS:      N/A
                      TYPE:       LOGICAL
                      DIMENSION:  Scalar
                      ATTRIBUTES: INTENT(IN), OPTIONAL
 
  OPTIONAL OUTPUTS:
        n_Profiles:   The number of profiles for which data was read.
                      UNITS:      N/A
                      TYPE:       INTEGER
                      DIMENSION:  Scalar
                      ATTRIBUTES: OPTIONAL, INTENT(OUT)
 
  FUNCTION RESULT:
        Error_Status: The return value is an integer defining the error status.
                      The error codes are defined in the Message_Handler module.
                      If == SUCCESS, the file read was successful
                         == FAILURE, an unrecoverable error occurred.
                      UNITS:      N/A
                      TYPE:       INTEGER
                      DIMENSION:  Scalar
 
  \end{alltt}

\subsection{\texttt{CRTM\_Geometry\_WriteFile} interface}
  \label{sec:CRTM_Geometry_WriteFile_interface}
  \begin{alltt}
 
  NAME:
        CRTM_Geometry_WriteFile
 
  PURPOSE:
        Function to write CRTM Geometry object files.
 
  CALLING SEQUENCE:
        Error_Status = CRTM_Geometry_WriteFile( Filename     , &
                                                Geometry     , &
                                                Quiet = Quiet  )
 
  INPUTS:
        Filename:     Character string specifying the name of the
                      Geometry format data file to write.
                      UNITS:      N/A
                      TYPE:       CHARACTER(*)
                      DIMENSION:  Scalar
                      ATTRIBUTES: INTENT(IN)
 
        Geometry:     CRTM Geometry object array containing the Geometry
                      data to write.
                      UNITS:      N/A
                      TYPE:       CRTM_Geometry_type
                      DIMENSION:  Rank-1
                      ATTRIBUTES: INTENT(IN)
 
  OPTIONAL INPUTS:
        Quiet:        Set this logical argument to suppress INFORMATION
                      messages being printed to stdout
                      If == .FALSE., INFORMATION messages are OUTPUT [DEFAULT].
                         == .TRUE.,  INFORMATION messages are SUPPRESSED.
                      If not specified, default is .FALSE.
                      UNITS:      N/A
                      TYPE:       LOGICAL
                      DIMENSION:  Scalar
                      ATTRIBUTES: INTENT(IN), OPTIONAL
 
  FUNCTION RESULT:
        Error_Status: The return value is an integer defining the error status.
                      The error codes are defined in the Message_Handler module.
                      If == SUCCESS, the file write was successful
                         == FAILURE, an unrecoverable error occurred.
                      UNITS:      N/A
                      TYPE:       INTEGER
                      DIMENSION:  Scalar
 
  SIDE EFFECTS:
        - If the output file already exists, it is overwritten.
        - If an error occurs during *writing*, the output file is deleted before
          returning to the calling routine.
 
  \end{alltt}



\clearpage
\section{\RTSolution{} Structure}
%==================================
\label{sec:rtsolution_structure}

\begin{figure}[htp]
  \centering
  \doublebox{
  \begin{minipage}[b]{6.5in}
    \begin{alltt}
  TYPE :: CRTM_RTSolution_type
    ! Allocation indicator
    LOGICAL :: Is_Allocated = .FALSE.
    ! Dimensions
    INTEGER :: n_Layers = 0  ! K
    ! Internal variables. Users do not need to worry about these.
    LOGICAL :: Scattering_Flag = .TRUE.
    INTEGER :: n_Full_Streams  = 0
    INTEGER :: n_Stokes        = 0
    ! Forward radiative transfer intermediate results for a single channel
    !    These components are not defined when they are used as TL, AD
    !    and K variables
    REAL(fp) :: Surface_Emissivity      = ZERO
    REAL(fp) :: Up_Radiance             = ZERO
    REAL(fp) :: Down_Radiance           = ZERO
    REAL(fp) :: Down_Solar_Radiance     = ZERO
    REAL(fp) :: Surface_Planck_Radiance = ZERO
    REAL(fp), ALLOCATABLE :: Upwelling_Radiance(:)   ! K
    ! The layer optical depths
    REAL(fp), ALLOCATABLE :: Layer_Optical_Depth(:)  ! K
    ! Radiative transfer results for a single channel/node
    REAL(fp) :: Radiance               = ZERO
    REAL(fp) :: Brightness_Temperature = ZERO
  END TYPE CRTM_RTSolution_type
    \end{alltt}
  \end{minipage}
  }
  \caption{CRTM\_RTSolution\_type structure definition.}
  \label{fig:CRTM_RTSolution_type_structure}
\end{figure}


\begin{table}[htp]
  \centering
  \begin{tabular}{l p{7cm} c c}
    \hline
    \sffamily\textbf{Component} & \sffamily\textbf{Description} & \sffamily\textbf{Units} & \sffamily\textbf{Dimensions} \\
    \hline\hline
    \texttt{n\_Layers}    & Number of atmospheric profile layers (\texttt{K}) & N/A & Scalar \\
    \texttt{Surface\_Emissivity      }  & The computed surface emissivity & N/A & Scalar \\
    \texttt{Up\_Radiance             }  & The atmospheric portion of the upwelling radiance & \radunit & Scalar \\
    \texttt{Down\_Radiance           }  & The atmospheric portion of the downwelling radiance & \radunit & Scalar \\
    \texttt{Down\_Solar\_Radiance    }  & The downwelling direct solar radiance & \radunit & Scalar \\
    \texttt{Surface\_Planck\_Radiance}  & The surface radiance & \radunit & Scalar \\
    \texttt{Upwelling\_Radiance      }  & The upwelling radiance profile, including the reflected downwelling and surface contributions. & \radunit & \texttt{K} \\
    \texttt{Layer\_Optical\_Depth    }  & The layer optical depth profile & N/A & \texttt{K} \\
    \texttt{Radiance                 }  & The sensor radiance & \radunit & Scalar \\
    \texttt{Brightness\_Temperature  }  & The sensor brightness temperature & Kelvin & Scalar \\
    \hline
  \end{tabular}
  \caption{CRTM \RTSolution{} structure component description}
  \label{tab:rtsolution_structure}
\end{table}

% RTSolution structure methods
%--------------------------
\clearpage
\subsection{\texttt{CRTM\_RTSolution\_Associated} interface}
  \label{sec:CRTM_RTSolution_Associated_interface}
  \begin{alltt}
 
  NAME:
        CRTM_RTSolution_Associated
 
  PURPOSE:
        Elemental function to test the status of the allocatable components
        of a CRTM RTSolution object.
 
  CALLING SEQUENCE:
        Status = CRTM_RTSolution_Associated( RTSolution )
 
  OBJECTS:
        RTSolution:   RTSolution structure which is to have its member's
                      status tested.
                      UNITS:      N/A
                      TYPE:       CRTM_RTSolution_type
                      DIMENSION:  Scalar or any rank
                      ATTRIBUTES: INTENT(IN)
 
  FUNCTION RESULT:
        Status:       The return value is a logical value indicating the
                      status of the RTSolution members.
                        .TRUE.  - if the array components are allocated.
                        .FALSE. - if the array components are not allocated.
                      UNITS:      N/A
                      TYPE:       LOGICAL
                      DIMENSION:  Same as input RTSolution argument
 
  \end{alltt}

\subsection{\texttt{CRTM\_RTSolution\_Compare} interface}
  \label{sec:CRTM_RTSolution_Compare_interface}
  \begin{alltt}
  NAME:
        CRTM_RTSolution_Compare
 
  PURPOSE:
        Elemental function to compare two CRTM_RTSolution objects to within
        a user specified number of significant figures.
 
  CALLING SEQUENCE:
        is_comparable = CRTM_RTSolution_Compare( x, y, n_SigFig=n_SigFig )
 
  OBJECTS:
        x, y:          Two CRTM RTSolution objects to be compared.
                       UNITS:      N/A
                       TYPE:       CRTM_RTSolution_type
                       DIMENSION:  Scalar or any rank
                       ATTRIBUTES: INTENT(IN)
 
  OPTIONAL INPUTS:
        n_SigFig:      Number of significant figure to compare floating point
                       components.
                       UNITS:      N/A
                       TYPE:       INTEGER
                       DIMENSION:  Scalar or same as input
                       ATTRIBUTES: INTENT(IN), OPTIONAL
 
  FUNCTION RESULT:
        is_equal:      Logical value indicating whether the inputs are equal.
                       UNITS:      N/A
                       TYPE:       LOGICAL
                       DIMENSION:  Same as inputs.
  \end{alltt}

\subsection{\texttt{CRTM\_RTSolution\_Create} interface}
  \label{sec:CRTM_RTSolution_Create_interface}
  \begin{alltt}
 
  NAME:
        CRTM_RTSolution_Create
  
  PURPOSE:
        Elemental subroutine to create an instance of the CRTM RTSolution object.
 
  CALLING SEQUENCE:
        CALL CRTM_RTSolution_Create( RTSolution, n_Layers )
 
  OBJECTS:
        RTSolution:   RTSolution structure.
                      UNITS:      N/A
                      TYPE:       CRTM_RTSolution_type
                      DIMENSION:  Scalar or any rank
                      ATTRIBUTES: INTENT(OUT)
 
  INPUTS:
        n_Layers:     Number of layers for which there is RTSolution data.
                      Must be > 0.
                      UNITS:      N/A
                      TYPE:       INTEGER
                      DIMENSION:  Same as RTSolution object
                      ATTRIBUTES: INTENT(IN)
 
  \end{alltt}

\subsection{\texttt{CRTM\_RTSolution\_DefineVersion} interface}
  \label{sec:CRTM_RTSolution_DefineVersion_interface}
  \begin{alltt}
 
  NAME:
        CRTM_RTSolution_DefineVersion
 
  PURPOSE:
        Subroutine to return the module version information.
 
  CALLING SEQUENCE:
        CALL CRTM_RTSolution_DefineVersion( Id )
 
  OUTPUTS:
        Id:            Character string containing the version Id information
                       for the module.
                       UNITS:      N/A
                       TYPE:       CHARACTER(*)
                       DIMENSION:  Scalar
                       ATTRIBUTES: INTENT(OUT)
 
  \end{alltt}

\subsection{\texttt{CRTM\_RTSolution\_Destroy} interface}
  \label{sec:CRTM_RTSolution_Destroy_interface}
  \begin{alltt}
 
  NAME:
        CRTM_RTSolution_Destroy
 
  PURPOSE:
        Elemental subroutine to re-initialize CRTM RTSolution objects.
 
  CALLING SEQUENCE:
        CALL CRTM_RTSolution_Destroy( RTSolution )
 
  OBJECTS:
        RTSolution:   Re-initialized RTSolution structure.
                      UNITS:      N/A
                      TYPE:       CRTM_RTSolution_type
                      DIMENSION:  Scalar OR any rank
                      ATTRIBUTES: INTENT(OUT)
 
  \end{alltt}

\subsection{\texttt{CRTM\_RTSolution\_Inspect} interface}
  \label{sec:CRTM_RTSolution_Inspect_interface}
  \begin{alltt}
 
  NAME:
        CRTM_RTSolution_Inspect
 
  PURPOSE:
        Subroutine to print the contents of a CRTM RTSolution object to stdout.
 
  CALLING SEQUENCE:
        CALL CRTM_RTSolution_Inspect( RTSolution )
 
  INPUTS:
        RTSolution:    CRTM RTSolution object to display.
                       UNITS:      N/A
                       TYPE:       CRTM_RTSolution_type
                       DIMENSION:  Scalar
                       ATTRIBUTES: INTENT(IN)
 
  \end{alltt}

\subsection{\texttt{CRTM\_RTSolution\_IOVersion} interface}
  \label{sec:CRTM_RTSolution_IOVersion_interface}
  \begin{alltt}
 
  NAME:
        CRTM_RTSolution_IOVersion
 
  PURPOSE:
        Subroutine to return the module version information.
 
  CALLING SEQUENCE:
        CALL CRTM_RTSolution_IOVersion( Id )
 
  OUTPUTS:
        Id:            Character string containing the version Id information
                       for the module.
                       UNITS:      N/A
                       TYPE:       CHARACTER(*)
                       DIMENSION:  Scalar
                       ATTRIBUTES: INTENT(OUT)
 
  \end{alltt}

\subsection{\texttt{CRTM\_RTSolution\_InquireFile} interface}
  \label{sec:CRTM_RTSolution_InquireFile_interface}
  \begin{alltt}
 
  NAME:
        CRTM_RTSolution_InquireFile
 
  PURPOSE:
        Function to inquire CRTM RTSolution object files.
 
  CALLING SEQUENCE:
        Error_Status = CRTM_RTSolution_InquireFile( Filename               , &
                                                    n_Channels = n_Channels, &
                                                    n_Profiles = n_Profiles  )
 
  INPUTS:
        Filename:       Character string specifying the name of a
                        CRTM RTSolution data file to read.
                        UNITS:      N/A
                        TYPE:       CHARACTER(*)
                        DIMENSION:  Scalar
                        ATTRIBUTES: INTENT(IN)
 
  OPTIONAL OUTPUTS:
        n_Channels:     The number of spectral channels for which there is
                        data in the file.
                        UNITS:      N/A
                        TYPE:       INTEGER
                        DIMENSION:  Scalar
                        ATTRIBUTES: OPTIONAL, INTENT(OUT)
 
        n_Profiles:     The number of profiles in the data file.
                        UNITS:      N/A
                        TYPE:       INTEGER
                        DIMENSION:  Scalar
                        ATTRIBUTES: OPTIONAL, INTENT(OUT)
 
  FUNCTION RESULT:
        Error_Status:   The return value is an integer defining the error status.
                        The error codes are defined in the Message_Handler module.
                        If == SUCCESS, the file inquire was successful
                           == FAILURE, an unrecoverable error occurred.
                        UNITS:      N/A
                        TYPE:       INTEGER
                        DIMENSION:  Scalar
 
  \end{alltt}

\subsection{\texttt{CRTM\_RTSolution\_ReadFile} interface}
  \label{sec:CRTM_RTSolution_ReadFile_interface}
  \begin{alltt}
 
  NAME:
        CRTM_RTSolution_ReadFile
 
  PURPOSE:
        Function to read CRTM RTSolution object files.
 
  CALLING SEQUENCE:
        Error_Status = CRTM_RTSolution_ReadFile( Filename                , &
                                                 RTSolution              , &
                                                 Quiet      = Quiet      , &
                                                 n_Channels = n_Channels , &
                                                 n_Profiles = n_Profiles , &
 
  INPUTS:
        Filename:     Character string specifying the name of an
                      RTSolution format data file to read.
                      UNITS:      N/A
                      TYPE:       CHARACTER(*)
                      DIMENSION:  Scalar
                      ATTRIBUTES: INTENT(IN)
 
  OUTPUTS:
        RTSolution:   CRTM RTSolution object array containing the RTSolution
                      data.
                      UNITS:      N/A
                      TYPE:       CRTM_RTSolution_type
                      DIMENSION:  Rank-2 (n_Channels x n_Profiles)
                      ATTRIBUTES: INTENT(OUT)
 
  OPTIONAL INPUTS:
        Quiet:        Set this logical argument to suppress INFORMATION
                      messages being printed to stdout
                      If == .FALSE., INFORMATION messages are OUTPUT [DEFAULT].
                         == .TRUE.,  INFORMATION messages are SUPPRESSED.
                      If not specified, default is .FALSE.
                      UNITS:      N/A
                      TYPE:       LOGICAL
                      DIMENSION:  Scalar
                      ATTRIBUTES: INTENT(IN), OPTIONAL
 
  OPTIONAL OUTPUTS:
        n_Channels:   The number of channels for which data was read.
                      UNITS:      N/A
                      TYPE:       INTEGER
                      DIMENSION:  Scalar
                      ATTRIBUTES: OPTIONAL, INTENT(OUT)
 
        n_Profiles:   The number of profiles for which data was read.
                      UNITS:      N/A
                      TYPE:       INTEGER
                      DIMENSION:  Scalar
                      ATTRIBUTES: OPTIONAL, INTENT(OUT)
 
 
  FUNCTION RESULT:
        Error_Status: The return value is an integer defining the error status.
                      The error codes are defined in the Message_Handler module.
                      If == SUCCESS, the file read was successful
                         == FAILURE, an unrecoverable error occurred.
                      UNITS:      N/A
                      TYPE:       INTEGER
                      DIMENSION:  Scalar
 
  \end{alltt}

\subsection{\texttt{CRTM\_RTSolution\_WriteFile} interface}
  \label{sec:CRTM_RTSolution_WriteFile_interface}
  \begin{alltt}
 
  NAME:
        CRTM_RTSolution_WriteFile
 
  PURPOSE:
        Function to write CRTM RTSolution object files.
 
  CALLING SEQUENCE:
        Error_Status = CRTM_RTSolution_WriteFile( Filename     , &
                                                  RTSolution   , &
                                                  Quiet = Quiet  )
 
  INPUTS:
        Filename:     Character string specifying the name of the
                      RTSolution format data file to write.
                      UNITS:      N/A
                      TYPE:       CHARACTER(*)
                      DIMENSION:  Scalar
                      ATTRIBUTES: INTENT(IN)
 
        RTSolution:   CRTM RTSolution object array containing the RTSolution
                      data.
                      UNITS:      N/A
                      TYPE:       CRTM_RTSolution_type
                      DIMENSION:  Rank-2 (n_Channels x n_Profiles)
                      ATTRIBUTES: INTENT(IN)
 
  OPTIONAL INPUTS:
        Quiet:        Set this logical argument to suppress INFORMATION
                      messages being printed to stdout
                      If == .FALSE., INFORMATION messages are OUTPUT [DEFAULT].
                         == .TRUE.,  INFORMATION messages are SUPPRESSED.
                      If not specified, default is .FALSE.
                      UNITS:      N/A
                      TYPE:       LOGICAL
                      DIMENSION:  Scalar
                      ATTRIBUTES: INTENT(IN), OPTIONAL
 
  FUNCTION RESULT:
        Error_Status: The return value is an integer defining the error status.
                      The error codes are defined in the Message_Handler module.
                      If == SUCCESS, the file write was successful
                         == FAILURE, an unrecoverable error occurred.
                      UNITS:      N/A
                      TYPE:       INTEGER
                      DIMENSION:  Scalar
 
  SIDE EFFECTS:
        - If the output file already exists, it is overwritten.
        - If an error occurs during *writing*, the output file is deleted before
          returning to the calling routine.
 
  \end{alltt}



\clearpage
\section{\Options{} Structure}
%=============================
\label{sec:options_structure}

\begin{figure}[htp]
  \centering
  \doublebox{
  \begin{minipage}[b]{6.5in}
    \begin{alltt}
  TYPE :: CRTM_Options_type
    ! Allocation indicator
    LOGICAL :: Is_Allocated = .FALSE.

    ! Input checking on by default
    LOGICAL :: Check_Input = .TRUE.

    ! User defined MW water emissivity algorithm
    LOGICAL :: Use_Old_MWSSEM = .FALSE.

    ! Antenna correction application
    LOGICAL :: Use_Antenna_Correction = .FALSE.

    ! NLTE radiance correction is ON by default
    LOGICAL :: Apply_NLTE_Correction = .TRUE.

    ! RT Algorithm is set to ADA by default
    INTEGER(Long) :: RT_Algorithm_Id = RT_ADA

    ! Aircraft flight level pressure
    ! Value > 0 turns "on" the aircraft option
    REAL(Double) :: Aircraft_Pressure = -ONE

    ! User defined number of RT solver streams (streams up + streams down)
    LOGICAL       :: Use_n_Streams = .FALSE.
    INTEGER(Long) :: n_Streams = 0

    ! Scattering switch. Default is for
    ! Cloud/Aerosol scattering to be included.
    LOGICAL :: Include_Scattering = .TRUE.

    ! Cloud cover overlap id is set to averaging type by default
    INTEGER(Long) :: Overlap_Id = DEFAULT_OVERLAP_ID

    ! User defined emissivity/reflectivity
    ! ...Dimensions
    INTEGER(Long) :: n_Channels = 0  ! L dimension
    ! ...Index into channel-specific components
    INTEGER(Long) :: Channel = 0
    ! ...Emissivity optional arguments
    LOGICAL :: Use_Emissivity = .FALSE.
    REAL(Double), ALLOCATABLE :: Emissivity(:)  ! L
    ! ...Direct reflectivity optional arguments
    LOGICAL :: Use_Direct_Reflectivity = .FALSE.
    REAL(Double), ALLOCATABLE :: Direct_Reflectivity(:) ! L

    ! SSU instrument input
    TYPE(SSU_Input_type) :: SSU

    ! Zeeman-splitting input
    TYPE(Zeeman_Input_type) :: Zeeman

  END TYPE CRTM_Options_type
    \end{alltt}
  \end{minipage}
  }
  \caption{CRTM\_Options\_type structure definition.}
  \label{fig:CRTM_Options_type_structure}
\end{figure}


\begin{table}[htp]
  \centering
  \begin{tabular}{l p{8.5cm} c c}
    \hline
    \sffamily\textbf{Component} & \sffamily\textbf{Description} & \sffamily\textbf{Units} & \sffamily\textbf{Dimensions} \\
    \hline\hline
    \texttt{Check\_Input}                 & Logical switch to enable or disable input data checking. If:

    \parbox{7cm}{\hspace{0.5cm}\texttt{.FALSE.}: No input data check.
    
                 \hspace{0.5cm}\texttt{.TRUE. }: Input data \emph{is} checked {\footnotesize [DEFAULT]}.}
     & N/A & Scalar \\
    \texttt{n\_Channels}                  & Number of sensor channels (\texttt{L}). & N/A & Scalar \\
    \texttt{Channel}                      & Index into channel-specific components. & N/A & Scalar \\
    \texttt{Use\_Emissivity}              & Logical switch to apply user-defined surface emissivity. If:

    \parbox{7cm}{\hspace{0.5cm}\texttt{.FALSE.}: Calculate emissivity {\footnotesize [DEFAULT]}.
    
                 \hspace{0.5cm}\texttt{.TRUE. }: Use user-defined emissivity}
     & N/A & Scalar \\
    \texttt{Emissivity}                   & User-defined surface emissivity for each sensor channel. & N/A & \texttt{L} \\
    \texttt{Use\_Direct\_Reflectivity}   & Logical switch to apply user-defined reflectivity for downwelling source (e.g. solar). This switch is ignored unless the \texttt{Use\_Emissivity} switch is also set. If:

    \parbox{7cm}{\hspace{0.5cm}\texttt{.FALSE.}: Calculate reflectivity {\footnotesize [DEFAULT]}.
    
                 \hspace{0.5cm}\texttt{.TRUE. }: Use user-defined reflectivity}
     & N/A & Scalar \\
    \texttt{Direct\_Reflectivity}         & User-defined direct reflectivity for downwelling source for each sensor channel. & N/A & \texttt{L} \\
    \texttt{Use\_Antenna\_Correction}     & Logical switch to apply antenna correction for the AMSU-A, AMSU-B, and MHS sensors. Note that for this switch to be effective in the CRTM call, the FOV field of the input \Geometry{} structure must be set and the antenna correction coefficients must be present in the sensor \SpcCoeff{} datafile. If:

    \parbox{7cm}{\hspace{0.5cm}\texttt{.FALSE.}: No correction {\footnotesize [DEFAULT]}.
    
                 \hspace{0.5cm}\texttt{.TRUE. }: Apply antenna correction.}
     & N/A & Scalar \\
    \texttt{SSU}                          & Structure component containing optional SSU sensor-specific input. See section \ref{sec:ssu_input_structure}. & N/A & Scalar \\
    \texttt{Zeeman}                       & Structure component containing optional input for those sensors where Zeeman-splitting is an issue for high-peaking channels. See section \ref{sec:zeeman_input_structure}. & N/A & Scalar \\
    \hline
  \end{tabular}
  \caption{CRTM \Options{} structure component description}
  \label{tab:options_structure}
\end{table}

% Options structure methods
%--------------------------
\clearpage
\subsection{\texttt{CRTM\_Options\_Associated} interface}
  \label{sec:CRTM_Options_Associated_interface}
  \begin{alltt}
 
  NAME:
        CRTM_Options_Associated
 
  PURPOSE:
        Elemental function to test the status of the allocatable components
        of a CRTM Options object.
 
  CALLING SEQUENCE:
        Status = CRTM_Options_Associated( Options )
 
  OBJECTS:
        Options:      Options structure which is to have its member's
                      status tested.
                      UNITS:      N/A
                      TYPE:       CRTM_Options_type
                      DIMENSION:  Scalar or any rank
                      ATTRIBUTES: INTENT(IN)
 
  FUNCTION RESULT:
        Status:       The return value is a logical value indicating the
                      status of the Options members.
                        .TRUE.  - if the array components are allocated.
                        .FALSE. - if the array components are not allocated.
                      UNITS:      N/A
                      TYPE:       LOGICAL
                      DIMENSION:  Same as input Options argument
 
  \end{alltt}

\subsection{\texttt{CRTM\_Options\_Create} interface}
  \label{sec:CRTM_Options_Create_interface}
  \begin{alltt}
 
  NAME:
        CRTM_Options_Create
 
  PURPOSE:
        Elemental subroutine to create an instance of the CRTM Options object.
 
  CALLING SEQUENCE:
        CALL CRTM_Options_Create( Options, n_Channels )
 
  OBJECTS:
        Options:      Options structure.
                      UNITS:      N/A
                      TYPE:       CRTM_Options_type
                      DIMENSION:  Scalar or any rank
                      ATTRIBUTES: INTENT(OUT)
 
  INPUTS:
        n_Channels:   Number of channels for which there is Options data.
                      Must be > 0.
                      This dimension only applies to the emissivity-related
                      components.
                      UNITS:      N/A
                      TYPE:       INTEGER
                      DIMENSION:  Same as Options object
                      ATTRIBUTES: INTENT(IN)
 
  \end{alltt}

\subsection{\texttt{CRTM\_Options\_DefineVersion} interface}
  \label{sec:CRTM_Options_DefineVersion_interface}
  \begin{alltt}
 
  NAME:
        CRTM_Options_DefineVersion
 
  PURPOSE:
        Subroutine to return the module version information.
 
  CALLING SEQUENCE:
        CALL CRTM_Options_DefineVersion( Id )
 
  OUTPUTS:
        Id:            Character string containing the version Id information
                       for the module.
                       UNITS:      N/A
                       TYPE:       CHARACTER(*)
                       DIMENSION:  Scalar
                       ATTRIBUTES: INTENT(OUT)
 
  \end{alltt}

\subsection{\texttt{CRTM\_Options\_Destroy} interface}
  \label{sec:CRTM_Options_Destroy_interface}
  \begin{alltt}
 
  NAME:
        CRTM_Options_Destroy
 
  PURPOSE:
        Elemental subroutine to re-initialize CRTM Options objects.
 
  CALLING SEQUENCE:
        CALL CRTM_Options_Destroy( Options )
 
  OBJECTS:
        Options:      Re-initialized Options structure.
                      UNITS:      N/A
                      TYPE:       CRTM_Options_type
                      DIMENSION:  Scalar OR any rank
                      ATTRIBUTES: INTENT(OUT)
 
  \end{alltt}

\subsection{\texttt{CRTM\_Options\_Inspect} interface}
  \label{sec:CRTM_Options_Inspect_interface}
  \begin{alltt}
 
  NAME:
        CRTM_Options_Inspect
 
  PURPOSE:
        Subroutine to print the contents of a CRTM Options object to stdout.
 
  CALLING SEQUENCE:
        CALL CRTM_Options_Inspect( Options )
 
  INPUTS:
        Options:       CRTM Options object to display.
                       UNITS:      N/A
                       TYPE:       CRTM_Options_type
                       DIMENSION:  Scalar
                       ATTRIBUTES: INTENT(IN)
 
  \end{alltt}

\subsection{\texttt{CRTM\_Options\_IsValid} interface}
  \label{sec:CRTM_Options_IsValid_interface}
  \begin{alltt}
 
  NAME:
        CRTM_Options_IsValid
 
  PURPOSE:
        Non-pure function to perform some simple validity checks on a
        CRTM Options object. 
 
        If invalid data is found, a message is printed to stdout.
 
  CALLING SEQUENCE:
        result = CRTM_Options_IsValid( opt )
 
          or
 
        IF ( CRTM_Options_IsValid( opt ) ) THEN....
 
  OBJECTS:
        opt:       CRTM Options object which is to have its
                   contents checked.
                   UNITS:      N/A
                   TYPE:       CRTM_Options_type
                   DIMENSION:  Scalar
                   ATTRIBUTES: INTENT(IN)
 
  FUNCTION RESULT:
        result:    Logical variable indicating whether or not the input
                   passed the check.
                   If == .FALSE., Options object is unused or contains
                                  invalid data.
                      == .TRUE.,  Options object can be used in CRTM.
                   UNITS:      N/A
                   TYPE:       LOGICAL
                   DIMENSION:  Scalar
 
  \end{alltt}



\clearpage
\section{\SSUInput{} Structure}
%=============================
\label{sec:ssu_input_structure}
The \SSUInput{} structure is a component of the \Options{} input structure. Note in figure \ref{fig:SSU_Input_type_structure} that the structure is declared as \texttt{PRIVATE}. As such, the only way to set values in, or get values from, the structure is via the \hyperref[sec:SSU_Input_SetValue_interface]{\texttt{SSU\_Input\_SetValue}} or \hyperref[sec:SSU_Input_GetValue_interface]{\texttt{SSU\_Input\_GetValue}} subroutines respectively.

\begin{figure}[htp]
  \centering
  \doublebox{
  \begin{minipage}[b]{6.5in}
    \begin{alltt}
  TYPE :: SSU_Input_type
    PRIVATE
    ! Time in decimal year (e.g. 2009.08892694 corresponds to 11:00 Feb. 2, 2009)
    REAL(fp) :: Time = ZERO
    ! SSU CO2 cell pressures (hPa)
    REAL(fp) :: Cell_Pressure(MAX_N_CHANNELS) = ZERO
  END TYPE SSU_Input_type
    \end{alltt}
  \end{minipage}
  }
  \caption{SSU\_Input\_type structure definition.}
  \label{fig:SSU_Input_type_structure}
\end{figure}


\begin{table}[htp]
  \centering
  \begin{tabular}{l p{7cm} c c}
    \hline
    \sffamily\textbf{Component} & \sffamily\textbf{Description} & \sffamily\textbf{Units} & \sffamily\textbf{Dimensions} \\
    \hline\hline
    \texttt{Time}           & Time in decimal year corresponding to SSU observation. & N/A & Scalar \\
    \texttt{Cell\_Pressure} & The SSU CO\subscript{2} cell pressures. & hPa & \texttt{MAX\_N\_CHANNELS} (3) \\
    \hline
  \end{tabular}
  \caption{CRTM \SSUInput{} structure component description}
  \label{tab:ssu_input_structure}
\end{table}

\subsection{\texttt{SSU\_Input\_CellPressureIsSet} interface}
  \label{sec:SSU_Input_CellPressureIsSet_interface}
  \begin{alltt}
 
  NAME:
        SSU_Input_CellPressureIsSet
 
  PURPOSE:
        Elemental function to determine if SSU_Input object cell pressures
        are set (i.e. > zero).
 
  CALLING SEQUENCE:
        result = SSU_Input_CellPressureIsSet( ssu )
 
          or
 
        IF ( SSU_Input_CellPressureIsSet( ssu ) ) THEN
          ...
        END IF
 
  OBJECTS:
        ssu:       SSU_Input object for which the cell pressures
                   are to be tested.
                   UNITS:      N/A
                   TYPE:       SSU_Input_type
                   DIMENSION:  Scalar or any rank
                   ATTRIBUTES: INTENT(IN)
 
  FUNCTION RESULT:
        result:    Logical variable indicating whether or not all the
                   SSU cell pressures are set.
                   If == .FALSE., cell pressure values are <= 0.0hPa and
                                  thus are considered to be NOT set or valid.
                      == .TRUE.,  cell pressure values are > 0.0hPa and
                                  thus are considered to be set and valid.
                   UNITS:      N/A
                   TYPE:       LOGICAL
                   DIMENSION:  Scalar
 
  \end{alltt}

\subsection{\texttt{SSU\_Input\_DefineVersion} interface}
  \label{sec:SSU_Input_DefineVersion_interface}
  \begin{alltt}
 
  NAME:
        SSU_Input_DefineVersion
 
  PURPOSE:
        Subroutine to return the module version information.
 
  CALLING SEQUENCE:
        CALL SSU_Input_DefineVersion( Id )
 
  OUTPUTS:
        Id:            Character string containing the version Id information
                       for the module.
                       UNITS:      N/A
                       TYPE:       CHARACTER(*)
                       DIMENSION:  Scalar
                       ATTRIBUTES: INTENT(OUT)
 
  \end{alltt}

\subsection{\texttt{SSU\_Input\_GetValue} interface}
  \label{sec:SSU_Input_GetValue_interface}
  \begin{alltt}
 
  NAME:
        SSU_Input_GetValue
  
  PURPOSE:
        Elemental subroutine to Get the values of SSU_Input
        object components.
 
  CALLING SEQUENCE:
        CALL SSU_Input_GetValue( SSU_Input                    , &
                                 Channel       = Channel      , & 
                                 Time          = Time         , & 
                                 Cell_Pressure = Cell_Pressure, & 
                                 n_Channels    = n_Channels     ) 
 
  OBJECTS:
        SSU_Input:            SSU_Input object for which component values
                              are to be set.
                              UNITS:      N/A
                              TYPE:       SSU_Input_type
                              DIMENSION:  Scalar or any rank
                              ATTRIBUTES: INTENT(IN OUT)
 
  OPTIONAL INPUTS:
        Channel:              SSU channel for which the CO2 cell pressure
                              is required.
                              UNITS:      N/A
                              TYPE:       INTEGER
                              DIMENSION:  Scalar or same as SSU_Input
                              ATTRIBUTES: INTENT(IN), OPTIONAL
 
  OPTIONAL OUTPUTS:
        Time:                 SSU instrument mission time.
                              UNITS:      decimal year
                              TYPE:       REAL(fp)
                              DIMENSION:  Scalar or same as SSU_Input
                              ATTRIBUTES: INTENT(OUT), OPTIONAL
 
        Cell_Pressure:        SSU channel CO2 cell pressure. Must be
                              specified with the Channel optional input
                              dummy argument.
                              UNITS:      hPa
                              TYPE:       REAL(fp)
                              DIMENSION:  Scalar or same as SSU_Input
                              ATTRIBUTES: INTENT(OUT), OPTIONAL
 
        n_Channels:           Number of SSU channels..
                              UNITS:      N/A
                              TYPE:       INTEGER
                              DIMENSION:  Scalar or same as SSU_Input
                              ATTRIBUTES: INTENT(OUT), OPTIONAL
 
  \end{alltt}

\subsection{\texttt{SSU\_Input\_Inspect} interface}
  \label{sec:SSU_Input_Inspect_interface}
  \begin{alltt}
 
  NAME:
        SSU_Input_Inspect
 
  PURPOSE:
        Subroutine to print the contents of an SSU_Input object to stdout.
 
  CALLING SEQUENCE:
        CALL SSU_Input_Inspect( ssu )
 
  INPUTS:
        ssu:           SSU_Input object to display.
                       UNITS:      N/A
                       TYPE:       SSU_Input_type
                       DIMENSION:  Scalar
                       ATTRIBUTES: INTENT(IN)
 
  \end{alltt}

\subsection{\texttt{SSU\_Input\_IsValid} interface}
  \label{sec:SSU_Input_IsValid_interface}
  \begin{alltt}
 
  NAME:
        SSU_Input_IsValid
 
  PURPOSE:
        Non-pure function to perform some simple validity checks on a
        SSU_Input object. 
 
        If invalid data is found, a message is printed to stdout.
 
  CALLING SEQUENCE:
        result = SSU_Input_IsValid( ssu )
 
          or
 
        IF ( SSU_Input_IsValid( ssu ) ) THEN....
 
  OBJECTS:
        ssu:       SSU_Input object which is to have its
                   contents checked.
                   UNITS:      N/A
                   TYPE:       SSU_Input_type
                   DIMENSION:  Scalar
                   ATTRIBUTES: INTENT(IN)
 
  FUNCTION RESULT:
        result:    Logical variable indicating whether or not the input
                   passed the check.
                   If == .FALSE., object is unused or contains
                                  invalid data.
                      == .TRUE.,  object can be used.
                   UNITS:      N/A
                   TYPE:       LOGICAL
                   DIMENSION:  Scalar
 
  \end{alltt}

\subsection{\texttt{SSU\_Input\_SetValue} interface}
  \label{sec:SSU_Input_SetValue_interface}
  \begin{alltt}
 
  NAME:
        SSU_Input_SetValue
 
  PURPOSE:
        Elemental subroutine to set the values of SSU_Input
        object components.
 
  CALLING SEQUENCE:
        CALL SSU_Input_SetValue( SSU_Input                    , &
                                 Time          = Time         , &
                                 Cell_Pressure = Cell_Pressure, &
                                 Channel       = Channel        )
 
  OBJECTS:
        SSU_Input:            SSU_Input object for which component values
                              are to be set.
                              UNITS:      N/A
                              TYPE:       SSU_Input_type
                              DIMENSION:  Scalar or any rank
                              ATTRIBUTES: INTENT(IN OUT)
 
  OPTIONAL INPUTS:
        Time:                 SSU instrument mission time.
                              UNITS:      decimal year
                              TYPE:       REAL(fp)
                              DIMENSION:  Scalar or same as SSU_Input
                              ATTRIBUTES: INTENT(IN), OPTIONAL
 
        Cell_Pressure:        SSU channel CO2 cell pressure. Must be
                              specified with the Channel optional dummy
                              argument.
                              UNITS:      hPa
                              TYPE:       REAL(fp)
                              DIMENSION:  Scalar or same as SSU_Input
                              ATTRIBUTES: INTENT(IN), OPTIONAL
 
        Channel:              SSU channel for which the CO2 cell pressure
                              is to be set. Must be specified with the
                              Cell_Pressure optional dummy argument.
                              UNITS:      N/A
                              TYPE:       INTEGER
                              DIMENSION:  Scalar or same as SSU_Input
                              ATTRIBUTES: INTENT(IN), OPTIONAL
 
  \end{alltt}


\clearpage
\section{\ZeemanInput{} Structure}
%=============================
\label{sec:zeeman_input_structure}
The \ZeemanInput{} structure is a component of the \Options{} input structure. Note in figure \ref{fig:Zeeman_Input_type_structure} that the structure is declared as \texttt{PRIVATE}. As such, the only way to set values in, or get values from, the structure is via the \hyperref[sec:Zeeman_Input_SetValue_interface]{\texttt{Zeeman\_Input\_SetValue}} or \hyperref[sec:Zeeman_Input_GetValue_interface]{\texttt{Zeeman\_Input\_GetValue}} subroutines respectively.

\begin{figure}[htp]
  \centering
  \doublebox{
  \begin{minipage}[b]{6.5in}
    \begin{alltt}
  TYPE :: Zeeman_Input_type
    PRIVATE
    ! Earth magnetic field strength in Gauss
    REAL(fp) :: Be = DEFAULT_MAGENTIC_FIELD
    ! Cosine of the angle between the Earth
    ! magnetic field and wave propagation direction                     
    REAL(fp) :: Cos_ThetaB = ZERO
    ! Cosine of the azimuth angle of the Be vector.
    REAL(fp) :: Cos_PhiB = ZERO
    ! Doppler frequency shift caused by Earth-rotation.
    REAL(fp) :: Doppler_Shift = ZERO 
  END TYPE Zeeman_Input_type
    \end{alltt}
  \end{minipage}
  }
  \caption{Zeeman\_Input\_type structure definition.}
  \label{fig:Zeeman_Input_type_structure}
\end{figure}


\begin{table}[htp]
  \centering
  \begin{tabular}{l p{7cm} c c}
    \hline
    \sffamily\textbf{Component} & \sffamily\textbf{Description} & \sffamily\textbf{Units} & \sffamily\textbf{Dimensions} \\
    \hline\hline
    \texttt{Be}           & Earth magnetic field strength. & Gauss & Scalar \\
    \texttt{Cos\_ThetaB}  & Cosine of the angle between the Earth magnetic field and wave propagation direction. & N/A & Scalar \\
    \texttt{Cos\_PhiB}    & Cosine of the azimuth angle of the $\mathbf{B}_e$ vector in the $(\mathbf{v}, \mathbf{h}, \mathbf{k})$ coordinates system, where $\mathbf{v}$, $\mathbf{h}$ and $\mathbf{k}$ comprise a right-hand orthogonal system, similar to the $(\mathbf{x}, \mathbf{y}, \mathbf{z})$ Cartesian coordinates. The $\mathbf{h}$ vector is normal to the plane containing the $\mathbf{k}$ and $\mathbf{z}$ vectors, where $\mathbf{k}$ points to the wave propagation direction and $\mathbf{z}$ points to the zenith. $\mathbf{h} = (\mathbf{z} \times \mathbf{k})/|\mathbf{z} \times \mathbf{k}|$. The azimuth angle is the angle on the $(\mathbf{v}, \mathbf{h})$ plane from the positive $\mathbf{v}$ axis to the projected line of the $\mathbf{B}_e$ vector on this plane, positive counterclockwise. & N/A & Scalar \\
    \texttt{Doppler\_Shift}  & Doppler frequency shift caused by Earth-rotation (positive towards sensor). A zero value means no frequency shift. & KHz & Scalar \\
    \hline
  \end{tabular}
  \caption{CRTM \ZeemanInput{} structure component description}
  \label{tab:zeeman_input_structure}
\end{table}

\subsection{\texttt{Zeeman\_Input\_DefineVersion} interface}
  \label{sec:Zeeman_Input_DefineVersion_interface}
  \begin{alltt}
 
  NAME:
        Zeeman_Input_DefineVersion
 
  PURPOSE:
        Subroutine to return the module version information.
 
  CALLING SEQUENCE:
        CALL Zeeman_Input_DefineVersion( Id )
 
  OUTPUTS:
        Id:            Character string containing the version Id information
                       for the module.
                       UNITS:      N/A
                       TYPE:       CHARACTER(*)
                       DIMENSION:  Scalar
                       ATTRIBUTES: INTENT(OUT)
 
  \end{alltt}

\subsection{\texttt{Zeeman\_Input\_GetValue} interface}
  \label{sec:Zeeman_Input_GetValue_interface}
  \begin{alltt}
 
  NAME:
        Zeeman_Input_GetValue
  
  PURPOSE:
        Elemental subroutine to get the values of Zeeman_Input
        object components.
 
  CALLING SEQUENCE:
        CALL Zeeman_Input_GetValue( Zeeman_Input                   , &
                                    Field_Strength = Field_Strength, & 
                                    Cos_ThetaB     = Cos_ThetaB    , & 
                                    Cos_PhiB       = Cos_PhiB      , &
                                    Doppler_Shift  = Doppler_Shift   )
 
  OBJECTS:
        Zeeman_Input:         Zeeman_Input object for which component values
                              are to be set.
                              UNITS:      N/A
                              TYPE:       Zeeman_Input_type
                              DIMENSION:  Scalar or any rank
                              ATTRIBUTES: INTENT(IN OUT)
 
  OPTIONAL OUTPUTS:
        Field_Strength:       Earth's magnetic filed strength
                              UNITS:      Gauss
                              TYPE:       REAL(fp)
                              DIMENSION:  Scalar or same as Zeeman_Input
                              ATTRIBUTES: INTENT(OUT), OPTIONAL
 
        Cos_ThetaB:           Cosine of the angle between the Earth magnetic
                              field and wave propagation vectors.
                              UNITS:      N/A
                              TYPE:       REAL(fp)
                              DIMENSION:  Scalar or same as Zeeman_Input
                              ATTRIBUTES: INTENT(OUT), OPTIONAL
 
        Cos_PhiB:             Cosine of the azimuth angle of the Earth magnetic
                              field vector.
                              UNITS:      N/A
                              TYPE:       REAL(fp)
                              DIMENSION:  Scalar or same as Zeeman_Input
                              ATTRIBUTES: INTENT(OUT), OPTIONAL
 
        Doppler_Shift:        Doppler frequency shift caused by Earth-rotation.
                              Positive towards sensor.
                              UNITS:      KHz
                              TYPE:       REAL(fp)
                              DIMENSION:  Scalar or same as Zeeman_Input
                              ATTRIBUTES: INTENT(OUT), OPTIONAL
 
  \end{alltt}

\subsection{\texttt{Zeeman\_Input\_Inspect} interface}
  \label{sec:Zeeman_Input_Inspect_interface}
  \begin{alltt}
 
  NAME:
        Zeeman_Input_Inspect
 
  PURPOSE:
        Subroutine to print the contents of an Zeeman_Input object to stdout.
 
  CALLING SEQUENCE:
        CALL Zeeman_Input_Inspect( z )
 
  INPUTS:
        z:             Zeeman_Input object to display.
                       UNITS:      N/A
                       TYPE:       Zeeman_Input_type
                       DIMENSION:  Scalar
                       ATTRIBUTES: INTENT(IN)
 
  \end{alltt}

\subsection{\texttt{Zeeman\_Input\_IsValid} interface}
  \label{sec:Zeeman_Input_IsValid_interface}
  \begin{alltt}
 
  NAME:
        Zeeman_Input_IsValid
 
  PURPOSE:
        Non-pure function to perform some simple validity checks on a
        Zeeman_Input object. 
 
        If invalid data is found, a message is printed to stdout.
 
  CALLING SEQUENCE:
        result = Zeeman_Input_IsValid( z )
 
          or
 
        IF ( Zeeman_Input_IsValid( z ) ) THEN....
 
  OBJECTS:
        z:         Zeeman_Input object which is to have its
                   contents checked.
                   UNITS:      N/A
                   TYPE:       Zeeman_Input_type
                   DIMENSION:  Scalar
                   ATTRIBUTES: INTENT(IN)
 
  FUNCTION RESULT:
        result:    Logical variable indicating whether or not the input
                   passed the check.
                   If == .FALSE., object is unused or contains
                                  invalid data.
                      == .TRUE.,  object can be used.
                   UNITS:      N/A
                   TYPE:       LOGICAL
                   DIMENSION:  Scalar
 
  \end{alltt}

\subsection{\texttt{Zeeman\_Input\_SetValue} interface}
  \label{sec:Zeeman_Input_SetValue_interface}
  \begin{alltt}
 
  NAME:
        Zeeman_Input_SetValue
  
  PURPOSE:
        Elemental subroutine to set the values of Zeeman_Input
        object components.
 
  CALLING SEQUENCE:
        CALL Zeeman_Input_SetValue( Zeeman_Input                   , &
                                    Field_Strength = Field_Strength, & 
                                    Cos_ThetaB     = Cos_ThetaB    , & 
                                    Cos_PhiB       = Cos_PhiB      , &
                                    Doppler_Shift  = Doppler_Shift   )
 
  OBJECTS:
        Zeeman_Input:         Zeeman_Input object for which component values
                              are to be set.
                              UNITS:      N/A
                              TYPE:       Zeeman_Input_type
                              DIMENSION:  Scalar or any rank
                              ATTRIBUTES: INTENT(IN OUT)
 
  OPTIONAL INPUTS:
        Field_Strength:       Earth's magnetic filed strength
                              UNITS:      Gauss
                              TYPE:       REAL(fp)
                              DIMENSION:  Scalar or same as Zeeman_Input
                              ATTRIBUTES: INTENT(IN), OPTIONAL
 
        Cos_ThetaB:           Cosine of the angle between the Earth magnetic
                              field and wave propagation vectors.
                              UNITS:      N/A
                              TYPE:       REAL(fp)
                              DIMENSION:  Scalar or same as Zeeman_Input
                              ATTRIBUTES: INTENT(IN), OPTIONAL
 
        Cos_PhiB:             Cosine of the azimuth angle of the Earth magnetic
                              field vector.
                              UNITS:      N/A
                              TYPE:       REAL(fp)
                              DIMENSION:  Scalar or same as Zeeman_Input
                              ATTRIBUTES: INTENT(IN), OPTIONAL
 
        Doppler_Shift:        Doppler frequency shift caused by Earth-rotation.
                              Positive towards sensor.
                              UNITS:      KHz
                              TYPE:       REAL(fp)
                              DIMENSION:  Scalar or same as Zeeman_Input
                              ATTRIBUTES: INTENT(IN), OPTIONAL
 
  \end{alltt}

