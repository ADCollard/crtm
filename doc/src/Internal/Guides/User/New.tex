\chapter*{What's New in v2.1}
%===========================
\addcontentsline{toc}{chapter}{What's New in v2.1}

\section*{New Science}
%---------------------
\addcontentsline{toc}{section}{New Science}

\begin{description}
\item[Updated microwave sea surface emissivity model] The FASTEM4/5 microwave sea surface emissivity models have been implemented. FASTEM5 is the default (via a file loaded during initialisation) and FASTEM4 \citep{QLiu_2011} can be selected by specifying the appropriate data file during CRTM initialisation. The previous model, a combination of FASTEM1 \citep{Fastem1} and a low frequency model \citep{Kazumori_2008}, can still be invoked via the options input to the CRTM functions.

\item[Non-LTE for hyperspectral infrared sensors] A model to correct daytime radiances for the non-LTE effect in the shortwave infrared channels has been implemented. Currently the correction is applied only to the hyperspectral infrared sensors; AIRS (Aqua), IASI (MetOp-A/B), and CrIS (Suomi NPP).

\item[Successive Order of Interaction (SOI) radiative transfer algorithm] An alternative radiative transfer (RT) solution algorithm \citep{SOI_1} has been implemented and can be selected for use via the options input to the CRTM functions. The default RT solver still remains the Advanced Doubling-Adding (ADA) method (REF!!!).

\end{description}


\section*{New Functionality}
%---------------------------
\addcontentsline{toc}{section}{New Functionality}

\begin{description}
\item[Aerosol optical depth functions] Separate functions to compute just the aerosol optical depth have been implemented. The new main level forward, tangent-linear, adjoint, and K-matrix functions are \texttt{CRTM\_AOD()}, \texttt{CRTM\_AOD\_TL()}, \texttt{CRTM\_AOD\_AD()}, and \texttt{CRTM\_AOD\_K()} respectively. See section(REF!!) for the function interfaces.

\item[Channel subsetting] To allow users to select which channels of a sensor will be processed, a channel subsetting function has been added. This subsetting operates on the \ChannelInfo structure and is invoked by passing the list of required channel numbers to a new \texttt{CRTM\_ChannelInfo\_Subset()} function. See section (REF!!) for the function interface and section (REF!!) for usage examples.

\item[Number of streams option] For scattering atmospheres the current method to determine the number of streams to be employed in the radiative transfer calculation is based upon the Mie parameter. Generally this methodology yields a higher number of streams than is necessary. A better ``stream selection'' method is under development and is slated for the v2.2 CRTM release. Part of this work led to the implementation of an \texttt{n\_Streams} option - that is, the user can explicitly state the number of streams they wish to use for scattering calculations and override any value determined internally. The user-define number of streams is set via the options input to the CRTM functions.

\item[Scattering switch option for clouds and aerosols] This implements a user-selectable switch to ``skip'' the scattering computations and only compute the cloud and aerosol absorption component when clouds and aerosols are present. The scattering switch is set via the options input to the CRTM functions.

\item[Aircraft instrument capability] The ability to simulate an aircraft instrument has been implemented in the CRTM. The user indicates that the calculation is for an aircraft instrument by specifying the flight level pressure in the options input to the CRTM functions. Note, however, that no spectral or transmittance coefficients are available for aircraft instruments. If you wish to run the CRTM for a particular aircraft sensor (microwave, infrared or visible) email the CRTM developers at \href{mailto:ncep.list.emc.jcsda_crtm.support@noaa.gov}{ncep.list.emc.jcsda\_crtm.support@noaa.gov}.

\item[Options structure I/O] Previously, the CRTM \Options structure was different from the other user accessible CRTM structures (e.g. \Atmosphere, \Surface, \Geometry, etc) in that there were no means to write and read the structure to/from file. This oversight has been corrected. See section(REF!!) for the function interfaces.

\end{description}


\section*{Interface Changes}
%---------------------------
\addcontentsline{toc}{section}{Interface Changes}
\label{sec:new_interface_changes}

\begin{description}
\item[Surface type specification changes] The specification of surface type in the CRTM surface structure was previously hardwired to use the NPOESS land surface classification scheme (infrared and visible spectral regions only). For users that employed a different land surface classification scheme, in particular those from USGS or IGBP, it meant there was a classification scheme remapping that was required to assign the ``correct'' NPOESS surface type for a particular USGS or IGBP surface type. To avoid the need to do this remapping, the land surface reflectivity data has now been provided in terms of three surface classification schemes: NPOESS (the default), USGS, and IGBP. These are loaded into the CRTM during the initialization stage.

Previously land surface type parameters such as \texttt{SCRUB} or \texttt{BROADLEAF\_FOREST} were available to refer to a unique surface type index that was used to reference a look up table of spectral reflectances. Now, however, the list of allowable surface types can be different based on the classification scheme with which the CRTM was initialized, and thus the numeric index for a surface type in the list is no longer unique to that surface type. This means there can no longer be a list of pre-specified parameterized surface types like there was with v2.0.x of the CRTM.

Tables \ref{tab:npoess_surface_type_classifications}, \ref{tab:usgs_surface_type_classifications}, and \ref{tab:igbp_surface_type_classifications} show the surface types, and their index, available for the NPOESS, USGS, and IGBP land surface classification schemes respectively.



\item[Emissivity model initialisation file changes] words

\end{description}


To migrate from the CRTM v2.0.x initialisation and surface type specification to that implemented in v2.1, see Appendix \ref{sec:migration_path}, ``\hyperref[sec:migration_path]{Migration Path from REL-2.0 to REL-2.1}.''


