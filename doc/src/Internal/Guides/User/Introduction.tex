\chapter{Introduction}
%=====================

What version does this apply to?

Overview of CRTM; nomenclature (AtmAbsorption, SfcOptics, CloudScatter, RTSolution etc)

Flowcharts



\section{Conventions}
%====================
\label{sec:conventions}
The following are conventions that have been adhered to in the current release of the CRTM framework. They are guidelines intended to make understanding the code at a glance easier, to provide a recognisable ``look and feel'', and to minimise name space clashes.

\subsection{Naming of Structure Types and Instances of Structures}
%-----------------------------------------------------------------
The derived data type, or structure\footnote{The terms ``derived type'' and ``structure'' are used interchangably in this document.} type, naming convention adopted for use in the CRTM is, 

\hspace{0.5cm}\f{[CRTM\_]}\textit{name}\f{\_type} 

where \textit{name} is an identifier that indicates for what a structure is to be used. All structure type names are suffixed with ``\f{\_type}'' and CRTM-specific structure types are prefixed with ``\f{CRTM\_}''. Some examples are,
\begin{ttfamily}
  \begin{verbatim}
  SpcCoeff_type
  CRTM_Atmosphere_type
  CRTM_RTSolution_type\end{verbatim}
\end{ttfamily}
An instance of a structure is then referred to via its \textit{name}, or some sort of derivate of its \textit{name}. Some structure declarations examples are,
\begin{ttfamily}
  \begin{verbatim}
  TYPE(SpcCoeff_type)        :: SpcCoeff
  TYPE(CRTM_Atmosphere_type) :: atm, atm_K
  TYPE(CRTM_RTSolution_type) :: rts, rts_K\end{verbatim}
\end{ttfamily}
where the K-matrix structure variables are identified with a ``\f{\_K}'' suffix. Similarly, tangent-linear and adjoint variables are suffixed with ``\f{\_TL}'' or ``\f{\_AD}'' respectively.

\subsection{Naming of Definition Modules}
%----------------------------------------
Modules containing structure type definitions are termed \textit{definition modules}. These modules contain the actual structure definitions as well as various utility procedures to allocate, destroy, copy etc. structures of the designated type. The naming convention adopted for definition modules in the CRTM is, 

\hspace{0.5cm}\f{[CRTM\_]}\textit{name}\f{\_Define} 

where, as with the structure type names, all definition module names are suffixed with ``\f{\_Define}'' and CRTM-specific definition modules are prefixed with ``\f{CRTM\_}''. Some examples are,
\begin{ttfamily}
  \begin{verbatim}
  SpcCoeff_Define
  CRTM_Atmosphere_Define
  CRTM_RTSolution_Define\end{verbatim}
\end{ttfamily}
The actual source code files for these modules have the same name with a ``\f{.f90}'' suffix.

\subsection{Naming of I/O Modules}
%---------------------------------
Modules containing the routines that read and write data from and to files are, naturally, termed \textit{I/O modules}. The naming convention adopted for application modules in the CRTM is, 

\hspace{0.5cm}\f{CRTM\_}\textit{name}

Some examples are,
\begin{ttfamily}
  \begin{verbatim}
  CRTM_AtmAbsorption
  CRTM_SfcOptics
  CRTM_RTSolution\end{verbatim}
\end{ttfamily}
However, in this case, \textit{name} does not necessarilty refer just to a structure type. Separate application modules are used as required to split up tasks in manageable (and easily maintained) chunks. For example, separate modules have been provided to contain the cloud and aerosol optical property retrieval; similarly separate modules handle different surface types for different instrument types in computing surface optics.

\subsection{Naming of Application Modules}
%-----------------------------------------
Modules containing the routines that perform the calculations for the various components of the CRTM are termed \textit{application modules}. The naming convention adopted for application modules in the CRTM is, 

\hspace{0.5cm}\f{CRTM\_}\textit{name}

Some examples are,
\begin{ttfamily}
  \begin{verbatim}
  CRTM_AtmAbsorption
  CRTM_SfcOptics
  CRTM_RTSolution\end{verbatim}
\end{ttfamily}
However, in this case, \textit{name} does not necessarilty refer just to a structure type. Separate application modules are used as required to split up tasks in manageable (and easily maintained) chunks. For example, separate modules have been provided to contain the cloud and aerosol optical property retrieval; similarly separate modules handle different surface types for different instrument types in computing surface optics.

Again, the actual source code files for these modules have the same name with a ``\f{.f90}'' suffix. Note that not all definition modules have a corresponding application module since some structures (e.g. SpcCoeff structures) are simply data containers.





\section{Components}
%===================
The radiative transfer problem is divided into three broad categories: atmospheric optics, surface optics and radiative transfer.

\subsection{Atmospheric Optics}
%------------------------------
(\AtmOptics) Includes computation of the absorption by atmospheric gases (\AtmAbsorption) and scattering and absorption by both clouds (\CloudScatter) and aerosols (\AerosolScatter).

  Concerned with the interpolation of a lut of data to obtain the optical properties of clouds and aerosols for the current set of parameters.


\subsection{Surface Optics}
%------------------------------
  Separate models for each gross surface type for each spectral type (IR or MW).

(\SfcOptics) Includes the computation of surface emissivity and reflectivity for four gross surface types (land, sea, snow, and ice).

\subsection{Radiative Transfer Solution}
%------------------------------
  The solution of the RT problem using the various bits.
(\RTSolution) Takes the \AtmOptics{} and \SfcOptics{} data and solves the radiative transfer problem.


\section{Models}
%===============
FWD, tL, AD, and K-Matrix

\begin{subequations}
  \begin{eqnarray}
    \mathbf{T_{B}},\mathbf{R} &=& \mathbf{F}(\mathbf{T},\mathbf{q},T_{s},...)\\\nonumber\\
    \mathbf{\delta{T_{B}}},\mathbf{\delta{R}} &=& \mathbf{H}(\mathbf{T},\mathbf{q},T_{s},...\mathbf{\delta{T}},\mathbf{\delta{q}},\delta{T_{s}},...)\\\nonumber\\
    \mathbf{\dstar{T}},\mathbf{\dstar{q}},\dstar{T_{s}},... &=& \mathbf{H^T}(\mathbf{T},\mathbf{q},T_{s},...\mathbf{\dstar{T_{B}}},\mathbf{\dstar{R}})\\\nonumber\\
    \mathbf{\dstar{T}}_l,\mathbf{\dstar{q}}_l,\dstar{T_{s,l}},... &=& \mathbf{K}(\mathbf{T},\mathbf{q},T_{s},...\mathbf{\dstar{T_{B}}},\mathbf{\dstar{R}})\textrm{ for }l=1,2,...,L
  \end{eqnarray}
\end{subequations}


\section{Design Framework}
%=========================

\begin{figure}[htp]
  \centering
  \input{graphics/Flowcharts/CRTM_Flowcharts.pstex_t}
  \caption{Flowchart of the CRTM Forward and K-Matrix models.}
  \label{fig:fwd_k_flowchart}
\end{figure}

