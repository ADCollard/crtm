\chapter{How to use the CRTM library}
%====================================

error status description

\section{Quick Start}
%====================

Quick start guide via examples. Refer to the following sections for more information about the structures and interfaces.

The first thing you need to do is identify what sensors you want to use. The list of supported sensors, along with their associated \texttt{Sensor\_Id}'s, are shown in appendix BLAH


%GOES-R ABI & abi\_gr 
%Aqua AIRS 281 & airs281\_aqua 
%Aqua AIRS 324 & airs324\_aqua 
%Aqua AIRS Module-1a & airsM1a\_aqua 
%Aqua AIRS Module-1b & airsM1b\_aqua 
%Aqua AIRS Module-2a & airsM2a\_aqua 
%Aqua AIRS Module-2b & airsM2b\_aqua 
%Aqua AIRS Module-3  & airsM3\_aqua 
%Aqua AIRS Module-4a & airsM4a\_aqua 
%Aqua AIRS Module-4b & airsM4b\_aqua 
%Aqua AIRS Module-4c & airsM4c\_aqua 
%Aqua AIRS Module-4d & airsM4d\_aqua 
%Aqua AIRS Module-5  & airsM5\_aqua 
%Aqua AIRS Module-6  & airsM6\_aqua 
%Aqua AIRS Module-7  & airsM7\_aqua 
%Aqua AIRS Module-8  & airsM8\_aqua 
%Aqua AIRS Module-9  & airsM9\_aqua 
%Aqua AIRS Module-10 & airsM10\_aqua 
%Aqua AIRS Module-11 & airsM11\_aqua 
%Aqua AIRS Module-12 & airsM12\_aqua 
%Aqua AIRS & airs\_aqua 
% & avhrr2\_n06 
% & avhrr2\_n07 
% & avhrr2\_n08 
% & avhrr2\_n09 
% & avhrr2\_n10 
% & avhrr2\_n11 
% & avhrr2\_n12 
% & avhrr2\_n14 
% & avhrr2\_tirosn 
% & avhrr3\_metop-a 
% & avhrr3\_n15 
% & avhrr3\_n16 
% & avhrr3\_n17 
% & avhrr3\_n18 
% & hirs2\_n06 
% & hirs2\_n07 
% & hirs2\_n08 
% & hirs2\_n09 
% & hirs2\_n10 
% & hirs2\_n11 
% & hirs2\_n12 
% & hirs2\_n14 
% & hirs2\_tirosn 
% & hirs3\_n15 
% & hirs3\_n16 
% & hirs3\_n17 
% & hirs4\_metop-a 
% & hirs4\_n18 
% & iasi300\_metop-a 
% & iasi316\_metop-a 
% & iasi616\_metop-a 
% & iasiB1\_metop-a 
% & iasiB2\_metop-a 
% & iasiB3\_metop-a 
% & iasi\_metop-a 
% & imgr\_g08 
% & imgr\_g09 
% & imgr\_g10 
% & imgr\_g11 
% & imgr\_g12 
% & imgr\_g13 
% & imgr\_mt1r 
% & modis\_aqua 
% & modis\_terra 
% & seviri\_m08 
% & seviri\_m09 
% & seviri\_m10 
% & sndr\_g08 
% & sndr\_g09 
% & sndr\_g10 
% & sndr\_g11 
% & sndr\_g12 
% & sndr\_g13 
% & ssu\_n06 
% & ssu\_n07 
% & ssu\_n08 
% & ssu\_n09 
% & ssu\_n11 
% & ssu\_n14 
% & ssu\_tirosn 
% & vissrDetA\_gms5 
% & amsre\_aqua 
% & amsua\_aqua 
% & amsua\_metop-a 
% & amsua\_metop-b 
% & amsua\_metop-c 
% & amsua\_n15 
% & amsua\_n16 
% & amsua\_n17 
% & amsua\_n18 
% & amsua\_n19 
% & amsub\_n15 
% & amsub\_n16 
% & amsub\_n17 
% & atms\_c1 
% & hsb\_aqua 
% & mhs\_metop-a 
% & mhs\_metop-b 
% & mhs\_metop-c 
% & mhs\_n18 
% & mhs\_n19 
% & msu\_n06 
% & msu\_n07 
% & msu\_n08 
% & msu\_n09 
% & msu\_n10 
% & msu\_n11 
% & msu\_n12 
% & msu\_n14 
% & msu\_tirosn 
% & ssmi\_f08 
% & ssmi\_f10 
% & ssmi\_f11 
% & ssmi\_f13 
% & ssmi\_f14 
% & ssmi\_f15 
% & ssmis\_f16 
% & ssmt1\_f13 
% & ssmt1\_f15 
% & ssmt2\_f14 
% & ssmt2\_f15 
% & windsat\_coriolis 


\section{Interface Descriptions}
%===============================

\subsection{CRTM Initialisation}
%-------------------------------

\begin{figure}[htp]
  \centering
  \doublebox{
  \begin{minipage}[b]{6.5in}
    \begin{ttfamily}
      \begin{verbatim}
  FUNCTION CRTM_Init( ChannelInfo      , &  ! Output, N
                      Sensor_ID        , &  ! Optional input, N
                      CloudCoeff_File  , &  ! Optional input
                      AerosolCoeff_File, &  ! Optional input
                      EmisCoeff_File   , &  ! Optional input
                      File_Path        , &  ! Optional input
                      Quiet            , &  ! Optional input
                      Process_ID       , &  ! Optional input
                      Output_Process_ID, &  ! Optional input
                      RCS_Id           , &  ! Revision control
                      Message_Log      ) &  ! Error messaging
                    RESULT( Error_Status )

    ! Arguments
    TYPE(CRTM_ChannelInfo_type), INTENT(IN OUT) :: ChannelInfo(:)  ! N
    CHARACTER(*),      OPTIONAL, INTENT(IN)     :: Sensor_ID(:)    ! N
    CHARACTER(*),      OPTIONAL, INTENT(IN)     :: CloudCoeff_File
    CHARACTER(*),      OPTIONAL, INTENT(IN)     :: AerosolCoeff_File
    CHARACTER(*),      OPTIONAL, INTENT(IN)     :: EmisCoeff_File
    CHARACTER(*),      OPTIONAL, INTENT(IN)     :: File_Path
    INTEGER     ,      OPTIONAL, INTENT(IN)     :: Quiet
    INTEGER     ,      OPTIONAL, INTENT(IN)     :: Process_ID
    INTEGER     ,      OPTIONAL, INTENT(IN)     :: Output_Process_ID
    CHARACTER(*),      OPTIONAL, INTENT(OUT)    :: RCS_Id
    CHARACTER(*),      OPTIONAL, INTENT(IN)     :: Message_Log
    ! Function result
    INTEGER :: Error_Status
      \end{verbatim}
    \end{ttfamily}
    \centering
    \begin{tabular}{c|c|c|c}
      \textbf{Argument} & \textbf{Description}                     & \textbf{Rank}   & \textbf{Intent} \\
      \hline\hline
      ChannelInfo        & Sensor and channel info structure       & n\_Sensors (N)  & IN OUT      \\
      \hline
      Sensor\_Id         & Sensor identification string            & n\_Sensors (N)  & IN      \\
      \hline
      CloudCoeff\_File   & Cloud optical property LUT filename     & Scalar          & IN      \\
      \hline
      AerosolCoeff\_File & Aerosol optical property LUT filename   & Scalar          & IN      \\
      \hline
      EmisCoeff\_File    & IR sea surface emissivity LUT filename  & Scalar          & IN      \\
      \hline
      File\_Path         & Path to *Coeff files                    & Scalar          & IN      \\
      \hline
      Quiet              & Keyword to control information message output & Scalar         & IN      \\
      \hline
      Process\_Id        & MPI process Id                          & Scalar          & IN      \\
      \hline
      Output\_Process\_Id & MPI process Id for message output      & Scalar          & IN      \\
      \hline
      RCS\_Id            & Version control ID for the module      & Scalar           & OUT     \\
      \hline
      Message\_Log       & Log message filename                   & Scalar           & IN      \\
    \end{tabular}
  \end{minipage}
  }
  \caption{CRTM Initialisation interface and argument description.}
  \label{fig:init_interface}
\end{figure}

\subsection{Forward Model}
%-------------------------

\begin{figure}[htp]
  \centering
  \doublebox{
  \begin{minipage}[b]{6.5in}
    \begin{ttfamily}
      \begin{verbatim}
  FUNCTION CRTM_Forward( Atmosphere  , &  ! Input, M
                         Surface     , &  ! Input, M    
                         GeometryInfo, &  ! Input, M    
                         ChannelInfo , &  ! Input, N
                         RTSolution  , &  ! Output, L x M   
                         Options     , &  ! Optional input, M    
                         RCS_Id      , &  ! Revision control
                         Message_Log ) &  ! Error messaging
                       RESULT( Error_Status )
    ! Arguments
    TYPE(CRTM_Atmosphere_type),        INTENT(IN)     :: Atmosphere(:)     ! M
    TYPE(CRTM_Surface_type),           INTENT(IN)     :: Surface(:)        ! M
    TYPE(CRTM_GeometryInfo_type),      INTENT(IN OUT) :: GeometryInfo(:)   ! M
    TYPE(CRTM_ChannelInfo_type),       INTENT(IN)     :: ChannelInfo(:)    ! N 
    TYPE(CRTM_RTSolution_type),        INTENT(IN OUT) :: RTSolution(:,:)   ! L x M
    TYPE(CRTM_Options_type), OPTIONAL, INTENT(IN)     :: Options(:)        ! M
    CHARACTER(*),            OPTIONAL, INTENT(OUT)    :: RCS_Id
    CHARACTER(*),            OPTIONAL, INTENT(IN)     :: Message_Log
    ! Function result
    INTEGER :: Error_Status
      \end{verbatim}
    \end{ttfamily}
    \centering
    \begin{tabular}{c|c|c|c}
      \textbf{Argument} & \textbf{Description}                    & \textbf{Rank}    & \textbf{Intent} \\
      \hline\hline
      Atmosphere         & Atmospheric state                      & n\_Profiles (M)  & IN      \\
      \hline
      Surface            & Surface state                          & n\_Profiles (M)  & IN      \\
      \hline
      GeometryInfo       & Geometry information (e.g. angles)     & n\_Profiles (M)  & IN      \\
      \hline
      ChannelInfo        & Sensor channel information             & n\_Sensors (N)   & IN      \\
      \hline
      RTSolution         & Radiative transfer solution            & n\_Channels x n\_Profiles                 & IN OUT  \\
      \hline
      Options            & Structure for optional input           & n\_Profiles (M)  & IN      \\
      \hline
      RCS\_Id            & Version control ID for the module      & Scalar           & OUT     \\
      \hline
      Message\_Log       & Log message filename                   & Scalar           & IN      \\
    \end{tabular}
  \end{minipage}
  }
  \caption{CRTM Forward model interface and argument description.}
  \label{fig:fwd_interface}
\end{figure}


\subsection{K-Matrix Model}
%--------------------------

\begin{figure}[htp]
  \centering
  \doublebox{
  \begin{minipage}[b]{6.5in}
    \begin{ttfamily}
      \begin{verbatim}
  FUNCTION CRTM_K_Matrix( Atmosphere  , &  ! FWD Input, M
                          Surface     , &  ! FWD Input, M
                          RTSolution_K, &  ! K   Input, L x M   
                          GeometryInfo, &  ! Input, M
                          ChannelInfo , &  ! Input, N  
                          Atmosphere_K, &  ! K   Output, L x M
                          Surface_K   , &  ! K   Output, L x M
                          RTSolution  , &  ! FWD Output, L x M
                          Options     , &  ! Optional FWD input, M
                          RCS_Id      , &  ! Revision control
                          Message_Log ) &  ! Error messaging
                        RESULT( Error_Status )
    ! Arguments
    TYPE(CRTM_Atmosphere_type)       , INTENT(IN)     :: Atmosphere(:)     ! M
    TYPE(CRTM_Surface_type)          , INTENT(IN)     :: Surface(:)        ! M
    TYPE(CRTM_RTSolution_type)       , INTENT(IN OUT) :: RTSolution_K(:,:) ! L x M
    TYPE(CRTM_GeometryInfo_type)     , INTENT(IN OUT) :: GeometryInfo(:)   ! M
    TYPE(CRTM_ChannelInfo_type)      , INTENT(IN)     :: ChannelInfo(:)    ! N
    TYPE(CRTM_Atmosphere_type)       , INTENT(IN OUT) :: Atmosphere_K(:,:) ! L x M
    TYPE(CRTM_Surface_type)          , INTENT(IN OUT) :: Surface_K(:,:)    ! L x M
    TYPE(CRTM_RTSolution_type)       , INTENT(IN OUT) :: RTSolution(:,:)   ! L x M
    TYPE(CRTM_Options_type), OPTIONAL, INTENT(IN)     :: Options(:)        ! M
    CHARACTER(*),            OPTIONAL, INTENT(OUT)    :: RCS_Id
    CHARACTER(*),            OPTIONAL, INTENT(IN)     :: Message_Log
    ! Function result
    INTEGER :: Error_Status
      \end{verbatim}
    \end{ttfamily}
    \centering
    \begin{tabular}{c|c|c|c}
      \textbf{Argument} & \textbf{Description}                    & \textbf{Rank}    & \textbf{Intent} \\
      \hline\hline
      Atmosphere         & Atmospheric state                      & n\_Profiles (M)  & IN      \\
      \hline
      Surface            & Surface state                          & n\_Profiles (M)  & IN      \\
      \hline
      RTSolution\_K      & Adjoint radiative transfer solution    & n\_Channels (L) x n\_Profiles (M) & IN OUT  \\
      \hline
      GeometryInfo       & Geometry information (e.g. angles)     & n\_Profiles (M)  & IN      \\
      \hline
      ChannelInfo        & Sensor channel information             & n\_Sensors (N)   & IN      \\
      \hline
      Atmosphere\_K      & Atmospheric state Jacobians            & n\_Channels (L) x n\_Profiles (M) & IN OUT  \\
      \hline
      Surface\_K         & Surface state Jacobians                & n\_Channels (L) x n\_Profiles (M) & IN OUT  \\
      \hline
      RTSolution         & Radiative transfer solution            & n\_Channels (L) x n\_Profiles (M) & IN OUT      \\
      \hline
      Options            & Structure for optional input           & n\_Profiles (M)  & IN      \\
      \hline
      RCS\_Id            & Version control ID for the module      & Scalar           & OUT     \\
      \hline
      Message\_Log       & Log message filename                   & Scalar           & IN      \\
    \end{tabular}
  \end{minipage}
  }
  \caption{CRTM K-Matrix model interface and argument description.}
  \label{fig:k_interface}
\end{figure}


\subsection{CRTM Destruction}
%----------------------------

\begin{figure}[htp]
  \centering
  \doublebox{
  \begin{minipage}[b]{6.5in}
    \begin{ttfamily}
      \begin{verbatim}
  FUNCTION CRTM_Destroy( ChannelInfo , &  ! Output
                         Process_ID  , &  ! Optional input
                         RCS_Id      , &  ! Revision control
                         Message_Log ) &  ! Error messaging
                       RESULT ( Error_Status )
    ! Arguments
    TYPE(CRTM_ChannelInfo_type), INTENT(IN OUT) :: ChannelInfo(:)
    INTEGER     ,      OPTIONAL, INTENT(IN)     :: Process_ID
    CHARACTER(*),      OPTIONAL, INTENT(OUT)    :: RCS_Id
    CHARACTER(*),      OPTIONAL, INTENT(IN)     :: Message_Log
    ! Function result
    INTEGER :: Error_Status
      \end{verbatim}
    \end{ttfamily}
    \centering
    \begin{tabular}{c|c|c|c}
      \textbf{Argument} & \textbf{Description}                     & \textbf{Rank}   & \textbf{Intent} \\
      \hline\hline
      ChannelInfo        & Sensor and channel info structure       & n\_Sensors (N)  & IN OUT      \\
      \hline
      Process\_Id        & MPI process Id                          & Scalar          & IN      \\
      \hline
      RCS\_Id            & Version control ID for the module      & Scalar           & OUT     \\
      \hline
      Message\_Log       & Log message filename                   & Scalar           & IN      \\
    \end{tabular}
  \end{minipage}
  }
  \caption{CRTM Destruction interface and argument description.}
  \label{fig:destroy_interface}
\end{figure}

