\chapter{How to use the CRTM library}
%====================================
\section{Step by Step Guide}
%===========================
This section will hopefully get you started using the CRTM library as quickly as possible. Refer to the following sections for more information about the structures and interfaces.

The examples shown here assume you are processing one sensor at a time. The CRTM can handle multiple sensors at once, but specifying the input information in a simple way is difficult; e.g. the \hyperref[sec:geometryinfo_structure]{\GeometryInfo} structure that is used to specify the sensor viewing geometry -- even sensors on the same platform typically have different numbers of fields-of-view (FOVs) per scan. For multiple sensor processing, we'll assume they will be separately processed in parallel.

Because there are many variations in what information is known ahead of time (and by ``ahead of time'' we mean at compile-time of your code), let's approach this via examples for a fixed number of atmospheric profiles, and a known sensors. It is left as an exercise to the reader to tailor calls to the CRTM in their application code according to their particular needs.

With regards to sensor identification, the CRTM uses a character string -- refered to as the \f{Sensor\_Id} -- to distinguish sensors and platforms. The lists of currently supported sensors, along with their associated \f{Sensor\_Id}'s, are shown in appendix \ref{sec:sensor_id}.


\newcounter{step}
\newcounter{example}[subsection]

\subsection{Step 1: Access the CRTM module}
%------------------------------------------
\stepcounter{step}
\label{sec:access_step}
All of the CRTM user procedures, parameters, and derived data type definitions are accessible via the container module \f{CRTM\_Module}. Thus, one needs to put the following statement in any calling program, module or procedure,

\qquad\f{USE CRTM\_Module}

Once you become familiar with the components of the CRTM you require, you can also specify an \f{ONLY} clause with the \f{USE} statement,

\qquad\f{USE CRTM\_Module}[\f{, ONLY:}\textit{only-list}]

where \textit{only-list} is a list of the symbols you want to ``import'' from \f{CRTM\_Module}. This latter form is the preferred style for self-documenting your code; e.g. when you give the code to someone else, they will be able to identify from which module various symbols in your code originate.


\subsection{Step 2: Declare the CRTM structures}
%-----------------------------------------------
\refstepcounter{step}
\label{sec:declare_step}
To compute satellite radiances you need to declare structures for the following information,\vspace{-2ex}
\begin{enumerate}
  \item Atmospheric profile data such as pressure, temperature, absorber amounts, clouds, aerosols, etc. Handled using the \hyperref[sec:atmosphere_structure]{\Atmosphere} structure.
  \item Surface data such as type of surface, temperature, surface type specific parameters etc. Handled using the \hyperref[sec:surface_structure]{\Surface} structure.
  \item Geometry information such as sensor scan angle, zenith angle, etc. Handled using the \hyperref[sec:geometryinfo_structure]{\GeometryInfo} structure.
  \item Instrument information, particularly which instrument(s), or sensor(s)\footnote{The terms ``instrument'' and ``sensor'' are used interchangeably in this document.}, you want to simulate. Handled using the \hyperref[sec:channelinfo_structure]{\ChannelInfo} structure.
  \item Results of the radiative transfer calculation. Handled using the \hyperref[sec:rtsolution_structure]{\RTSolution} structure.
  \item Optional inputs. Handled using the \hyperref[sec:options_structure]{\Options} structure.
\end{enumerate}

Let's assume you want to process, say, 50 profiles for the NOAA-18 AMSU-A sensor which has 15 channels. The forward model declarations would look something like,
\begin{alltt}
  ! Processing parameters
  INTEGER     , PARAMETER :: N_SENSORS  =  1
  INTEGER     , PARAMETER :: N_CHANNELS = 15
  INTEGER     , PARAMETER :: N_PROFILES = 50
  CHARACTER(*), PARAMETER :: SID(N_SENSORS) = (/'amsua_n18'/)
  TYPE(\hyperref[fig:CRTM_ChannelInfo_type_structure]{CRTM_ChannelInfo_type})  :: chInfo(N_SENSORS)  
  TYPE(\hyperref[fig:CRTM_GeometryInfo_type_structure]{CRTM_GeometryInfo_type}) :: gInfo(N_PROFILES)  
  ! Forward declarations
  TYPE(\hyperref[fig:CRTM_Atmosphere_type_structure]{CRTM_Atmosphere_type})   :: atm(N_PROFILES)
  TYPE(\hyperref[fig:CRTM_Surface_type_structure]{CRTM_Surface_type})      :: sfc(N_PROFILES)
  TYPE(\hyperref[fig:CRTM_RTSolution_type_structure]{CRTM_RTSolution_type})   :: rts(N_CHANNELS,N_PROFILES)\end{alltt}
If you are also interested in calling the K-matrix model, you will also need the following declarations,
\begin{alltt}
  ! K-Matrix declarations
  TYPE(\hyperref[fig:CRTM_Atmosphere_type_structure]{CRTM_Atmosphere_type})   :: atm_K(N_CHANNELS,N_PROFILES)
  TYPE(\hyperref[fig:CRTM_Surface_type_structure]{CRTM_Surface_type})      :: sfc_K(N_CHANNELS,N_PROFILES)
  TYPE(\hyperref[fig:CRTM_RTSolution_type_structure]{CRTM_RTSolution_type})   :: rts_K(N_CHANNELS,N_PROFILES)\end{alltt}


\subsection{Step 3: Initialise the CRTM}
%---------------------------------------
\stepcounter{step}
\label{sec:init_step}
The CRTM is initialised by calling the \hyperref[sec:CRTM_Init_interface]{\f{CRTM\_Init()}} function. This loads all the various coefficient data used by CRTM components into memory for later use. We'll assume that all the required datafiles reside in the subdirectory \f{./coeff\_data} and follow on from the example of Step \ref{sec:declare_step}. The CRTM initialisation is profile independent, so we're only dealing with sensor information here. The CRTM initialisation function call looks like,
\begin{alltt}
  INTEGER :: errStatus
  ....
  errStatus = \hyperref[sec:CRTM_Init_interface]{CRTM_Init}( chInfo, Sensor_Id=SID, File_Path='./coeff_data' )
  IF ( errStatus /= SUCCESS ) THEN 
    \textrm{\textit{handle error...}}
  END IF\end{alltt}

Here we see for the first time how the CRTM functions let you know if they were successful. As you can see the \hyperref[sec:CRTM_Init_interface]{\f{CRTM\_Init()}} function result is an error status that is checked against a parameterised integer error code, \f{SUCCESS}. The function result should \emph{not} be tested against the actual value of the error code, just its parameterised name. Other available error code parameters are \f{FAILURE}, \f{WARNING}, and \f{INFORMATION} -- although the latter is never used as a function result.


\subsection{Step 4: Allocate the CRTM structures}
%------------------------------------------------
\stepcounter{step}
Now we need to allocate the \emph{internal} components of the various CRTM structures where necessary to hold the input or output data. In this case, functions are used to perform these ``internal'' allocations. The function naming convention is \f{CRTM\_Allocate\_}\textit{name} where, for typical usage, the CRTM structures that need to be allocated are the \hyperref[sec:atmosphere_structure]{\Atmosphere}, \hyperref[sec:rtsolution_structure]{\RTSolution} and, if used, \hyperref[sec:options_structure]{\Options} structures. Potentially, the \hyperref[sec:sensordata_structure]{\SensorData} component of the \hyperref[sec:surface_structure]{\Surface} structure may also need to be allocated to allow for input of sensor observations for some of the NESDIS microwave surface emissivity models.

\subsubsection{Allocation of the Atmosphere structures}
%......................................................
First, we'll allocate the atmosphere structures to the required dimensions. The forward variable is allocated like so:
\begin{alltt}
  ! Allocate the forward atmosphere structure
  errStatus = \hyperref[sec:CRTM_Allocate_Atmosphere_interface]{CRTM_Allocate_Atmosphere}( n_Layers   , &  ! Input
                                        N_ABSORBERS, &  ! Input (always 2)
                                        n_Clouds   , &  ! Input
                                        n_Aerosols , &  ! Input
                                        atm          )  ! Output
  IF ( errStatus /= SUCCESS ) THEN 
    \textrm{\textit{handle error...}}
  END IF\end{alltt}
and the K-matrix variable is allocated by looping over all profiles,
\begin{alltt}
  ! Allocate the K-matrix atmosphere structure
  DO m = 1, N_PROFILES
    errStatus = \hyperref[sec:CRTM_Allocate_Atmosphere_interface]{CRTM_Allocate_Atmosphere}( n_Layers   , &  ! Input
                                          N_ABSORBERS, &  ! Input (always 2)
                                          n_Clouds   , &  ! Input
                                          n_Aerosols , &  ! Input
                                          atm_k(:,m)   )  ! Output
    IF ( errStatus /= SUCCESS ) THEN 
      \textrm{\textit{handle error...}}
    END IF
  END DO\end{alltt}
Note that the number of absorbers is in all capitals. In the CRTM, this style convention indicates a parameter. A parameter is used for the number of absorbers in the current CRTM release because the number of absorbers is fixed at two: water vapour and ozone. Future CRTM releases will allow more flexibility in selecting the number of absorbers, but currently the number must be set to two.


\subsection{Step 5: Fill the CRTM input structures with data}
%------------------------------------------------------------
\stepcounter{step}
This step simply entails filling the input \texttt{atm}, \texttt{sfc} and \texttt{gInfo} structures with the required information. However, there are some issues that need to be mentioned:
\begin{itemize}
  \item In the CRTM, all profile layering is from top-of-atmosphere (TOA) to surface (SFC). So, for an atmospheric profile layered as $k = 1,2,...,K$, layer 1 is the TOA layer and layer $K$ is the SFC layer.
  \item In the \hyperref[sec:atmosphere_structure]{\Atmosphere} structure, the \texttt{Climatology} component is not yet used.
  \item In the \hyperref[sec:atmosphere_structure]{\Atmosphere} structure, \emph{both} the level and layer pressure profiles must be specified.
  \item In the \hyperref[sec:atmosphere_structure]{\Atmosphere} structure, the absorber profile data units \emph{must} be mass mixing ratio for water vapour and ppmv for ozone. The \texttt{Absorber\_Units} component is not yet utilised to allow conversion of different user-supplied concentration units.
  \item In the \hyperref[sec:surface_structure]{\Surface} structure, the sum of the coverage types \emph{must} add up to 1.0.
  \item In the \hyperref[sec:geometryinfo_structure]{\GeometryInfo} structure, the sensor zenith and sensor scan angles should be consistent.
  \item Graphical definitions of the \hyperref[sec:geometryinfo_structure]{\GeometryInfo} structure sensor scan, sensor zenith, sensor azimuth, source zenith, and source azimuth angles are shown in figures \ref{fig:gInfo_sensor_scan_angle}, \ref{fig:gInfo_sensor_zenith_angle}, \ref{fig:gInfo_sensor_azimuth_angle}, \ref{fig:gInfo_source_zenith_angle}, and  \ref{fig:gInfo_source_azimuth_angle} respectively.
\end{itemize}
For the K-matrix structures, you should zero the K-matrix \emph{outputs}, \texttt{atm\_K}, \texttt{sfc\_K},
\begin{alltt}
  ! Zero the K-matrix OUTPUT structures
  CALL \hyperref[sec:CRTM_Zero_Atmosphere_interface]{CRTM_Zero_Atmosphere}( atm_K )
  CALL \hyperref[sec:CRTM_Zero_Surface_interface]{CRTM_Zero_Surface}( sfc_K )\end{alltt}
and initialise the K-matrix \emph{input}, \texttt{rts\_K}, to provide you with the derivatives you want. For example, if you want the \texttt{atm\_K}, \texttt{sfc\_K} outputs to contain brightness temperature derivatives, you should initialise \texttt{rts\_K} like so,
\begin{alltt}
  ! Initialise the K-Matrix INPUT to provide dTb/dx derivatives
  rts_K%Radiance = ZERO
  rts_K%Brightness_Temperature = ONE\end{alltt}
Alternatively, if you want radiance derivatives returned in \texttt{atm\_K} and \texttt{sfc\_K}, the \texttt{rts\_K} structure should be initialised like so,  
\begin{alltt}
  ! Initialise the K-Matrix INPUT to provide dR/dx derivatives
  rts_K%Radiance = ONE
  rts_K%Brightness_Temperature = ZERO\end{alltt}

For computational efficiency, there is minimal input data checking done so the GIGO\footnote{Garbage In, Garbage Out} principle applies.


\subsection{Step 6: Call the required CRTM function}
%---------------------------------------------------
\stepcounter{step}
At this point, much of the prepatory heavy lifting has been done. The CRTM function calls themselves are quite simple. For the forward model we do,
\begin{alltt}
  Error_Status = \hyperref[sec:CRTM_Forward_interface]{CRTM_Forward}( atm   , & ! Input
                               sfc   , & ! Input
                               gInfo , & ! Input
                               chInfo, & ! Input
                               rts     ) ! Output
  IF ( Error_Status /= SUCCESS ) THEN 
    \textrm{\textit{handle error...}}
  END IF\end{alltt}
and for the K-matrix model, the calling syntax is,
\begin{alltt}
  Error_Status = \hyperref[sec:CRTM_K_Matrix_interface]{CRTM_K_Matrix}( atm   , & ! Forward  input  
                                sfc   , & ! Forward  input    
                                rts_K , & ! K-matrix input    
                                gInfo , & ! Input  
                                chInfo, & ! Input  
                                atm_K , & ! K-matrix output
                                sfc_K , & ! K-matrix output  
                                rts     ) ! Forward  output
  IF ( Error_Status /= SUCCESS ) THEN 
    \textrm{\textit{handle error...}}
  END IF\end{alltt}
Note that the K-matrix model also returns the forward model radiances. The \hyperref[sec:CRTM_Tangent_Linear_interface]{tangent-linear} and \hyperref[sec:CRTM_Adjoint_interface]{adjoint} models have similar call structures and will not be shown here.


\subsection{Step 7: Destroy the CRTM and cleanup}
%------------------------------------------------
\stepcounter{step}
The last step is to cleanup. This involves calling the CRTM destruction function
\begin{alltt}
  Error_Status = \hyperref[sec:CRTM_Destroy_interface]{CRTM_Destroy}( chInfo )
  IF ( Error_Status /= SUCCESS ) THEN 
    \textrm{\textit{handle error...}}
  END IF\end{alltt}
to deallocate all the shared coefficient data, as well as calling the individual structure destroy functions to deallocate as required. For the example here, that entails deallocating the forward and K-matrix \hyperref[sec:atmosphere_structure]{\Atmosphere}  structures, \texttt{atm} and \texttt{atm\_K},
\begin{alltt}
  Error_Status = \hyperref[sec:CRTM_Destroy_Atmosphere_interface]{CRTM_Destroy_Atmosphere}(atm_K)
  IF ( Error_Status /= SUCCESS ) THEN 
    \textrm{\textit{handle error...}}
  END IF
  
  Error_Status = \hyperref[sec:CRTM_Destroy_Atmosphere_interface]{CRTM_Destroy_Atmosphere}(atm)
  IF ( Error_Status /= SUCCESS ) THEN 
    \textrm{\textit{handle error...}}
  END IF\end{alltt}


\section{Interface Descriptions}
%===============================

\subsection{\texttt{CRTM\_Init} interface}
  \label{sec:CRTM_Init_interface}
  \begin{alltt}
 
  NAME:
        CRTM_Init
 
  PURPOSE:
        Function to initialise the CRTM.
 
  CALLING SEQUENCE:
        Error_Status = CRTM_Init( Sensor_ID  , &
                                  ChannelInfo, &
                                  CloudCoeff_File    = CloudCoeff_File   , &
                                  AerosolCoeff_File  = AerosolCoeff_File , &
                                  Load_CloudCoeff    = Load_CloudCoeff   , &
                                  Load_AerosolCoeff  = Load_AerosolCoeff , &
                                  IRwaterCoeff_File  = IRwaterCoeff_File , &
                                  IRlandCoeff_File   = IRlandCoeff_File  , &
                                  IRsnowCoeff_File   = IRsnowCoeff_File  , &
                                  IRiceCoeff_File    = IRiceCoeff_File   , &
                                  VISwaterCoeff_File = VISwaterCoeff_File, &
                                  VISlandCoeff_File  = VISlandCoeff_File , &
                                  VISsnowCoeff_File  = VISsnowCoeff_File , &
                                  VISiceCoeff_File   = VISiceCoeff_File  , &
                                  MWwaterCoeff_File  = MWwaterCoeff_File , &
                                  File_Path          = File_Path         , &
                                  Quiet              = Quiet             , &
                                  Process_ID         = Process_ID        , &
                                  Output_Process_ID  = Output_Process_ID   )
 
  INPUTS:
        Sensor_ID:          List of the sensor IDs (e.g. hirs3_n17, amsua_n18,
                            ssmis_f16, etc) with which the CRTM is to be
                            initialised. These sensor ids are used to construct
                            the sensor specific SpcCoeff and TauCoeff filenames
                            containing the necessary coefficient data, i.e.
                              <Sensor_ID>.SpcCoeff.bin
                            and
                              <Sensor_ID>.TauCoeff.bin
                            for each sensor Id in the list.
                            UNITS:      N/A
                            TYPE:       CHARACTER(*)
                            DIMENSION:  Rank-1 (n_Sensors)
                            ATTRIBUTES: INTENT(IN), OPTIONAL
 
  OUTPUTS:
        ChannelInfo:        ChannelInfo structure array populated based on
                            the contents of the coefficient files and the
                            user inputs.
                            UNITS:      N/A
                            TYPE:       CRTM_ChannelInfo_type
                            DIMENSION:  Same as input Sensor_Id argument
                            ATTRIBUTES: INTENT(OUT)
 
  OPTIONAL INPUTS:
        CloudCoeff_File:    Name of the data file containing the cloud optical
                            properties data for scattering calculations.
                            Available datafiles:
                            - CloudCoeff.bin  [DEFAULT]
                            UNITS:      N/A
                            TYPE:       CHARACTER(*)
                            DIMENSION:  Scalar
                            ATTRIBUTES: INTENT(IN), OPTIONAL
 
        AerosolCoeff_File:  Name of the data file containing the aerosol optical
                            properties data for scattering calculations.
                            Available datafiles:
                            - AerosolCoeff.bin  [DEFAULT]
                            UNITS:      N/A
                            TYPE:       CHARACTER(*)
                            DIMENSION:  Scalar
                            ATTRIBUTES: INTENT(IN), OPTIONAL
 
        Load_CloudCoeff:    Set this logical argument for not loading the CloudCoeff data
                            to save memory space under the clear conditions
                            If == .FALSE., the CloudCoeff data will not be loaded;
                               == .TRUE.,  the CloudCoeff data will be loaded.
                            If not specified, default is .TRUE. (will be loaded)
                            UNITS:      N/A
                            TYPE:       LOGICAL
                            DIMENSION:  Scalar
                            ATTRIBUTES: INTENT(IN), OPTIONAL
 
        Load_AerosolCoeff:  Set this logical argument for not loading the AerosolCoeff data
                            to save memory space under the clear conditions
                            If == .FALSE., the AerosolCoeff data will not be loaded;
                               == .TRUE.,  the AerosolCoeff data will be loaded.
                            If not specified, default is .TRUE. (will be loaded)
                            UNITS:      N/A
                            TYPE:       LOGICAL
                            DIMENSION:  Scalar
                            ATTRIBUTES: INTENT(IN), OPTIONAL
 
        MWwaterCoeff_File:  Name of the data file containing the coefficient
                            data for the microwave water emissivity model.
                            Available datafiles:
                            - FASTEM5.MWwater.EmisCoeff.bin  [DEFAULT]
                            - FASTEM4.MWwater.EmisCoeff.bin
                            UNITS:      N/A
                            TYPE:       CHARACTER(*)
                            DIMENSION:  Scalar
                            ATTRIBUTES: INTENT(IN), OPTIONAL
 
        IRwaterCoeff_File:  Name of the data file containing the coefficient
                            data for the infrared water emissivity model.
                            Available datafiles:
                            - Nalli.IRwater.EmisCoeff.bin  [DEFAULT]
                            - WuSmith.IRwater.EmisCoeff.bin
                            If not specified the Nalli datafile is read.
                            UNITS:      N/A
                            TYPE:       CHARACTER(*)
                            DIMENSION:  Scalar
                            ATTRIBUTES: INTENT(IN), OPTIONAL
 
        IRlandCoeff_File:   Name of the data file containing the coefficient
                            data for the infrared land emissivity model.
                            Available datafiles:
                            - NPOESS.IRland.EmisCoeff.bin  [DEFAULT]
                            - IGBP.IRland.EmisCoeff.bin
                            - USGS.IRland.EmisCoeff.bin
                            UNITS:      N/A
                            TYPE:       CHARACTER(*)
                            DIMENSION:  Scalar
                            ATTRIBUTES: INTENT(IN), OPTIONAL
 
        IRsnowCoeff_File:   Name of the data file containing the coefficient
                            data for the infrared snow emissivity model.
                            Available datafiles:
                            - NPOESS.IRsnow.EmisCoeff.bin  [DEFAULT]
                            - IGBP.IRsnow.EmisCoeff.bin
                            - USGS.IRsnow.EmisCoeff.bin
                            UNITS:      N/A
                            TYPE:       CHARACTER(*)
                            DIMENSION:  Scalar
                            ATTRIBUTES: INTENT(IN), OPTIONAL
 
        IRiceCoeff_File:    Name of the data file containing the coefficient
                            data for the infrared ice emissivity model.
                            Available datafiles:
                            - NPOESS.IRice.EmisCoeff.bin  [DEFAULT]
                            - IGBP.IRice.EmisCoeff.bin
                            - USGS.IRice.EmisCoeff.bin
                            UNITS:      N/A
                            TYPE:       CHARACTER(*)
                            DIMENSION:  Scalar
                            ATTRIBUTES: INTENT(IN), OPTIONAL
 
        VISwaterCoeff_File: Name of the data file containing the coefficient
                            data for the visible water emissivity model.
                            Available datafiles:
                            - NPOESS.VISwater.EmisCoeff.bin  [DEFAULT]
                            - IGBP.VISwater.EmisCoeff.bin
                            - USGS.VISwater.EmisCoeff.bin
                            UNITS:      N/A
                            TYPE:       CHARACTER(*)
                            DIMENSION:  Scalar
                            ATTRIBUTES: INTENT(IN), OPTIONAL
 
        VISlandCoeff_File:  Name of the data file containing the coefficient
                            data for the visible land emissivity model.
                            Available datafiles:
                            - NPOESS.VISland.EmisCoeff.bin  [DEFAULT]
                            - IGBP.VISland.EmisCoeff.bin
                            - USGS.VISland.EmisCoeff.bin
                            UNITS:      N/A
                            TYPE:       CHARACTER(*)
                            DIMENSION:  Scalar
                            ATTRIBUTES: INTENT(IN), OPTIONAL
 
        VISsnowCoeff_File:  Name of the data file containing the coefficient
                            data for the visible snow emissivity model.
                            Available datafiles:
                            - NPOESS.VISsnow.EmisCoeff.bin  [DEFAULT]
                            - IGBP.VISsnow.EmisCoeff.bin
                            - USGS.VISsnow.EmisCoeff.bin
                            UNITS:      N/A
                            TYPE:       CHARACTER(*)
                            DIMENSION:  Scalar
                            ATTRIBUTES: INTENT(IN), OPTIONAL
 
        VISiceCoeff_File:   Name of the data file containing the coefficient
                            data for the visible ice emissivity model.
                            Available datafiles:
                            - NPOESS.VISice.EmisCoeff.bin  [DEFAULT]
                            - IGBP.VISice.EmisCoeff.bin
                            - USGS.VISice.EmisCoeff.bin
                            UNITS:      N/A
                            TYPE:       CHARACTER(*)
                            DIMENSION:  Scalar
                            ATTRIBUTES: INTENT(IN), OPTIONAL
 
        File_Path:          Character string specifying a file path for the
                            input data files. If not specified, the current
                            directory is the default.
                            UNITS:      N/A
                            TYPE:       CHARACTER(*)
                            DIMENSION:  Scalar
                            ATTRIBUTES: INTENT(IN), OPTIONAL
 
        Quiet:              Set this logical argument to suppress INFORMATION
                            messages being printed to stdout
                            If == .FALSE., INFORMATION messages are OUTPUT [DEFAULT].
                               == .TRUE.,  INFORMATION messages are SUPPRESSED.
                            If not specified, default is .FALSE.
                            UNITS:      N/A
                            TYPE:       LOGICAL
                            DIMENSION:  Scalar
                            ATTRIBUTES: INTENT(IN), OPTIONAL
 
        Process_ID:         Set this argument to the MPI process ID that this
                            function call is running under. This value is used
                            solely for controlling INFORMATION message output.
                            If MPI is not being used, ignore this argument.
                            This argument is ignored if the Quiet argument is set.
                            UNITS:      N/A
                            TYPE:       INTEGER
                            DIMENSION:  Scalar
                            ATTRIBUTES: INTENT(IN), OPTIONAL
 
        Output_Process_ID:  Set this argument to the MPI process ID in which
                            all INFORMATION messages are to be output. If
                            the passed Process_ID value agrees with this value
                            the INFORMATION messages are output.
                            This argument is ignored if the Quiet argument
                            is set.
                            UNITS:      N/A
                            TYPE:       INTEGER
                            DIMENSION:  Scalar
                            ATTRIBUTES: INTENT(IN), OPTIONAL
 
  FUNCTION RESULT:
        Error_Status:       The return value is an integer defining the error
                            status. The error codes are defined in the
                            Message_Handler module.
                            If == SUCCESS the CRTM initialisation was successful
                               == FAILURE an unrecoverable error occurred.
                            UNITS:      N/A
                            TYPE:       INTEGER
                            DIMENSION:  Scalar
 
  SIDE EFFECTS:
        All public data arrays accessed by this module and its dependencies
        are overwritten.
 
  \end{alltt}

\subsection{\texttt{CRTM\_Forward} interface}
  \label{sec:CRTM_Forward_interface}
  \begin{alltt}
 
  NAME:
        CRTM_Forward
 
  PURPOSE:
        Function that calculates top-of-atmosphere (TOA) radiances
        and brightness temperatures for an input atmospheric profile or
        profile set and user specified satellites/channels.
 
  CALLING SEQUENCE:
        Error_Status = CRTM_Forward( Atmosphere       , &
                                     Surface          , &
                                     Geometry         , &
                                     ChannelInfo      , &
                                     RTSolution       , &
                                     Options = Options  )
 
  INPUTS:
        Atmosphere:     Structure containing the Atmosphere data.
                        UNITS:      N/A
                        TYPE:       CRTM_Atmosphere_type
                        DIMENSION:  Rank-1 (n_Profiles)
                        ATTRIBUTES: INTENT(IN)
 
        Surface:        Structure containing the Surface data.
                        UNITS:      N/A
                        TYPE:       CRTM_Surface_type
                        DIMENSION:  Same as input Atmosphere structure
                        ATTRIBUTES: INTENT(IN)
 
        Geometry:       Structure containing the view geometry
                        information.
                        UNITS:      N/A
                        TYPE:       CRTM_Geometry_type
                        DIMENSION:  Same as input Atmosphere structure
                        ATTRIBUTES: INTENT(IN)
 
        ChannelInfo:    Structure returned from the CRTM_Init() function
                        that contains the satellite/sensor channel index
                        information.
                        UNITS:      N/A
                        TYPE:       CRTM_ChannelInfo_type
                        DIMENSION:  Rank-1 (n_Sensors)
                        ATTRIBUTES: INTENT(IN)
 
  OUTPUTS:
        RTSolution:     Structure containing the soluition to the RT equation
                        for the given inputs.
                        UNITS:      N/A
                        TYPE:       CRTM_RTSolution_type
                        DIMENSION:  Rank-2 (n_Channels x n_Profiles)
                        ATTRIBUTES: INTENT(IN OUT)
 
  OPTIONAL INPUTS:
        Options:        Options structure containing the optional arguments
                        for the CRTM.
                        UNITS:      N/A
                        TYPE:       CRTM_Options_type
                        DIMENSION:  Same as input Atmosphere structure
                        ATTRIBUTES: INTENT(IN), OPTIONAL
 
  FUNCTION RESULT:
        Error_Status:   The return value is an integer defining the error status.
                        The error codes are defined in the Message_Handler module.
                        If == SUCCESS the computation was sucessful
                           == FAILURE an unrecoverable error occurred
                        UNITS:      N/A
                        TYPE:       INTEGER
                        DIMENSION:  Scalar
 
  COMMENTS:
        - The Options optional input structure argument contains
          spectral information (e.g. emissivity) that must have the same
          spectral dimensionality (the "L" dimension) as the output
          RTSolution structure.
 
  \end{alltt}

\subsection{\texttt{CRTM\_Tangent\_Linear} interface}
  \label{sec:CRTM_Tangent_Linear_interface}
  \begin{alltt}
 
  NAME:
        CRTM_Tangent_Linear
 
  PURPOSE:
        Function that calculates tangent-linear top-of-atmosphere (TOA)
        radiances and brightness temperatures for an input atmospheric
        profile or profile set and user specified satellites/channels.
 
  CALLING SEQUENCE:
        Error_Status = CRTM_Tangent_Linear( Atmosphere             , &
                                            Surface                , &
                                            Atmosphere_TL          , &
                                            Surface_TL             , &
                                            GeometryInfo           , &
                                            ChannelInfo            , &
                                            RTSolution             , &
                                            RTSolution_TL          , &
                                            Options    =Options    , &
                                            RCS_Id     =RCS_Id     , &
                                            Message_Log=Message_Log  )
 
  INPUT ARGUMENTS:
        Atmosphere:     Structure containing the Atmosphere data.
                        UNITS:      N/A
                        TYPE:       CRTM_Atmosphere_type
                        DIMENSION:  Rank-1 (n_Profiles)
                        ATTRIBUTES: INTENT(IN)
 
        Surface:        Structure containing the Surface data.
                        UNITS:      N/A
                        TYPE:       CRTM_Surface_type
                        DIMENSION:  Same as input Atmosphere structure
                        ATTRIBUTES: INTENT(IN)
 
        Atmosphere_TL:  Structure containing the tangent-linear Atmosphere data.
                        UNITS:      N/A
                        TYPE:       CRTM_Atmosphere_type
                        DIMENSION:  Same as input Atmosphere structure
                        ATTRIBUTES: INTENT(IN)
 
        Surface_TL:     Structure containing the tangent-linear Surface data.
                        UNITS:      N/A
                        TYPE:       CRTM_Surface_type
                        DIMENSION:  Same as input Atmosphere structure
                        ATTRIBUTES: INTENT(IN)
 
        GeometryInfo:   Structure containing the view geometry
                        information.
                        UNITS:      N/A
                        TYPE:       CRTM_GeometryInfo_type
                        DIMENSION:  Same as input Atmosphere structure
                        ATTRIBUTES: INTENT(IN)
 
        ChannelInfo:    Structure returned from the CRTM_Init() function
                        that contains the satellite/sensor channel index
                        information.
                        UNITS:      N/A
                        TYPE:       CRTM_ChannelInfo_type
                        DIMENSION:  Rank-1 (n_Sensors)
                        ATTRIBUTES: INTENT(IN)
 
  OUTPUT ARGUMENTS:
        RTSolution:     Structure containing the solution to the RT equation
                        for the given inputs.
                        UNITS:      N/A
                        TYPE:       CRTM_RTSolution_type
                        DIMENSION:  Rank-2 (n_Channels x n_Profiles)
                        ATTRIBUTES: INTENT(IN OUT)
 
        RTSolution_TL:  Structure containing the solution to the tangent-
                        linear RT equation for the given inputs.
                        UNITS:      N/A
                        TYPE:       CRTM_RTSolution_type
                        DIMENSION:  Rank-2 (n_Channels x n_Profiles)
                        ATTRIBUTES: INTENT(IN OUT)
 
  OPTIONAL INPUT ARGUMENTS:
        Options:        Options structure containing the optional forward model
                        arguments for the CRTM.
                        UNITS:      N/A
                        TYPE:       CRTM_Options_type
                        DIMENSION:  Same as input Atmosphere structure
                        ATTRIBUTES: INTENT(IN), OPTIONAL
 
        Message_Log:    Character string specifying a filename in which any
                        messages will be logged. If not specified, or if an
                        error occurs opening the log file, the default action
                        is to output messages to the screen.
                        UNITS:      N/A
                        TYPE:       CHARACTER(*)
                        DIMENSION:  Scalar
                        ATTRIBUTES: INTENT(IN), OPTIONAL
 
  OPTIONAL OUTPUT ARGUMENTS:
        RCS_Id:         Character string containing the Revision Control
                        System Id field for the module.
                        UNITS:      N/A
                        TYPE:       CHARACTER(*)
                        DIMENSION:  Scalar
                        ATTRIBUTES: INTENT(OUT), OPTIONAL
 
  FUNCTION RESULT:
        Error_Status:   The return value is an integer defining the error status.
                        The error codes are defined in the Message_Handler module.
                        If == SUCCESS the computation was sucessful
                           == FAILURE an unrecoverable error occurred
                        UNITS:      N/A
                        TYPE:       INTEGER
                        DIMENSION:  Scalar
 
  COMMENTS:
        - The Options optional input structure arguments contain
          spectral information (e.g. emissivity) that must have the same
          spectral dimensionality (the "L" dimension) as the output
          RTSolution structures.
 
        - The INTENT on the output RTSolution arguments are IN OUT rather
          than just OUT. This is necessary because the arguments may be defined
          upon input. To prevent memory leaks, the IN OUT INTENT is a must.
 
  \end{alltt}

\subsection{\texttt{CRTM\_Adjoint} interface}
  \label{sec:CRTM_Adjoint_interface}
  \begin{alltt}
 
  NAME:
        CRTM_Adjoint
 
  PURPOSE:
        Function that calculates the adjoint of top-of-atmosphere (TOA)
        radiances and brightness temperatures for an input atmospheric
        profile or profile set and user specified satellites/channels.
 
  CALLING SEQUENCE:
        Error_Status = CRTM_Adjoint( Atmosphere             , &
                                     Surface                , &
                                     RTSolution_AD          , &
                                     GeometryInfo           , &
                                     ChannelInfo            , &
                                     Atmosphere_AD          , &
                                     Surface_AD             , &
                                     RTSolution             , &
                                     Options    =Options    , &
                                     RCS_Id     =RCS_Id     , &
                                     Message_Log=Message_Log  )
 
  INPUT ARGUMENTS:
        Atmosphere:     Structure containing the Atmosphere data.
                        UNITS:      N/A
                        TYPE:       CRTM_Atmosphere_type
                        DIMENSION:  Rank-1 (n_Profiles)
                        ATTRIBUTES: INTENT(IN)
 
        Surface:        Structure containing the Surface data.
                        UNITS:      N/A
                        TYPE:       CRTM_Surface_type
                        DIMENSION:  Same as input Atmosphere structure
                        ATTRIBUTES: INTENT(IN)
 
        RTSolution_AD:  Structure containing the RT solution adjoint inputs.
                        **NOTE: On EXIT from this function, the contents of
                                this structure may be modified (e.g. set to
                                zero.)
                        UNITS:      N/A
                        TYPE:       CRTM_RTSolution_type
                        DIMENSION:  Rank-2 (n_Channels x n_Profiles)
                        ATTRIBUTES: INTENT(IN OUT)
 
        GeometryInfo:   Structure containing the view geometry
                        information.
                        UNITS:      N/A
                        TYPE:       CRTM_GeometryInfo_type
                        DIMENSION:  Same as input Atmosphere argument
                        ATTRIBUTES: INTENT(IN)
 
        ChannelInfo:    Structure returned from the CRTM_Init() function
                        that contains the satellite/sensor channel index
                        information.
                        UNITS:      N/A
                        TYPE:       CRTM_ChannelInfo_type
                        DIMENSION:  Rank-1 (n_Sensors)
                        ATTRIBUTES: INTENT(IN)
 
  OPTIONAL INPUT ARGUMENTS:
        Options:        Options structure containing the optional forward model
                        arguments for the CRTM.
                        UNITS:      N/A
                        TYPE:       CRTM_Options_type
                        DIMENSION:  Same as input Atmosphere structure
                        ATTRIBUTES: INTENT(IN), OPTIONAL
 
        Message_Log:    Character string specifying a filename in which any
                        messages will be logged. If not specified, or if an
                        error occurs opening the log file, the default action
                        is to output messages to the screen.
                        UNITS:      N/A
                        TYPE:       CHARACTER(*)
                        DIMENSION:  Scalar
                        ATTRIBUTES: INTENT(IN), OPTIONAL
 
  OUTPUT ARGUMENTS:
        Atmosphere_AD:  Structure containing the adjoint Atmosphere data.
                        **NOTE: On ENTRY to this function, the contents of
                                this structure should be defined (e.g.
                                initialized to some value based on the
                                position of this function in the call chain.)
                        UNITS:      N/A
                        TYPE:       CRTM_Atmosphere_type
                        DIMENSION:  Same as input Atmosphere argument
                        ATTRIBUTES: INTENT(IN OUT)
 
        Surface_AD:     Structure containing the tangent-linear Surface data.
                        **NOTE: On ENTRY to this function, the contents of
                                this structure should be defined (e.g.
                                initialized to some value based on the
                                position of this function in the call chain.)
                        UNITS:      N/A
                        TYPE:       CRTM_Surface_type
                        DIMENSION:  Same as input Atmosphere argument
                        ATTRIBUTES: INTENT(IN OUT)
 
        RTSolution:     Structure containing the solution to the RT equation
                        for the given inputs.
                        UNITS:      N/A
                        TYPE:       CRTM_RTSolution_type
                        DIMENSION:  Same as input RTSolution_AD argument
                        ATTRIBUTES: INTENT(IN OUT)
 
  OPTIONAL OUTPUT ARGUMENTS:
        RCS_Id:         Character string containing the Revision Control
                        System Id field for the module.
                        UNITS:      N/A
                        TYPE:       CHARACTER(*)
                        DIMENSION:  Scalar
                        ATTRIBUTES: INTENT(OUT), OPTIONAL
 
  FUNCTION RESULT:
        Error_Status:   The return value is an integer defining the error status.
                        The error codes are defined in the Message_Handler module.
                        If == SUCCESS the computation was sucessful
                           == FAILURE an unrecoverable error occurred
                        UNITS:      N/A
                        TYPE:       INTEGER
                        DIMENSION:  Scalar
 
  SIDE EFFECTS:
       Note that the input adjoint arguments are modified upon exit, and
       the output adjoint arguments must be defined upon entry. This is
       a consequence of the adjoint formulation where, effectively, the
       chain rule is being used and this function could reside anywhere
       in the chain of derivative terms.
 
  COMMENTS:
        - The Options optional structure arguments contain
          spectral information (e.g. emissivity) that must have the same
          spectral dimensionality (the "L" dimension) as the RTSolution
          structures.
 
        - The INTENT on the output RTSolution, Atmosphere_AD, and
          Surface_AD arguments are IN OUT rather than just OUT. This is
          necessary because the arguments should be defined upon input.
          To prevent memory leaks, the IN OUT INTENT is a must.
 
  \end{alltt}

\subsection{\texttt{CRTM\_K\_Matrix} interface}
  \label{sec:CRTM_K_Matrix_interface}
  \begin{alltt}
 
  NAME:
        CRTM_K_Matrix
 
  PURPOSE:
        Function that calculates the K-matrix of top-of-atmosphere (TOA)
        radiances and brightness temperatures for an input atmospheric
        profile or profile set and user specified satellites/channels.
 
  CALLING SEQUENCE:
        Error_Status = CRTM_K_Matrix( Atmosphere             , &
                                      Surface                , &
                                      RTSolution_K           , &
                                      GeometryInfo           , &
                                      ChannelInfo            , &
                                      Atmosphere_K           , &
                                      Surface_K              , &
                                      RTSolution             , &
                                      Options    =Options    , &
                                      RCS_Id     =RCS_Id     , &
                                      Message_Log=Message_Log  )
 
  INPUT ARGUMENTS:
        Atmosphere:     Structure containing the Atmosphere data.
                        UNITS:      N/A
                        TYPE:       CRTM_Atmosphere_type
                        DIMENSION:  Rank-1 (n_Profiles)
                        ATTRIBUTES: INTENT(IN)
 
        Surface:        Structure containing the Surface data.
                        UNITS:      N/A
                        TYPE:       CRTM_Surface_type
                        DIMENSION:  Same as input Atmosphere argument.
                        ATTRIBUTES: INTENT(IN)
 
        RTSolution_K:   Structure containing the RT solution K-matrix inputs.
                        **NOTE: On EXIT from this function, the contents of
                                this structure may be modified (e.g. set to
                                zero.)
                        UNITS:      N/A
                        TYPE:       CRTM_RTSolution_type
                        DIMENSION:  Rank-2 (n_Channels x n_Profiles)
                        ATTRIBUTES: INTENT(IN OUT)
 
        GeometryInfo:   Structure containing the view geometry
                        information.
                        UNITS:      N/A
                        TYPE:       CRTM_GeometryInfo_type
                        DIMENSION:  Same as input Atmosphere argument
                        ATTRIBUTES: INTENT(IN)
 
        ChannelInfo:    Structure returned from the CRTM_Init() function
                        that contains the satellite/sesnor channel index
                        information.
                        UNITS:      N/A
                        TYPE:       CRTM_ChannelInfo_type
                        DIMENSION:  Rank-1 (n_Sensors)
                        ATTRIBUTES: INTENT(IN)
 
  OPTIONAL INPUT ARGUMENTS:
        Options:        Options structure containing the optional forward model
                        arguments for the CRTM.
                        UNITS:      N/A
                        TYPE:       CRTM_Options_type
                        DIMENSION:  Same as input Atmosphere structure
                        ATTRIBUTES: INTENT(IN), OPTIONAL
 
        Message_Log:    Character string specifying a filename in which any
                        messages will be logged. If not specified, or if an
                        error occurs opening the log file, the default action
                        is to output messages to the screen.
                        UNITS:      N/A
                        TYPE:       CHARACTER(*)
                        DIMENSION:  Scalar
                        ATTRIBUTES: INTENT(IN), OPTIONAL
 
  OUTPUT ARGUMENTS:
        Atmosphere_K:   Structure containing the K-matrix Atmosphere data.
                        **NOTE: On ENTRY to this function, the contents of
                                this structure should be defined (e.g.
                                initialized to some value based on the
                                position of this function in the call chain.)
                        UNITS:      N/A
                        TYPE:       CRTM_Atmosphere_type
                        DIMENSION:  Same as input RTSolution_K argument
                        ATTRIBUTES: INTENT(IN OUT)
 
        Surface_K:      Structure containing the tangent-linear Surface data.
                        **NOTE: On ENTRY to this function, the contents of
                                this structure should be defined (e.g.
                                initialized to some value based on the
                                position of this function in the call chain.)
                        UNITS:      N/A
                        TYPE:       CRTM_Surface_type
                        DIMENSION:  Same as input RTSolution_K argument
                        ATTRIBUTES: INTENT(IN OUT)
 
        RTSolution:     Structure containing the solution to the RT equation
                        for the given inputs.
                        UNITS:      N/A
                        TYPE:       CRTM_RTSolution_type
                        DIMENSION:  Same as input RTSolution_K argument
                        ATTRIBUTES: INTENT(IN OUT)
 
  OPTIONAL OUTPUT ARGUMENTS:
        RCS_Id:         Character string containing the Revision Control
                        System Id field for the module.
                        UNITS:      N/A
                        TYPE:       CHARACTER(*)
                        DIMENSION:  Scalar
                        ATTRIBUTES: INTENT(OUT), OPTIONAL
 
  FUNCTION RESULT:
        Error_Status:   The return value is an integer defining the error status.
                        The error codes are defined in the Message_Handler module.
                        If == SUCCESS the computation was sucessful
                           == FAILURE an unrecoverable error occurred
                        UNITS:      N/A
                        TYPE:       INTEGER
                        DIMENSION:  Scalar
 
  SIDE EFFECTS:
       Note that the input K-matrix arguments are modified upon exit, and
       the output K-matrix arguments must be defined upon entry. This is
       a consequence of the K-matrix formulation where, effectively, the
       chain rule is being used and this funtion could reside anywhere
       in the chain of derivative terms.
 
  COMMENTS:
        - The Options optional structure arguments contain
          spectral information (e.g. emissivity) that must have the same
          spectral dimensionality (the "L" dimension) as the RTSolution
          structures.
 
        - The INTENT on the output RTSolution, Atmosphere_K, and Surface_K,
          arguments are IN OUT rather than just OUT. This is necessary because
          the arguments should be defined upon input. To prevent memory leaks,
          the IN OUT INTENT is a must.
 
  \end{alltt}

\subsection{\texttt{CRTM\_Destroy} interface}
  \label{sec:CRTM_Destroy_interface}
  \begin{alltt}
 
  NAME:
        CRTM_Destroy
 
  PURPOSE:
        Function to deallocate all the shared data arrays allocated and
        populated during the CRTM initialization.
 
  CALLING SEQUENCE:
        Error_Status = CRTM_Destroy( ChannelInfo            , &
                                     Process_ID = Process_ID  )
 
  OUTPUTS:
        ChannelInfo:  Reinitialized ChannelInfo structure.
                      UNITS:      N/A
                      TYPE:       CRTM_ChannelInfo_type
                      DIMENSION:  Scalar
                      ATTRIBUTES: INTENT(IN OUT)
 
  OPTIONAL INPUTS:
        Process_ID:   Set this argument to the MPI process ID that this
                      function call is running under. This value is used
                      solely for controlling message output. If MPI is not
                      being used, ignore this argument.
                      UNITS:      N/A
                      TYPE:       INTEGER
                      DIMENSION:  Scalar
                      ATTRIBUTES: INTENT(IN), OPTIONAL
 
  FUNCTION RESULT:
        Error_Status: The return value is an integer defining the error
                      status. The error codes are defined in the
                      Message_Handler module.
                      If == SUCCESS the CRTM deallocations were successful
                         == FAILURE an unrecoverable error occurred.
                      UNITS:      N/A
                      TYPE:       INTEGER
                      DIMENSION:  Scalar
 
  SIDE EFFECTS:
        All CRTM shared data arrays and structures are deallocated.
 
  COMMENTS:
        Note the INTENT on the output ChannelInfo argument is IN OUT rather than
        just OUT. This is necessary because the argument may be defined upon
        input. To prevent memory leaks, the IN OUT INTENT is a must.
 
  \end{alltt}



