\chapter{How to use the CRTM library}
%====================================
\section{Step by Step Guide}
%===========================
This section will hopefully get you started using the CRTM library as quickly as possible. Refer to the following sections for more information about the structures and interfaces.

The examples shown here assume you are processing one sensor at a time. The CRTM can handle multiple sensors at once, but specifying the input information in a simple way is difficult; e.g. the \hyperref[sec:geometryinfo_structure]{\GeometryInfo} structure that is used to specify the sensor viewing geometry -- even sensors on the same platform typically have different numbers of fields-of-view (FOVs) per scan. For multiple sensor processing, we'll assume they will be separately processed in parallel.

Because there are many variations in what information is known ahead of time (and by ``ahead of time'' we mean at compile-time of your code), let's approach this via examples for a fixed number of atmospheric profiles, and a known sensors. It is left as an exercise to the reader to tailor calls to the CRTM in their application code according to their particular needs.

With regards to sensor identification, the CRTM uses a character string -- refered to as the \f{Sensor\_Id} -- to distinguish sensors and platforms. The lists of currently supported sensors, along with their associated \f{Sensor\_Id}'s, are shown in appendix \ref{sec:sensor_id}.


\newcounter{step}
\newcounter{example}[subsection]

\subsection{Step 1: Access the CRTM module}
%------------------------------------------
\stepcounter{step}
\label{sec:access_step}
All of the CRTM user procedures, parameters, and derived data type definitions are accessible via the container module \f{CRTM\_Module}. Thus, one needs to put the following statement in any calling program, module or procedure,

\qquad\f{USE CRTM\_Module}

Once you become familiar with the components of the CRTM you require, you can also specify an \f{ONLY} clause with the \f{USE} statement,

\qquad\f{USE CRTM\_Module}[\f{, ONLY:}\textit{only-list}]

where \textit{only-list} is a list of the symbols you want to ``import'' from \f{CRTM\_Module}. This latter form is the preferred style for self-documenting your code; e.g. when you give the code to someone else, they will be able to identify from which module various symbols in your code originate.


\subsection{Step 2: Declare the CRTM structures}
%-----------------------------------------------
\refstepcounter{step}
\label{sec:declare_step}
To compute satellite radiances you need to declare structures for the following information,\vspace{-2ex}
\begin{enumerate}
  \item Atmospheric profile data such as pressure, temperature, absorber amounts, clouds, aerosols, etc. Handled using the \hyperref[sec:atmosphere_structure]{\Atmosphere} structure.
  \item Surface data such as type of surface, temperature, surface type specific parameters etc. Handled using the \hyperref[sec:surface_structure]{\Surface} structure.
  \item Geometry information such as sensor scan angle, zenith angle, etc. Handled using the \hyperref[sec:geometryinfo_structure]{\GeometryInfo} structure.
  \item Instrument information, particularly which instrument(s), or sensor(s)\footnote{The terms ``instrument'' and ``sensor'' are used interchangeably in this document.}, you want to simulate. Handled using the \hyperref[sec:channelinfo_structure]{\ChannelInfo} structure.
  \item Results of the radiative transfer calculation. Handled using the \hyperref[sec:rtsolution_structure]{\RTSolution} structure.
  \item Optional inputs. Handled using the \hyperref[sec:options_structure]{\Options} structure.
\end{enumerate}

Let's assume you want to process, say, 50 profiles for the NOAA-18 AMSU-A sensor which has 15 channels. The declarations would look something like,
\begin{alltt}
  ! Processing parameters
  INTEGER     , PARAMETER :: N_SENSORS  =  1
  INTEGER     , PARAMETER :: N_CHANNELS = 15
  INTEGER     , PARAMETER :: N_PROFILES = 50
  CHARACTER(*), PARAMETER :: SID(N_SENSORS) = (/'amsua_n18'/)
  ! Declarations
  TYPE(CRTM_Atmosphere_type)   :: atm(N_PROFILES)
  TYPE(CRTM_Surface_type)      :: sfc(N_PROFILES)
  TYPE(CRTM_GeometryInfo_type) :: gInfo(N_PROFILES)  
  TYPE(CRTM_ChannelInfo_type)  :: chInfo(N_SENSORS)  
  TYPE(CRTM_RTSolution_type)   :: rts(N_CHANNELS,N_PROFILES)\end{alltt}


\subsection{Step 3: Initialise the CRTM}
%---------------------------------------
\stepcounter{step}
\label{sec:init_step}
The CRTM is initialised by calling the \hyperref[fig:init_interface]{\f{CRTM\_Init()}} function in the \f{CRTM\_LifeCycle.f90} module. This loads all the various coefficient data used by CRTM components into memory for later use. We'll assume that all the required datafiles reside in the subdirectory \f{./coeff\_data} and follow on from the example of Step \ref{sec:declare_step}. The CRTM initialisation is profile independent, so we're only dealing with sensor information here. The CRTM initialisation function call looks like,
\begin{alltt}
  INTEGER :: errStatus
  ....
  errStatus = CRTM_Init( chInfo, Sensor_Id=SID, File_Path='./coeff_data' )
  IF ( errStatus /= SUCCESS ) THEN 
    \textrm{\textit{handle error...}}
  END IF\end{alltt}

Here we see for the first time how the CRTM functions let you know if they were successful. As you can see the \hyperref[fig:init_interface]{\f{CRTM\_Init()}} function result is an error status that is checked against a parameterised integer error code, \f{SUCCESS}. The function result should \emph{not} be tested against the actual value of the error code, just its parameterised name. Other available error code parameters are \f{FAILURE}, \f{WARNING}, and \f{INFORMATION} -- although the latter is never used as a function result.


\subsection{Step 4: Allocate the CRTM structures}
%------------------------------------------------
\stepcounter{step}
Now we need to allocate the \emph{internal} components of the various CRTM structures where necessary to hold the input or output data. In this case, functions are used to perform these ``internal'' allocations. The function naming convention is \f{CRTM\_Allocate\_}\textit{name} where, for typical usage, the CRTM structures that need to be allocated are the \hyperref[sec:atmosphere_structure]{\Atmosphere}, \hyperref[sec:rtsolution_structure]{\RTSolution} and, if used, \hyperref[sec:options_structure]{\Options} structures. Potentially, the \hyperref[sec:sensordata_structure]{\SensorData} component of the \hyperref[sec:surface_structure]{\Surface} structure may also need to be allocated to allow for input of sensor observations for some of the NESDIS microwave surface emissivity models.

\subsubsection{Allocation of the Atmosphere structure}
%.....................................................
First, we'll allocate the atmosphere structure to the required dimensions, like so:
\begin{alltt}
  ! Allocate the atmosphere structure
  errStatus = CRTM_Allocate_Atmosphere( n_Layers   , &  ! Input
                                        N_ABSORBERS, &  ! Input (always 2)
                                        n_Clouds   , &  ! Input
                                        n_Aerosols , &  ! Input
                                        atm          )  ! Output
  IF ( errStatus /= SUCCESS ) THEN 
    CALL Display_Message( PROGRAM_NAME, &
                          'Error allocating the atmosphere structure', & 
                          FAILURE )
    STOP
  END IF\end{alltt}
Note that the number of absorbers is in all capitals. In the CRTM, this style convention indicates a parameter. A parameter is used for the number of absorbers in the current CRTM release because the number of absorbers is fixed at two: water vapour and ozone. Future CRTM releases will allow more flexibility in selecting the number of absorbers, but currently the number must be set to two.

If 










\subsection{Step 6: Fill the CRTM input structures with data}
%------------------------------------------------------------
\stepcounter{step}


\subsection{Step 7: Call the required CRTM function}
%---------------------------------------------------
\stepcounter{step}


\subsection{Step 8: Destroy the CRTM}
%------------------------------------
\stepcounter{step}



\section{Interface Descriptions}
%===============================

\subsection{CRTM Initialisation}
%-------------------------------

\begin{figure}[htp]
  \centering
  \doublebox{
  \begin{minipage}[b]{6.5in}
    \begin{alltt}
  FUNCTION CRTM_Init( ChannelInfo      , &
                      Sensor_ID        , &
                      CloudCoeff_File  , &
                      AerosolCoeff_File, &
                      EmisCoeff_File   , &
                      File_Path        , &
                      Quiet            , &
                      Process_ID       , &
                      Output_Process_ID, &
                      RCS_Id           , &
                      Message_Log      ) &
                    RESULT( Error_Status )
    ! Arguments
    TYPE(CRTM_ChannelInfo_type), INTENT(IN OUT) :: ChannelInfo(:)
    CHARACTER(*),      OPTIONAL, INTENT(IN)     :: Sensor_ID(:)
    CHARACTER(*),      OPTIONAL, INTENT(IN)     :: CloudCoeff_File
    CHARACTER(*),      OPTIONAL, INTENT(IN)     :: AerosolCoeff_File
    CHARACTER(*),      OPTIONAL, INTENT(IN)     :: EmisCoeff_File
    CHARACTER(*),      OPTIONAL, INTENT(IN)     :: File_Path
    INTEGER     ,      OPTIONAL, INTENT(IN)     :: Quiet
    INTEGER     ,      OPTIONAL, INTENT(IN)     :: Process_ID
    INTEGER     ,      OPTIONAL, INTENT(IN)     :: Output_Process_ID
    CHARACTER(*),      OPTIONAL, INTENT(OUT)    :: RCS_Id
    CHARACTER(*),      OPTIONAL, INTENT(IN)     :: Message_Log
    ! Function result
    INTEGER :: Error_Status
    \end{alltt}
    \centering
    \begin{tabular}{p{3.25cm} p{6.5cm} p{1.75cm} p{2.5cm}}
      \hline
      \tblhd{Argument}             & \tblhd{Description}                             & \tblhd{Rank} & \tblhd{Intent} \\
      \hline\hline
      \f{ChannelInfo}              & Sensor and channel info structure               & $N$          & Output \\
      \optarg{Sensor\_Id}          & \textit{Sensor identification string}           & $N$          & \textit{Input}  \\
      \optarg{CloudCoeff\_File}    & \textit{Cloud optical property LUT filename}    & Scalar       & \textit{Input}  \\
      \optarg{AerosolCoeff\_File}  & \textit{Aerosol optical property LUT filename}  & Scalar       & \textit{Input}  \\
      \optarg{EmisCoeff\_File}     & \textit{IR sea surface emissivity LUT filename} & Scalar       & \textit{Input}  \\
      \optarg{File\_Path}          & \textit{Path to *Coeff files}                   & Scalar       & \textit{Input}  \\
      \optarg{Quiet}               & \textit{Keyword to control info message output} & Scalar       & \textit{Input}  \\
      \optarg{Process\_Id}         & \textit{MPI process Id}                         & Scalar       & \textit{Input}  \\
      \optarg{Output\_Process\_Id} & \textit{MPI process Id for message output}      & Scalar       & \textit{Input}  \\
      \optarg{RCS\_Id}             & \textit{Version control ID for the module}      & Scalar       & \textit{Output} \\
      \optarg{Message\_Log}        & \textit{Log message filename}                   & Scalar       & \textit{Input} 
    \end{tabular}
  \end{minipage}
  }
  \caption{CRTM Initialisation interface and argument description.}
  \label{fig:init_interface}
\end{figure}

\subsection{Forward Model}
%-------------------------

\begin{figure}[htp]
  \centering
  \doublebox{
  \begin{minipage}[b]{6.5in}
    \begin{alltt}
  FUNCTION CRTM_Forward( Atmosphere  , &
                         Surface     , &
                         GeometryInfo, &
                         ChannelInfo , &
                         RTSolution  , &
                         Options     , &    
                         RCS_Id      , &
                         Message_Log ) &
                       RESULT( Error_Status )
    ! Arguments
    TYPE(CRTM_Atmosphere_type),        INTENT(IN)     :: Atmosphere(:)
    TYPE(CRTM_Surface_type),           INTENT(IN)     :: Surface(:)
    TYPE(CRTM_GeometryInfo_type),      INTENT(IN OUT) :: GeometryInfo(:)
    TYPE(CRTM_ChannelInfo_type),       INTENT(IN)     :: ChannelInfo(:)
    TYPE(CRTM_RTSolution_type),        INTENT(IN OUT) :: RTSolution(:,:)
    TYPE(CRTM_Options_type), OPTIONAL, INTENT(IN)     :: Options(:)
    CHARACTER(*),            OPTIONAL, INTENT(OUT)    :: RCS_Id
    CHARACTER(*),            OPTIONAL, INTENT(IN)     :: Message_Log
    ! Function result
    INTEGER :: Error_Status
    \end{alltt}
    \centering
    \begin{tabular}{p{3.25cm} p{6.5cm} p{1.75cm} p{2.5cm}}
      \hline
      \tblhd{Argument}       & \tblhd{Description}                           & \tblhd{Rank} & \tblhd{Intent} \\
      \hline\hline
      \f{Atmosphere}         & Atmospheric state                             & $M$          & Input  \\
      \f{Surface}            & Surface state                                 & $M$          & Input  \\
      \f{GeometryInfo}       & Geometry information (e.g. angles)            & $M$          & Input  \\
      \f{ChannelInfo}        & Sensor channel information                    & $N$          & Input  \\
      \f{RTSolution}         & Radiative transfer solution                   & $L \times M$ & Output \\
      \optarg{Options}       & \textit{Structure containing optional inputs} & $M$          & \textit{Input}  \\
      \optarg{RCS\_Id}       & \textit{Version control ID for the module}    & Scalar       & \textit{Output} \\
      \optarg{Message\_Log}  & \textit{Log message filename}                 & Scalar       & \textit{Input} 
    \end{tabular}
  \end{minipage}
  }
  \caption{CRTM Forward model interface and argument description.}
  \label{fig:fwd_interface}
\end{figure}


\subsection{K-Matrix Model}
%--------------------------

\begin{figure}[htp]
  \centering
  \doublebox{
  \begin{minipage}[b]{6.5in}
    \begin{alltt}
  FUNCTION CRTM_K_Matrix( Atmosphere  , &
                          Surface     , &
                          RTSolution_K, &
                          GeometryInfo, &
                          ChannelInfo , &
                          Atmosphere_K, &
                          Surface_K   , &
                          RTSolution  , &
                          Options     , &
                          RCS_Id      , &
                          Message_Log ) &
                        RESULT( Error_Status )
    ! Arguments
    TYPE(CRTM_Atmosphere_type)       , INTENT(IN)     :: Atmosphere(:)
    TYPE(CRTM_Surface_type)          , INTENT(IN)     :: Surface(:)
    TYPE(CRTM_RTSolution_type)       , INTENT(IN OUT) :: RTSolution_K(:,:)
    TYPE(CRTM_GeometryInfo_type)     , INTENT(IN OUT) :: GeometryInfo(:)
    TYPE(CRTM_ChannelInfo_type)      , INTENT(IN)     :: ChannelInfo(:)
    TYPE(CRTM_Atmosphere_type)       , INTENT(IN OUT) :: Atmosphere_K(:,:)
    TYPE(CRTM_Surface_type)          , INTENT(IN OUT) :: Surface_K(:,:)
    TYPE(CRTM_RTSolution_type)       , INTENT(IN OUT) :: RTSolution(:,:)
    TYPE(CRTM_Options_type), OPTIONAL, INTENT(IN)     :: Options(:)
    CHARACTER(*),            OPTIONAL, INTENT(OUT)    :: RCS_Id
    CHARACTER(*),            OPTIONAL, INTENT(IN)     :: Message_Log
    ! Function result
    INTEGER :: Error_Status
    \end{alltt}
    \centering
    \begin{tabular}{p{3.25cm} p{6.5cm} p{1.75cm} p{2.5cm}}
      \hline
      \tblhd{Argument}      & \tblhd{Description}                           & \tblhd{Rank} & \tblhd{Intent} \\
      \hline\hline
      \f{Atmosphere}        & Atmospheric state                             & $M$          & FWD Input  \\
      \f{Surface}           & Surface state                                 & $M$          & FWD Input  \\
      \f{RTSolution\_K}     & Adjoint radiative transfer solution           & $L \times M$ & K   Input  \\
      \f{GeometryInfo}      & Geometry information (e.g. angles)            & $M$          & Input      \\
      \f{ChannelInfo}       & Sensor channel information                    & $N$          & Input      \\
      \f{Atmosphere\_K}     & Atmospheric state Jacobians                   & $L \times M$ & K   Output \\
      \f{Surface\_K}        & Surface state Jacobians                       & $L \times M$ & K   Output \\
      \f{RTSolution}        & Radiative transfer solution                   & $L \times M$ & FWD Output \\
      \optarg{Options}      & \textit{Structure containing optional inputs} & $M$          & \textit{Input}  \\
      \optarg{RCS\_Id}      & \textit{Version control ID for the module}    & Scalar       & \textit{Output} \\
      \optarg{Message\_Log} & \textit{Log message filename}                 & Scalar       & \textit{Input} 
    \end{tabular}
  \end{minipage}
  }
  \caption{CRTM K-Matrix model interface and argument description.}
  \label{fig:k_interface}
\end{figure}


\subsection{CRTM Destruction}
%----------------------------

\begin{figure}[htp]
  \centering
  \doublebox{
  \begin{minipage}[b]{16.5cm}
    \begin{alltt}
  FUNCTION CRTM_Destroy( ChannelInfo , &  ! Output
                         Process_ID  , &  ! Optional input
                         RCS_Id      , &  ! Revision control
                         Message_Log ) &  ! Error messaging
                       RESULT ( Error_Status )
    ! Arguments
    TYPE(CRTM_ChannelInfo_type), INTENT(IN OUT) :: ChannelInfo(:)
    INTEGER     ,      OPTIONAL, INTENT(IN)     :: Process_ID
    CHARACTER(*),      OPTIONAL, INTENT(OUT)    :: RCS_Id
    CHARACTER(*),      OPTIONAL, INTENT(IN)     :: Message_Log
    ! Function result
    INTEGER :: Error_Status
    \end{alltt}
    \centering
    \begin{tabular}{p{3.25cm} p{6.5cm} p{1.75cm} p{2.5cm}}
      \hline
      \tblhd{Argument}      & \tblhd{Description}                           & \tblhd{Rank} & \tblhd{Intent} \\
      \hline\hline
      \f{ChannelInfo}       & Sensor and channel info structure             & $N$          & In/Output       \\
      \optarg{Process\_Id}  & \textit{MPI process Id}                       & Scalar       & \textit{Input}  \\
      \optarg{RCS\_Id}      & \textit{Version control ID for the module}    & Scalar       & \textit{Output} \\
      \optarg{Message\_Log} & \textit{Log message filename}                 & Scalar       & \textit{Input} 
    \end{tabular}
  \end{minipage}
  }
  \caption{CRTM Destruction interface and argument description.}
  \label{fig:destroy_interface}
\end{figure}


\section{Filling input data structures}
%======================================

\begin{figure}[htp]
  \centering
  \input{graphics/gInfo/sensor_scan_angle.pstex_t}
  \caption{Definition of \GeometryInfo{} sensor scan angle component.}
  \label{fig:gInfo_sensor_scan_angle}
\end{figure}

\begin{figure}[htp]
  \centering
  \input{graphics/gInfo/sensor_zenith_angle.pstex_t}
  \caption{Definition of \GeometryInfo{} sensor zenith angle component.}
  \label{fig:gInfo_sensor_zenith_angle}
\end{figure}

\begin{figure}[htp]
  \centering
  \input{graphics/gInfo/sensor_azimuth_angle.pstex_t}
  \caption{Definition of \GeometryInfo{} sensor azimuth angle component.}
  \label{fig:gInfo_sensor_azimuth_angle}
\end{figure}

\begin{figure}[htp]
  \centering
  \input{graphics/gInfo/source_zenith_angle.pstex_t}
  \caption{Definition of \GeometryInfo{} source zenith angle component.}
  \label{fig:gInfo_source_zenith_angle}
\end{figure}

\begin{figure}[htp]
  \centering
  \input{graphics/gInfo/source_azimuth_angle.pstex_t}
  \caption{Definition of \GeometryInfo{} source azimuth angle component.}
  \label{fig:gInfo_source_azimuth_angle}
\end{figure}
