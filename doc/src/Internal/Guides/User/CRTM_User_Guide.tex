% The generic preamble
\documentclass[10pt,letterpaper,fleqn,titlepage]{article}

% Define packages to use
\usepackage{natbib}
\usepackage[dvips]{graphicx,color}
\usepackage{amsmath,amssymb}
\usepackage{bm}
\usepackage{caption}
\usepackage{xr}
\usepackage{ifthen}
\usepackage[dvipdfm,colorlinks,linkcolor=blue,citecolor=blue,urlcolor=blue]{hyperref}
\usepackage{fancybox}
\usepackage{textcomp}
\usepackage{alltt}
%\usepackage{floatflt}
%\usepackage{svn}


% Redefine default page
\setlength{\textheight}{9in}  % 1" above and below
\setlength{\textwidth}{6.75in}   % 0.5" left and right
\setlength{\oddsidemargin}{-0.25in}

% Redefine default paragraph
\setlength{\parindent}{0pt}
\setlength{\parskip}{1ex plus 0.5ex minus 0.2ex}

% Define caption width and default fonts
\setlength{\captionmargin}{0.5in}
\renewcommand{\captionfont}{\sffamily}
\renewcommand{\captionlabelfont}{\bfseries\sffamily}

% Define commands for super- and subscript in text mode
\newcommand{\superscript}[1]{\ensuremath{^\textrm{#1}}}
\newcommand{\subscript}[1]{\ensuremath{_\textrm{#1}}}

% Derived commands
\newcommand{\invcm}{\textrm{cm\superscript{-1}}}
\newcommand{\micron}{\ensuremath{\mu\textrm{m}}}

\newcommand{\df}{\ensuremath{\delta f}}
\newcommand{\Df}{\ensuremath{\Delta f}}
\newcommand{\dx}{\ensuremath{\delta x}}
\newcommand{\Dx}{\ensuremath{X_{max}}}
\newcommand{\Xeff}{\ensuremath{X_{eff}}}

\newcommand{\water}{\textrm{H\subscript{2}O}}
\newcommand{\carbondioxide}{\textrm{CO\subscript{2}}}
\newcommand{\ozone}{\textrm{O\subscript{3}}}

\newcommand{\taup}[1]{\ensuremath{\tau_{#1}}}
\newcommand{\efftaup}[1]{\ensuremath{\tau_{#1}^{*}}}

\newcommand{\textbfm}[1]{\boldmath\ensuremath{#1}\unboldmath}

\newcommand{\rb}[1]{\raisebox{1.5ex}[0pt]{#1}}

\newcommand{\f}[1]{\texttt{#1}}

% Define how equations are numbered
\numberwithin{equation}{section}
\numberwithin{figure}{section}
\numberwithin{table}{section}

% Define a command for title page author email footnote
\newcommand{\email}[1]
{%
  \renewcommand{\thefootnote}{\alph{footnote}}%
  \footnote{#1}
  \renewcommand{\thefootnote}{\arabic{footnote}}
}

% Define a command to print the Office Note subheading
\newcommand{\notesubheading}[1]
{%
  \ifthenelse{\equal{#1}{}}{}
  { {\Large\bfseries Office Note #1\par}%
    {\scriptsize \sc This is an unreviewed manuscript, primarily intended for informal}\\ 
    {\scriptsize \sc exchange of information among JCSDA researchers\par}%
  }
}

% Redefine the maketitle macro
\makeatletter
\def\docseries#1{\def\@docseries{#1}}
\def\docnumber#1{\def\@docnumber{#1}}
\renewcommand{\maketitle}
{%
  \thispagestyle{empty}
  \vspace*{1in}
  \begin{center}%
     \sffamily
     {\huge\bfseries Joint Center for Satellite Data Assimilation\par}%
     \notesubheading{\@docnumber}
  \end{center}
  \begin{flushleft}%
     \sffamily
     \vspace*{0.5in}
     {\Large\bfseries\ifthenelse{\equal{\@docseries}{}}{}{\@docseries: }\@title\par}%
     \medskip
     {\large\@author\par}%
     \medskip
     {\large\@date\par}%
     \bigskip\hrule\vspace*{2pc}%
  \end{flushleft}%
  \newpage
  \setcounter{footnote}{0}
}
\makeatother
\docseries{}
\docnumber{}


% Define a command for a DRAFT watermark
\usepackage{eso-pic}
\newcommand{\draftwatermark}
{
  \AddToShipoutPicture{%
    \definecolor{lightgray}{gray}{.85}
    \setlength{\unitlength}{1in}
    \put(2.5,3.5){%
      \rotatebox{45}{%
        \resizebox{4in}{1in}{%
          \textsf{\textcolor{lightgray}{DRAFT}}
        }
      }
    }
  }
}




% Define included documents
\includeonly{New,Introduction,Get,Build,Use,StructDef.appendix,SensorId.appendix,Utility.appendix}

% Define fixed font for the various component names
\newcommand{\Atmosphere}{\texttt{Atmosphere}}
\newcommand{\Cloud}{\texttt{Cloud}}
\newcommand{\Aerosol}{\texttt{Aerosol}}
\newcommand{\Surface}{\texttt{Surface}}
\newcommand{\Geometry}{\texttt{Geometry}}
\newcommand{\GeometryInfo}{\texttt{GeometryInfo}}
\newcommand{\ChannelInfo}{\texttt{ChannelInfo}}
\newcommand{\Options}{\texttt{Options}}
\newcommand{\SSUInput}{\texttt{SSU\_Input}}
\newcommand{\ZeemanInput}{\texttt{Zeeman\_Input}}
\newcommand{\AtmOptics}{\texttt{AtmOptics}}
\newcommand{\SfcOptics}{\texttt{SfcOptics}}
\newcommand{\RTSolution}{\texttt{RTSolution}}
\newcommand{\SensorData}{\texttt{SensorData}}
\newcommand{\AtmAbsorption}{\texttt{AtmAbsorption}}
\newcommand{\AtmScatter}{\texttt{AtmScatter}}
\newcommand{\CloudScatter}{\texttt{CloudScatter}}
\newcommand{\AerosolScatter}{\texttt{AerosolScatter}}
\newcommand{\SpcCoeff}{\texttt{SpcCoeff}}
% Generic fixed font command for Fortran95 syntax
\newcommand{\f}[1]{\texttt{#1}}
% Table heading format
\newcommand{\tblhd}[1]{\sffamily\textbf{#1}}

%Argument formatting
\newcommand{\inarg}[1]{\f{\textcolor{green}{#1}}}
\newcommand{\outarg}[1]{\f{\textcolor{red}{#1}}}
\newcommand{\optarg}[1]{\f{\textit{#1}}}

% Define the adjoint operator with correct spacing
\newcommand{\dstar}{\ensuremath{\delta^{*}\!}}
%Define macro for radiance units
\newcommand{\radunit}{mW/(m\ensuremath{^2}.sr.\invcm)}

% Title info
\title{v2.0 User Guide}
%\author{Paul van Delst\email{paul.vandelst@noaa.gov}\\JCSDA/EMC/SAIC}
\date{March, 2010}


%-------------------------------------------------------------------------------
%                            Ze document begins...
%-------------------------------------------------------------------------------
\begin{document}
\maketitle

%\draftwatermark

% The front matter
%=================
\thispagestyle{empty}
\vspace*{10cm}
\begin{center}
  {\sffamily\Large\bfseries Change History}
  \begin{table}[htp]
    \centering
    \begin{tabular}{|p{2cm}|p{3cm}|p{8cm}|}
      \hline
      \sffamily\textbf{Date} & \sffamily\textbf{Author} & \sffamily\textbf{Change}\\
      \hline\hline
      2009-01-30 & P.van Delst & Initial release.\\
      \hline
      2009-02-03 & P.van Delst & Updated chapter 2 with descriptions of the example code and coefficient tarballs. Added explanation of layering convention in chapter 4.\\
      \hline
      2010-03-12 & P.van Delst & Updated for v2.0.\\
      \hline
    \end{tabular}
  \end{table}
\end{center}
\clearpage

% The frontmatter
%================
\setcounter{page}{1}
\pagenumbering{roman}
  \tableofcontents\newpage
  \listoffigures\newpage
  \listoftables\newpage
  \chapter*{What's New in v2.0}
%===========================
\addcontentsline{toc}{chapter}{What's New in v2.0}

\section*{New Science}
%---------------------
\addcontentsline{toc}{section}{New Science}

\begin{description}
\item[Multiple transmittance algorithms] There are now two transmittance models available for use in the CRTM: ODAS (Optical Depth in Absorber Space), which is equivalent to the previous CompactOPTRAN algorithm; and ODPS (Optical Depth in Pressure Space) which is similar to the RTTOV-type of transmittance algorithm, except here OPTRAN is used for water vapor line absorption.

The algorithm is selectable by the user via the transmittance coefficient (\f{TauCoeff}) data file used to initialise the CRTM. This method, rather than a switch argument in the \hyperref[sec:CRTM_Init_interface]{\f{CRTM\_Init()}} function, was chosen to allow users to ``mix-and-match'' transmittance algorithms for different sensors in the same initialisation call.

\item[SSU-specific transmittance model] Similar to the multiple transmittance algorithm approach, a separate algorithm \emph{just} for the SSU instrument has been constructed. The algorithm is based on the ODAS approach, but with elements to account for the time-dependence of the SSU CO\subscript{2} cell pressures.

\item[Zeeman-splitting transmittance model for SSMIS upper-level channels] A separate algorithm is available to account for the change in absorption at very low pressures due to the Zeeman-splitting of absorption lines. Currently this algorithm has only been applied to the affected channels in the SSMIS instrument, 19-22.

\item[Visible sensor capability] The CRTM now supports radiative transfer for visible instruments/channels. The treatment of visible channels was handled in the CRTM framework by considering them separate instruments. The sensor identifier for these instruments/channels are differentiated from their infrared counterparts by a ``\f{v.}'' prefix. For example, while \f{modis\_aqua} is the sensor identifier for the infrared channels, \f{v.modis\_aqua} identifies the visible channels.

\item[Inclusion of Matrix Operator Method (MOM) in radiaitve transfer] To handle visible wavelength radiative transfer in the prescence of aerosols, the Advanced Doubling-Adding (ADA) algorithm was adapted to use the MOM technique \citep{Liu_1996}.

\item[Inclusion of additional infrared sea surface emissivity model] Files containing the emissivity data (\f{EmisCoeff}) for the \citet{Nalli_2008a} model are provided. Previously, only the \f{EmisCoeff} files for the \citet{WuSmith_1997} model were provided. Users can now select between the \citet{Nalli_2008a} or \citet{WuSmith_1997} models by specifying the requisite filename in the call to \hyperref[sec:CRTM_Init_interface]{\f{CRTM\_Init()}}.

\item[Surface BRDF for solar-affected shortwave IR channels] A bi-directional reflectance distribution function (BRDF) has been added to account for reflected solar in affected shortware infrared channels \citep{Breon_1993}.

\item[Reflectivity for downwelling infrared over water] The reflectivity for downwelling infrared  radiation over water surface has been changed from Lambertian to specular.

\item[Aerosol type changes] To account for changes in the handling of GOCART \citep{Chin_2002} aerosol model output, additional sea salt coarse modes were added to the list of allowed aerosol types. Also, the separate dry and wet types for organic and black carbon aerosols were combined, with a relatvie humidity of 0\% used to indicate the previous ``dry'' aerosol type. See table \ref{tab:aerosol_type} for the new list of accepted aerosol types.

\end{description}


\section*{Interface Changes}
%---------------------------
\addcontentsline{toc}{section}{Interface Changes}

\begin{description}
\item[CRTM Initialisation function] The changes to the \hyperref[sec:CRTM_Init_interface]{\f{CRTM\_Init()}} interface were relatively minor but do require calling codes to be modified:
  \begin{itemize}
  \item The \f{Sensor\_Id} argument is now mandatory. This argument is used to construct the sensor-specific \f{SpcCoeff} and \f{TauCoeff} filename and in the past was optional to allow for ``generic'' filenames. This is no longer allowed and generic \f{SpcCoeff} and \f{TauCoeff} files are no longer used.
  \item The loading of the \f{CloudCoeff} and \f{AerosolCoeff} datafiles containing the optical properties of cloud and aerosol particulates is no longer mandatory. For cloud-free CRTM runs, the load of the \f{CloudCoeff} and \f{AerosolCoeff} datafiles can be disabled via the optional \f{Load\_CloudCoeff} and \f{Load\_AerosolCoeff} arguments which are logical switches (true or false).
  \end{itemize}

\item[User accessible structures] The structures are defined as those that are used in the argument lists of the main CRTM functions (e.g. initialisation; the forward, tangent-linear, adjoint, and K-matrix models; and destruction). Changes were made to both the structure definitions and their procedures. To mitigate the possibility of memory leaks, the definitions of array members of structures have had their \f{POINTER} attribute replaced with \f{ALLOCATABLE}. This was a first step in preparation for use of Fortran2003 Object Oriented features in the CRTM (once Fortran2003 compiler become widely available), where the derived type structure definitions will be reclassified as objects and their procedures will be type-bound. To delineate this change from previous versions of CRTM the interfaces of the derived type procedures have been altered by:
  \begin{itemize}
  \item changing the procedure names to use the convention \f{CRTM\_}\textit{object}\f{\_}\textit{action} where an \textit{object} can be any of the user accessible CRTM derived types (e.g. \f{CRTM\_Atmosphere\_type}, \f{CRTM\_RTSolution\_type} etc), and the \textit{action} can be those defined operations for the structure (e.g. \f{Create}, \f{Destroy}, \f{Inspect}, etc).
  \item making the first dummy argument of the definition module procedures the derived type itself. This will eventually allow the procedures to be called via an instance of the derived type\footnote{Interested readers can investigate the \f{PASS} attribute that can be used in the \f{PROCEDURE} statement within derived type definitions in Fortran2003.}\footnote{The I/O functions do not yet follow this convention, since they are considered secondary to the definition module procedures used to manipulate the derived types.}
  \end{itemize}
  All of the current derived type definitions and their associated procedures and interfaces are shown in appendix \ref{sec:structure_and_interface_definition}.

\item[\Options{} structure specific changes] The additional changes made to the \hyperref[sec:options_structure]{\f{CRTM\_Options\_type}} definition:
  \begin{itemize}
  \item all usage on/off switches have been changed from integers (0/1) to logicals (true/false),
  \item a logical switch to control input checking, \f{Check\_Input}, has been added.
  \item structure components for \hyperref[sec:ssu_input_structure]{SSU-specific} and \hyperref[sec:zeeman_input_structure]{Zeeman model} input have been added.
  \end{itemize}
\end{description}

\pagenumbering{arabic}
\setcounter{page}{1}


% Include all the sections
%=========================
\chapter{Introduction}
%=====================


\section{Components}
%===================
\label{sec:components}

The LBLRTM I/O library is constructed around five data constructs, described in table \ref{tab:component_definitions}:
\begin{table}[htp]
  \centering
  \caption{The data constructs of the LBLRTM I/O library}
  \begin{tabular}{p{2.5cm} p{12cm}}
    \hline\\[-0.1cm]
    \sffamily\textbf{Component Name} & \sffamily\textbf{Description} \\
    \hline\hline\\[-0.2cm]
    \texttt{Fhdr}  & The file header construct that is present at the start of each layer of data. \\\\
    \texttt{Phdr}  & This is the panel header construct that is present at the start of each ``chunk'' of data (usually referred to as a ``panel''. See following.) \\\\
    \texttt{Panel} & This construct corresponds to a ``chunk'' of spectral data. An LBLRTM datafile is referred to as being a single- or double-panel file. The former means a single spectral quantity is present (e.g. optical depth), and the latter means that two spectral quantities are present (e.g. radiance and transmittances). \\\\
    \texttt{Layer} & This construct contains spectral data for the entire frequency range of an LBLRTM calculation for a single layer. The concept ``layer'' can correspond to the spectral data for individual atmospheric layers of the input profile, or to the final result for an entire atmosphere. \\\\
    \texttt{File}  & This contruct is true to its name. It corresponds to an entire datafile of data which may consist of a single layer or multiple layers, and for single- or double-panel spectral data. \\
  \hline
  \end{tabular}
  \label{tab:component_definitions}
\end{table}

Each component has a definition module to define the object and some basic methods to manipulate it, and an I/O module to read and write instances of the objects from/to file.

Two components -- the file header and panel header -- are standalone, but the others contain other components, i.e. the panel object contains panel headers; the layer object contains file headers; and the file object contains layers. A schematic illustration of how the actual LBLRTM datafile format relates to the component definitions is shown in figure \ref{fig:lblrtm_format}.

Note, however, that when an LBLRTM file is read, the individual panel ``chunks'' of spectral data are concatenated into a single spectrum. Thus the \Panel{} object itself is only used when reading from an LBLRTM file and is not used in the \File{} or \Layer{} objects.

\begin{figure}[htp]
  \centering
  \input{graphics/lblrtm_format.pstex_t}
  \caption{Schematic illustration of the LBLRTM single- and double-panel datafile format. A datafile can contain one, or multiple, layers of data.}
  \label{fig:lblrtm_format}
\end{figure}



\section{Conventions}
%====================
\label{sec:conventions}
The following are conventions that have been adhered to in the current release of the LBLRTM I/O library. They are guidelines intended to make understanding the code at a glance easier, to provide a recognisable ``look and feel'', and to minimise name space clashes.



\subsection{Naming of Objects and Instances of Objects}
%------------------------------------------------------

The object\footnote{The terms ``derived type'' and ``structure'' can also be used as the code is not yet fully OO - that's for future updates.} naming convention adopted for use in the LBLRTM I/O library is, 

\hspace{0.4cm}\f{LBLRTM\_}\textit{name}\f{\_type} 

where \textit{name} is an identifier for the particular component (e.g. panel header, layer, file, etc as listed in table \ref{tab:component_definitions}). All object type names are suffixed with ``\f{\_type}''. The ``\f{LBLRTM\_}'' prefix is to define a namespace to minimise name clashes. An instance of a object is then referred to via its \textit{name}, or some sort of derivate of its \textit{name}. Some object declarations examples are,

\begin{alltt}
  TYPE(\hyperref[fig:LBLRTM_File_type_structure]{LBLRTM_File_type}) :: sp_file, dp_file
  TYPE(\hyperref[fig:LBLRTM_Layer_type_structure]{LBLRTM_Layer_type}) :: layer\end{alltt}



\subsection{Naming of Definition Modules}
%----------------------------------------

Modules containing object definitions are termed \textit{definition modules}. These modules contain the actual object definitions as well as various utility procedures that operate on the object. The naming convention adopted for definition modules in the LBLRTM I/O library is, 

\hspace{0.4cm}\f{LBLRTM\_}\textit{name}\f{\_Define} 

where all definition module names are suffixed with ``\f{\_Define}''. The actual source code files for these modules have the same name with a ``\f{.f90}'' suffix.



\subsection{Naming of I/O Modules}
%---------------------------------

Modules containing all the object I/O procedures are termed, surprise, surprise, \textit{I/O modules}. These modules contain function to read and write LBLRTM format datafiles. The naming convention adopted for I/O modules in the LBLRTM I/O library is, 

\hspace{0.4cm}\f{LBLRTM\_}\textit{name}\f{\_IO} 

where all I/O module names are suffixed with ``\f{\_IO}''. As with the definition modules, the actual source code files for these modules have the same name with a ``\f{.f90}'' suffix.



\subsection{Standard Definition Module Procedures}
%-------------------------------------------------

The definition modules for the user-accessible objects (for practical purposes just \File, although \Layer, \Fhdr, \Panel, and \Phdr are accessible for now) contain a standard set of procedures for use with the object being defined. The naming convention for these procedures is,

\hspace{0.4cm}\f{LBLRTM\_}\textit{name}\f{\_}\textit{action}

where the available default actions for each procedure are listed in table \ref{tab:definition_module_default_procedures}. This is not an exhaustive list but procedures for the actions listed in table \ref{tab:definition_module_default_procedures} are generally going to be present.

The exception is that the objects with no allocatable components do not have a creation procedure.

\begin{table}[htp]
  \centering
  \caption{Default action procedures available in object definition modules. $^{\dagger}$Procedures not available for the \Fhdr{} and \Phdr{} objects. $^{\ddagger}$Procedure available only for the \Layer{} object.}
  \begin{tabular}{p{2.5cm} p{3.5cm} p{8.5cm}}
    \hline\\[-0.1cm]
    \sffamily\textbf{Action} & \sffamily\textbf{Type} & \sffamily\textbf{Description} \\
    \hline\hline\\[-0.2cm]
    \texttt{OPERATOR(==)}             & Elemental function   & Tests the equality of two structures. \\
    \texttt{OPERATOR(/=)}             & Elemental function   & Tests the inequality of two structures. \\
    \texttt{Associated}$^{\dagger}$   & Elemental function   & Tests if the object components have been allocated. \\
    \texttt{Create}$^{\dagger}$       & Elemental subroutine & Allocates any allocatable object components. \\
    \texttt{Destroy}                  & Elemental subroutine & Reinitialises an object. \\
    \texttt{DefineVersion}            & Subroutine           & Returns the module version information. \\
    \texttt{Frequency}$^{\ddagger}$   & Subroutine           & Compute and return the spectral frequency grid. \\
    \texttt{Inspect}                  & Subroutine           & Displays object contents to \texttt{stdout}. \\
    \texttt{IsValid}                  & Elemental function   & Tests if the object contains valid data. \\
    \texttt{SetValid}                 & Elemental subroutine & Flags the object as containing valid data. \\
  \hline
  \end{tabular}
  \label{tab:definition_module_default_procedures}
\end{table}

\begin{table}[htp]
  \centering
  \caption{Default action procedures available in object I/O modules.}
  \begin{tabular}{p{2.5cm} p{3.5cm} p{8.5cm}}
    \hline\\[-0.1cm]
    \sffamily\textbf{Action} & \sffamily\textbf{Type} & \sffamily\textbf{Description} \\
    \hline\hline\\[-0.2cm]
    \texttt{IOVersion} & Subroutine  & Returns the module version information. \\
    \texttt{Read}      & Function    & Loads an instance of an object with data read from file. \\
    \texttt{Write}     & Function    & Write an instance of an object to file. \\
  \hline
  \end{tabular}
  \label{tab:io_module_default_procedures}
\end{table}

Some examples of these procedure names are,

\begin{alltt}
  \hyperref[sec:LBLRTM_File_Associated_interface]{LBLRTM_File_Associated}
  \hyperref[sec:LBLRTM_File_IsValid_interface]{LBLRTM_File_IsValid}
  \hyperref[sec:LBLRTM_Layer_Destroy_interface]{LBLRTM_Layer_Destroy}
  \hyperref[sec:LBLRTM_Layer_Frequency_interface]{LBLRTM_Layer_Frequency}
  \hyperref[sec:LBLRTM_File_Inspect_interface]{LBLRTM_File_Inspect}\end{alltt}

The relational operators, \f{==} and \f{/=}, are implemented via overloaded \f{Equal} and \f{NotEqual} action procedures respectively, as is shown below for the \File{} structure,

\begin{alltt}
  INTERFACE OPERATOR(==)
    MODULE PROCEDURE LBLRTM_File_Equal
  END INTERFACE OPERATOR(==)

  INTERFACE OPERATOR(/=)
    MODULE PROCEDURE LBLRTM_File_NotEqual
  END INTERFACE OPERATOR(/=)\end{alltt}

For a complete list of the definition and I/O module procedures for use with the available objects, see appendix \ref{app:object_and_interface_definition}.



\chapter{How to obtain the CRTM}
%===============================

\section{CRTM ftp download site}
%===============================
The CRTM source code and coefficients are released as compressed tarballs\footnote{A compressed (e.g. gzip'd) tape archive (tar) file.} via the CRTM ftp site:

\hspace{1cm}\href{ftp://ftp.emc.ncep.noaa.gov/jcsda/CRTM}{\texttt{ftp://ftp.emc.ncep.noaa.gov/jcsda/CRTM/}}

The REL-2.0.2 release is available directly from

\hspace{1cm}\href{ftp://ftp.emc.ncep.noaa.gov/jcsda/CRTM/REL-2.0.2}{\texttt{ftp://ftp.emc.ncep.noaa.gov/jcsda/CRTM/REL-2.0.2}}

Also note that additional releases, e.g. beta or experimental branches, are also made available on this ftp site.


\section{Coefficient Data}
%=========================
The tarball labeled as ``\texttt{Coeffs}'' packages up all the transmittance, spectral, cloud, aerosol, and emissivity coefficient data needed by the CRTM. The coefficient tarball directory structure is organised by coefficient and format type as shown in figure \ref{fig:crtm_coefficients_dir}.

\begin{figure}[htb]
  \centering
  \input{graphics/Get/CRTM_Coefficients_dir.pstex_t}
  \caption{The CRTM coefficients tarball structure}
  \label{fig:crtm_coefficients_dir}
\end{figure}

Both big- and little-endian format files are provided to save users the trouble of switching what they use for their system\footnote{All of the supplied configurations for little-endian platforms described in Section \ref{sec:build} use compiler switches to default to big-endian format.}. Note in the TauCoeff directory there are two subdirectories: ODAS and ODPS. These directories correspond to the coefficient files for the different transmittance model algorithms. The user can select which algorithm to use by using the corresponding TauCoeff file.

To run the CRTM, all the required coefficient files need to be in the same path (see the  \hyperref[sec:CRTM_Init_interface]{CRTM initialisation function} description) so users will have to move/link the datafiles as required.

\chapter{How to build the CRTM library}
%======================================
\label{sec:build}

\section{Build Files}
%====================
The build system for the CRTM is currently quite unsophisticated. It consists of a number of make, include, and configuration files in the CRTM tarball hierarchy:

\begin{tabular}{l@{ : }p{4.75in}}
  \,\texttt{makefile} & The main makefile\\
  \,\texttt{make.macros} & The include file containing the defined macros.\\
  \,\texttt{make.rules} & The include file containing the suffix rules for compiling Fortran95/2003 source code.\\
  \,\texttt{configure} & The directory containing build environment definitions.\\
\end{tabular}

\section{Predefined Configuration Files}
%=======================================
The build makefiles now assumes that environment variables (envars) will be defined that describe the compilation and link environment. The envars that \emph{must} be defined are:

\begin{tabular}{l@{ : }p{4.75in}}
  \,\texttt{FC}        & the Fortran95/2003 compiler executable,\\
  \,\texttt{FC\_FLAGS} & the flags/switches provided to the Fortran compiler,\\
  \,\texttt{FL}        & the linker used to create the executable test/example programs, and\\
  \,\texttt{FL\_FLAGS} & the flags/switches provided to the linker.\\
\end{tabular}
  
Several shell source files are provided for the build environment definitions for the compilers to which we have access and have tested here at the JCSDA. These shell source files are in the \f{configure} subdirectory of the tarball. The configuration files provided are shown in table \ref{tab:supplied_configurations}. Both ``production'' and debug configurations are supplied, with the former using compiler switches to produce fast code and the latter using compiler switches to turn on all the available debugging capabilities. Note that the debug configurations will produce executables much slower than the production builds. 
\begin{table}[htp]
  \centering
  \begin{tabular}{clr@{.}lr@{.}l}
    \hline
    \sffamily\textbf{Platform} & \sffamily\textbf{Compiler} & \multicolumn{2}{c}{\sffamily\textbf{Production}} & \multicolumn{2}{c}{\sffamily\textbf{Debug}} \\
    \hline\hline
    \multirow{4}{*}{Linux} & GNU gfortran          & \texttt{gfortran}&\texttt{setup} & \texttt{gfortran\_debug}&\texttt{setup}\\
                           & Intel ifort           & \texttt{intel}&\texttt{setup}    & \texttt{intel\_debug}&\texttt{setup}   \\
                           & PGI pgf95             & \texttt{pgi}&\texttt{setup}      & \texttt{pgi\_debug}&\texttt{setup}     \\
                           & g95                   & \texttt{g95}&\texttt{setup}      & \texttt{g95\_debug}&\texttt{setup}     \\[0.2cm]
    IBM                    & AIX xlf95             & \texttt{xlf}&\texttt{setup}      & \texttt{xlf\_debug}&\texttt{setup}     \\
  \hline
  \end{tabular}
  \caption{Supplied configuration files for the CRTM library and test/example program build.}
  \label{tab:supplied_configurations}
\end{table}


\section{Compilation Environment Setup}
%======================================
To set the compilation envars for your CRTM build, you need to source the required ``setup'' file. For example, to use gfortran to build the CRTM you would type

\begin{verbatim}     . configure/gfortran.setup\end{verbatim}

in the main directory. Note the ``.'' and space preceding the filename. This should print out something like the following:
\begin{alltt}
  =========================================
   CRTM compilation environment variables:
     FC:       gfortran
     FC_FLAGS:  -c  -O3  -fconvert=big-endian  -ffast-math  -ffree-form
                -fno-second-underscore  -frecord-marker=4  -funroll-loops
                -ggdb  -static  -Wall 
     FL:       gfortran
     FL_FLAGS: 
  =========================================\end{alltt}

indicating the values to which the envars have been set.

Change the supplied setups to suit your needs. If you use a different compiler please consider submitting your compilation setup to be included in future releases.

Note that as of CRTM v2.0, the Fortran compiler needs to be compatible with the ISO TR-15581 Allocatable Enhancements update to Fortran95. Most current Fortran95 compilers do support TR-15581.


\section{Building the library}
%=============================
Once the compilation environment has been set, the CRTM library build is performed by simply typing,

\begin{verbatim}     make\end{verbatim}
   
If you are using the DEBUG compiler flags you may, unfortunately, see many warnings similar to:
\begin{alltt}
  Warning (137): Variable 'cosaz' at (1) is never used and never set
  Warning (112): Variable 'rlongitude' at (1) is set but never used
  Warning (140): Implicit conversion at (1) may cause precision loss
  Warning: Unused dummy argument 'group_index' at (1)   
  PGF90-I-0035-Predefined intrinsic scale loses intrinsic property
  etc..\end{alltt}

The actual format of the warning message depends on the compiler. We are working on eliminating these warning messages.


\section{Testing the library}
%============================
Several test/example programs exercising the forward and K-matrix functions have been supplied with the CRTM. To build and run all these tests, type,

\begin{verbatim}     make test\end{verbatim}
   
This process does generate a lot of output to screen so be prepared to scroll through it. Currently there are five forward model test, or example, programs:
\begin{alltt}
  test/forward/Example1_Simple
  test/forward/Example2_SSU
  test/forward/Example3_Zeeman
  test/forward/Example4_ODPS
  test/forward/Example5_ClearSky\end{alltt}
And there are four cases for the K-matrix model:
\begin{alltt}
  test/k_matrix/Example1_Simple
  test/k_matrix/Example2_SSU
  test/k_matrix/Example3_Zeeman
  test/k_matrix/Example4_ODPS\end{alltt}

Both the forward and K-matrix tests \emph{should} end with output that looks like:

\begin{alltt}
  ======================
  SUMMARY OF ALL RESULTS
  ======================

  Passed 14 of 14 tests.
  Failed 0 of 14 tests.\end{alltt}

Currently they both have the same number of tests. If you encounter failures you might see something like:

\begin{alltt}
  ======================
  SUMMARY OF ALL RESULTS
  ======================

  Passed 10 of 14 tests.
  Failed 4 of 14 tests.  <----<<<  **WARNING**\end{alltt}

Some important things to note about the tests:
\begin{itemize}
  \item The supplied results were generated using the gfortran DEBUG build.
  \item Comparisons between DEBUG and PRODUCTION builds can be different due to various compiler switches that modify floating point arithmetic (e.g. optimisation levels), or different hardware.
  \item For test failures, you can view the differences between the generated and supplied ASCII output files. For example, to view the K-matrix \f{Example1\_Simple} test case differences for the \f{amsua\_metop-a} sensor you would do something like:
\begin{alltt}
  $ cd test/k_matrix/Example1_Simple
  $ diff -u amsua_metop-a.output results/amsua_metop-a.output | more\end{alltt}

where the \f{amsua\_metop-a.output} file is generated during the test run, and the \f{results/amsua\_metop-a.output} file is supplied with the CRTM tarball.
\end{itemize}


\section{Installing the library}
%===============================
A very simple install target is specified in the supplied makefile to put all the necessary include files (the generated \texttt{*.mod} files containing all the procedure interface information) in an \texttt{/include} subdirectory and the library itself (the generated \texttt{libCRTM.a} file) in a \texttt{/lib} subdirectory. The make command is
\begin{verbatim}     make install\end{verbatim}
The \texttt{/include} and \texttt{/lib} subdirectories can then be copied/moved/linked to a more suitable location on your system, for example: \texttt{\$HOME/local/CRTM}

NOTE: Currently, running the tests also invokes this install target. That will change in future tarball releases so do not rely on the behaviour.


\section{Clean Up}
%=================
Two cleanup targets are provided in the makefile:

\begin{verbatim}     make clean\end{verbatim}
  
Removes all the compilation and link products from the \f{libsrc/} directory.

\begin{verbatim}     make distclean\end{verbatim} 
  
This does the same as the ``clean'' target but also deletes the library and include directories created by the ``install'' target.


\section{Linking to the library}
%===============================
Let's assume you've built the CRTM library and placed the \texttt{/include} and \texttt{/lib} subdirectories in your own local area, \texttt{\$HOME/local/CRTM}. In the makefile for your application that uses the CRTM, you will need to add
\begin{verbatim}     -I$HOME/local/CRTM/include\end{verbatim}
to your list of compilation switches, and the following to your list of link switches,
\begin{verbatim}     -L$HOME/local/CRTM/lib -lCRTM\end{verbatim}

\chapter{How to use the CRTM library}
%====================================
\section{Step by Step Guide}
%===========================
This section will hopefully get you started using the CRTM library as quickly as possible. Refer to the following sections for more information about the structures and interfaces.

Nearly all the examples shown here assume you are processing one sensor at a time. The CRTM can handle multiple sensors at once, but specifying the input information in a simple way is difficult; e.g. the GeometryInfo structure that is used to specify the sensor viewing geometry -- even sensors on the same platform typically have different numbers of fields-of-view (FOVs) per scan. For multiple sensor processing, we'll assume they will be separately processed in parallel.

Because there are many variations in what information is known ahead of time (and by ``ahead of time'' we mean at compile-time of your code), let's approach this via examples for fixed and variable numbers of atmospheric profiles, and known and unknown sensors. Different scenarios will be described indicating how the various CRTM structures should be declared and allocated, how the CRTM should be initialised, and how to call the forward and K-matrix models (use of the tangent-linear and adjoint models will be left as an exercise for the reader.) The scenarios are:\vspace{-2ex} 
\begin{enumerate}
  \item{Fixed number of atmospheric profiles; single \textit{known} sensor}
  \item{Variable number of atmospheric profiles; single \textit{known} sensor}
  \item{Fixed number of atmospheric profiles; single \textit{unknown} sensor}
  \item{Variable number of atmospheric profiles; multiple \textit{unknown} sensors}
\end{enumerate}
With regards to sensor identification, the CRTM uses a character string -- refered to as the \f{Sensor\_Id} -- to distinguish sensors and platforms. The lists of currently supported sensors, along with their associated \f{Sensor\_Id}'s, are shown in appendix \ref{sec:sensor_id}.


\newcounter{step}
\newcounter{example}[subsection]

\subsection{Step 1: Access the CRTM module}
%------------------------------------------
\stepcounter{step}
All of the CRTM user procedures, parameters, and derived data type definitions are accessible via the container module \f{CRTM\_Module}. Thus, one needs to put the following statement in any calling program, module or procedure,

\qquad\f{USE CRTM\_Module}

Once you become familiar with the components of the CRTM you require, you can also specify an \f{ONLY} clause with the \f{USE} statement,

\qquad\f{USE CRTM\_Module}[\f{, ONLY:}\textit{only-list}]

where \textit{only-list} is a list of the symbols you want to ``import'' from \f{CRTM\_Module}. This latter form is the preferred style for self-documenting your code; e.g. when you give the code to someone else, they will be able to identify from which module various symbols in your code originate.

\subsection{Step 2: Declare the CRTM structures}
%-----------------------------------------------
\refstepcounter{step}
\label{sec:declare_step}
To compute satellite radiances you need to declare structures for the following information,\vspace{-2ex}
\begin{enumerate}
  \item Atmospheric profile data such as pressure, temperature, absorber amounts, clouds, aerosols, etc. Handled using the \Atmosphere{} structure.
  \item Surface data such as type of surface, temperature, surface type specific parameters etc. Handled using the \Surface{} structure.
  \item Geometry information such as sensor scan angle, zenith angle, etc. Handled using the \GeometryInfo{} structure.
  \item Instrument information, particularly which instrument(s), or sensor(s)\footnote{The terms ``instrument'' and ``sensor'' are used interchangeably in this document.}, you want to simulate. Handled using the \ChannelInfo{} structure.
  \item Results of the radiative transfer calculation. Handled using the \RTSolution{} structure.
  \item Optional inputs. Handled using the \Options{} structure.
\end{enumerate}

\subsubsection{Example 1: Fixed number of profiles; single known sensor}
%.......................................................................
\refstepcounter{example}
\label{sec:declare_ex_fpsks}
Here let's assume you want to process, say, 50 profiles for the NOAA-18 AMSU-A sensor. That's a total of 15 channels. 
\begin{alltt}
  ! Processing parameters
  INTEGER, PARAMETER :: N_SENSORS  =  1
  INTEGER, PARAMETER :: N_CHANNELS = 15
  INTEGER, PARAMETER :: N_PROFILES = 50
  CHARACTER(*), PARAMETER :: SID(N_SENSORS) = (/'amsua_n18'/)
  ! Declarations
  TYPE(CRTM_Atmosphere_type)   :: atm(N_PROFILES)
  TYPE(CRTM_Surface_type)      :: sfc(N_PROFILES)
  TYPE(CRTM_GeometryInfo_type) :: gInfo(N_PROFILES)  
  TYPE(CRTM_ChannelInfo_type)  :: chInfo(N_SENSORS)  
  TYPE(CRTM_RTSolution_type)   :: rts(N_CHANNELS,N_PROFILES)\end{alltt}

The point we want to make here is that if you know how many channels you will be processing ahead of time, you can explicitly declare the sizes of the various structure arrays.

\subsubsection{Example 2: Variable number of profiles; single known sensor}
%..........................................................................
\refstepcounter{example}
\label{sec:declare_ex_vpsks}
Say you want to process the same instrument as the previous example, but you don't know ahead of time how many profiles you will be processing. In this case you will need to declare any profile-dependent arguments with the \f{ALLOCATABLE} attribute,
\begin{alltt}
  ! Processing parameters
  INTEGER, PARAMETER :: N_SENSORS  =  1
  INTEGER, PARAMETER :: N_CHANNELS = 15
  CHARACTER(*), PARAMETER :: SID(N_SENSORS) = (/'amsua_n18'/)
  ! Declarations
  TYPE(CRTM_Atmosphere_type)  , ALLOCATABLE :: atm(:)
  TYPE(CRTM_Surface_type)     , ALLOCATABLE :: sfc(:)
  TYPE(CRTM_GeometryInfo_type), ALLOCATABLE :: gInfo(:)  
  TYPE(CRTM_ChannelInfo_type)               :: chInfo(N_SENSORS)
  TYPE(CRTM_RTSolution_type)  , ALLOCATABLE :: rts(:,:)\end{alltt}


\subsubsection{Example 3: Fixed number of profiles; single unknown sensor}
%.........................................................................
\refstepcounter{example}
\label{sec:declare_ex_fpsus}
Let's assume you'll be serially processing multiple sensors for a fixed number of profiles. What this means is that you don't necessarily know the number of channels since they are typically different for different sensors. This means that the RTSolution argument, \f{rts}, has an unknown dimension (the number of channels) and must be declared with the \f{ALLOCATABLE} attribute,
\begin{alltt}
  ! Processing parameters
  INTEGER, PARAMETER :: N_SENSORS  =  1
  INTEGER, PARAMETER :: N_PROFILES = 50
  ! Declarations
  CHARACTER(STRLEN)                       :: sId(N_SENSORS)
  TYPE(CRTM_Atmosphere_type)              :: atm(N_PROFILES)
  TYPE(CRTM_Surface_type)                 :: sfc(N_PROFILES)
  TYPE(CRTM_GeometryInfo_type)            :: gInfo(N_PROFILES)
  TYPE(CRTM_ChannelInfo_type)             :: chInfo(N_SENSORS)
  TYPE(CRTM_RTSolution_type), ALLOCATABLE :: rts(:,:)\end{alltt}

Note the use of the parameter \f{STRLEN} in the declaration of the Sensor Id argument, \f{sId}. This character length definition is inherited from the \f{CRTM\_Module} and is used throughout the CRTM as the length of the Sensor Id string. Currently it's value is 20, but that could change in the future; suffice it to say you should use the \f{STRLEN} parameter for Sensor Id character string declarations. 

\subsubsection{Example 4: Variable number of profiles; multiple unknown sensors}
%...............................................................................
\refstepcounter{example}
\label{sec:declare_ex_vpmus}
This is a bit of a ``cover all your bases'' example, where we'll assume you'll be processing a variable number of profiles for an unknown number of sensors \textit{in parallel}\footnote{It doesn't make sense to process multiple sensors simultaneously in the same CRTM function call since each sensor will have more than likely have different viewing geometries even if their fields-of-view (FOVs) correspond geographically.}. This simply means everything is declared with the \f{ALLOCATABLE} attribute,
\begin{ttfamily}
  \begin{alltt}
  ! Declarations
  CHARACTER(STRLEN)           , ALLOCATABLE :: sId(:)
  TYPE(CRTM_Atmosphere_type)  , ALLOCATABLE :: atm(:)
  TYPE(CRTM_Surface_type)     , ALLOCATABLE :: sfc(:)
  TYPE(CRTM_GeometryInfo_type), ALLOCATABLE :: gInfo(:)  
  TYPE(CRTM_ChannelInfo_type) , ALLOCATABLE :: chInfo(:)
  TYPE(CRTM_RTSolution_type)  , ALLOCATABLE :: rts(:,:)\end{alltt}
\end{ttfamily}




\subsection{Step 3: Initialise the CRTM}
%---------------------------------------
\stepcounter{step}
\label{sec:init_step}
The CRTM is initialised by calling the \f{CRTM\_Init()} function in the \f{CRTM\_LifeCycle.f90} module. This loads all the various coefficient data used by CRTM components into memory for later use. We'll assume that all the required datafiles reside in the subdirectory \f{./coeff\_data} and follow on from the examples of Step \ref{sec:declare_step}. The CRTM initialisation is profile independent, so we're only dealing with sensor variability here.

\subsubsection{Example 1: Fixed number of sensors}
%.................................................
\refstepcounter{example}
\label{sec:init_ex_fns}
Here you're using the ChannelInfo and Sensor Id arrays as declared in examples \ref{sec:declare_ex_fpsks} through \ref{sec:declare_ex_fpsus} of Step \ref{sec:declare_step},
\begin{alltt}
  CHARACTER(*), PARAMETER :: PROGRAM_NAME='My Program Name'
  INTEGER :: errStatus
  ....
  errStatus = CRTM_Init( chInfo, Sensor_Id=SID, File_Path='./coeff_data' )
  IF ( errStatus /= SUCCESS ) THEN 
    CALL Display_Message( PROGRAM_NAME,'Error initializing CRTM',errStatus )
    STOP
  END IF\end{alltt}

Here we see for the first time how the CRTM functions let you know if they were successful. As you can see the \f{CRTM\_Init()} function result is an error status that is checked against a parameterised error code, \f{SUCCESS}. Other available error code parameters are \f{FAILURE}, \f{WARNING}, and \f{INFORMATION} -- although the latter is never used as a function result. Note that you don't \textit{have} to call the error message handler subroutine \f{Display\_Message}, it just outputs messages in a standard way (see appendix \ref{sec:utility_message_handler}) and is used throughout the CRTM code.

\subsubsection{Example 2: Variable number of sensors}
%....................................................
\refstepcounter{example}
\label{sec:init_ex_vns}
This example follows on from example \ref{sec:declare_ex_vpmus} of Step \ref{sec:declare_step}. For this sort of usage you will need to allocate the ChannelInfo and Sensor Id arguments to the number of sensors you wish to process as well as fill it. For illustrative purposes only, let's assume your calling program requires you to simply type in the number and list of sensors,
\begin{alltt}
  CHARACTER(256) :: Message
  INTEGER :: n, n_Sensors
  ....
  ! Allocate the channelInfo and Sensor Id array
  READ( *,'(/2x,"Enter the number of sensors to process: ")', ADVANCE="NO" ) n_Sensors
  ALLOCATE( chInfo(n_Sensors), sId(n_Sensors), STAT=errStatus )
  IF ( errStatus /= 0 ) THEN
    WRITE( Message,'("Error allocating the chInfo and sId arrays. STAT=",i0)' ) errStatus
    CALL Display_Message( PROGRAM_NAME,TRIM(Message),FAILURE )
    STOP
  END IF
  
  ! Get user sensor id input
  DO n = 1, n_Sensors
    READ( *,'(4x,"Enter sensor id #",i0,": ")', ADVANCE="NO" ) sId(n)
  END DO
  
  ! Initialise the CRTM for the dynamic sensor list
  errStatus = CRTM_Init( chInfo, Sensor_Id=sId, File_Path='./coeff_data' )
  IF ( errStatus /= SUCCESS ) THEN 
    CALL Display_Message( PROGRAM_NAME,'Error initializing CRTM',errStatus )
    STOP
  END IF\end{alltt}

This is, of course, a rather contrived example since it is guaranteed that typing in sensor identifiers by hand will get quite annoying very quickly.

One thing to note in the above example is the use of the error status variable \f{errStatus} for both standard errors (e.g. the allocation \f{STAT} result) and CRTM errors. In the former case, \f{errStatus} doesn't have any meaning for the \f{Display\_Message} subroutine\footnote{Indeed, the actual values of standard error codes can vary widely not only with the cause of the error, but also with compiler, operating system, and combinations thereof.}, hence the use of the actual fail parameter \f{FAILURE} in the call.


\subsection{Step 4: Allocate the CRTM structure arrays}
%------------------------------------------------------
\stepcounter{step}
\label{sec:alloc_arr_step}
If you've declared everything for a fixed number of profiles and sensors/channels (e.g. as shown in example \ref{sec:declare_ex_fpsks} of Step \ref{sec:declare_step}), then you can skip to the next step. Here we will allocate the various CRTM structure \textit{arrays} (not the structures themselves -- the distinction is important) to the required size.

\subsubsection{Example 1: Variable number of profiles; single known sensor}
%..........................................................................
\refstepcounter{example}
\label{sec:alloc_arr_ex_vpsks}
The following allocations apply to the declarations shown in example \ref{sec:declare_ex_vpsks} of step \ref{sec:declare_step}, where the \f{N\_CHANNELS} parameter is declared. For simplicity, we'll assume the number of profiles to process is entered by a user, but it would more likely be read from a file or passed as an argument, etc.
\begin{alltt}
  INTEGER :: n_Profiles
  ....
  ! Determine the number of profiles to process
  READ( *,'(/2x,"Enter the number of profiles to process: ")', ADVANCE="NO" ) n_Profiles
  
  ! Allocate the structure arrays
  ALLOCATE( atm(n_Profiles), &
            sfc(n_Profiles), &
            gInfo(n_Profiles), &
            rts(N_CHANNELS, n_Profiles), &
            STAT=errStatus )
  IF ( errStatus /= 0 ) THEN 
    WRITE( Message,'("Error allocating the structure arrays. STAT=",i0)' ) errStatus
    CALL Display_Message( PROGRAM_NAME,TRIM(Message),FAILURE )
    STOP
  END IF\end{alltt}


\subsubsection{Example 2: Fixed number of profiles; single unknown sensor}
%.........................................................................
\refstepcounter{example}
\label{sec:alloc_arr_ex_fpsus}
Here the only unknown is the number of channels since the sensor is ``unknown'' at compile time, as shown in example \ref{sec:declare_ex_fpsus} of Step \ref{sec:declare_step}. Thus only the RTSolution argument needs allocating, but using the number of channels determined from the initialisation step,
\begin{alltt}
  INTEGER :: n_Profiles
  ....
  ! Allocate the structure arrays
  ALLOCATE( \textcolor{blue}{rts(chInfo(1)%n_Channels}, n_Profiles), &
            STAT=errStatus )
  IF ( errStatus /= 0 ) THEN 
    WRITE( Message,'("Error allocating the structure arrays. STAT=",i0)' ) errStatus
    CALL Display_Message( PROGRAM_NAME, &
                          TRIM(Message), & 
                          FAILURE )
    STOP
  END IF\end{alltt}

The ChannelInfo structure contains the number of sensor channels (see appendix \ref{sec:channelinfo_structure} for its full definition). The only thing to point out here is that even though we are only processing a single sensor, we still have to indicate the \f{chInfo} element for that sensor, \f{chInfo(1)}, when accessing the number of channel component.


\subsubsection{Example 3: Variable number of profiles; multiple unknown sensors}
%...............................................................................
\refstepcounter{example}
\label{sec:alloc_arr_ex_vpmus}
Here everything is unknown. Note how the total number of channels is obtained in the example source code,
\begin{alltt}
  INTEGER :: n_Channels
  ....
  ! Determine the total number of channels for all the sensors
  \textcolor{blue}{n_Channels = SUM(chInfo%n_Channels)}
  
  ! Determine the number of profiles to process
  READ( *,'(/2x,"Enter the number of profiles to process: ")', ADVANCE="NO" ) n_Profiles
  
  ! Allocate the structure arrays
  ALLOCATE( atm(n_Profiles), &
            sfc(n_Profiles), &
            gInfo(n_Profiles), &
            rts(n_Channels, n_Profiles), &
            STAT=errStatus )
  IF ( errStatus /= 0 ) THEN 
    WRITE( Message,'("Error allocating the structure arrays. STAT=",i0)' ) errStatus
    CALL Display_Message( PROGRAM_NAME, &
                          TRIM(Message), & 
                          FAILURE )
    STOP
  END IF\end{alltt}


\subsection{Step 5: Allocate the CRTM structures}
%------------------------------------------------
\stepcounter{step}
The previous step involved allocating the structure arrays for the required number of profiles, channels, and/or sensors. This step involves the allocation of the \emph{internal} structure components where necessary to hold the input or output data. In this case, functions are used to perform these ``internal'' allocations. The function naming convention is \f{CRTM\_Allocate\_}\textit{name} where, for typical usage, the CRTM structures that need to be allocated are the Atmosphere, RTSolution and, if used, Options structures. 

\subsubsection{Example 1: Variable number of profiles; single known sensor}
%..........................................................................
\refstepcounter{example}
The following structure allocations follow the array allocations of example \ref{sec:alloc_arr_ex_vpsks} in step \ref{sec:alloc_arr_step}, where the \f{N\_CHANNELS} parameter is declared and the number of profiles is user-specified. The following assumes you'll want to allocate all the structures. First, we'll allocate the atmosphere structure to the required dimensions, like so:
\begin{alltt}
  ! Allocate the atmosphere structure
  errStatus = CRTM_Allocate_Atmosphere( n_Layers   , &  ! Input
                                        N_ABSORBERS, &  ! Input (always 2)
                                        n_Clouds   , &  ! Input
                                        n_Aerosols , &  ! Input
                                        atm          )  ! Output
  IF ( errStatus /= SUCCESS ) THEN 
    CALL Display_Message( PROGRAM_NAME, &
                          'Error allocating the atmosphere structure', & 
                          FAILURE )
    STOP
  END IF\end{alltt}
Note that the number of absorbers is in all capitals. In the CRTM, this style convention indicates a parameter. A parameter is used for the number of absorbers in the current CRTM release because the number of absorbers is fixed at two: water vapour and ozone. Future CRTM releases will allow more flexibility in selecting the number of absorbers, but currently the number must be set to two.

If 










\subsection{Step 6: Fill the CRTM input structures with data}
%------------------------------------------------------------
\stepcounter{step}


\subsection{Step 7: Call the required CRTM function}
%---------------------------------------------------
\stepcounter{step}


\subsection{Step 8: Destroy the CRTM}
%------------------------------------
\stepcounter{step}



\section{Interface Descriptions}
%===============================

\subsection{CRTM Initialisation}
%-------------------------------

\begin{figure}[htp]
  \centering
  \doublebox{
  \begin{minipage}[b]{6.5in}
    \begin{alltt}
  FUNCTION CRTM_Init( ChannelInfo      , &
                      Sensor_ID        , &
                      CloudCoeff_File  , &
                      AerosolCoeff_File, &
                      EmisCoeff_File   , &
                      File_Path        , &
                      Quiet            , &
                      Process_ID       , &
                      Output_Process_ID, &
                      RCS_Id           , &
                      Message_Log      ) &
                    RESULT( Error_Status )
    ! Arguments
    TYPE(CRTM_ChannelInfo_type), INTENT(IN OUT) :: ChannelInfo(:)
    CHARACTER(*),      OPTIONAL, INTENT(IN)     :: Sensor_ID(:)
    CHARACTER(*),      OPTIONAL, INTENT(IN)     :: CloudCoeff_File
    CHARACTER(*),      OPTIONAL, INTENT(IN)     :: AerosolCoeff_File
    CHARACTER(*),      OPTIONAL, INTENT(IN)     :: EmisCoeff_File
    CHARACTER(*),      OPTIONAL, INTENT(IN)     :: File_Path
    INTEGER     ,      OPTIONAL, INTENT(IN)     :: Quiet
    INTEGER     ,      OPTIONAL, INTENT(IN)     :: Process_ID
    INTEGER     ,      OPTIONAL, INTENT(IN)     :: Output_Process_ID
    CHARACTER(*),      OPTIONAL, INTENT(OUT)    :: RCS_Id
    CHARACTER(*),      OPTIONAL, INTENT(IN)     :: Message_Log
    ! Function result
    INTEGER :: Error_Status
    \end{alltt}
    \centering
    \begin{tabular}{p{3.25cm} p{6.5cm} p{1.75cm} p{2.5cm}}
      \hline
      \tblhd{Argument}             & \tblhd{Description}                             & \tblhd{Rank} & \tblhd{Intent} \\
      \hline\hline
      \f{ChannelInfo}              & Sensor and channel info structure               & $N$          & Output \\
      \optarg{Sensor\_Id}          & \textit{Sensor identification string}           & $N$          & \textit{Input}  \\
      \optarg{CloudCoeff\_File}    & \textit{Cloud optical property LUT filename}    & Scalar       & \textit{Input}  \\
      \optarg{AerosolCoeff\_File}  & \textit{Aerosol optical property LUT filename}  & Scalar       & \textit{Input}  \\
      \optarg{EmisCoeff\_File}     & \textit{IR sea surface emissivity LUT filename} & Scalar       & \textit{Input}  \\
      \optarg{File\_Path}          & \textit{Path to *Coeff files}                   & Scalar       & \textit{Input}  \\
      \optarg{Quiet}               & \textit{Keyword to control info message output} & Scalar       & \textit{Input}  \\
      \optarg{Process\_Id}         & \textit{MPI process Id}                         & Scalar       & \textit{Input}  \\
      \optarg{Output\_Process\_Id} & \textit{MPI process Id for message output}      & Scalar       & \textit{Input}  \\
      \optarg{RCS\_Id}             & \textit{Version control ID for the module}      & Scalar       & \textit{Output} \\
      \optarg{Message\_Log}        & \textit{Log message filename}                   & Scalar       & \textit{Input} 
    \end{tabular}
  \end{minipage}
  }
  \caption{CRTM Initialisation interface and argument description.}
  \label{fig:init_interface}
\end{figure}

\subsection{Forward Model}
%-------------------------

\begin{figure}[htp]
  \centering
  \doublebox{
  \begin{minipage}[b]{6.5in}
    \begin{alltt}
  FUNCTION CRTM_Forward( Atmosphere  , &
                         Surface     , &
                         GeometryInfo, &
                         ChannelInfo , &
                         RTSolution  , &
                         Options     , &    
                         RCS_Id      , &
                         Message_Log ) &
                       RESULT( Error_Status )
    ! Arguments
    TYPE(CRTM_Atmosphere_type),        INTENT(IN)     :: Atmosphere(:)
    TYPE(CRTM_Surface_type),           INTENT(IN)     :: Surface(:)
    TYPE(CRTM_GeometryInfo_type),      INTENT(IN OUT) :: GeometryInfo(:)
    TYPE(CRTM_ChannelInfo_type),       INTENT(IN)     :: ChannelInfo(:)
    TYPE(CRTM_RTSolution_type),        INTENT(IN OUT) :: RTSolution(:,:)
    TYPE(CRTM_Options_type), OPTIONAL, INTENT(IN)     :: Options(:)
    CHARACTER(*),            OPTIONAL, INTENT(OUT)    :: RCS_Id
    CHARACTER(*),            OPTIONAL, INTENT(IN)     :: Message_Log
    ! Function result
    INTEGER :: Error_Status
    \end{alltt}
    \centering
    \begin{tabular}{p{3.25cm} p{6.5cm} p{1.75cm} p{2.5cm}}
      \hline
      \tblhd{Argument}       & \tblhd{Description}                           & \tblhd{Rank} & \tblhd{Intent} \\
      \hline\hline
      \f{Atmosphere}         & Atmospheric state                             & $M$          & Input  \\
      \f{Surface}            & Surface state                                 & $M$          & Input  \\
      \f{GeometryInfo}       & Geometry information (e.g. angles)            & $M$          & Input  \\
      \f{ChannelInfo}        & Sensor channel information                    & $N$          & Input  \\
      \f{RTSolution}         & Radiative transfer solution                   & $L \times M$ & Output \\
      \optarg{Options}       & \textit{Structure containing optional inputs} & $M$          & \textit{Input}  \\
      \optarg{RCS\_Id}       & \textit{Version control ID for the module}    & Scalar       & \textit{Output} \\
      \optarg{Message\_Log}  & \textit{Log message filename}                 & Scalar       & \textit{Input} 
    \end{tabular}
  \end{minipage}
  }
  \caption{CRTM Forward model interface and argument description.}
  \label{fig:fwd_interface}
\end{figure}


\subsection{K-Matrix Model}
%--------------------------

\begin{figure}[htp]
  \centering
  \doublebox{
  \begin{minipage}[b]{6.5in}
    \begin{alltt}
  FUNCTION CRTM_K_Matrix( Atmosphere  , &
                          Surface     , &
                          RTSolution_K, &
                          GeometryInfo, &
                          ChannelInfo , &
                          Atmosphere_K, &
                          Surface_K   , &
                          RTSolution  , &
                          Options     , &
                          RCS_Id      , &
                          Message_Log ) &
                        RESULT( Error_Status )
    ! Arguments
    TYPE(CRTM_Atmosphere_type)       , INTENT(IN)     :: Atmosphere(:)
    TYPE(CRTM_Surface_type)          , INTENT(IN)     :: Surface(:)
    TYPE(CRTM_RTSolution_type)       , INTENT(IN OUT) :: RTSolution_K(:,:)
    TYPE(CRTM_GeometryInfo_type)     , INTENT(IN OUT) :: GeometryInfo(:)
    TYPE(CRTM_ChannelInfo_type)      , INTENT(IN)     :: ChannelInfo(:)
    TYPE(CRTM_Atmosphere_type)       , INTENT(IN OUT) :: Atmosphere_K(:,:)
    TYPE(CRTM_Surface_type)          , INTENT(IN OUT) :: Surface_K(:,:)
    TYPE(CRTM_RTSolution_type)       , INTENT(IN OUT) :: RTSolution(:,:)
    TYPE(CRTM_Options_type), OPTIONAL, INTENT(IN)     :: Options(:)
    CHARACTER(*),            OPTIONAL, INTENT(OUT)    :: RCS_Id
    CHARACTER(*),            OPTIONAL, INTENT(IN)     :: Message_Log
    ! Function result
    INTEGER :: Error_Status
    \end{alltt}
    \centering
    \begin{tabular}{p{3.25cm} p{6.5cm} p{1.75cm} p{2.5cm}}
      \hline
      \tblhd{Argument}      & \tblhd{Description}                           & \tblhd{Rank} & \tblhd{Intent} \\
      \hline\hline
      \f{Atmosphere}        & Atmospheric state                             & $M$          & FWD Input  \\
      \f{Surface}           & Surface state                                 & $M$          & FWD Input  \\
      \f{RTSolution\_K}     & Adjoint radiative transfer solution           & $L \times M$ & K   Input  \\
      \f{GeometryInfo}      & Geometry information (e.g. angles)            & $M$          & Input      \\
      \f{ChannelInfo}       & Sensor channel information                    & $N$          & Input      \\
      \f{Atmosphere\_K}     & Atmospheric state Jacobians                   & $L \times M$ & K   Output \\
      \f{Surface\_K}        & Surface state Jacobians                       & $L \times M$ & K   Output \\
      \f{RTSolution}        & Radiative transfer solution                   & $L \times M$ & FWD Output \\
      \optarg{Options}      & \textit{Structure containing optional inputs} & $M$          & \textit{Input}  \\
      \optarg{RCS\_Id}      & \textit{Version control ID for the module}    & Scalar       & \textit{Output} \\
      \optarg{Message\_Log} & \textit{Log message filename}                 & Scalar       & \textit{Input} 
    \end{tabular}
  \end{minipage}
  }
  \caption{CRTM K-Matrix model interface and argument description.}
  \label{fig:k_interface}
\end{figure}


\subsection{CRTM Destruction}
%----------------------------

\begin{figure}[htp]
  \centering
  \doublebox{
  \begin{minipage}[b]{16.5cm}
    \begin{alltt}
  FUNCTION CRTM_Destroy( ChannelInfo , &  ! Output
                         Process_ID  , &  ! Optional input
                         RCS_Id      , &  ! Revision control
                         Message_Log ) &  ! Error messaging
                       RESULT ( Error_Status )
    ! Arguments
    TYPE(CRTM_ChannelInfo_type), INTENT(IN OUT) :: ChannelInfo(:)
    INTEGER     ,      OPTIONAL, INTENT(IN)     :: Process_ID
    CHARACTER(*),      OPTIONAL, INTENT(OUT)    :: RCS_Id
    CHARACTER(*),      OPTIONAL, INTENT(IN)     :: Message_Log
    ! Function result
    INTEGER :: Error_Status
    \end{alltt}
    \centering
    \begin{tabular}{p{3.25cm} p{6.5cm} p{1.75cm} p{2.5cm}}
      \hline
      \tblhd{Argument}      & \tblhd{Description}                           & \tblhd{Rank} & \tblhd{Intent} \\
      \hline\hline
      \f{ChannelInfo}       & Sensor and channel info structure             & $N$          & In/Output       \\
      \optarg{Process\_Id}  & \textit{MPI process Id}                       & Scalar       & \textit{Input}  \\
      \optarg{RCS\_Id}      & \textit{Version control ID for the module}    & Scalar       & \textit{Output} \\
      \optarg{Message\_Log} & \textit{Log message filename}                 & Scalar       & \textit{Input} 
    \end{tabular}
  \end{minipage}
  }
  \caption{CRTM Destruction interface and argument description.}
  \label{fig:destroy_interface}
\end{figure}


\section{Filling input data structures}
%======================================

\begin{figure}[htp]
  \centering
  \input{graphics/gInfo/sensor_scan_angle.pstex_t}
  \caption{Definition of \GeometryInfo{} sensor scan angle component.}
  \label{fig:gInfo_sensor_scan_angle}
\end{figure}

\begin{figure}[htp]
  \centering
  \input{graphics/gInfo/sensor_zenith_angle.pstex_t}
  \caption{Definition of \GeometryInfo{} sensor zenith angle component.}
  \label{fig:gInfo_sensor_zenith_angle}
\end{figure}

\begin{figure}[htp]
  \centering
  \input{graphics/gInfo/sensor_azimuth_angle.pstex_t}
  \caption{Definition of \GeometryInfo{} sensor azimuth angle component.}
  \label{fig:gInfo_sensor_azimuth_angle}
\end{figure}

\begin{figure}[htp]
  \centering
  \input{graphics/gInfo/source_zenith_angle.pstex_t}
  \caption{Definition of \GeometryInfo{} source zenith angle component.}
  \label{fig:gInfo_source_zenith_angle}
\end{figure}

\begin{figure}[htp]
  \centering
  \input{graphics/gInfo/source_azimuth_angle.pstex_t}
  \caption{Definition of \GeometryInfo{} source azimuth angle component.}
  \label{fig:gInfo_source_azimuth_angle}
\end{figure}



% The references section
%=======================
\clearpage
\phantomsection
\addcontentsline{toc}{chapter}{Bibliography}
\bibliographystyle{plainnat}
\bibliography{bibliography}


% The appendices
%===============
\begin{appendix}
  \chapter{Structure definitions}
%==============================

\clearpage
\section{\ChannelInfo{} Structure}
%=================================
\label{sec:channelinfo_structure}

\begin{figure}[htp]
  \centering
  \doublebox{
  \begin{minipage}[b]{6.5in}
    \begin{ttfamily}
      \begin{verbatim}
  TYPE :: CRTM_ChannelInfo_type
    ! Dimension values
    INTEGER :: n_Channels = 0  ! L dimension
    ! Scalar data
    CHARACTER(STRLEN) :: Sensor_ID        = ' '
    INTEGER           :: WMO_Satellite_ID = INVALID_WMO_SATELLITE_ID
    INTEGER           :: WMO_Sensor_ID    = INVALID_WMO_SENSOR_ID
    INTEGER           :: Sensor_Index     = 0
    ! Array data
    INTEGER,  POINTER :: Sensor_Channel(:) => NULL()  ! L
    INTEGER,  POINTER :: Channel_Index(:)  => NULL()  ! L
  END TYPE CRTM_ChannelInfo_type\end{verbatim}
    \end{ttfamily}
  \end{minipage}
  }
  \caption{CRTM \ChannelInfo{} structure definition.}
  \label{fig:channelinfo_structure}
\end{figure}

% ChannelInfo component description table
\begin{table}[htp]
  \centering
  \begin{tabular}{l p{8cm} c c}
    \hline
    \sffamily\textbf{Component} & \sffamily\textbf{Description} & \sffamily\textbf{Units} & \sffamily\textbf{Dimensions} \\
    \hline\hline
    \texttt{n\_Channels}  & Number of sensor channels (\texttt{L}) & N/A & Scalar \\
    \texttt{Sensor\_Id} & The sensor and platform identifier character string & N/A & Scalar \\
    \texttt{WMO\_Satellite\_Id} & The WMO satellite identifier & N/A & Scalar \\
    \texttt{WMO\_Sensor\_Id} & The WMO sensor identifier & N/A & Scalar \\
    \texttt{Sensor\_Index} & The index of the current structure in an array. Set during CRTM initialisation & N/A & Scalar \\
    \texttt{Sensor\_Channel} & The channel numbers of the current sensor & N/A & \texttt{L}\\
    \texttt{Channel\_Index} & The indices of the selected channels for the current sensors. Currently, all channels are indexed. & N/A & \texttt{L}\\
    \hline
  \end{tabular}
  \caption{CRTM \ChannelInfo{} structure component description.}
  \label{tab:chanelinfo_structure}
\end{table}

\clearpage
\section{\Atmosphere{} Structure}
%================================
\label{sec:atmosphere_structure}

\begin{figure}[htp]
  \centering
  \doublebox{
  \begin{minipage}[b]{6.5in}
    \begin{ttfamily}
      \begin{verbatim}
  TYPE :: CRTM_Atmosphere_type
    ! Dimension values
    INTEGER :: n_Layers     = 0  ! K dimension
    INTEGER :: n_Absorbers  = 0  ! J dimension
    INTEGER :: n_Clouds     = 0  ! Nc dimension
    INTEGER :: n_Aerosols   = 0  ! Na dimension
    ! Number of added layers
    INTEGER :: n_Added_Layers = 0
    ! Climatology model associated with the profile
    INTEGER :: Climatology = INVALID_MODEL
    ! Absorber ID and units
    INTEGER, POINTER :: Absorber_ID(:)    => NULL() ! J
    INTEGER, POINTER :: Absorber_Units(:) => NULL() ! J
    ! Profile LEVEL and LAYER quantities
    REAL(fp), POINTER :: Level_Pressure(:) => NULL()  ! 0:K
    REAL(fp), POINTER :: Pressure(:)       => NULL()  ! K
    REAL(fp), POINTER :: Temperature(:)    => NULL()  ! K
    REAL(fp), POINTER :: Absorber(:,:)     => NULL()  ! K x J
    ! Clouds associated with each profile
    TYPE(CRTM_Cloud_type),   POINTER :: Cloud(:)   => NULL()  ! Nc
    ! Aerosols associated with each profile
    TYPE(CRTM_Aerosol_type), POINTER :: Aerosol(:) => NULL()  ! Na
  END TYPE CRTM_Atmosphere_type\end{verbatim}
    \end{ttfamily}
  \end{minipage}
  }
  \caption{CRTM \Atmosphere{} structure definition.}
  \label{fig:atmosphere_structure}
\end{figure}

% Atmosphere component description table
\begin{table}[htp]
  \centering
  \begin{tabular}{l p{7cm} c c}
    \hline
    \sffamily\textbf{Component} & \sffamily\textbf{Description} & \sffamily\textbf{Units} & \sffamily\textbf{Dimensions} \\
    \hline\hline
    \texttt{n\_Layers}    & Number of atmospheric profile layers (\texttt{K}) & N/A & Scalar \\
    \texttt{n\_Absorbers} & Number of atmospheric absorbers (\texttt{J}) & N/A & Scalar \\
    \texttt{n\_Clouds}    & Number of clouds (\texttt{Nc}) & N/A & Scalar \\
    \texttt{n\_Aerosols}  & Number of aerosols (\texttt{Na}) & N/A & Scalar \\
    \texttt{n\_Added\_Layers} & Number of layers added so as to extend the input profile to nominal TOA  & N/A & Scalar \\
    \texttt{Climatology} & Climatology of the profile (see table \ref{tab:climatology}) & N/A & Scalar \\
    \texttt{Absorber\_Id} & Gaseous absorber identifiers (see table \ref{tab:absorber_id}) & N/A & \texttt{J}\\
    \texttt{Absorber\_Units} & Gaseous absorber amount units (see table \ref{tab:absorber_units}) & N/A & \texttt{J}\\
    \texttt{Level\_Pressure} & Interface pressures of the profile layers & hPa & \texttt{0:K} \\
    \texttt{Pressure} & Average layer pressure & hPa & \texttt{K} \\
    \texttt{Temperatures} & Layer temperatures & Kelvin & \texttt{K} \\
    \texttt{Absorbers} & Layer absorber amount & Variable & \texttt{K} $\times$ \texttt{J}\\
    \texttt{Cloud} & Cloud structure array & N/A & \texttt{Nc}\\
    \texttt{Aerosol} & Aerosol structure array & N/A & \texttt{Na}\\
    \hline
  \end{tabular}
  \caption{CRTM \Atmosphere{} structure component description.}
  \label{tab:atmosphere_structure}
\end{table}

% Climatology table
\begin{table}[htp]
  \centering
  \begin{tabular}{c l}
    \hline
    \sffamily\textbf{Climatology Type} & \sffamily\textbf{Parameter} \\
    \hline\hline
             Tropical          &  \texttt{TROPICAL}\\              
        Midlatitude summer     &  \texttt{MIDLATITUDE\_SUMMER}\\
        Midlatitude winter     &  \texttt{MIDLATITUDE\_WINTER}\\
         Subarctic summer      &  \texttt{SUBARCTIC\_SUMMER}\\
         Subarctic winter      &  \texttt{SUBARCTIC\_WINTER}\\
     U.S. Standard Atmosphere  &  \texttt{US\_STANDARD\_ATMOSPHERE}\\
    \hline 
  \end{tabular}
  \caption{CRTM \Atmosphere{} structure valid \texttt{Climatology} definitions. The same set as defined for LBLRTM is used.}
  \label{tab:climatology}
\end{table}

% Absorber id table
\begin{table}
  \centering
  \begin{tabular}{ c l c c l }
    \hline
    \sffamily\textbf{Molecule} & \sffamily\textbf{Parameter} & \hspace{0.5cm} & \sffamily\textbf{Molecule} & \sffamily\textbf{Parameter}\\
    \hline\hline
     H\subscript{2}O  & \texttt{H2O\_ID}  & \hspace{0.5cm} & HI                           & \texttt{HI\_ID}    \\
     CO\subscript{2}  & \texttt{CO2\_ID}  & \hspace{0.5cm} & ClO                          & \texttt{ClO\_ID}   \\
     O\subscript{3}   & \texttt{O3\_ID}   & \hspace{0.5cm} & OCS                          & \texttt{OCS\_ID}   \\
     N\subscript{2}O  & \texttt{N2O\_ID}  & \hspace{0.5cm} & H\subscript{2}CO             & \texttt{H2CO\_ID}  \\
     CO               & \texttt{CO\_ID}   & \hspace{0.5cm} & HOCl                         & \texttt{HOCl\_ID}  \\
     CH\subscript{4}  & \texttt{CH4\_ID}  & \hspace{0.5cm} & N\subscript{2}               & \texttt{N2\_ID}    \\
     O\subscript{2}   & \texttt{O2\_ID}   & \hspace{0.5cm} & HCN                          & \texttt{HCN\_ID}   \\
     NO               & \texttt{NO\_ID}   & \hspace{0.5cm} & CH\subscript{3}l             & \texttt{CH3l\_ID}  \\
     SO\subscript{2}  & \texttt{SO2\_ID}  & \hspace{0.5cm} & H\subscript{2}O\subscript{2} & \texttt{H2O2\_ID}  \\
     NO\subscript{2}  & \texttt{NO2\_ID}  & \hspace{0.5cm} & C\subscript{2}H\subscript{2} & \texttt{C2H2\_ID}  \\
     NH\subscript{3}  & \texttt{NH3\_ID}  & \hspace{0.5cm} & C\subscript{2}H\subscript{6} & \texttt{C2H6\_ID}  \\
     HNO\subscript{3} & \texttt{HNO3\_ID} & \hspace{0.5cm} & PH\subscript{3}              & \texttt{PH3\_ID}   \\
     OH               & \texttt{OH\_ID}   & \hspace{0.5cm} & COF\subscript{2}             & \texttt{COF2\_ID}  \\
     HF               & \texttt{HF\_ID}   & \hspace{0.5cm} & SF\subscript{6}              & \texttt{SF6\_ID}   \\
     HCl              & \texttt{HCl\_ID}  & \hspace{0.5cm} & H\subscript{2}S              & \texttt{H2S\_ID}   \\
     HBr              & \texttt{HBr\_ID}  & \hspace{0.5cm} & HCOOH                        & \texttt{HCOOH\_ID} \\
    \hline
  \end{tabular}
  \caption{CRTM \Atmosphere{} structure valid \texttt{Absorber\_ID} definitions. The same molecule set as defined for HITRAN is used.}
  \label{tab:absorber_id}
\end{table}

% Absorber units table
\begin{table}
  \centering
  \begin{tabular}{c l}
    \hline
    \sffamily\textbf{Units} & \sffamily\textbf{Parameter} \\
    \hline\hline
     Volume mixing ratio, ppmv                       & \texttt{VOLUME\_MIXING\_RATIO\_UNITS} \\
     Number density, cm$^{-3}$                       & \texttt{NUMBER\_DENSITY\_UNITS} \\
     Mass mixing ratio, g/kg                         & \texttt{MASS\_MIXING\_RATIO\_UNITS} \\
     Mass density, g.m$^{-3}$                        & \texttt{MASS\_DENSITY\_UNITS} \\
     Partial pressure, hPa                           & \texttt{PARTIAL\_PRESSURE\_UNITS} \\
     Dewpoint temperature, K  \textbf{(H$\mathbf{_2}$O ONLY)} & \texttt{DEWPOINT\_TEMPERATURE\_K\_UNITS} \\
     Dewpoint temperature, C  \textbf{(H$\mathbf{_2}$O ONLY)} & \texttt{DEWPOINT\_TEMPERATURE\_C\_UNITS} \\
     Relative humidity, \%    \textbf{(H$\mathbf{_2}$O ONLY)} & \texttt{RELATIVE\_HUMIDITY\_UNITS} \\
     Specific amount, g/g                            & \texttt{SPECIFIC\_AMOUNT\_UNITS} \\
     Integrated path, mm                             & \texttt{INTEGRATED\_PATH\_UNITS} \\
    \hline
  \end{tabular}
  \caption{CRTM \Atmosphere{} structure valid \texttt{Absorber\_Units} definitions. The same set as defined for LBLRTM is used.}
  \label{tab:absorber_units}
\end{table}

\clearpage
\subsection{\Cloud{} Structure}
%------------------------------
\label{sec:cloud_structure}

\begin{figure}[htp]
  \centering
  \doublebox{
  \begin{minipage}[b]{6.5in}
    \begin{ttfamily}
      \begin{verbatim}
  TYPE :: CRTM_Cloud_type
    ! Dimension values
    INTEGER :: n_Layers = 0  ! K dimension.
    ! Number of added layers
    INTEGER :: n_Added_Layers = 0
    ! Cloud type
    INTEGER :: Type = NO_CLOUD
    ! Cloud state variables
    REAL(fp), POINTER :: Effective_Radius(:)   => NULL() ! K
    REAL(fp), POINTER :: Effective_Variance(:) => NULL() ! K
    REAL(fp), POINTER :: Water_Content(:)      => NULL() ! K
  END TYPE CRTM_Cloud_type\end{verbatim}
    \end{ttfamily}
  \end{minipage}
  }
  \caption{CRTM \Cloud{} structure definition.}
  \label{fig:cloud_structure}
\end{figure}

% Cloud component description table
\begin{table}[htp]
  \centering
  \begin{tabular}{l p{7cm} c c}
    \hline
    \sffamily\textbf{Component} & \sffamily\textbf{Description} & \sffamily\textbf{Units} & \sffamily\textbf{Dimensions} \\
    \hline\hline
    \texttt{n\_Layers} & Number of atmospheric profile layers (\texttt{K}) & N/A & Scalar \\
    \texttt{n\_Added\_Layers} & Number of layers added so as to extend the input profile to nominal TOA  & N/A & Scalar \\
    \texttt{Type} & Type of cloud (see table \ref{tab:cloud_type}) & N/A & Scalar \\
    \texttt{Effective\_Radius} & Cloud particle $r_{eff}$ profile & \micron & \texttt{K} \\
    \texttt{Effective\_Variance} & Cloud particle $\sigma_{eff}$ profile & \micron$^2$ & \texttt{K} \\
    \texttt{Water\_Content} & Cloud water content profile & kg.m$^{-2}$ & \texttt{K} \\
    \hline
  \end{tabular}
  \caption{CRTM \Cloud{} structure component description.}
  \label{tab:cloud_structure}
\end{table}

% Cloud type table
\begin{table}
  \centering
  \begin{tabular}{cc l}
    \hline
    \sffamily\textbf{Cloud Type} & \hspace{0.5cm} & \sffamily\textbf{Parameter} \\
    \hline\hline
     Water   & \hspace{0.5cm} & \texttt{WATER\_CLOUD}\\
     Ice     & \hspace{0.5cm} & \texttt{ICE\_CLOUD}\\
     Rain    & \hspace{0.5cm} & \texttt{RAIN\_CLOUD}\\
     Snow    & \hspace{0.5cm} & \texttt{SNOW\_CLOUD}\\
     Graupel & \hspace{0.5cm} & \texttt{GRAUPEL\_CLOUD}\\
     Hail    & \hspace{0.5cm} & \texttt{HAIL\_CLOUD}\\
    \hline
  \end{tabular}
  \caption{CRTM \Cloud{} structure valid \texttt{Type} definitions.}
  \label{tab:cloud_type}
\end{table}

\clearpage
\subsection{\Aerosol{} Structure}
%--------------------------------
\label{sec:aerosol_structure}


\begin{figure}[htp]
  \centering
  \doublebox{
  \begin{minipage}[b]{6.5in}
    \begin{ttfamily}
      \begin{verbatim}
  TYPE :: CRTM_Aerosol_type
    ! Dimensions
    INTEGER :: n_Layers = 0  ! K dimension
    ! Number of added layers    
    INTEGER :: n_Added_Layers = 0
    ! Aerosol type
    INTEGER :: Type = NO_AEROSOL
    ! Aerosol state variables
    REAL(fp), POINTER :: Effective_Radius(:) => NULL()  ! K
    REAL(fp), POINTER :: Concentration(:)    => NULL()  ! K
  END TYPE CRTM_Aerosol_type\end{verbatim}
    \end{ttfamily}
  \end{minipage}
  }
  \caption{CRTM \Aerosol{} structure definition.}
  \label{fig:aerosol_structure}
\end{figure}

% Aerosol component description table
\begin{table}[htp]
  \centering
  \begin{tabular}{l p{7cm} c c}
    \hline
    \sffamily\textbf{Component} & \sffamily\textbf{Description} & \sffamily\textbf{Units} & \sffamily\textbf{Dimensions} \\
    \hline\hline
    \texttt{n\_Layers} & Number of atmospheric profile layers (\texttt{K}) & N/A & Scalar \\
    \texttt{n\_Added\_Layers} & Number of layers added so as to extend the input profile to nominal TOA  & N/A & Scalar \\
    \texttt{Type} & Type of Aerosol (see table \ref{tab:aerosol_type}) & N/A & Scalar \\
    \texttt{Effective\_Radius} & Aerosol particle $r_{eff}$ profile & \micron & \texttt{K} \\
    \texttt{Concentration} & Aerosol concentration profile & kg.m$^{-2}$ & \texttt{K} \\
    \hline
  \end{tabular}
  \caption{CRTM \Aerosol{} structure component description.}
  \label{tab:aerosol_structure}
\end{table}

% Aerosol type table
\begin{table}
  \centering
  \begin{tabular}{c l}
    \hline
    \sffamily\textbf{Aerosol Type} & \sffamily\textbf{Parameter} \\
    \hline\hline
           Dust         &  \texttt{DUST\_AEROSOL}\\
       Sea salt SSAM\footnote{SSAM $\equiv$ sea salt accumulation mode, $r_{eff}\sim$0.5-5.0\micron.} &  \texttt{SEASALT\_SSAM\_AEROSOL}\\
       Sea salt SSCM\footnote{SSCM $\equiv$ sea salt coarse mode, $r_{eff}\sim$5.0-30\micron} &  \texttt{SEASALT\_SSCM\_AEROSOL}\\
     Dry organic carbon &  \texttt{DRY\_ORGANIC\_CARBON\_AEROSOL}\\
     Wet organic carbon &  \texttt{WET\_ORGANIC\_CARBON\_AEROSOL}\\
      Dry black carbon  &  \texttt{DRY\_BLACK\_CARBON\_AEROSOL}\\
      Wet black carbon  &  \texttt{WET\_BLACK\_CARBON\_AEROSOL}\\
          Sulfate       &  \texttt{SULFATE\_AEROSOL}\\
    \hline
  \end{tabular}
  \caption{CRTM \Aerosol{} structure valid \texttt{Type} definitions.}
  \label{tab:aerosol_type}
\end{table}


\clearpage
\section{\Surface{} Structure}
%=============================
\label{sec:surface_structure}

\begin{figure}[htp]
  \centering
  \doublebox{
  \begin{minipage}[b]{6.5in}
    \begin{ttfamily}
      \begin{verbatim}
  TYPE :: CRTM_Surface_type
    ! Dimension values
    INTEGER :: n_Sensors    = 0  ! N dimension
    ! Gross type of surface determined by coverage
    REAL(fp) :: Land_Coverage  = ZERO
    REAL(fp) :: Water_Coverage = ZERO
    REAL(fp) :: Snow_Coverage  = ZERO
    REAL(fp) :: Ice_Coverage   = ZERO
    ! Surface type independent data
    REAL(fp) :: Wind_Speed     = DEFAULT_WIND_SPEED
    REAL(fp) :: Wind_Direction = DEFAULT_WIND_DIRECTION
    ! Land surface type data
    INTEGER  :: Land_Type             = DEFAULT_LAND_TYPE
    REAL(fp) :: Land_Temperature      = DEFAULT_LAND_TEMPERATURE
    REAL(fp) :: Soil_Moisture_Content = DEFAULT_SOIL_MOISTURE_CONTENT
    REAL(fp) :: Canopy_Water_Content  = DEFAULT_CANOPY_WATER_CONTENT
    REAL(fp) :: Vegetation_Fraction   = DEFAULT_VEGETATION_FRACTION
    REAL(fp) :: Soil_Temperature      = DEFAULT_SOIL_TEMPERATURE
    ! Water type data
    INTEGER  :: Water_Type        = DEFAULT_WATER_TYPE
    REAL(fp) :: Water_Temperature = DEFAULT_WATER_TEMPERATURE
    REAL(fp) :: Salinity          = DEFAULT_SALINITY
    ! Snow surface type data
    INTEGER  :: Snow_Type        = DEFAULT_SNOW_TYPE
    REAL(fp) :: Snow_Temperature = DEFAULT_SNOW_TEMPERATURE
    REAL(fp) :: Snow_Depth       = DEFAULT_SNOW_DEPTH
    REAL(fp) :: Snow_Density     = DEFAULT_SNOW_DENSITY
    REAL(fp) :: Snow_Grain_Size  = DEFAULT_SNOW_GRAIN_SIZE
    ! Ice surface type data
    INTEGER  :: Ice_Type        = DEFAULT_ICE_TYPE
    REAL(fp) :: Ice_Temperature = DEFAULT_ICE_TEMPERATURE
    REAL(fp) :: Ice_Thickness   = DEFAULT_ICE_THICKNESS
    REAL(fp) :: Ice_Density     = DEFAULT_ICE_DENSITY
    REAL(fp) :: Ice_Roughness   = DEFAULT_ICE_ROUGHNESS
    ! SensorData containing channel brightness temperatures
    TYPE(CRTM_SensorData_type) :: SensorData  ! N
  END TYPE CRTM_Surface_type\end{verbatim}
    \end{ttfamily}
  \end{minipage}
  }
  \caption{CRTM \Surface{} structure definition.}
  \label{fig:surface_structure}
\end{figure}

\begin{table}[htp]
  \centering
  \begin{tabular}{l p{7cm} c c}
    \hline
    \sffamily\textbf{Component} & \sffamily\textbf{Description} & \sffamily\textbf{Units} & \sffamily\textbf{Dimensions} \\
    \hline\hline
    \texttt{n\_Sensors} & The number of sensors for which data is provided inside the SensorData structure & N/A & Scalar \\
    \hline
    \texttt{Land\_Coverage}  & Fraction of the FOV that is land surface & N/A & Scalar \\
    \texttt{Water\_Coverage} & Fraction of the FOV that is water surface & N/A & Scalar \\
    \texttt{Snow\_Coverage}  & Fraction of the FOV that is snow surface & N/A & Scalar \\
    \texttt{Ice\_Coverage}   & Fraction of the FOV that is ice surface & N/A & Scalar \\
    \hline
    \texttt{Wind\_Speed}     & Surface wind speed & m.s$^{-1}$ & Scalar \\
    \texttt{Wind\_Direction} & Surface wind direction & deg. E from N & Scalar \\
    \hline
    \texttt{Land\_Type}              & Land surface type & N/A & Scalar \\
    \texttt{Land\_Temperature}       & Land surface temperature & Kelvin & Scalar \\
    \texttt{Soil\_Moisture\_Content} & Volumetric water content of the soil & g.cm$^{-3}$ & Scalar \\
    \texttt{Canopy\_Water\_Content}  & Gravimetric water content of the canopy & g.cm$^{-3}$ & Scalar \\
    \texttt{Vegetation\_Fraction}    & Vegetation fraction of the surface & \% & Scalar \\
    \texttt{Soil\_Temperature}       & Soil temperature & Kelvin & Scalar \\
    \hline
    \texttt{Water\_Type}        & Water surface type & N/A & Scalar \\
    \texttt{Water\_Temperature} & Water surface temperature & Kelvin & Scalar \\
    \texttt{Salinity}           & Water salinity & \textperthousand & Scalar \\
    \hline
    \texttt{Snow\_Type}        & Snow surface type & N/A & Scalar \\ 
    \texttt{Snow\_Temperature} & Snow surface temperature & Kelvin & Scalar \\ 
    \texttt{Snow\_Depth}       & Snow depth & mm & Scalar \\ 
    \texttt{Snow\_Density}     & Snow density & g.m$^{-3}$ & Scalar \\ 
    \texttt{Snow\_Grain\_Size} & Snow grain size & mm & Scalar \\ 
    \hline
    \texttt{Ice\_Type}        & Ice surface type & N/A & Scalar \\ 
    \texttt{Ice\_Temperature} & Ice surface temperature & Kelvin & Scalar \\ 
    \texttt{Ice\_Thickness}   & Thickness of ice & mm & Scalar \\ 
    \texttt{Ice\_Density}     & Density of ice & g.m$^{-3}$ & Scalar \\ 
    \texttt{Ice\_Roughness}   & Measure of the surface roughness of the ice & N/A & Scalar \\ 
    \hline
    \texttt{SensorData} & Satellite sensor data required for some surface emissivity algorithms & N/A & Scalar \\ 
    \hline
  \end{tabular}
  \caption{CRTM \Surface{} structure component description.}
  \label{tav:surface_structure}
\end{table}

% Default surface value table
\begin{table}[htp]
  \centering
  \begin{tabular}{l c c}
    \hline
    \sffamily\textbf{Parameter} & \sffamily\textbf{Value}  & \sffamily\textbf{Units} \\
    \hline\hline
    \multicolumn{3}{c}{\textsf{Surface type independent data}}\\
    \hline
    \texttt{DEFAULT\_WIND\_SPEED}             & 5.0        & m.s$^{-1}$\\
    \texttt{DEFAULT\_WIND\_DIRECTION}         & 0.0        & deg. E from N\\[0.2cm]
    \multicolumn{3}{c}{\textsf{Land surface type data}}\\
    \hline
    \texttt{DEFAULT\_LAND\_TYPE}              & \texttt{GRASS\_SOIL}& N/A \\
    \texttt{DEFAULT\_LAND\_TEMPERATURE}       & 283.0      & K\\
    \texttt{DEFAULT\_SOIL\_MOISTURE\_CONTENT} & 0.05       & g.cm$^{-3}$\\
    \texttt{DEFAULT\_CANOPY\_WATER\_CONTENT}  & 0.05       & g.cm$^{-3}$\\
    \texttt{DEFAULT\_VEGETATION\_FRACTION}    & 0.3        & 30\%\\
    \texttt{DEFAULT\_SOIL\_TEMPERATURE}       & 283.0      & K\\[0.2cm]
    \multicolumn{3}{c}{\textsf{Water type data}}\\
    \hline
    \texttt{DEFAULT\_WATER\_TYPE}             & \texttt{SEA\_WATER} & N/A\\
    \texttt{DEFAULT\_WATER\_TEMPERATURE}      & 283.0      & K\\
    \texttt{DEFAULT\_SALINITY}                & 33.0       & ppmv\\[0.2cm]
    \multicolumn{3}{c}{\textsf{Snow surface type data}}\\
    \hline
    \texttt{DEFAULT\_SNOW\_TYPE}              & \texttt{NEW\_SNOW}  & N/A\\
    \texttt{DEFAULT\_SNOW\_TEMPERATURE}       & 263.0      & K\\
    \texttt{DEFAULT\_SNOW\_DEPTH}             & 50.0       & mm\\
    \texttt{DEFAULT\_SNOW\_DENSITY}           & 0.2        & g.cm$^{-3}$\\
    \texttt{DEFAULT\_SNOW\_GRAIN\_SIZE}       & 2.0        & mm\\[0.2cm]
    \multicolumn{3}{c}{\textsf{Ice surface type data}}\\
    \hline
    \texttt{DEFAULT\_ICE\_TYPE}               & \texttt{FRESH\_ICE} & N/A\\
    \texttt{DEFAULT\_ICE\_TEMPERATURE}        & 263.0      & K\\
    \texttt{DEFAULT\_ICE\_THICKNESS}          & 10.0       & mm\\
    \texttt{DEFAULT\_ICE\_DENSITY}            & 0.9        & g.cm$^{-3}$\\
    \texttt{DEFAULT\_ICE\_ROUGHNESS}          & 0.0        & N/A\\
    \hline
  \end{tabular}
  \caption{CRTM \Surface{} structure default values.}
  \label{tab:surface_default}
\end{table}

% Land surface type table
\begin{table}[htp]
  \centering
  \begin{tabular}{c l}
    \hline
    \sffamily\textbf{Land Type} & \sffamily\textbf{Parameter} \\
    \hline\hline
          Compacted soil      & \texttt{COMPACTED\_SOIL} \\
            Tilled soil       & \texttt{TILLED\_SOIL} \\
              Sand            & \texttt{SAND} \\
              Rock            & \texttt{ROCK} \\
     Irrigated low vegetation & \texttt{IRRIGATED\_LOW\_VEGETATION} \\
           Meadow grass       & \texttt{MEADOW\_GRASS} \\
              Scrub           & \texttt{SCRUB} \\
         Broadleaf forest     & \texttt{BROADLEAF\_FOREST} \\
           Pine forest        & \texttt{PINE\_FOREST} \\
             Tundra           & \texttt{TUNDRA} \\
           Grass soil         & \texttt{GRASS\_SOIL} \\
       Broadleaf-pine forest  & \texttt{BROADLEAF\_PINE\_FOREST} \\
           Grass scrub        & \texttt{GRASS\_SCRUB} \\
            Oil grass         & \texttt{OIL\_GRASS} \\
          Urban concrete      & \texttt{URBAN\_CONCRETE} \\
            Pine brush        & \texttt{PINE\_BRUSH} \\
          Broadleaf brush     & \texttt{BROADLEAF\_BRUSH} \\
             Wet soil         & \texttt{WET\_SOIL} \\
            Scrub soil        & \texttt{SCRUB\_SOIL} \\
      Broadleaf(70)-Pine(30)  & \texttt{BROADLEAF70\_PINE30} \\
    \hline
  \end{tabular}
  \caption{CRTM \Surface{} structure valid \texttt{Land\_Type} definitions.}
  \label{tab:surface_land_type}
\end{table}

% Water surface type table
\begin{table}[htp]
  \centering
  \begin{tabular}{cc l}
    \hline
    \sffamily\textbf{Water Type} & \hspace{0.5cm} & \sffamily\textbf{Parameter} \\
    \hline\hline
      Sea water  & \hspace{0.5cm} &  \texttt{SEA\_WATER} \\     
     Fresh water & \hspace{0.5cm} &  \texttt{FRESH\_WATER} \\   
    \hline
  \end{tabular}
  \caption{CRTM \Surface{} structure valid \texttt{Water\_Type} definitions.}
  \label{tab:surface_water_type}
\end{table}

% Snow surface type table
\begin{table}[htp]
  \centering
  \begin{tabular}{c l}
    \hline
    \sffamily\textbf{Snow Type} & \sffamily\textbf{Parameter} \\
    \hline\hline
         Wet snow          &   \texttt{WET\_SNOW} \\           
      Grass after snow     &   \texttt{GRASS\_AFTER\_SNOW} \\   
        Powder snow        &   \texttt{POWDER\_SNOW} \\        
         RS snow(A)        &   \texttt{RS\_SNOW\_A} \\          
         RS snow(B)        &   \texttt{RS\_SNOW\_B} \\          
         RS snow(C)        &   \texttt{RS\_SNOW\_C} \\          
         RS snow(D)        &   \texttt{RS\_SNOW\_D} \\          
         RS snow(E)        &   \texttt{RS\_SNOW\_E} \\          
      Thin Crust snow      &   \texttt{THIN\_CRUST\_SNOW} \\    
      Thick crust snow     &   \texttt{THICK\_CRUST\_SNOW } \\  
        Shallow snow       &   \texttt{SHALLOW\_SNOW} \\       
         Deep snow         &   \texttt{DEEP\_SNOW} \\          
        Crust snow         &   \texttt{CRUST\_SNOW} \\         
        Medium snow        &   \texttt{MEDIUM\_SNOW} \\        
     Bottom crust snow(A)  &   \texttt{BOTTOM\_CRUST\_SNOW\_A} \\
     Bottom crust snow(B)  &   \texttt{BOTTOM\_CRUST\_SNOW\_B} \\
    \hline
  \end{tabular}
  \caption{CRTM \Surface{} structure valid \texttt{Snow\_Type} definitions.}
  \label{tab:surface_snow_type}
\end{table}

% Ice surface type table
\begin{table}[htp]
  \centering
  \begin{tabular}{c l}
    \hline
    \sffamily\textbf{Ice Type} & \sffamily\textbf{Parameter} \\
    \hline\hline
            Fresh ice        &   \texttt{FRESH\_ICE} \\       
        First year sea ice   &   \texttt{FIRST\_YEAR\_SEA\_ICE} \\
      Multiple year sea ice  &   \texttt{MULTI\_YEAR\_SEA\_ICE} \\
            Ice floe         &   \texttt{ICE\_FLOE} \\            
            Ice ridge        &   \texttt{ICE\_RIDGE} \\           
    \hline
  \end{tabular}
  \caption{CRTM \Surface{} structure valid \texttt{Ice\_Type} definitions.}
  \label{tab:surface_ice_type}
\end{table}


\clearpage
\subsection{\SensorData{} Structure}
%-----------------------------------
\label{sec:sensordata_structure}

\begin{figure}[htp]
  \centering
  \doublebox{
  \begin{minipage}[b]{6.5in}
    \begin{ttfamily}
      \begin{verbatim}
  TYPE, PUBLIC :: CRTM_SensorData_type
    ! Dimension values
    INTEGER :: n_Channels = 0  ! L
    ! The WMO sensor ID of the sensor for which the data is to be used
    INTEGER :: Select_WMO_Sensor_Id = INVALID_WMO_SENSOR_ID
    ! The data sensor IDs and channels
    CHARACTER(STRLEN), POINTER :: Sensor_Id(:)        => NULL() ! L
    INTEGER,           POINTER :: WMO_Satellite_ID(:) => NULL() ! L
    INTEGER,           POINTER :: WMO_Sensor_ID(:)    => NULL() ! L
    INTEGER,           POINTER :: Sensor_Channel(:)   => NULL() ! L
    ! The sensor brightness temperatures
    REAL(fp),          POINTER :: Tb(:) => NULL() ! L
  END TYPE CRTM_SensorData_type\end{verbatim}
    \end{ttfamily}
  \end{minipage}
  }
  \caption{CRTM \SensorData{} structure definition.}
  \label{fig:sensordata_structure}
\end{figure}

% SensorData component description table
\begin{table}[htp]
  \centering
  \begin{tabular}{l p{7cm} c c}
    \hline
    \sffamily\textbf{Component} & \sffamily\textbf{Description} & \sffamily\textbf{Units} & \sffamily\textbf{Dimensions} \\
    \hline\hline
    \texttt{n\_Channels} & Number of channels to use in SfcOptics emissivty algorithms (\texttt{L}) & N/A & Scalar \\
    \texttt{Select\_WMO\_Sensor\_Id} & The WMO Sensor Id value of the sensor for which the data is to be used. & N/A & Scalar \\
    \texttt{Sensor\_Id} & The sensor id character string for each channel of data & N/A & \texttt{L} \\
    \texttt{WMO\_Satellite\_Id} & The WMO satellite Id for each channel of data & N/A & \texttt{L} \\
    \texttt{WMO\_Sensor\_Id} & The WMO sensor Id for each channel of data & N/A & \texttt{L} \\
    \texttt{Sensor\_Channel} & The channel number for each channel of data & N/A & \texttt{L} \\
    \texttt{Tb} & The brightness temperature measurements for each channel & Kelvin & \texttt{L} \\
    \hline
  \end{tabular}
  \caption{CRTM \SensorData{} structure component description.}
  \label{tab:sensordata_structure}
\end{table}



\clearpage
\section{\GeometryInfo{} Structure}
%==================================
\label{sec:geometryinfo_structure}

\begin{figure}[htp]
  \centering
  \doublebox{
  \begin{minipage}[b]{6.5in}
    \begin{ttfamily}
      \begin{verbatim}
  TYPE :: CRTM_GeometryInfo_type
    ! User Input
    ! ----------
    ! Earth location
    REAL(fp) :: Longitude        = ZERO
    REAL(fp) :: Latitude         = ZERO
    REAL(fp) :: Surface_Altitude = ZERO
    ! Field of view index (1-nFOV)
    INTEGER  :: iFOV = 0
    ! Sensor angle information
    REAL(fp) :: Sensor_Scan_Angle    = ZERO
    REAL(fp) :: Sensor_Zenith_Angle  = ZERO
    REAL(fp) :: Sensor_Azimuth_Angle = ZERO 
    ! Source angle information
    REAL(fp) :: Source_Zenith_Angle  = 100.0_fp  ! Below horizon
    REAL(fp) :: Source_Azimuth_Angle = ZERO
    ! Flux angle information
    REAL(fp) :: Flux_Zenith_Angle = DIFFUSIVITY_ANGLE
    ! Derived from User Input
    ! -----------------------
    ! Default distance ratio
    REAL(fp) :: Distance_Ratio = EARTH_RADIUS/(EARTH_RADIUS + SATELLITE_HEIGHT)
    ! Sensor angle information
    REAL(fp) :: Sensor_Scan_Radian    = ZERO
    REAL(fp) :: Sensor_Zenith_Radian  = ZERO
    REAL(fp) :: Sensor_Azimuth_Radian = ZERO
    REAL(fp) :: Secant_Sensor_Zenith  = ZERO
    ! Source angle information
    REAL(fp) :: Source_Zenith_Radian  = ZERO
    REAL(fp) :: Source_Azimuth_Radian = ZERO
    REAL(fp) :: Secant_Source_Zenith  = ZERO
    ! Flux angle information
    REAL(fp) :: Flux_Zenith_Radian = DIFFUSIVITY_RADIAN
    REAL(fp) :: Secant_Flux_Zenith = SECANT_DIFFUSIVITY
  END TYPE CRTM_GeometryInfo_type\end{verbatim}
    \end{ttfamily}
  \end{minipage}
  }
  \caption{CRTM \GeometryInfo{} structure definition.}
  \label{fig:geometryinfo_structure}
\end{figure}

% GeometryInfo component description table
\begin{table}[htp]
  \centering
  \begin{tabular}{l p{7cm} c c}
    \hline
    \sffamily\textbf{Component} & \sffamily\textbf{Description} & \sffamily\textbf{Units} & \sffamily\textbf{Dimensions} \\
    \hline\hline
    \texttt{Longitude}               & Earth longitude & degrees East (0$\rightarrow$360) & Scalar \\
    \texttt{Latitude}                & Earth latitude  & degrees North (-90$\rightarrow$+90) & Scalar \\
    \texttt{Surface\_Altitude}       & Altitude of the Earth's surface at the specified lon/lat location & metres (m) & Scalar \\
    \texttt{iFOV}                    & The scan line FOV index & N/A & Scalar \\
    \texttt{Sensor\_Scan\_Angle}     & The sensor scan angle from nadir. See fig.\ref{fig:gInfo_sensor_scan_angle} & degrees & Scalar \\
    \texttt{Sensor\_Zenith\_Angle}   & The sensor zenith angle of the FOV. See fig.\ref{fig:gInfo_sensor_zenith_angle} & degrees & Scalar \\
    \texttt{Sensor\_Azimuth\_Angle}  & The sensor azimuth angle is the angle subtended by the horizontal projection of a direct line from the satellite to the FOV and the North-South axis measured clockwise from North. See fig.\ref{fig:gInfo_sensor_azimuth_angle} & degrees from North & Scalar \\
    \texttt{Source\_Zenith\_Angle}   & The source zenith angle. The source is typically the Sun (IR/VIS) or Moon (MW/VIS) [only solar source valid in current release] See fig.\ref{fig:gInfo_source_zenith_angle} & degrees & Scalar \\
    \texttt{Source\_Azimuth\_Angle}  & The source azimuth angle is the angle subtended by the horizontal projection of a direct line from the source to the FOV and the North-South axis measured clockwise from North. See fig.\ref{fig:gInfo_source_azimuth_angle} & degrees from North & Scalar \\
    \texttt{Flux\_Zenith\_Angle}     & The zenith angle used to approximate downwelling flux transmissivity. If not set, the default value is that of the diffusivity approximation, such that $\sec(F) = 5/3$. Maximum allowed value is determined from $\sec(F) = 9/4$ & degres & Scalar \\
    \hline
  \end{tabular}
  \caption{Description of CRTM \GeometryInfo{} structure components defined by the user.}
  \label{tab:user_defined_geometryinfo_structure}
\end{table}

\begin{table}[htp]
  \centering
  \begin{tabular}{l p{7cm} c c}
    \hline
    \sffamily\textbf{Component} & \sffamily\textbf{Description} & \sffamily\textbf{Units} & \sffamily\textbf{Dimensions} \\
    \hline\hline
    \texttt{Distance\_Ratio}         & The ratio of the radius of the earth at the FOV location to the sum of the radius of the earth at nadir, and the satellite altitude;
    
     \mbox{\hspace{1cm}$r = \displaystyle\frac{R_e(FOV)}{R_e(nadir) + h}$}.
     
     Note that this quantity is actually computed using the user input sensor scan and zenith angles; 
     
     \mbox{\hspace{1cm}$r = \displaystyle\frac{\sin(\theta_{scan})}{\sin(\theta_{zenith})}$} & N/A & Scalar \\
    \texttt{Sensor\_Scan\_Radian}    & The sensor scan angle in radians & radians & Scalar \\
    \texttt{Sensor\_Zenith\_Radian}  & The sensor scan angle in radians & radians & Scalar \\
    \texttt{Sensor\_Azimuth\_Radian} & The sensor azimuth angle in radians & radians & Scalar \\
    \texttt{Secant\_Sensor\_Zenith}  & The secant of the sensor zenith angle & N/A & Scalar \\
    \texttt{Source\_Zenith\_Radian}  & The source zenith angle in radians & radians & Scalar \\
    \texttt{Source\_Azimuth\_Radian} & The source azimuth angle in radians & radians & Scalar \\
    \texttt{Secant\_Source\_Zenith}  & The secant of the source zenith angle & N/A & Scalar \\
    \texttt{Flux\_Zenith\_Radian}    & The flux zenith angle in radians & radians & Scalar \\
    \texttt{Secant\_Flux\_Zenith}    & The secant of the flux zenith angle & N/A & Scalar \\
    \hline
  \end{tabular}
  \caption{Description of CRTM \GeometryInfo{} structure components derived from user inputs.}
  \label{tab:derived_geometryinfo_structure}
\end{table}


\clearpage
\section{\RTSolution{} Structure}
%==================================
\label{sec:rtsolution_structure}

\begin{figure}[htp]
  \centering
  \doublebox{
  \begin{minipage}[b]{6.5in}
    \begin{ttfamily}
      \begin{verbatim}
  TYPE :: CRTM_RTSolution_type
    ! Dimension values
    INTEGER :: n_Layers = 0  ! K
    ! Forward radiative transfer intermediate results for a single channel
    !    These components are not defined when they are used as TL, AD
    !    and K variables
    REAL(fp)          :: Surface_Emissivity      = ZERO
    REAL(fp)          :: Up_Radiance             = ZERO
    REAL(fp)          :: Down_Radiance           = ZERO
    REAL(fp)          :: Down_Solar_Radiance     = ZERO
    REAL(fp)          :: Surface_Planck_Radiance = ZERO
    REAL(fp), POINTER :: Upwelling_Radiance(:)  => NULL()  ! K
    ! The layer optical depths
    REAL(fp), POINTER :: Layer_Optical_Depth(:) => NULL()  ! K
    ! Radiative transfer results for a single channel/node
    REAL(fp) :: Radiance               = ZERO
    REAL(fp) :: Brightness_Temperature = ZERO
  END TYPE CRTM_RTSolution_type\end{verbatim}
    \end{ttfamily}
  \end{minipage}
  }
  \caption{CRTM \RTSolution{} structure definition.}
  \label{fig:rtsolution_structure}
\end{figure}

\begin{table}[htp]
  \centering
  \begin{tabular}{l p{7cm} c c}
    \hline
    \sffamily\textbf{Component} & \sffamily\textbf{Description} & \sffamily\textbf{Units} & \sffamily\textbf{Dimensions} \\
    \hline\hline
    \texttt{n\_Layers}    & Number of atmospheric profile layers (\texttt{K}) & N/A & Scalar \\
    \texttt{Surface\_Emissivity      }  & The computed surface emissivity & N/A & Scalar \\
    \texttt{Up\_Radiance             }  & The atmospheric portion of the upwelling radiance & \radunit & Scalar \\
    \texttt{Down\_Radiance           }  & The atmospheric portion of the downwelling radiance & \radunit & Scalar \\
    \texttt{Down\_Solar\_Radiance    }  & The downwelling direct solar radiance & \radunit & Scalar \\
    \texttt{Surface\_Planck\_Radiance}  & The surface radiance & \radunit & Scalar \\
    \texttt{Upwelling\_Radiance      }  & The upwelling radiance profile, including the reflected downwelling and surface contributions. & \radunit & \texttt{K} \\
    \texttt{Layer\_Optical\_Depth    }  & The layer optical depth profile & N/A & \texttt{K} \\
    \texttt{Radiance                 }  & The sensor radiance & \radunit & Scalar \\
    \texttt{Brightness\_Temperature  }  & The sensor brightness temperature & Kelvin & Scalar \\
    \hline
  \end{tabular}
  \caption{CRTM \RTSolution{} structure component description}
  \label{tab:rtsolution_structure}
\end{table}


\clearpage
\section{\Options{} Structure}
%=============================
\label{sec:options_structure}

\begin{figure}[htp]
  \centering
  \doublebox{
  \begin{minipage}[b]{6.5in}
    \begin{ttfamily}
      \begin{verbatim}
  TYPE :: CRTM_Options_type
    ! Dimension values
    INTEGER :: n_Channels = 0  ! L dimension
    ! Index into channel-specific components
    INTEGER :: Channel = 0
    ! Emissivity optional arguments
    INTEGER           :: Emissivity_Switch =  NOT_SET
    REAL(fp), POINTER :: Emissivity(:)     => NULL() ! L
    ! Direct reflectivity optional arguments
    INTEGER           :: Direct_Reflectivity_Switch =  NOT_SET
    REAL(fp), POINTER :: Direct_Reflectivity(:)     => NULL() ! L
    ! Antenna correction application
    INTEGER :: Antenna_Correction = NOT_SET
  END TYPE CRTM_Options_type\end{verbatim}
    \end{ttfamily}
  \end{minipage}
  }
  \caption{CRTM \Options{} structure definition.}
  \label{fig:options_structure}
\end{figure}

\begin{table}[htp]
  \centering
  \begin{tabular}{l p{7cm} c c}
    \hline
    \sffamily\textbf{Component} & \sffamily\textbf{Description} & \sffamily\textbf{Units} & \sffamily\textbf{Dimensions} \\
    \hline\hline
    \texttt{n\_Channels}                  & Number of sensor channels (\texttt{L}). & N/A & Scalar \\
    \texttt{Channel}                      & Index into channel-specifi components. & N/A & Scalar \\
    \texttt{Emissivity\_Switch}           & Switch to apply user-defined surface emissivity. Valid values:

    \parbox{7cm}{\hspace{0.5cm}\texttt{NOT\_SET}: Calculate emissivity (default).
    
                 \hspace{0.5cm}\texttt{SET}: Use user-defined emissivity}
     & N/A & Scalar \\
    \texttt{Emissivity}                   & User-defined surface emissivity for each sensor channel. & N/A & \texttt{L} \\
    \texttt{Direct\_Reflectivity\_Switch} & Switch to apply user-defined reflectivity for downwelling source (e.g. solar). This switch is ignored unless the \texttt{Emissivity\_Switch} is also set. Valid values:

    \parbox{7cm}{\hspace{0.5cm}\texttt{NOT\_SET}: Calculate reflectivity (default).
    
                 \hspace{0.5cm}\texttt{SET}: Use user-defined reflectivity}
     & N/A & Scalar \\
    \texttt{Direct\_Reflectivity}         & User-defined direct reflectivity for downwelling source for each sensor channel. & N/A & \texttt{L} \\
    \texttt{Antenna\_Correction}          & Switch to apply antenna correction for select microwave instruments. & N/A & Scalar \\
    \hline
  \end{tabular}
  \caption{CRTM \Options{} structure component description}
  \label{tab:options_structure}
\end{table}


  \chapter{Valid Sensor Identifiers}
%=================================
\label{sec:sensor_id}

This section contains a table detailing the instruments for which there are CRTM coefficients. For most sensors there are transmittance coefficient (\texttt{TauCoeff}) datafiles for both the Optical Depth in Absorber Space (ODAS; also known as Compact-OPTRAN) and Optical Depth in Pressure Space (ODPS) transmittance algorithms. All visible and SSU channels have only ODAS coefficients.

\clearpage
\begin{center}
\begin{longtable}{c c c c}
  \caption[CRTM sensor identifiers and the availability of ODAS or ODPS TauCoeff files]{CRTM sensor identifiers and the availability of ODAS or ODPS TauCoeff files}
  \label{tab:sensor_id} \\

  % Header for first page
  \hline\hline \\[-2ex]
    \multicolumn{1}{c}{\sffamily\textbf{Instrument}} &
    \multicolumn{1}{c}{\sffamily\textbf{Sensor Id}} &
    \multicolumn{1}{c}{\sffamily\textbf{ODAS available}} &
    \multicolumn{1}{c}{\sffamily\textbf{ODPS available}} \\[0.5ex] \hline
    \\[-1.8ex]
  \endfirsthead

  % Header for the remaing page(s) of the table
  \multicolumn{4}{c}{\sffamily\textbf{{\tablename} \thetable{}} -- Continued} \\[0.5ex]
  \hline\hline \\[-2ex]
      \multicolumn{1}{c}{\sffamily\textbf{Instrument}} &
      \multicolumn{1}{c}{\sffamily\textbf{Sensor Id}} &
      \multicolumn{1}{c}{\sffamily\textbf{ODAS available}} &
      \multicolumn{1}{c}{\sffamily\textbf{ODPS available}} \\[0.5ex] \hline
      \\[-1.8ex]
  \endhead
  
  %This is the footer for all pages except the last page of the table...
  \hline\hline 
    \multicolumn{4}{l}{{Continued on Next Page\ldots}} \\
  \endfoot
  
  %This is the footer for the last page of the table...
  \\[-1.8ex] \hline \hline
  \endlastfoot
  
  % Now the data
  Envisat AATSR                      & \texttt{aatsr\_envisat}      &  yes     &  yes       \\
  GOES-R ABI                         & \texttt{abi\_gr}             &  yes     &  yes       \\
  Aqua AIRS (281ch. subset)          & \texttt{airs281\_aqua}       &  yes     &  yes       \\
  Aqua AIRS (324ch. subset)          & \texttt{airs324\_aqua}       &  yes     &  yes       \\
  Aqua AIRS (all channels)           & \texttt{airs2378\_aqua}      &  yes     &  yes       \\
  Aqua AIRS Module-1a                & \texttt{airsM1a\_aqua}       &  yes     &  yes       \\
  Aqua AIRS Module-1b                & \texttt{airsM1b\_aqua}       &  yes     &  yes       \\
  Aqua AIRS Module-2a                & \texttt{airsM2a\_aqua}       &  yes     &  yes       \\
  Aqua AIRS Module-2b                & \texttt{airsM2b\_aqua}       &  yes     &  yes       \\
  Aqua AIRS Module-3                 & \texttt{airsM3\_aqua}        &  yes     &  yes       \\
  Aqua AIRS Module-4a                & \texttt{airsM4a\_aqua}       &  yes     &  yes       \\
  Aqua AIRS Module-4b                & \texttt{airsM4b\_aqua}       &  yes     &  yes       \\
  Aqua AIRS Module-4c                & \texttt{airsM4c\_aqua}       &  yes     &  yes       \\
  Aqua AIRS Module-4d                & \texttt{airsM4d\_aqua}       &  yes     &  yes       \\
  Aqua AIRS Module-5                 & \texttt{airsM5\_aqua}        &  yes     &  yes       \\
  Aqua AIRS Module-6                 & \texttt{airsM6\_aqua}        &  yes     &  yes       \\
  Aqua AIRS Module-7                 & \texttt{airsM7\_aqua}        &  yes     &  yes       \\
  Aqua AIRS Module-8                 & \texttt{airsM8\_aqua}        &  yes     &  yes       \\
  Aqua AIRS Module-9                 & \texttt{airsM9\_aqua}        &  yes     &  yes       \\
  Aqua AIRS Module-10                & \texttt{airsM10\_aqua}       &  yes     &  yes       \\
  Aqua AIRS Module-11                & \texttt{airsM11\_aqua}       &  yes     &  yes       \\
  Aqua AIRS Module-12                & \texttt{airsM12\_aqua}       &  yes     &  yes       \\
  Aqua AMSR-E                        & \texttt{amsre\_aqua}         &  yes     &  yes       \\
  Aqua AMSU-A                        & \texttt{amsua\_aqua}         &  yes     &  yes       \\
  NOAA-15 AMSU-A                     & \texttt{amsua\_n15}          &  yes     &  yes       \\
  NOAA-16 AMSU-A                     & \texttt{amsua\_n16}          &  yes     &  yes       \\
  NOAA-17 AMSU-A                     & \texttt{amsua\_n17}          &  yes     &  yes       \\
  NOAA-18 AMSU-A                     & \texttt{amsua\_n18}          &  yes     &  yes       \\
  NOAA-19 AMSU-A                     & \texttt{amsua\_n19}          &  yes     &  yes       \\
  MetOp-A AMSU-A                     & \texttt{amsua\_metop-a}      &  yes     &  yes       \\
  MetOp-B AMSU-A                     & \texttt{amsua\_metop-b}      &  yes     &  yes       \\
  MetOp-C AMSU-A                     & \texttt{amsua\_metop-c}      &  yes     &  yes       \\
  NOAA-15 AMSU-B                     & \texttt{amsub\_n15}          &  yes     &  yes       \\
  NOAA-16 AMSU-B                     & \texttt{amsub\_n16}          &  yes     &  yes       \\
  NOAA-17 AMSU-B                     & \texttt{amsub\_n17}          &  yes     &  yes       \\
  NPP ATMS                           & \texttt{atms\_npp}           &  yes     &  yes       \\
  ERS-1 ATSR                         & \texttt{atsr1\_ers1}         &  yes     &  yes       \\
  ERS-2 ATSR                         & \texttt{atsr2\_ers2}         &  yes     &  yes       \\
  TIROS-N AVHRR/2                    & \texttt{avhrr2\_tirosn}      &  yes     &  yes       \\
  NOAA-06 AVHRR/2                    & \texttt{avhrr2\_n06}         &  yes     &  yes       \\
  NOAA-07 AVHRR/2                    & \texttt{avhrr2\_n07}         &  yes     &  yes       \\
  NOAA-08 AVHRR/2                    & \texttt{avhrr2\_n08}         &  yes     &  yes       \\
  NOAA-09 AVHRR/2                    & \texttt{avhrr2\_n09}         &  yes     &  yes       \\
  NOAA-10 AVHRR/2                    & \texttt{avhrr2\_n10}         &  yes     &  yes       \\
  NOAA-11 AVHRR/2                    & \texttt{avhrr2\_n11}         &  yes     &  yes       \\
  NOAA-12 AVHRR/2                    & \texttt{avhrr2\_n12}         &  yes     &  yes       \\
  NOAA-14 AVHRR/2                    & \texttt{avhrr2\_n14}         &  yes     &  yes       \\
  NOAA-15 AVHRR/3                    & \texttt{avhrr3\_n15}         &  yes     &  yes       \\
  NOAA-16 AVHRR/3                    & \texttt{avhrr3\_n16}         &  yes     &  yes       \\
  NOAA-17 AVHRR/3                    & \texttt{avhrr3\_n17}         &  yes     &  yes       \\
  NOAA-18 AVHRR/3                    & \texttt{avhrr3\_n18}         &  yes     &  yes       \\
  NOAA-19 AVHRR/3                    & \texttt{avhrr3\_n19}         &  yes     &  yes       \\
  MetOp-A AVHRR/3                    & \texttt{avhrr3\_metop-a}     &  yes     &  yes       \\
  MetOp-B AVHRR/3                    & \texttt{avhrr3\_metop-b}     &  yes     &  yes       \\
  NPP CrIS (374ch. subset)           & \texttt{cris374\_npp}        &  yes     &  yes       \\
  NPP CrIS (399ch. subset)           & \texttt{cris399\_npp}        &  yes     &  yes       \\
  NPP CrIS (all channels)            & \texttt{cris1305\_npp}       &  yes     &  yes       \\
  NPP CrIS Band 1                    & \texttt{crisB1\_npp}         &  yes     &  yes       \\
  NPP CrIS Band 2                    & \texttt{crisB2\_npp}         &  yes     &  yes       \\
  NPP CrIS Band 3                    & \texttt{crisB3\_npp}         &  yes     &  yes       \\
  GPM GMI                            & \texttt{gmi\_gpm}            &  yes     &  yes       \\
  TIROS-N HIRS/2                     & \texttt{hirs2\_tirosn}       &  yes     &  yes       \\
  NOAA-06 HIRS/2                     & \texttt{hirs2\_n06}          &  yes     &  yes       \\
  NOAA-07 HIRS/2                     & \texttt{hirs2\_n07}          &  yes     &  yes       \\
  NOAA-08 HIRS/2                     & \texttt{hirs2\_n08}          &  yes     &  yes       \\
  NOAA-09 HIRS/2                     & \texttt{hirs2\_n09}          &  yes     &  yes       \\
  NOAA-10 HIRS/2                     & \texttt{hirs2\_n10}          &  yes     &  yes       \\
  NOAA-11 HIRS/2                     & \texttt{hirs2\_n11}          &  yes     &  yes       \\
  NOAA-12 HIRS/2                     & \texttt{hirs2\_n12}          &  yes     &  yes       \\
  NOAA-14 HIRS/2                     & \texttt{hirs2\_n14}          &  yes     &  yes       \\
  NOAA-15 HIRS/3                     & \texttt{hirs3\_n15}          &  yes     &  yes       \\
  NOAA-16 HIRS/3                     & \texttt{hirs3\_n16}          &  yes     &  yes       \\
  NOAA-17 HIRS/3                     & \texttt{hirs3\_n17}          &  yes     &  yes       \\
  NOAA-18 HIRS/4                     & \texttt{hirs4\_n18}          &  yes     &  yes       \\
  NOAA-19 HIRS/4                     & \texttt{hirs4\_n19}          &  yes     &  yes       \\
  MetOp-A HIRS/4                     & \texttt{hirs4\_metop-a}      &  yes     &  yes       \\
  MetOp-B HIRS/4                     & \texttt{hirs4\_metop-b}      &  yes     &  yes       \\
  Aqua HSB                           & \texttt{hsb\_aqua}           &  yes     &  yes       \\
  MetOp-A IASI (300ch. subset)       & \texttt{iasi300\_metop-a}    &  yes     &  yes       \\
  MetOp-A IASI (316ch. subset)       & \texttt{iasi316\_metop-a}    &  yes     &  yes       \\
  MetOp-A IASI (616ch. subset)       & \texttt{iasi616\_metop-a}    &  yes     &  yes       \\
  MetOp-A IASI (all channels)        & \texttt{iasi8461\_metop-a}   &  yes     &  yes       \\
  MetOp-A IASI Band 1                & \texttt{iasiB1\_metop-a}     &  yes     &  yes       \\
  MetOp-A IASI Band 2                & \texttt{iasiB2\_metop-a}     &  yes     &  yes       \\
  MetOp-A IASI Band 3                & \texttt{iasiB3\_metop-a}     &  yes     &  yes       \\
  MetOp-B IASI (300ch. subset)       & \texttt{iasi300\_metop-b}    &  yes     &  yes       \\
  MetOp-B IASI (316ch. subset)       & \texttt{iasi316\_metop-b}    &  yes     &  yes       \\
  MetOp-B IASI (616ch. subset)       & \texttt{iasi616\_metop-b}    &  yes     &  yes       \\
  MetOp-B IASI (all channels)        & \texttt{iasi8461\_metop-b}   &  yes     &  yes       \\
  MetOp-B IASI Band 1                & \texttt{iasiB1\_metop-b}     &  yes     &  yes       \\
  MetOp-B IASI Band 2                & \texttt{iasiB2\_metop-b}     &  yes     &  yes       \\
  MetOp-B IASI Band 3                & \texttt{iasiB3\_metop-b}     &  yes     &  yes       \\
  GOES-08 Imager                     & \texttt{imgr\_g08}           &  yes     &  yes       \\
  GOES-09 Imager                     & \texttt{imgr\_g09}           &  yes     &  yes       \\
  GOES-10 Imager                     & \texttt{imgr\_g10}           &  yes     &  yes       \\
  GOES-11 Imager                     & \texttt{imgr\_g11}           &  yes     &  yes       \\
  GOES-12 Imager                     & \texttt{imgr\_g12}           &  yes     &  yes       \\
  GOES-13 Imager                     & \texttt{imgr\_g13}           &  yes     &  yes       \\
  GOES-14 Imager                     & \texttt{imgr\_g14}           &  yes     &  yes       \\
  GOES-15 Imager                     & \texttt{imgr\_g15}           &  yes     &  yes       \\
  MTSAT-1R Imager                    & \texttt{imgr\_mt1r}          &  yes     &  yes       \\
  MTSAT-2 Imager                     & \texttt{imgr\_mt2}           &  yes     &  yes       \\
  Fengyun-3a IRAS                    & \texttt{iras\_fy3a}          &  yes     &  yes       \\
  Fengyun-3b IRAS                    & \texttt{iras\_fy3b}          &  yes     &  yes       \\
  Megha-Tropiques MADRAS             & \texttt{madras\_meghat}      &  yes     &  yes       \\
  Fengyun-3a MERSI                   & \texttt{mersi\_fy3a}         &  yes     &  yes       \\
  NOAA-18 MHS                        & \texttt{mhs\_n18}            &  yes     &  yes       \\
  NOAA-19 MHS                        & \texttt{mhs\_n19}            &  yes     &  yes       \\
  MetOp-A MHS                        & \texttt{mhs\_metop-a}        &  yes     &  yes       \\
  MetOp-B MHS                        & \texttt{mhs\_metop-b}        &  yes     &  yes       \\
  MetOp-C MHS                        & \texttt{mhs\_metop-c}        &  yes     &  yes       \\
  COMS-1 MI (low patch)              & \texttt{mi-l\_coms}          &  yes     &  yes       \\
  COMS-1 MI (medium patch)           & \texttt{mi-m\_coms}          &  yes     &  yes       \\
  Aqua MODIS                         & \texttt{modis\_aqua}         &  yes     &  yes       \\
  Terra MODIS                        & \texttt{modis\_terra}        &  yes     &  yes       \\
  TIROS-N MSU                        & \texttt{msu\_tirosn}         &  yes     &  yes       \\
  NOAA-06 MSU                        & \texttt{msu\_n06}            &  yes     &  yes       \\
  NOAA-07 MSU                        & \texttt{msu\_n07}            &  yes     &  yes       \\
  NOAA-08 MSU                        & \texttt{msu\_n08}            &  yes     &  yes       \\
  NOAA-09 MSU                        & \texttt{msu\_n09}            &  yes     &  yes       \\
  NOAA-10 MSU                        & \texttt{msu\_n10}            &  yes     &  yes       \\
  NOAA-11 MSU                        & \texttt{msu\_n11}            &  yes     &  yes       \\
  NOAA-12 MSU                        & \texttt{msu\_n12}            &  yes     &  yes       \\
  NOAA-14 MSU                        & \texttt{msu\_n14}            &  yes     &  yes       \\
  Meteosat-3 MVIRI (backup)          & \texttt{mviriBKUP\_m03}      &  no      &  yes       \\
  Meteosat-4 MVIRI (backup)          & \texttt{mviriBKUP\_m04}      &  no      &  yes       \\
  Meteosat-5 MVIRI (backup)          & \texttt{mviriBKUP\_m05}      &  no      &  yes       \\
  Meteosat-6 MVIRI (backup)          & \texttt{mviriBKUP\_m06}      &  no      &  yes       \\
  Meteosat-7 MVIRI (backup)          & \texttt{mviriBKUP\_m07}      &  no      &  yes       \\
  Meteosat-3 MVIRI (nominal)         & \texttt{mviriNOM\_m03}       &  no      &  yes       \\
  Meteosat-4 MVIRI (nominal)         & \texttt{mviriNOM\_m04}       &  no      &  yes       \\
  Meteosat-5 MVIRI (nominal)         & \texttt{mviriNOM\_m05}       &  no      &  yes       \\
  Meteosat-6 MVIRI (nominal)         & \texttt{mviriNOM\_m06}       &  no      &  yes       \\
  Meteosat-7 MVIRI (nominal)         & \texttt{mviriNOM\_m07}       &  no      &  yes       \\
  Fengyun-3a MWHS                    & \texttt{mwhs\_fy3a}          &  yes     &  yes       \\
  Fengyun-3b MWHS                    & \texttt{mwhs\_fy3b}          &  yes     &  yes       \\
  Fengyun-3a MWRI                    & \texttt{mwri\_fy3a}          &  yes     &  yes       \\
  Fengyun-3b MWRI                    & \texttt{mwri\_fy3b}          &  yes     &  yes       \\
  Fengyun-3a MWTS                    & \texttt{mwts\_fy3a}          &  yes     &  yes       \\
  Fengyun-3b MWTS                    & \texttt{mwts\_fy3b}          &  yes     &  yes       \\
  Megha-Tropiques SAPHIR             & \texttt{saphir\_meghat}      &  yes     &  yes       \\
  Meteosat-08 SEVIRI                 & \texttt{seviri\_m08}         &  yes     &  yes       \\
  Meteosat-09 SEVIRI                 & \texttt{seviri\_m09}         &  yes     &  yes       \\
  Meteosat-10 SEVIRI                 & \texttt{seviri\_m10}         &  yes     &  yes       \\
  GOES-10 Sounder (Detector 1)       & \texttt{sndrD1\_g10}         &  yes     &  yes       \\
  GOES-10 Sounder (Detector 2)       & \texttt{sndrD2\_g10}         &  yes     &  yes       \\
  GOES-10 Sounder (Detector 3)       & \texttt{sndrD3\_g10}         &  yes     &  yes       \\
  GOES-10 Sounder (Detector 4)       & \texttt{sndrD4\_g10}         &  yes     &  yes       \\
  GOES-11 Sounder (Detector 1)       & \texttt{sndrD1\_g11}         &  yes     &  yes       \\
  GOES-11 Sounder (Detector 2)       & \texttt{sndrD2\_g11}         &  yes     &  yes       \\
  GOES-11 Sounder (Detector 3)       & \texttt{sndrD3\_g11}         &  yes     &  yes       \\
  GOES-11 Sounder (Detector 4)       & \texttt{sndrD4\_g11}         &  yes     &  yes       \\
  GOES-12 Sounder (Detector 1)       & \texttt{sndrD1\_g12}         &  yes     &  yes       \\
  GOES-12 Sounder (Detector 2)       & \texttt{sndrD2\_g12}         &  yes     &  yes       \\
  GOES-12 Sounder (Detector 3)       & \texttt{sndrD3\_g12}         &  yes     &  yes       \\
  GOES-12 Sounder (Detector 4)       & \texttt{sndrD4\_g12}         &  yes     &  yes       \\
  GOES-13 Sounder (Detector 1)       & \texttt{sndrD1\_g13}         &  yes     &  yes       \\
  GOES-13 Sounder (Detector 2)       & \texttt{sndrD2\_g13}         &  yes     &  yes       \\
  GOES-13 Sounder (Detector 3)       & \texttt{sndrD3\_g13}         &  yes     &  yes       \\
  GOES-13 Sounder (Detector 4)       & \texttt{sndrD4\_g13}         &  yes     &  yes       \\
  GOES-14 Sounder (Detector 1)       & \texttt{sndrD1\_g14}         &  yes     &  yes       \\
  GOES-14 Sounder (Detector 2)       & \texttt{sndrD2\_g14}         &  yes     &  yes       \\
  GOES-14 Sounder (Detector 3)       & \texttt{sndrD3\_g14}         &  yes     &  yes       \\
  GOES-14 Sounder (Detector 4)       & \texttt{sndrD4\_g14}         &  yes     &  yes       \\
  GOES-15 Sounder (Detector 1)       & \texttt{sndrD1\_g15}         &  yes     &  yes       \\
  GOES-15 Sounder (Detector 2)       & \texttt{sndrD2\_g15}         &  yes     &  yes       \\
  GOES-15 Sounder (Detector 3)       & \texttt{sndrD3\_g15}         &  yes     &  yes       \\
  GOES-15 Sounder (Detector 4)       & \texttt{sndrD4\_g15}         &  yes     &  yes       \\
  GOES-08 Sounder                    & \texttt{sndr\_g08}           &  yes     &  yes       \\
  GOES-09 Sounder                    & \texttt{sndr\_g09}           &  yes     &  yes       \\
  GOES-10 Sounder                    & \texttt{sndr\_g10}           &  yes     &  yes       \\
  GOES-11 Sounder                    & \texttt{sndr\_g11}           &  yes     &  yes       \\
  GOES-12 Sounder                    & \texttt{sndr\_g12}           &  yes     &  yes       \\
  GOES-13 Sounder                    & \texttt{sndr\_g13}           &  yes     &  yes       \\
  GOES-14 Sounder                    & \texttt{sndr\_g14}           &  yes     &  yes       \\
  GOES-15 Sounder                    & \texttt{sndr\_g15}           &  yes     &  yes       \\
  DMSP-08 SSM/I                      & \texttt{ssmi\_f08}           &  yes     &  yes       \\
  DMSP-10 SSM/I                      & \texttt{ssmi\_f10}           &  yes     &  yes       \\
  DMSP-11 SSM/I                      & \texttt{ssmi\_f11}           &  yes     &  yes       \\
  DMSP-13 SSM/I                      & \texttt{ssmi\_f13}           &  yes     &  yes       \\
  DMSP-14 SSM/I                      & \texttt{ssmi\_f14}           &  yes     &  yes       \\
  DMSP-15 SSM/I                      & \texttt{ssmi\_f15}           &  yes     &  yes       \\
  DMSP-16 SSMIS                      & \texttt{ssmis\_f16}          &  yes     &  yes       \\
  DMSP-17 SSMIS                      & \texttt{ssmis\_f17}          &  yes     &  yes       \\
  DMSP-18 SSMIS                      & \texttt{ssmis\_f18}          &  yes     &  yes       \\
  DMSP-19 SSMIS                      & \texttt{ssmis\_f19}          &  yes     &  yes       \\
  DMSP-20 SSMIS                      & \texttt{ssmis\_f20}          &  yes     &  yes       \\
  DMSP-13 SSM/T-1                    & \texttt{ssmt1\_f13}          &  yes     &  yes       \\
  DMSP-15 SSM/T-1                    & \texttt{ssmt1\_f15}          &  yes     &  yes       \\
  DMSP-14 SSM/T-2                    & \texttt{ssmt2\_f14}          &  yes     &  yes       \\
  DMSP-15 SSM/T-2                    & \texttt{ssmt2\_f15}          &  yes     &  yes       \\
  TIROS-N SSU                        & \texttt{ssu\_tirosn}         &  yes     &  yes       \\ 
  NOAA-06 SSU                        & \texttt{ssu\_n06}            &  yes     &  yes       \\
  NOAA-07 SSU                        & \texttt{ssu\_n07}            &  yes     &  yes       \\
  NOAA-08 SSU                        & \texttt{ssu\_n08}            &  yes     &  yes       \\
  NOAA-09 SSU                        & \texttt{ssu\_n09}            &  yes     &  yes       \\
  NOAA-11 SSU                        & \texttt{ssu\_n11}            &  yes     &  yes       \\
  NOAA-14 SSU                        & \texttt{ssu\_n14}            &  yes     &  yes       \\
  TRMM TMI                           & \texttt{tmi\_trmm}           &  yes     &  yes       \\
  GOES-R ABI (visible)               & \texttt{v.abi\_gr}           &  yes     &  no        \\
  NOAA-15 AVHRR/3 (visible)          & \texttt{v.avhrr3\_n15}       &  yes     &  no        \\
  NOAA-16 AVHRR/3 (visible)          & \texttt{v.avhrr3\_n16}       &  yes     &  no        \\
  NOAA-17 AVHRR/3 (visible)          & \texttt{v.avhrr3\_n17}       &  yes     &  no        \\
  NOAA-18 AVHRR/3 (visible)          & \texttt{v.avhrr3\_n18}       &  yes     &  no        \\
  NOAA-19 AVHRR/3 (visible)          & \texttt{v.avhrr3\_n19}       &  yes     &  no        \\
  MetOp-A AVHRR/3 (visible)          & \texttt{v.avhrr3\_metop-a}   &  yes     &  no        \\
  MetOp-B AVHRR/3 (visible)          & \texttt{v.avhrr3\_metop-b}   &  yes     &  no        \\
  GOES-11 Imager (visible)           & \texttt{v.imgr\_g11}         &  yes     &  no        \\
  GOES-12 Imager (visible)           & \texttt{v.imgr\_g12}         &  yes     &  no        \\
  GOES-13 Imager (visible)           & \texttt{v.imgr\_g13}         &  yes     &  no        \\
  GOES-14 Imager (visible)           & \texttt{v.imgr\_g14}         &  yes     &  no        \\
  GOES-15 Imager (visible)           & \texttt{v.imgr\_g15}         &  yes     &  no        \\
  MTSAT-2 Imager (visible)           & \texttt{v.imgr\_mt2}         &  yes     &  no        \\
  Aqua MODIS (visible)               & \texttt{v.modis\_aqua}       &  yes     &  no        \\
  Terra MODIS (visible)              & \texttt{v.modis\_terra}      &  yes     &  no        \\
  Meteosat-08 SEVIRI (visible)       & \texttt{v.seviri\_m08}       &  yes     &  no        \\
  Meteosat-09 SEVIRI (visible)       & \texttt{v.seviri\_m09}       &  yes     &  no        \\
  Meteosat-10 SEVIRI (visible)       & \texttt{v.seviri\_m10}       &  yes     &  no        \\
  NPP VIIRS Imager, HiRes (visible)  & \texttt{v.viirs-i\_npp}      &  yes     &  no        \\
  NPP VIIRS Imager, ModRes (visible) & \texttt{v.viirs-m\_npp}      &  yes     &  no        \\
  GOES-4 VAS                         & \texttt{vas\_g04}            &  no      &  yes       \\
  GOES-5 VAS                         & \texttt{vas\_g05}            &  no      &  yes       \\
  GOES-6 VAS                         & \texttt{vas\_g06}            &  no      &  yes       \\
  GOES-7 VAS                         & \texttt{vas\_g07}            &  no      &  yes       \\
  NPP VIIRS Imager, HiRes            & \texttt{viirs-i\_npp}        &  yes     &  yes       \\
  NPP VIIRS Imager, ModRes           & \texttt{viirs-m\_npp}        &  yes     &  yes       \\
  Fengyun-3a VIRR                    & \texttt{virr\_fy3a}          &  yes     &  yes       \\
  GMS-5 VISSR (Detector A)           & \texttt{vissrDetA\_gms5}     &  yes     &  yes       \\
  GMS-5 VISSR (Detector B)           & \texttt{vissrDetB\_gms5}     &  no      &  yes       \\
  Kalpana-1 VHRR                     & \texttt{vhrr\_kalpana1}      &  yes     &  yes       \\
  ITOS VTPR-S1                       & \texttt{vtprS1\_itos}        &  yes     &  yes       \\
  ITOS VTPR-S2                       & \texttt{vtprS2\_itos}        &  yes     &  yes       \\
  ITOS VTPR-S3                       & \texttt{vtprS3\_itos}        &  yes     &  yes       \\
  ITOS VTPR-S4                       & \texttt{vtprS4\_itos}        &  yes     &  yes       \\
  Coriolis WindSat                   & \texttt{windsat\_coriolis}   &  yes     &  yes       \\

\end{longtable}
\end{center}

%  \chapter{Utility Software Description}
%=====================================

\clearpage
\section{\f{Type\_Kinds} Module}
%===============================
\label{sec:utility_type_kinds}


\clearpage
\section{\f{File\_Utility} Module}
%=================================
\label{sec:utility_file_utility}


\clearpage
\section{\f{Message\_Handler} Module}
%====================================
\label{sec:utility_message_handler}

\end{appendix}

\end{document}

