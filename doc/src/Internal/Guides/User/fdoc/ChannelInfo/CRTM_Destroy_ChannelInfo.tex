\subsection{\texttt{CRTM\_Destroy\_ChannelInfo} interface}
  \label{sec:CRTM_Destroy_ChannelInfo_interface}
  \begin{alltt}
 
  NAME:
        CRTM_Destroy_ChannelInfo
  
  PURPOSE:
        Function to re-initialize the scalar and pointer members of a CRTM
        ChannelInfo data structures.
 
  CALLING SEQUENCE:
        Error_Status = CRTM_Destroy_ChannelInfo( ChannelInfo            , &
                                                 Message_Log=Message_Log  )
  
  OUTPUT ARGUMENTS:
        ChannelInfo:  Re-initialized ChannelInfo structure.
                      UNITS:      N/A
                      TYPE:       CRTM_ChannelInfo_type
                      DIMENSION:  Scalar
                      ATTRIBUTES: INTENT(IN OUT)
 
  OPTIONAL INPUT ARGUMENTS:
        Message_Log:  Character string specifying a filename in which any
                      Messages will be logged. If not specified, or if an
                      error occurs opening the log file, the default action
                      is to output Messages to standard output.
                      UNITS:      N/A
                      TYPE:       CHARACTER(*)
                      DIMENSION:  Scalar
                      ATTRIBUTES: INTENT(IN), OPTIONAL
 
  FUNCTION RESULT:
        Error_Status: The return value is an integer defining the error status.
                      The error codes are defined in the ERROR_HANDLER module.
                      If == SUCCESS the structure re-initialisation was successful
                         == FAILURE - an error occurred, or
                                    - the structure internal allocation counter
                                      is not equal to zero (0) upon exiting this
                                      function. This value is incremented and
                                      decremented for every structure allocation
                                      and deallocation respectively.
                      UNITS:      N/A
                      TYPE:       INTEGER
                      DIMENSION:  Scalar
 
  COMMENTS:
        Note the INTENT on the output ChannelInfo argument is IN OUT rather than
        just OUT. This is necessary because the argument may be defined upon
        input. To prevent memory leaks, the IN OUT INTENT is a must.
 
  \end{alltt}
