\subsection{\texttt{CRTM\_Destroy} interface}
  \label{sec:CRTM_Destroy_interface}
  \begin{alltt}
 
  NAME:
        CRTM_Destroy
 
  PURPOSE:
        Function to deallocate all the shared data arrays allocated and
        populated during the CRTM initialization.
 
  CALLING SEQUENCE:
        Error_Status = CRTM_Destroy( ChannelInfo            , &
                                     Process_ID = Process_ID  )
 
  OUTPUTS:
        ChannelInfo:  Reinitialized ChannelInfo structure.
                      UNITS:      N/A
                      TYPE:       CRTM_ChannelInfo_type
                      DIMENSION:  Rank-1
                      ATTRIBUTES: INTENT(IN OUT)
 
  OPTIONAL INPUTS:
        Process_ID:   Set this argument to the MPI process ID that this
                      function call is running under. This value is used
                      solely for controlling message output. If MPI is not
                      being used, ignore this argument.
                      UNITS:      N/A
                      TYPE:       INTEGER
                      DIMENSION:  Scalar
                      ATTRIBUTES: INTENT(IN), OPTIONAL
 
  FUNCTION RESULT:
        Error_Status: The return value is an integer defining the error
                      status. The error codes are defined in the
                      Message_Handler module.
                      If == SUCCESS the CRTM deallocations were successful
                         == FAILURE an unrecoverable error occurred.
                      UNITS:      N/A
                      TYPE:       INTEGER
                      DIMENSION:  Scalar
 
  SIDE EFFECTS:
        All CRTM shared data arrays and structures are deallocated.
 
  COMMENTS:
        Note the INTENT on the output ChannelInfo argument is IN OUT rather than
        just OUT. This is necessary because the argument may be defined upon
        input. To prevent memory leaks, the IN OUT INTENT is a must.
 
  \end{alltt}
