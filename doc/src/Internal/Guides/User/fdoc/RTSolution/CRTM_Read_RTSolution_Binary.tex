\subsection{\texttt{CRTM\_Read\_RTSolution\_Binary} interface}
  \label{sec:CRTM_Read_RTSolution_Binary_interface}
  \begin{alltt}
 
  NAME:
        CRTM_Read_RTSolution_Binary
 
  PURPOSE:
        Function to read Binary format CRTM RTSolution structure files.
 
  CALLING SEQUENCE:
        Error_Status = CRTM_Read_RTSolution_Binary( Filename               , &
                                                    RTSolution             , &
                                                    Quiet      =Quiet      , &
                                                    n_Channels =n_Channels , &
                                                    n_Profiles =n_Profiles , &
                                                    RCS_Id     =RCS_Id     , &
                                                    Message_Log=Message_Log  )
 
  INPUT ARGUMENTS:
        Filename:     Character string specifying the name of an
                      RTSolution format data file to read.
                      UNITS:      N/A
                      TYPE:       CHARACTER(*)
                      DIMENSION:  Scalar
                      ATTRIBUTES: INTENT(IN)
 
  OUTPUT ARGUMENTS:
        RTSolution:   Structure array containing the RTSolution data. Note 
                      the rank is CHANNELS x PROFILES.
                      UNITS:      N/A
                      TYPE:       CRTM_RTSolution_type
                      DIMENSION:  Rank-2 (L x M)
                      ATTRIBUTES: INTENT(IN OUT)
 
 
  OPTIONAL INPUT ARGUMENTS:
        Quiet:        Set this argument to suppress INFORMATION messages
                      being printed to standard output (or the message
                      log file if the Message_Log optional argument is
                      used.) By default, INFORMATION messages are printed.
                      If QUIET = 0, INFORMATION messages are OUTPUT.
                         QUIET = 1, INFORMATION messages are SUPPRESSED.
                      UNITS:      N/A
                      TYPE:       INTEGER
                      DIMENSION:  Scalar
                      ATTRIBUTES: INTENT(IN), OPTIONAL
 
        Message_Log:  Character string specifying a filename in which any
                      messages will be logged. If not specified, or if an
                      error occurs opening the log file, the default action
                      is to output messages to standard output.
                      UNITS:      N/A
                      TYPE:       CHARACTER(*)
                      DIMENSION:  Scalar
                      ATTRIBUTES: INTENT(IN), OPTIONAL
 
  OPTIONAL OUTPUT ARGUMENTS:
        n_Channels:   The number of channels for which data was read.
                      UNITS:      N/A
                      TYPE:       INTEGER
                      DIMENSION:  Scalar
                      ATTRIBUTES: OPTIONAL, INTENT(OUT)
 
        n_Profiles:   The number of profiles for which data was read.
                      UNITS:      N/A
                      TYPE:       INTEGER
                      DIMENSION:  Scalar
                      ATTRIBUTES: OPTIONAL, INTENT(OUT)
 
        RCS_Id:       Character string containing the version control Id
                      field for the module.
                      UNITS:      N/A
                      TYPE:       CHARACTER(*)
                      DIMENSION:  Scalar
                      ATTRIBUTES: OPTIONAL, INTENT(OUT)
 
  FUNCTION RESULT:
        Error_Status: The return value is an integer defining the error status.
                      The error codes are defined in the Message_Handler module.
                      If == SUCCESS the Binary file read was successful
                         == FAILURE an unrecoverable error occurred.
                      UNITS:      N/A
                      TYPE:       INTEGER
                      DIMENSION:  Scalar
 
  COMMENTS:
        Note the INTENT on the output RTSolution argument is IN OUT rather
        than just OUT. This is necessary because the argument may be defined on
        input. To prevent memory leaks, the IN OUT INTENT is a must.
 
  \end{alltt}
