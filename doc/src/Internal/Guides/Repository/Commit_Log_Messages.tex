\chapter{Commit Log Messages}
%============================
The purpose of this section is to describe the convention for commit log message formats. This may seem overly meticulous, but the goal is to use the repository commit messages to form the change log for CRTM releases. The change log should show the history of the devleopment of the CRTM and as more developers contribute to the CRTM directly by committing to the repository, the log message format should not differ from one developer to the next.

It is conceivable that at some point in the future the subversion log outputs will be automatically processed via a script to create the change log file for distribution with a CRTM release--or posting on a web page--so developers should endeavour to adhere to this formatting standard so as make parsing the log output easier.

Nearly all of the advice and format descriptions in this section are either taken directly or paraphrased from either the \href{http://www.gnu.org/prep/standards/html_node/Change-Logs.html#Change-Logs}{Change Logs section} of the GNU Coding Standards, \citet{GNU_Coding_Standards}, or the \href{http://www.gnu.org/software/guile/changelogs/changelogs.html}{Change Log Guidelines section} of the GNU guile project \citet{guile_home}.

Some generic points for good log messages (taken from \citep{guile_home}) are:
\begin{enumerate}
  \item Log messages should consist of complete sentences, not fragments. Sentence fragments can be ambiguous. Fragments like ``Initial commit'' for a new file, or ``Added function'' for a new function are acceptable, because they are standard idioms.
  \item Log messages should mention every file changed, as well as mention by name every function and/or subroutine changed. Some common sense exceptions,
  \begin{itemize}
    \item For trivial changes (e.g. renaming a variable), all affected procedures do not have to be listed.
    \item For a complete rewrite of a file, a log entry description such as ``Rewritten'' is acceptable.
  \end{itemize}
  \item Group log message entries in ``paragraphs'', where each paragraph describes a set of changes with a single goal.
  \item Do not abbreviate filenames or procedure names. It makes the log message output difficult to search for changes to these files and procedures.
\end{enumerate}

Specific formats requirements with examples follow.


\section{Log message format}
%---------------------------
An example of a log message format for a CRTM commit is shown in figure \ref{fig:trunk_commit_log_format}, starting with a header line that describes the CRTM category (in this case \texttt{src}, but see table \ref{tab:trunk_category_description} for all the current categories) and the relative source file location (here \texttt{Utility/InstrumentInfo/SpcCoeff}), followed by descriptions of the changes being committed.

\begin{figure}[htp]
  \centering
  \doublebox{
  \begin{minipage}[b]{6.5in}
    \begin{ttfamily}
      \begin{verbatim}
      
  src:Utility/InstrumentInfo/SpcCoeff subdirectory.
  * SpcCoeff_Define.f90 (Associated_SpcCoeff): Removed Skip_AC optional argument.
    (Assign_SpcCoeff): Removed Skip_AC actual argument in call to Associated_SpcCoeff.
      \end{verbatim}
    \end{ttfamily}
  \end{minipage}
  }
  \caption{Commit log message format for a commit to the \texttt{trunk}.}
  \label{fig:trunk_commit_log_format}
\end{figure}

Each entry is bulleted using the ``*'' character, followed by the filename (or list of filenames). Functions and subroutines are surrounded by parentheses. Always use the specific procedure name in the source code, not the generic (or overloaded) procedure name. Additionally, if similar changes were made to many procedures such that the list doesn't fit on a single line, close the parentheses before the line break and reopen them on the next line continuing with the procedure list. This makes the modified procedures easier to search for in the log messages\footnote{A lesson the author learned the hard way as you will undoubtedly encounter log messages that do not do this and these cases tend to break simple searching commands or scripts.}

An example of a multiple entry log message is shown in figure \ref{fig:trunk_commit_log_format_multi_entry}. Note the separate $<$\textit{category}$>$:$<$\textit{directory}$>$ header lines

\begin{figure}[htp]
  \centering
  \doublebox{
  \begin{minipage}[b]{6.5in}
    \begin{ttfamily}
      \begin{verbatim}
  src:Statistics/FitStats subdirectory.
  * FitStats_Define.f90: Made the maximum number of predictors a public entity.
  * FitStats_netCDF.f90: Major rewrite. The various netCDF utility modules are no
    longer used.

  src:Statistics/FitStats/Test_FitStats subdirectory.
  * Makefile, make.dependencies: Updated to reflect changes to the FitStats_netCDF
    module.
  * Test_FitStats.f90: Decreased the number of loops used in the memory leak checks for
    use with valgrind.

  src:Statistics/FitStats/FitStats_ASCII2NC subdirectory.
  * FitStats_ASCII2NC.f90: Modified for use with microwave statistics files where there
    is no ozone component.
  * Makefile, make.dependencies: Updated to reflect changes in the main FitStats modules.
      \end{verbatim}
    \end{ttfamily}
  \end{minipage}
  }
  \caption{Multiple entry commit log message format for a commit to the \f{trunk}.}
  \label{fig:trunk_commit_log_format_multi_entry}
\end{figure}


\section{Branch creation log message format}
%-------------------------------------------
When a branch is initially created the log message should state the branch name, and also identify the revision and source from which it was created. The log message for the creation of the CRTM v1.2 release branch is shown in figure \ref{fig:branch_create_log_format}.
\begin{figure}[htp]
  \centering
  \doublebox{
  \begin{minipage}[b]{6.5in}
    \begin{ttfamily}
      \begin{verbatim}
      
RB-1.2 branch. Created from r2376 trunk.
      \end{verbatim}
    \end{ttfamily}
  \end{minipage}
  }
  \caption{Commit log message format when creating a branch.}
  \label{fig:branch_create_log_format}
\end{figure}

As is clear, the name of the branch is \f{RB-1.2} and it was created from revision 2376 of the trunk. It is useful to list the source since a branch \textit{may} be created from another branch, although this practice is generally discouraged.


\section{Branch commit log message format}
%-----------------------------------------
When committing to a branch, the log message format is the same as for the trunk, \textit{except} that the branch name should \textit{always} be listed first. Doing this allows searching of the log messages for all instances of commits to a particular branch. An example of a branch commit log message is shown in figure \ref{fig:branch_commit_log_format}.
\begin{figure}[htp]
  \centering
  \doublebox{
  \begin{minipage}[b]{6.5in}
    \begin{ttfamily}
      \begin{verbatim}
      
RB-1.2 branch.
  src:Surface subdirectory.
  * CRTM_Surface_Binary_IO.f90 (Read_Surface_Record, Write_Surface_Record): Updated I/O
    statements that contained references to the SensorData structure components to be
    consistent with the structure definition updates from r2572.
      \end{verbatim}
    \end{ttfamily}
  \end{minipage}
  }
  \caption{Commit log message format for a commit to a branch, in this case the RB-1.2 branch.}
  \label{fig:branch_commit_log_format}
\end{figure}

\section{Merge log message format}
%---------------------------------
When merging changes from a branch to the trunk (or vice versa), the range of revisions merged should be specified in the log message. An example of a merge log message format is shown in figure \ref{fig:merge_commit_log_format}

\begin{figure}[htp]
  \centering
  \doublebox{
  \begin{minipage}[b]{6.5in}
    \begin{ttfamily}
      \begin{verbatim}
      
Merged RB-1.2 branch r2377:2444 into the trunk.
      \end{verbatim}
    \end{ttfamily}
  \end{minipage}
  }
  \caption{Commit log message format for a merge from a branch, in this case the RB-1.2 branch, to the trunk}
  \label{fig:merge_commit_log_format}
\end{figure}

Thus, the log message contains a record of what was merged, what revisions were merged, and what they were merged into. Inspection of the log messages informs developers at what revisions future merges should begin (in the example of figure \ref{fig:merge_commit_log_format}, that would be r2445).

