\section{Introduction}

Currently there are two separate atmospheric optics structures: \f{AtmAbsorption}, used to hold the gaseous absorption optical properties; and \f{AtmScatter}, used to hold the cloud and aerosol absorption and scattering optical properties. In the CRTM, these two structures are filled by their respective application modules, and are them combined into a spearate \f{AtmScatter} structure that is passed to the radiative transfer module. This document details the consolidation of these structures into a generic atmospheric optics structure.

The current definitions for the \f{AtmAbsorption} and \f{AtmScatter} structures are shown in figures \ref{fig:CRTM_AtmAbsorption_type_structure} and \ref{fig:CRTM_AtmScatter_type_structure} respectively. The simplicity of the \f{AtmAbsorption} structure definition will make the structure consolidation itself in the definition modules trivial as the only member of the \f{AtmAbsorption} structure already exist in name in the \f{AtmScatter} structure. The main issue is thus the combination of application modules in the various models (forward, tangent-linear, adjoint, and K-matrix.)

\begin{figure}[htp]
  \centering
  \doublebox{
  \begin{minipage}[b]{6.5in}
    \begin{alltt}
    
  TYPE :: CRTM_AtmAbsorption_type
    INTEGER :: n_Allocates = 0
    ! Dimensions
    INTEGER :: n_Layers = 0  ! K dimension
    ! Structure members
    REAL(fp), DIMENSION(:), POINTER :: Optical_Depth => NULL() ! K
  END TYPE CRTM_AtmAbsorption_type
    \end{alltt}
  \end{minipage}
  }
  \caption{Initial CRTM\_AtmAbsorption\_type structure definition.}
  \label{fig:CRTM_AtmAbsorption_type_structure}
\end{figure}

\begin{figure}[htp]
  \centering
  \doublebox{
  \begin{minipage}[b]{6.5in}
    \begin{alltt}
    
  TYPE :: CRTM_AtmScatter_type
    INTEGER :: n_Allocates = 0
    ! Dimensions
    INTEGER :: n_Layers           = 0  ! K dimension
    INTEGER :: Max_Legendre_Terms = 0  ! Ic dimension
    INTEGER :: n_Legendre_Terms   = 0  ! IcUse dimension
    INTEGER :: Max_Phase_Elements = 0  ! Ip dimension
    INTEGER :: n_Phase_Elements   = 0  ! IpUse dimension
    ! Scalar components
    INTEGER :: lOffset = 0   ! start position in array for Legendre coefficients 
    ! Array components
    REAL(fp), POINTER :: Optical_Depth(:)         => NULL() ! K
    REAL(fp), POINTER :: Single_Scatter_Albedo(:) => NULL() ! K
    REAL(fp), POINTER :: Asymmetry_Factor(:)      => NULL() ! K
    REAL(fp), POINTER :: Delta_Truncation(:)      => NULL() ! K
    REAL(fp), POINTER :: Phase_Coefficient(:,:,:) => NULL() ! 0:Ic x Ip x K
  END TYPE CRTM_AtmScatter_type
    \end{alltt}
  \end{minipage}
  }
  \caption{Initial CRTM\_AtmScatter\_type structure definition.}
  \label{fig:CRTM_AtmScatter_type_structure}
\end{figure}

In the current CRTM, the \f{AtmAbsorption} structure is filled in the \f{CRTM\_Compute\_AtmAbsorption()} family of procedures and the \f{AtmScatter} structures for cloud and aerosol scattering, \f{CloudScatter} and \f{AerosolScatter} respectively, are filled in the \f{CRTM\_Compute\_CloudScatter()} and \f{CRTM\_Compute\_AerosolScatter()} family of procedures. A separate \f{AtmScatter} structure, named \f{AtmOptics}, is generated by combining the \f{AtmAbsorption}, \f{CloudScatter} and \f{AerosolScatter} structure components using the \f{CRTM\_Combine\_AtmOptics()} procedures. This \f{AtmOptics} structure is passed to the radiative transfer procedures.

The goal is to do the above with a generic \f{AtmOptics} structure, combining the various optical properties as required in the application modules themselves. This will simplify the source code as well as minimise the need for separate structure allocations.


