\section{Current (r2664) Methodology}

\subsection{Computations performed in the \f{AtmAbsorption} module}

Aside from the details of the gas absorption model, this is the ``simplest'' computation: that of optical depth. Effectively, the optical depth, $\tau$, for layer $k$ and absorbing consituent $j$ for a spectral channel $l$ is computed using a simple regression equation,
\begin{equation}
  \tau_{k,j,l} = c_{0,k,j,l} + \sum_{i=1}^{N} c_{i,k,j,l} X_{i,k,j,l}
\end{equation}
where the $c_i$ are the regression coefficients determined from fitting channel transmittances with a number of atmospheric profile-based predictors, $X_i$.

The resultant total optical depth profile is loaded into the \f{AtmAbsorption} structure of figure \ref{fig:CRTM_AtmAbsorption_type_structure}


\subsection{Computations performed in the \f{CloudScatter} and \f{AerosolScatter} module}

The cloud optical properties are determined by interpolating a lookup table (LUT) which is a function of cloud type, particle effective radius, and, for microwave frequencies, temperature. For aerosols, the LUT is a function of aerosol type and particle effective radius only. The interpolated quantities are mass extinction coefficient ($k_e$)\footnote{Do not confuse the standard symbol for mass extincton coefficient, $k_e$ with the symbol $k$ used to reference a particular atmospheric layer. The latter symbol has no subscript.}, single scatter albedo ($\varpi$), asymmetry parameter ($g$), and the phase matrix coefficients ($p$).

The volume scattering coefficient, $\beta_s$, is then computed for the $i^{th}$ cloud or aerosol,
\begin{equation}
  \beta_{s,i} = \rho_{i} \varpi_{i} k_{e,i}
\end{equation}
where $\rho$ is either the cloud water content or aerosol concentration in g.m\superscript{-2}, and is accumulated \emph{by layer} for all cloud or aerosol input
\begin{equation}
  \beta_{s,Total} = \sum_{i=1}^{N_s}\beta_{s,i}
\end{equation}
where $N_s$ is either the number of clouds or aerosols in the layer.

The optical depth for both absorption and scattering, $\tau$, is also accumulated,
\begin{equation}
  \tau = \sum_{i=1}^{N_s}\rho_{i}  k_{e,i}
  \label{eqn:tau_scat}
\end{equation}
Since this computation is done for a given layer, the optical depth is the same as the volume extinction coefficient, $\beta_e$. Usually,
\begin{equation}
  \tau = \beta_e \Delta z
\end{equation}
but since we are working with height/thickness independent quantities we can say,
\begin{equation}
  \tau \equiv \beta_e
  \label{eqn:equiv_tau}
\end{equation}
where the summation of the number of clouds or aerosols (as shown in equation \ref{eqn:tau_scat}) is implied.

The total asymmetry factor is also accumulated,
\begin{equation}
  g = \sum_{i=1}^{N_c}g_i \beta_{s,i}
\end{equation}
as is the phase coefficients for the $l^{th}$ Legendre term and $m^{th}$ phase matrix element,
\begin{equation}
  p_{l,m} = \sum_{i=1}^{N_c}p_{i,l,m} \beta_{s,i}
\end{equation}

The asymmetry factor is then normalised with the total volume sacattering coefficients, 
\begin{equation}
  g = \frac{g}{\beta_{s,Total}}
\end{equation}
If the number of phase matrix elements is greater than zero, then the phase coefficients are also normalised,
\begin{equation}
  p_{l,m} = \frac{p_{l,m}}{\beta_{s,Total}}
\end{equation}
unless the number of Legendre terms, $N_L$, is less than two, then the Henyey-Greenstein phase function is used,
\begin{eqnarray}
  p_{1,1} &=& 1.5g\\
  p_{2,1} &=& 0.0
\end{eqnarray}
The final step is the normalisation requirements,
\begin{eqnarray}
  p_{0,1} &=& 0.5\\
  \varpi  &=& \frac{\beta_{s,Total}}{\tau}\\
  \delta  &=& p_{N_L,1}
\end{eqnarray}
The above calculations are performed identically for cloud and aerosol scattering.


\subsection{Combination performed in the \f{AtmOptics} module}

First, the scattering variables are initialised,
\begin{equation}
  \begin{array}{l@{\,=\,}l}
    \overline{\varpi} & 0\vspace{0.25em}\\
    \overline{g}      & 0\vspace{0.25em}\\
    \overline{\delta} & 0\vspace{0.25em}\\
    \overline{p}      & 0
  \end{array}
\end{equation}
If there is no scattering, the optical depth is set to the gas absorption optical depth and control is returned to the calling procedure.

If there is significant scattering, the components are combined as shown below.

\subsubsection{Optical depth}
\begin{equation}
  \tau_{total} = \tau_{gas} + \tau_{cloud} + \tau_{aerosol}
\end{equation}
where
\begin{tabular}{l@{ = }l}
  $\tau_{gas}$     & layer optical depth due to gas absorption,\\
  $\tau_{cloud}$   & layer optical depth due to absorption and scattering by clouds,\\
  $\tau_{aerosol}$ & layer optical depth due to absorption and scattering by aerosols.
\end{tabular}

\subsubsection{Single scatter albedo}
The single scatter albedo, $\varpi$, is defined as
\begin{equation}
  \varpi = \frac{\beta_s}{\beta_e}
  \label{eqn:ssa}
\end{equation}
The volume scattering coefficients, $\beta_s$, can be computed from equation \ref{eqn:ssa} substituting in equation \ref{eqn:equiv_tau},
\begin{eqnarray}
  \beta_s &=     & \varpi \beta_e \nonumber\\
          &\equiv& \varpi \tau
\end{eqnarray}
The total volume scattering coefficient is simply a sum of that due to clouds and aerosols,
\begin{equation}
  \beta_{s,total} = \beta_{s,cloud} + \beta_{s,aerosol}
\end{equation}
The weighted average single scatter albedo is computed using,
\begin{eqnarray}
  \overline{\varpi} &=& \frac{(\varpi\tau)_{cloud} + (\varpi\tau)_{aerosol}}{\tau_{gas} + \tau_{cloud} + \tau_{aerosol}}\nonumber\\
                    &=& \frac{\beta_{s,total}}{\tau_{total}}
\end{eqnarray}


\subsubsection{Phase coefficients, asymmetry factor, and delta truncation}
The phase matrix elements, asymmetry factor and delta truncation factor are then computed differently depending on the type of phase function,\vspace{1em}

\begin{minipage}[t]{3.25in}
  \begin{center}
    \textbf{Specified phase functions}
  \end{center}
  For the $l^{th}$ Legendre term and $i^{th}$ phase element,
  \begin{equation}
    \overline{p}_{l,i} = \frac{(p_{l,i}\beta_{s})_{cloud} + (p_{l,i}\beta_{s})_{aerosol}}{\beta_{s,total}}
  \end{equation}
  and the asymmetry factor and delta truncation are simply,
  \begin{eqnarray}
    \overline{g}      &=& \frac{2\overline{p}_{1,1}}{3}\\
    \overline{\delta} &=& \overline{p}_{L,1}\label{eqn:spf_deltatrunc}
  \end{eqnarray}
\end{minipage}\hfill\vline\hfill
\begin{minipage}[t]{3.25in}
  \begin{center}
    \textbf{Henyey-Greenstein phase function}
  \end{center}
  The HG phase coefficients for the $l^{th}$ Legendre term are,
  \begin{equation}
    \overline{p}_l = \left\{
    \begin{array}{cl}
    0.5 & \text{for } l=0\vspace{0.5em}\\
    \displaystyle\frac{2l+1}{2}\cdot\frac{\overline{g}^l-\overline{\delta}}{1-\overline{\delta}} & \text{for } l=1,...,L-1
    \end{array} \right.
  \end{equation}
  The asymmetry factor is updated using,
  \begin{equation}
    \overline{g} = \frac{\overline{g}-\overline{\delta}}{1-\overline{\delta}}{}{}
  \end{equation}
  and the ``average'' delta truncation is then given by,
  \begin{equation}
    \overline{\delta} = \frac{(g\beta_{s})_{cloud} + (g\beta_{s})_{aerosol}} {\beta_{s,total}}
    \label{eqn:hgpf_deltatrunc}
  \end{equation}
  Note the order of the above calculations: the ``old'' asymmetry factor and delta truncation are used to compute the phase coefficients and updated asymmetry factor.
\end{minipage}

\subsubsection{Delta function adjustment of optical depth and single scatter albedo}
The optical depth and single scatter albedo are then modified for the delta-function adjustment of either equation \ref{eqn:spf_deltatrunc} or \ref{eqn:hgpf_deltatrunc},
\begin{eqnarray}
  \overline{\tau}   &=& (1 - \overline{\delta}\overline{\varpi})\overline{\tau}\\
  \overline{\varpi} &=& \frac{(1 - \overline{\delta})\overline{\varpi}}{1 - \overline{\delta}\overline{\varpi}}
\end{eqnarray}







