\section{Tangent-linear and Adjoint testing of complex square root}
%==================================================================

During testing of the entire emissivity model, the TL/AD tests consistently failed. The problem was tracked to down to the adjoint of the Fresnel reflectivities. Note that the standalone TL/AD Fresnel tests were all successful. 

\subsection{Simple Test Case}
%----------------------------
A simplified test case was run where the reflectivity results used ($r_v$ and $r_h$) were the real and imaginary parts of the square root of the complex permittivity,
\begin{eqnarray}
  z &=& \sqrt{\epsilon}\nonumber\\
  r_v &=& \Re\{z\}\label{eqn:fwd_sqrt}\\
  r_h &=& \Im\{z\}\nonumber
\end{eqnarray}
and implemented in the code as shown in figure \ref{fig:fwd_code_intrinsic}.

\begin{figure}[htp]
  \centering
  \doublebox{
  \begin{minipage}[b]{4in}
    \begin{ttfamily}
      \begin{verbatim}
  COMPLEX :: permittivity, z
  REAL :: rv, rh
  ...
  z  = SQRT(permittivity)
  rv = REAL(z)
  rh = AIMAG(z)\end{verbatim}
    \end{ttfamily}
  \end{minipage}
  }
  \caption{Test case forward implementation using the \texttt{SQRT()} instrinsic function to compute the complex square root.}
  \label{fig:fwd_code_intrinsic}
\end{figure}

The tangent-linear form of equation \ref{eqn:fwd_sqrt} that was used was,
\begin{eqnarray}
  \delta{z} &=& \frac{\de}{2\sqrt{\epsilon}}\nonumber\\
  \delta{r_v} &=& \Re\{\delta{z}\}\label{eqn:tl_sqrt}\\
  \delta{r_h} &=& \Im\{\delta{z}\}\nonumber
\end{eqnarray}
and implemented as shown in figure \ref{fig:tl_code_intrinsic}.

\begin{figure}[htp]
  \centering
  \doublebox{
  \begin{minipage}[b]{4in}
    \begin{ttfamily}
      \begin{verbatim}
  COMPLEX :: permittivity_tl, z_tl
  REAL :: rv_tl, rh_tl
  ...
  z_tl  = 0.5 * permittivity_TL / z
  rv_tl = REAL(z_tl)
  rh_tl = AIMAG(z_tl)\end{verbatim}
    \end{ttfamily}
  \end{minipage}
  }
  \caption{Test case tangent-linear implementation using the \texttt{SQRT()} instrinsic function to compute the complex square root.}
  \label{fig:tl_code_intrinsic}
\end{figure}

The adjoint form of equation \ref{eqn:tl_sqrt} is,
\begin{eqnarray}
  \dstar z &=& \dstar r_v - i\,\dstar r_h\nonumber\\
  \dstar\epsilon &=& \frac{\dstar z}{2\sqrt{\epsilon}}
  \label{eqn:ad_sqrt}
\end{eqnarray}
and implemented in the adjoint code as shown in figure \ref{fig:ad_code_intrinsic}.

\begin{figure}[htp]
  \centering
  \doublebox{
  \begin{minipage}[b]{4in}
    \begin{ttfamily}
      \begin{verbatim}
  REAL :: rv_ad, rh_ad
  COMPLEX :: permittivity_ad, z_ad
  ...
  z_ad = CMPLX(rv_ad,-rh_ad)
  rv_ad = 0.0; rh_ad = 0.0
  permittivity_AD = 0.5 * z_AD / z\end{verbatim}
    \end{ttfamily}
  \end{minipage}
  }
  \caption{Test case adjoint implementation using the \texttt{SQRT()} instrinsic function to compute the complex square root.}
  \label{fig:ad_code_intrinsic}
\end{figure}

Running the FWD/TL and TL/AD tests using the above simplified representations for the Fresnel reflectivities, the FWD/TL tests consistently passed and the TL/AD tests consistently failed. Because the test is relatively simple, it appears that there is some misunderstanding of how the adjoint is being constructed.


\subsection{Using an Explicit Expression for the Complex Square Root}
%--------------------------------------------------------------------

To try to determine the cause of the TL/AD failures the complex square root was computed explicitly using,
\begin{equation}
  \sqrt{x + i\,y} = \sqrt{\frac{r+x}{2}} + i\;\textrm{SIGN}(y)\sqrt{\frac{r-x}{2}}
  \label{eqn:compute_sqrt}
\end{equation}
where
\begin{equation*}
  r = \sqrt{x^2 + y^2}
\end{equation*}
and
\begin{equation*}
  \textrm{SIGN}(y) = \begin{cases}
                       -1 & \textrm{if } y < 0\\
                       \hfill 1 & \textrm{if } y \ge 1
                     \end{cases}
\end{equation*}

Using equation \ref{eqn:compute_sqrt} in equation \ref{eqn:fwd_sqrt}, we get a series of expressions for the simplified test case,
\begin{eqnarray}
  x &=& \Re\{\epsilon\}\nonumber\\
  y &=& \Im\{\epsilon\}\nonumber\\
  r &=& \sqrt{x^2 + y^2}\label{eqn:fwd_sqrt_explicit}\\
  r_v &=& \sqrt{\frac{r+x}{2}}\nonumber\\
  r_h &=& \textrm{SIGN}(y)\sqrt{\frac{r-x}{2}}\nonumber
\end{eqnarray}
These are implemented as,

\begin{figure}[htp]
  \centering
  \doublebox{
  \begin{minipage}[b]{4in}
    \begin{ttfamily}
      \begin{verbatim}
  COMPLEX :: permittivity
  REAL :: x, y, r, sgn
  REAL :: rv, rh
  ...
  x = REAL(permittivity)
  y = AIMAG(permittivity)
  r = SQRT(x**2 + y**2)
  IF ( y < 0.0 ) THEN
    sgn = -1.0
  ELSE
    sgn =  1.0
  END IF
  rv =     SQRT((r+x)/2.0)
  rh = sgn*SQRT((r-x)/2.0)\end{verbatim}
    \end{ttfamily}
  \end{minipage}
  }
  \caption{Test case forward implementation using an explicit expression to compute the complex square root.}
  \label{fig:fwd_code_explicit}
\end{figure}

The tangent-linear form of equation \ref{eqn:fwd_sqrt_explicit} is thus,
\begin{eqnarray}
  \delta{x} &=& \Re\{\de\}\nonumber\\
  \delta{y} &=& \Im\{\de\}\nonumber\\
  \delta{r} &=& \frac{x\,\delta{x} + y\,\delta{y}}{r}\label{eqn:tl_sqrt_explicit}\\
  \delta{r_v} &=& \frac{\delta{r} + \delta{x}}{4\sqrt{\displaystyle\frac{r+x}{2}}}\nonumber\\
  \delta{r_h} &=& \textrm{SIGN}(y)\frac{\delta{r} - \delta{x}}{4\sqrt{\displaystyle\frac{r-x}{2}}}\nonumber
\end{eqnarray}
and are implemented as shown in figure \ref{fig:tl_code_explicit}.

\begin{figure}[htp]
  \centering
  \doublebox{
  \begin{minipage}[b]{4in}
    \begin{ttfamily}
      \begin{verbatim}
  COMPLEX :: permittivity_tl
  REAL :: x_tl, y_tl, r_tl
  REAL :: rv_tl, rh_tl
  ...
  x_tl = REAL(permittivity_tl)
  y_tl = AIMAG(permittivity_tl)
  r_tl = (x*x_tl + y*y_tl)/r
  rv_tl =     (r_tl+x_tl)/(4.0*SQRT((r+x)/2.0))
  rh_tl = sgn*(r_tl-x_tl)/(4.0*SQRT((r-x)/2.0))\end{verbatim}
    \end{ttfamily}
  \end{minipage}
  }
  \caption{Test case tangent-linear implementation using an explicit expression to compute the complex square root.}
  \label{fig:tl_code_explicit}
\end{figure}

The adjoints of the $\delta r_h$ terms in equation \ref{eqn:tl_sqrt_explicit} are,
\begin{eqnarray}
  \dstar x &=& \displaystyle\frac{-\textrm{SIGN}(y)}{4\sqrt{\displaystyle\frac{r-x}{2}}}\, \dstar r_h\nonumber\\
  & & \label{eqn:ad_sqrt_explicit1}\\
  \dstar r &=& \displaystyle\frac{\textrm{SIGN}(y)}{4\sqrt{\displaystyle\frac{r-x}{2}}}\, \dstar r_h\nonumber
\end{eqnarray}
the adjoints of the $\delta r_v$ terms are,
\begin{eqnarray}
  \dstar x &=& \displaystyle\frac{1}{4\sqrt{\displaystyle\frac{r+x}{2}}}\, \dstar r_v\nonumber\\
  & & \label{eqn:ad_sqrt_explicit2}\\
  \dstar r &=& \displaystyle\frac{1}{4\sqrt{\displaystyle\frac{r+x}{2}}}\, \dstar r_v\nonumber
\end{eqnarray}
the adjoints of the $\delta r$ terms are,
\begin{eqnarray}
  \dstar y &=& \displaystyle\frac{y}{r}\, \dstar r\nonumber\\
  & & \label{eqn:ad_sqrt_explicit3}\\
  \dstar x &=& \displaystyle\frac{x}{r}\, \dstar r\nonumber
\end{eqnarray}
with the final adjoint of the $\delta x$ and $\delta y$ terms being
\begin{equation}
  \dstar\epsilon = \dstar x + i\,\dstar y
  \label{eqn:ad_sqrt_explicit4}
\end{equation}
The adjoint equations \ref{eqn:ad_sqrt_explicit1} to \ref{eqn:ad_sqrt_explicit4} are implemented as shown in figure \ref{fig:ad_code_explicit}
\begin{figure}[htp]
  \centering
  \doublebox{
  \begin{minipage}[b]{4in}
    \begin{ttfamily}
      \begin{verbatim}
  REAL :: rv_ad, rh_ad
  REAL :: x_ad, y_ad, r_ad
  COMPLEX :: permittivity_ad
  ...
  ! Adjoint of rh_tl
  x_ad = -sgn/(4.0*SQRT((r-x)/2.0)) * rh_ad
  r_ad =  sgn/(4.0*SQRT((r-x)/2.0)) * rh_ad
  rh_ad = zero
  ! Adjoint of rv_tl
  x_ad = x_ad + 1.0/(4.0*SQRT((r+x)/2.0)) * rv_ad
  r_ad = r_ad + 1.0/(4.0*SQRT((r+x)/2.0)) * rv_ad
  rv_ad = zero
  ! Adjoint of r_tl
  y_ad =        (y/r) * r_ad
  x_ad = x_ad + (x/r) * r_ad
  r_ad = zero
  ! Adjoint of x_tl and y_tl
  permittivity_ad = CMPLX(x_ad,y_ad)\end{verbatim}
    \end{ttfamily}
  \end{minipage}
  }
  \caption{Test case adjoint implementation using an explicit expression to compute the complex square root.}
  \label{fig:ad_code_explicit}
\end{figure}

Running the FWD/TL and TL/AD tests using the simplified representations for the Fresnel reflectivities derived via an explicit expression for the complex square root, both the FWD/TL and TL/AD tests consistently passed.

In addition, when the forward and tangent-linear code was reverted to that using the intrinsic \texttt{SQRT()} form of figures \ref{fig:fwd_code_intrinsic} and \ref{fig:tl_code_intrinsic} respectively - that is, \textit{only} the adjoint code used the explicit form - all the FWD/TL and TL/AD tests still passed.

