\section{Conclusions}
%====================
The low frequency microwave sea surface emissivity model has been shown to be internally consistent across its forward, tangent-linear, and adjoint forms.

Validating the forward/tangent-linear model consistency for frequencies greater than 15GHz proved slightly difficult due to the impact that the noisy ocean height variance LUT data had on the resultant emissivities due to the applied small-scale correction. Smoothing the ocean height variance data did decrease the test residuals, but did not eliminate particular features associated with interpolation across LUT hingepoints. Visual inspection of a selection of FWD/TL residuals for various temperatures, salinities, and wind speeds was required to verify the tests. No rigorous objective method was found that could be applied successfully for all combinations of inputs.

Validation of the tangent-linear/adjoint model was comparatively easy in that all test residuals could be objectively compared to within numerical precision. All model components passed this test. 

The impact of the updated microwave sea surface emissivity model on computed brightness temperatures in the CRTM can be quite large, 20-30K, for those channels that are senstive to the surface. The largest portion of the change is due to the emissivity model in the current v1.1 release of the CRTM being Fastem1 which is known to not handle low frequencies very well. However, for the very lowest frequencies tested, the low frequency model still produces an additional brightness temperature difference from Fastem3 of the order of 4-8K. 

