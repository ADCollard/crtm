\section{CrIS (Cross-track Infrared Sounder)}
%============================================

\subsection{Spectral and interferometric domain transforms}
%----------------------------------------------------------
In generating the line-by-line transmittances for a Fourier Transform Interferometer (FTS), we assume an idealised instrument. The relationships between the spectral frequency interval and bandwidth, {\df} and {\Df}, and the effective interferogram sampling interval and maximum optical path difference, {\dx} and {\Dx}, can be simply expressed by,
\begin{equation}\df = \frac{1}{2.\Dx}\label{eqn:df_deltax}\end{equation}
\begin{equation}\Df = \frac{1}{2.\dx}\label{eqn:deltaf_dx}\end{equation}
(Fully general definitions are more nuanced than this, but for this exercise, the above will suffice. See \citet{Bell_1972} for an exhaustive treatment.)

The relationship between the number of spectral (SPC) and double-sided interferogram (IFG) points is given by,
\begin{equation}N_{IFG} = 2.(N_{SPC}-1)\end{equation}
\begin{equation}N_{SPC} = \left(\frac{N_{IFG}}{2}\right) - 1\end{equation}

These relationships are shown schematically in figure \ref{fig:X_F_defn}. Note that maximum positive abscissa at $x = \Dx$, or $f = f_{2}$, does not have an equivalent negative point pairing.
\begin{figure}[htp]
  \centering
  \input{graphics/X_F_definition.pstex_t}
  \caption{Schematic illustration of interferogram and spectrum point ordering. The circles represent positive delays or frequencies, and the squares negative. (ZPD=Zero Path Difference)}
  \label{fig:X_F_defn}
\end{figure}

\subsection{Band Frequencies}
%----------------------------
It assumed that the CrIS radiances that are to be modeled have been corrected for self-apodisation effects and detector nonlinearities, and have been resampled to the frequency grids shown in table \ref{tab:cris_band_f}.
\begin{table}[htp]
  \centering
  \begin{tabular}{|c|c|c|c|c|c|}
    \hline
    \textbf{Band} & \bfrequency{min} & \bfrequency{max} & \boldmath$\Delta f$\unboldmath & \boldmath$\delta f$\unboldmath & \boldmath$N_{SPC}$\unboldmath       \\
                  & (\invcm)         & (\invcm)         & (\invcm)   & (\invcm)   & \textbf{Number of points} \\
    \hline\hline
    1 (LW) &  650.0 & 1095.0 & 445.0 & 0.625 & 713 \\
    2 (MW) & 1210.0 & 1750.0 & 540.0 & 1.25  & 433 \\
    3 (SW) & 2155.0 & 2550.0 & 395.0 & 2.5   & 159 \\
    \hline
  \end{tabular}
  \caption{Frequency range and spacing for the CrIS bands. From \cite{CrIS_SDR_ATBD}.}
  \label{tab:cris_band_f}
\end{table}
