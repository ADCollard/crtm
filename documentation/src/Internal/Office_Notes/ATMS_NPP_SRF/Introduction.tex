\section{Introduction}
%=====================
This document describes the pre-processing applied to the Suomi National Polar-orbiting Partnership (Suomi-NPP) Advanced Technology Microwave Sounder (ATMS) spectral response functions (SRFs), obtained from \cite{ATMS_SRF_Data}, to prepare for use in the CRTM processing chain. The SRFs are used to generate channel central frequencies, as well as in the convolution of monochromatic quantities such as Planck radiances or line-by-line (LBL) model generated transmittances to produce such things as polychromatic correction coefficients and instrument resolution transmittances. The latter, for a diverse set of atmospheric profiles, are then regressed against a set of predictors to produce the fast transmittance model coefficients used by the CRTM.


\subsection{Conversion from decibel to relative response}
%--------------------------------------------------------
All of the ATMS response data were in units of decibels ($\phi_{dB}$). For use with the LBL model they were converted to a relative response ($\phi_{rel}$) using the following equation,
\begin{equation}
  \phi_{rel} = 10^{\displaystyle(\phi_{dB}-\max(\phi_{dB}))/10}
\end{equation} 


\subsection{Computation of the channel central frequency}
%--------------------------------------------------------
The computed ATMS channel central frequencies, $\nu_0$, are the first moments of the defined SRF,
\begin{equation}
  \nu_0 = \frac{\displaystyle\int{\nu\:\phi_{rel}(\nu)\ud \nu}}{\displaystyle\int{\phi_{rel}(\nu)\ud \nu}}
\end{equation}
For multiple passband channels, each band is numerically integrated and summed to give the total. It should also be noted that all calculations (in the SRF processing, but also in the CRTM itself) are done in frequency units of cm\superscript{-1}. Conversion to units of GHz is done for display purposes only.


\subsection{Computation of polychromatic correction coefficients}
%----------------------------------------------------------------
In the CRTM, the conversion of \emph{channel resolution} radiances to brightness temperatures has to take the passband widths into account. For any channel, the regression relation to be solved is

\begin{equation}
  a_0 + a_1T + \ldots = \frac{\displaystyle k_1}{\displaystyle \ln\left[\frac{k_2}{R(T)}+1\right]} = Y(T)
\end{equation}
where
\begin{equation}
  \begin{array}{r@{\;=\;}l}
         T &\mbox{brightness temperature} \\
        a_j&\mbox{regression coefficients} \\
    k_1,k_2&\mbox{Planck coefficients} \\
       R(T)&\mbox{channel radiance} \\
       Y(T)&\mbox{``effective'' brightness temperature}
  \end{array}
\end{equation}
and the channel radiances used to determine the effective temperatures, $Y(T)$, are computed the usual way
\begin{equation}
  R(T) = \frac{\displaystyle\int{B(T,\nu)\:\phi_{rel}(\nu)\ud \nu}}{\displaystyle\int{\phi_{rel}(\nu)\ud \nu}}
\end{equation}
The quantity minimised to obtain the $a_j$ coefficients is
\begin{equation}
  \left[ \sum_{j=0}^{M}a_j T_{i} - Y(T_{i}) \right]^2 \quad\mbox{for}\quad T_i = 150K, \ldots, 340K \;\mbox{ in 5K steps.}
\end{equation}
Currently the number of coefficients is fixed at two (i.e. $M=1$).
