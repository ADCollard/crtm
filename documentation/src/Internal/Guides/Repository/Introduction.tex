\chapter{Introduction}
%=====================
This document describes the CRTM \subversion{} repository and the \trac{} Software Configuration Management (SCM) and Project Management (PM) system to aid in organising CRTM future development and dealing with current issues.

This guide will discuss how to access the repository and trac pages, and describe how to set up your local environment to allow you to build the CRTM library, and any associated programs, directly in your working copy. It is assumed the user is familiar with \subversion{}.

The CRTM is one of many projects in main EMC repository on the subversion server \texttt{svnemc.ncep.noaa.gov}. The location of the actual CRTM repository is\\ \hspace*{1cm}\href{https://svnemc.ncep.noaa.gov/projects/crtm}{\f{https://svnemc.ncep.noaa.gov/projects/crtm}}.\\
It is this URL that should be used for initial checkouts from the repository, or for subversion operations directly on the repository (e.g. creation of branches or tags).
 
The day-to-day graphical user interface to the CRTM repository is via the CRTM \trac{} webpage at\\ \hspace*{1cm}\href{https://svnemc.ncep.noaa.gov/trac/crtm}{\f{https://svnemc.ncep.noaa.gov/trac/crtm}}.\\
All descriptions of the CRTM repository will be described via the functionality of the \trac{} system.

