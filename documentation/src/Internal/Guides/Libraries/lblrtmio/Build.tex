\chapter{How to build the LBLRTM I/O library}
%============================================
%============================================
\label{chapter:build}

\section{Configuration}
%======================

The LBLRTM I/O tarball directory structure looks like:
\begin{alltt} ./
  |-README  .................. library build instructions
  |-configure  ............... configuration script
  |-Makefile.in  ............. makefile template
  `-libsrc/  ................. library source files\end{alltt}

The build system for the LBLRTM I/O library uses a GNU autoconf-generated configure script.

You run the configuration script like so:

\begin{verbatim}$ ./configure --prefix=install-directory\end{verbatim}

The \f{--prefix} switch sets the installation directory and defaults to \f{/usr/local} so make sure you set it to a directory in which you have write access.

By default, the LBLRTM I/O library is set to read files generated from a double-precision LBLRTM executable (what is used in the CRTM project by default). If you want to read datafiles from a ``standard'' single precision LBLRTM executable, it needs to be enabled in the configuration:

\begin{verbatim}$ ./configure --enable-single --prefix=install-directory\end{verbatim}

The switch between single-and double-precision capability is achieved via preprocessing that redefines the default real and integer kind types. By default the configuration step sets preprocessing macros \f{INT\_SIZE} and \f{REAL\_SIZE} to a value of 8 (indicating byte size) to allow for reading of double-precision files. The \f{--enable-single} switch sets the value of those macros to 4 for use with single-precision files.

Note, however, that the test datafiles provided with the LBLRTM I/O library are double-precision files generated on a little-endian machine. So if you build the single-precision version of the LBLRTM I/O library the associated test will fail.


\subsection{Supported compilers}
%-------------------------------

The LBLRTM I/O library configuration is set up for the following four compilers:
\begin{itemize}
  \item ifort
  \item gfortran
  \item xlf2003
  \item pgf95
\end{itemize}

For these compilers, the configuration file will use the correct compiler switches to promote the single-precision real and integer variables to double-precision by default. For any other compilers, please contact CRTM support\email{ncep.list.emc.jcsda\_crtm.support@noaa.gov} for information on how to get the library built.



\section{Building the library}
%=============================

Once the \f{configure} script has been run successfully, to start building the library simply type

\begin{verbatim}$ make\end{verbatim}



\section{Checking the library build}
%===================================

To run the accompanying tests using the just-built library, simply type

\begin{verbatim}$ make check\end{verbatim}

This will build and run any tests. The current output from the (successful) test runs looks like:
\begin{verbatim}
         **********************************************************
                               check_lblrtmio

          Check/example program for the LBLRTM File I/O functions
          using

          LBLRTM I/O library version: v0.1.0
         **********************************************************


         Test reading a single layer, single panel LBLRTM file...

     LBLRTM_Utility::File_Open(INFORMATION) : Set for DOUBLE-precision LBLRTM files
     LBLRTM_File_IO::Read(INFORMATION) : Reading layer #1...
     LBLRTM_Layer_IO::Read(INFORMATION) :   Reading spectral chunk #1...
     LBLRTM_Panel_IO::Read(INFORMATION) :     Reading spectrum #1...
     LBLRTM_Layer_IO::Read(INFORMATION) :   Reading spectral chunk #2...
     LBLRTM_Panel_IO::Read(INFORMATION) :     Reading spectrum #1...

     ...etc...

          Test reading a single layer, double panel LBLRTM file...

     LBLRTM_Utility::File_Open(INFORMATION) : Set for DOUBLE-precision LBLRTM files
     LBLRTM_File_IO::Read(INFORMATION) : Reading layer #1...
     LBLRTM_Layer_IO::Read(INFORMATION) :   Reading spectral chunk #1...
     LBLRTM_Panel_IO::Read(INFORMATION) :     Reading spectrum #1...
     LBLRTM_Panel_IO::Read(INFORMATION) :     Reading spectrum #2...
     LBLRTM_Layer_IO::Read(INFORMATION) :   Reading spectral chunk #2...
     LBLRTM_Panel_IO::Read(INFORMATION) :     Reading spectrum #1...
     LBLRTM_Panel_IO::Read(INFORMATION) :     Reading spectrum #2...

     ...etc...

         Test reading some layers from a multiple layer, double panel LBLRTM file...

     LBLRTM_Utility::File_Open(INFORMATION) : Set for DOUBLE-precision LBLRTM files
     LBLRTM_File_IO::Read(INFORMATION) : Reading layer #1...
     LBLRTM_Layer_IO::Read(INFORMATION) :   Reading spectral chunk #1...
     LBLRTM_Panel_IO::Read(INFORMATION) :     Reading spectrum #1...
     LBLRTM_Panel_IO::Read(INFORMATION) :     Reading spectrum #2...
     LBLRTM_Layer_IO::Read(INFORMATION) :   Reading spectral chunk #2...
     LBLRTM_File_IO::Read(INFORMATION) : Reading layer #2...
     LBLRTM_Layer_IO::Read(INFORMATION) :   Reading spectral chunk #1...
     LBLRTM_Panel_IO::Read(INFORMATION) :     Reading spectrum #1...
     LBLRTM_Panel_IO::Read(INFORMATION) :     Reading spectrum #2...

     ...etc...

      TEST SUCCESSFUL!
\end{verbatim}

As mentioned previously, the test datafiles used in the above check target are little-endian double-precision datafiles.



\section{Installing the library}
%===============================

To install the library, type:

\begin{verbatim}$ make install\end{verbatim}

Installation of the library \textit{always} occurs into its own directory within the directory specified by the \f{--prefix} switch. The name of the installation directory follows the convention:

\textit{library-name}\f{\_}\textit{version}

So, if a library version (say, v1.0.0) build was configured with \f{--prefix=\$PWD}, then the installation directory will be

\begin{verbatim}${PWD}/lblrtmio_v1.0.0\end{verbatim}



\section{Linking to the library}
%===============================

Let's assume the above install was moved into \f{\${HOME}/local}. To use the library in your own application, the usual environment variables would be modified something like:
\begin{verbatim}
  libroot="${HOME}/local/lblrtmio_v1.0.0"
  FCFLAGS="-I${libroot}/include ${FCFLAGS}"
  LDFLAGS="-L${libroot}/lib ${LDFLAGS}"
  LIBS="-llblrtmio"
\end{verbatim}

(with appropriate syntax changes for \f{csh})



\section{Uninstalling the library}
%=================================

To uninstall the library (assuming you haven't moved the installation directory contents somewhere else) you can type:

\begin{verbatim}$ make uninstall\end{verbatim}

This will \emph{delete} the created installation directory. So, for a library version, say, v1.0.0, if your configure script invocation was something like

\f{\$ ./configure --prefix=\${PWD} }...other command line arguments...

then the \f{uninstall} target will delete the \f{\${PWD}/lblrtmio\_v1.0.0} directory.



\section{Cleaning up}
%====================

Two targets are provided for cleaning up after the build. To remove all of the build products type

\begin{verbatim}$ make clean\end{verbatim}

To also remove all of the configuration products (i.e. the generated \f{makefile}s) type

\begin{verbatim}$ make distclean\end{verbatim}


\section{Feedback and contact information}
%=========================================

That's pretty much it. Any questions or bug reports can be sent to CRTM Support.

\href{mailto:ncep.list.emc.jcsda_crtm.support@noaa.gov}{\f{ncep.list.emc.jcsda\_crtm.support@noaa.gov}}

If you have problems building the library please include the generated \f{config.log} file in your email correspondence.
