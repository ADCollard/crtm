% Preamble
% ========
\documentclass[10pt,letterpaper]{article}

% Define packages to use
\usepackage{natbib}
\usepackage[dvips]{graphicx,color}
\usepackage{amsmath,amssymb}
\usepackage{caption}
\usepackage[dvipdfm,colorlinks,linkcolor=blue,citecolor=blue,urlcolor=blue]{hyperref}

% Redefine default page
\setlength{\textheight}{9in}  % 1" above and below
\setlength{\textwidth}{6.75in}   % 0.5" left and right
\setlength{\oddsidemargin}{-0.25in}

% Redefine default paragraph
\setlength{\parindent}{0pt}
\setlength{\parskip}{1ex plus 0.5ex minus 0.2ex}

% Define caption width and default fonts
\setlength{\captionmargin}{0.5in}
\renewcommand{\captionfont}{\sffamily}
\renewcommand{\captionlabelfont}{\bfseries\sffamily}

% Define commands for super- and subscript in text mode
\newcommand{\superscript}[1]{\ensuremath{^\textrm{#1}}}
\newcommand{\subscript}[1]{\ensuremath{_\textrm{#1}}}


% Document
% ========
\begin{document}
\thispagestyle{empty}

% Heading
\begin{center}
  {\Large\bfseries Overview of the\\Joint Center for Satellite Data Assimilation (JCSDA)\vspace{0.1em}\\Community Radiative Transfer Model (CRTM)}\vspace{1.0em}
  
  {\bfseries Paul F. van Delst\superscript{a,e}, Yong Han\superscript{f}, Quanhua Liu\superscript{b,f}, Fuzhong Weng\superscript{f}, John C. Derber\superscript{e}, Yong Chen\superscript{c,f}, David N. Groff\superscript{a,e}, Ronald Vogel\superscript{d,f}, Banghua Yan\superscript{b,f}, and Tong Zhu\superscript{c,f}}\vspace{1.0em}
  
  a. SAIC\\
  b. Perot Systems Inc.\\
  c. CIRA\\
  d. IMSG\\
  e. NOAA/NWS/NCEP/EMC\\
  f. NOAA/NESDIS/STAR\vspace{1.0em}
  
  Submitted for the 1\superscript{st} Numerical Weather and Climate Modeling Workshop.
\end{center}

% Main text
The CRTM, used in the NCEP/EMC data assimilation systems to simulate satellite radiance observations and their adjoint, is designed around three broad categories: atmospheric optics, surface optics, and radiative transfer.

The atmospheric optics components of the CRTM compute absorption by gaseous constituents, and scattering and absorption by both clouds and aerosols. The gaseous absorption component computes the optical depth of the absorbing constituents in the atmosphere given the pressure, temperature, and absorber concentration profiles. The scattering component simply interpolates look-up-tables (LUTs) of optical properties, such as mass extinction coefficient and single scatter albedo for various cloud and aerosol types, that are then used in the radiative transfer component.

The surface optics category includes the computation of surface emissivity and reflectivity for four gross surface types (land, water, snow, and ice). Each gross surface type has a specified number of specific surface types associated with it. The CRTM utilises separate models for each gross surface type for each spectral type (infrared and microwave). These models can be physical models, database/LUT models, or models that derive emissivities from window channel radiances.

The radiative transfer components use the computed atmospheric and surface optics values to solve the radiative transfer problem in either clear or scattering atmospheres.

The gas absorption model fitting errors are of the order 0.1K (bias and RMS) with the current CompactOPTRAN algorithm. Recent updates have decreased these errors significantly for some broadband instruments. 

Regarding the bias in cloudy calculations, that is not known at the moment since large biases can be expected from uncertainties in cloud particle sizes and cloud fractions. The verification of cloudy and aerosol-laden radiances from the CRTM is still being investigated.


%\bibliographystyle{plainnat}
%\bibliography{bibliography}
\end{document}

